\documentclass[12pt,a4paper]{article}

% Paquetes
\usepackage[utf8]{inputenc}
\usepackage[T1]{fontenc}
\usepackage[spanish]{babel}
\usepackage{enumitem}
\usepackage{geometry}
\geometry{margin=2.5cm}

\title{ Colegio Técnico Profesional Santa Lucía}
\author{Preparación Ciencias 2026}
\date{}

\begin{document}
\maketitle

\section*{Preguntas de práctica}

\subsection*{Parte A: Preguntas de ejemplo oficiales}

\begin{enumerate}[label=\textbf{\arabic*.}]

\item Anita se sintió enferma, su mamá le mide la temperatura y obtiene un valor de 38 grados centígrados por lo que decide llevarla al médico.  
¿Cómo se llama el instrumento que usó la mamá de Anita para medir su temperatura?  
\begin{enumerate}[label=(\Alph*)]
\item Calorímetro  
\item Termómetro  
\item Escala  
\end{enumerate}

\item Instrumento utilizado en la ciencia para observar objetos que son sumamente pequeños, como células o bacterias, y que no se pueden ver a simple vista.  
\begin{enumerate}[label=(\Alph*)]
\item Microscopio  
\item Telescopio  
\item Láser  
\end{enumerate}

\item La relación que existe cuando un organismo se beneficia de otro y no le produce daño ni le da ningún beneficio. Por ejemplo: cuando las aves construyen sus nidos en un árbol.  
\begin{enumerate}[label=(\Alph*)]
\item Competencia  
\item Comensalismo  
\item Mutualismo  
\end{enumerate}

\item La reforestación permite el aumento de organismos autótrofos que producen oxígeno. Estos organismos renuevan el oxígeno de los ecosistemas.  
La información anterior corresponde a los organismos conocidos como:  
\begin{enumerate}[label=(\Alph*)]
\item Descomponedores  
\item Productores  
\item Heterótrofos  
\end{enumerate}

\item Claudia utilizó agua y agregó sirope para hacer un refresco. Las partes del refresco no se podían distinguir entre sí.  
¿Cómo se clasifica el refresco preparado?  
\begin{enumerate}[label=(\Alph*)]
\item Mezcla homogénea  
\item Compuesto  
\item Mezcla heterogénea  
\end{enumerate}

\end{enumerate}

% ========================
\subsection*{Parte B: Preguntas adicionales}

\begin{enumerate}[label=\textbf{\arabic*.}, resume]

\item ¿Cuál de las siguientes opciones describe mejor la diferencia entre calor y temperatura?  
\begin{enumerate}[label=(\Alph*)]
\item El calor es una forma de energía en tránsito y la temperatura mide el grado de agitación de las partículas.  
\item El calor mide la cantidad de movimiento de los átomos y la temperatura es energía en tránsito.  
\item Ambos son lo mismo.  
\end{enumerate}

\item Cuando la luz incide en un espejo plano y rebota manteniendo el mismo ángulo, se produce el fenómeno de:  
\begin{enumerate}[label=(\Alph*)]
\item Refracción  
\item Reflexión  
\item Dispersión  
\end{enumerate}

\item Un lápiz dentro de un vaso con agua parece estar “quebrado” debido al fenómeno de:  
\begin{enumerate}[label=(\Alph*)]
\item Reflexión  
\item Refracción  
\item Absorción  
\end{enumerate}

\item Las abejas recolectan néctar de las flores y, al mismo tiempo, ayudan a polinizarlas. Este tipo de relación se llama:  
\begin{enumerate}[label=(\Alph*)]
\item Parasitismo  
\item Mutualismo  
\item Comensalismo  
\end{enumerate}

\item Los hongos y bacterias que descomponen restos de materia orgánica se clasifican como:  
\begin{enumerate}[label=(\Alph*)]
\item Descomponedores  
\item Consumidores primarios  
\item Productores  
\end{enumerate}

\item El aire que respiramos (mezcla de nitrógeno, oxígeno y otros gases) se clasifica como:  
\begin{enumerate}[label=(\Alph*)]
\item Mezcla homogénea  
\item Mezcla heterogénea  
\item Compuesto  
\end{enumerate}
\newpage
\item Los peces pequeños que nadan cerca de tiburones y se alimentan de restos de comida sin afectar al tiburón muestran:  
\begin{enumerate}[label=(\Alph*)]
\item Competencia  
\item Comensalismo  
\item Mutualismo  
\end{enumerate}

\item Un ejemplo de productor en un ecosistema acuático es:  
\begin{enumerate}[label=(\Alph*)]
\item El fitoplancton  
\item Los peces  
\item Los cangrejos  
\end{enumerate}

\item El agua salada del mar es una:  
\begin{enumerate}[label=(\Alph*)]
\item Mezcla homogénea  
\item Mezcla heterogénea  
\item Sustancia pura  
\end{enumerate}

\item Un aumento de temperatura en un cuerpo implica que:  
\begin{enumerate}[label=(\Alph*)]
\item Las partículas están más agitadas y aumenta su energía cinética promedio.  
\item El cuerpo ganó masa.  
\item El cuerpo se transformó en otro material.  
\end{enumerate}

\end{enumerate}

% ========================
\subsection*{Parte C: Preguntas similares}

\begin{enumerate}[label=\textbf{\arabic*.}, resume]

\item ¿Cuál de los siguientes materiales es un ejemplo de sustancia pura?  
\begin{enumerate}[label=(\Alph*)]
\item Agua destilada  
\item Aire  
\item Leche  
\end{enumerate}

\item Cuando un rayo de luz pasa del aire al vidrio y cambia su dirección, se produce:  
\begin{enumerate}[label=(\Alph*)]
\item Reflexión  
\item Refracción  
\item Difracción  
\end{enumerate}

\item Los leones cazan cebras en la sabana africana. Este tipo de relación es:  
\begin{enumerate}[label=(\Alph*)]
\item Depredación  
\item Mutualismo  
\item Comensalismo  
\end{enumerate}
\newpage
\item El oxígeno que respiramos es un:  
\begin{enumerate}[label=(\Alph*)]
\item Elemento  
\item Compuesto  
\item Mezcla  
\end{enumerate}

\item Los caracoles que comen hojas son:  
\begin{enumerate}[label=(\Alph*)]
\item Productores  
\item Consumidores primarios  
\item Descomponedores  
\end{enumerate}

\item La luz que incide sobre un vidrio empañado no lo atraviesa totalmente, sino que se dispersa. El vidrio se clasifica como:  
\begin{enumerate}[label=(\Alph*)]
\item Transparente  
\item Translúcido  
\item Opaco  
\end{enumerate}

\item El azúcar disuelta en agua se clasifica como:  
\begin{enumerate}[label=(\Alph*)]
\item Mezcla homogénea  
\item Mezcla heterogénea  
\item Sustancia pura  
\end{enumerate}

\item Los parásitos intestinales que viven dentro de un organismo huésped se benefician, pero dañan al huésped. Esta relación es:  
\begin{enumerate}[label=(\Alph*)]
\item Mutualismo  
\item Parasitismo  
\item Comensalismo  
\end{enumerate}

\item Los restos de animales muertos que son transformados en nutrientes del suelo corresponden a la acción de:  
\begin{enumerate}[label=(\Alph*)]
\item Productores  
\item Descomponedores  
\item Consumidores secundarios  
\end{enumerate}

\item La unidad básica para medir la temperatura en el Sistema Internacional es:  
\begin{enumerate}[label=(\Alph*)]
\item Grado Celsius (\(^\circ\)C)  
\item Kelvin (K)  
\item Grado Fahrenheit (\(^\circ\)F)  
\end{enumerate}

\end{enumerate}

% ========================
\newpage
\section*{Respuestas}

\begin{enumerate}
\item (B) Termómetro  
\item (A) Microscopio  
\item (B) Comensalismo  
\item (B) Productores  
\item (A) Mezcla homogénea  
\item (A) El calor es energía en tránsito y la temperatura mide el grado de agitación.  
\item (B) Reflexión  
\item (B) Refracción  
\item (B) Mutualismo  
\item (A) Descomponedores  
\item (A) Mezcla homogénea  
\item (B) Comensalismo  
\item (A) Fitoplancton  
\item (A) Mezcla homogénea  
\item (A) Las partículas están más agitadas.  
\item (A) Agua destilada  
\item (B) Refracción  
\item (A) Depredación  
\item (A) Elemento  
\item (B) Consumidores primarios  
\item (B) Translúcido  
\item (A) Mezcla homogénea  
\item (B) Parasitismo  
\item (B) Descomponedores  
\item (B) Kelvin (K)  
\end{enumerate}

\end{document}