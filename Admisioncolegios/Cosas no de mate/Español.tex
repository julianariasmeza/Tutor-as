\documentclass[12pt]{article}

% -------------------- Paquetes básicos --------------------
\usepackage[spanish]{babel}
\usepackage[utf8]{inputenc}
\usepackage[T1]{fontenc}
\usepackage[a4paper, margin=2.3cm]{geometry}
\usepackage{setspace}
\onehalfspacing
\usepackage{parskip}
\setlength{\parindent}{0pt}
% === Encabezados y pies de página ===
\usepackage{fancyhdr}
\pagestyle{fancy}
\fancyhf{} % limpia encabezados y pies
\rhead{Realizado por Julián Arias Meza} % esquina derecha
\cfoot{\thepage} % número de página en el centro del pie
% Tablas y listas
\usepackage{booktabs}
\usepackage{array}
\usepackage{enumitem}
\setlist[itemize]{topsep=3pt,itemsep=2pt}
\setlist[enumerate]{topsep=6pt,itemsep=4pt}

% Colores e hipervínculos (opcional)
\usepackage{xcolor}
\definecolor{azul}{RGB}{25,80,160}
\usepackage{hyperref}
\hypersetup{colorlinks=true, linkcolor=azul, urlcolor=azul}

% Comandos útiles
\newcommand{\ejemplo}[1]{\textit{Ejemplo:} #1}
\newcommand{\clave}[1]{\textbf{\textcolor{azul}{#1}}}

\begin{document}

\begin{center}
  {\LARGE \textbf{Gramática básica del español (Primaria)}}\\[2mm]
  {\large Adverbios, preposiciones, conjunciones y otras clases de palabras}
\end{center}

\vspace{0.5em}

Este material explica con palabras simples y ejemplos cotidianos algunas clases de palabras del español. Incluye una tabla–resumen y actividades para practicar.

% ================== ADVERBIOS ==================
\section*{1. Adverbios}
\textbf{¿Qué son?} Palabras que acompañan a verbos, adjetivos u otros adverbios y dicen \textbf{cómo, cuándo, dónde o cuánto} ocurre algo.

\textbf{Tipos frecuentes:}
\begin{itemize}
  \item \textbf{Lugar:} aquí, allá, cerca, lejos.
  \item \textbf{Tiempo:} ayer, hoy, mañana, siempre.
  \item \textbf{Modo:} bien, mal, rápido, despacio.
  \item \textbf{Cantidad:} mucho, poco, demasiado.
\end{itemize}

\ejemplo{Pedro corre \clave{rápido}. \quad Mañana iremos \clave{allí}.}

% ================== PREPOSICIONES ==================
\section*{2. Preposiciones}
\textbf{¿Qué son?} Palabras pequeñas que \textbf{unen} otras palabras y muestran relación de \textbf{lugar, tiempo, causa o compañía}.

\textbf{Más usadas en primaria:}\\
\emph{a, ante, bajo, con, contra, de, desde, durante, en, entre, hacia, hasta, mediante, para, por, según, sin, sobre, tras}.

\ejemplo{El libro está \clave{sobre} la mesa. \quad Voy \clave{con} mi mamá. \quad Caminamos \clave{hacia} el parque.}
\newpage
% ================== CONJUNCIONES ==================
\section*{3. Conjunciones}
\textbf{¿Qué son?} Palabras que \textbf{unen} palabras u oraciones.

\textbf{Tipos comunes:}
\begin{itemize}
  \item \textbf{Copulativas (suman ideas):} y, e, ni.\\
  \ejemplo{María \clave{y} Juan juegan.}
  \item \textbf{Disyuntivas (opción):} o, u.\\
  \ejemplo{¿Querés jugo \clave{o} agua?}
  \item \textbf{Adversativas (oponen):} pero, sino, mas.\\
  \ejemplo{Quiero ir, \clave{pero} estoy cansado.}
\end{itemize}

(Otras que podés presentar más adelante: \emph{porque, pues, ya que} —causales—; \emph{por tanto, así que} —consecutivas—; \emph{si} —condicional—).

% ================== OTRAS CLASES ==================
\section*{4. Otras clases de palabras (repaso rápido)}
\begin{itemize}
  \item \textbf{Sustantivo:} nombra personas, animales, cosas o ideas.\\
  \ejemplo{perro, niña, montaña, amor.}
  \item \textbf{Adjetivo:} describe al sustantivo (dice cómo es).\\
  \ejemplo{El perro \clave{grande} duerme.}
  \item \textbf{Verbo:} expresa acción o estado.\\
  \ejemplo{Pedro \clave{corre}.}
\end{itemize}

% ================== TABLA RESUMEN ==================
\section*{5. Tabla–resumen}
\renewcommand{\arraystretch}{1.25}
\begin{tabular}{@{}p{3.2cm}p{6.2cm}p{6.2cm}@{}}
\toprule
\textbf{Tipo de palabra} & \textbf{¿Qué hace?} & \textbf{Ejemplo} \\
\midrule
\textbf{Adverbio} & Dice cómo, cuándo, dónde o cuánto. & Camina \textbf{rápido}. / Iremos \textbf{mañana}. \\
\textbf{Preposición} & Une palabras; indica lugar, tiempo, causa, compañía. & El gato está \textbf{en} la caja. / Voy \textbf{con} mamá. \\
\textbf{Conjunción} & Une palabras u oraciones. & Juan \textbf{y} María cantan. / ¿Té \textbf{o} café? \\
\textbf{Sustantivo} & Nombra seres, cosas o ideas. & \textbf{niño}, \textbf{perro}, \textbf{casa}. \\
\textbf{Adjetivo} & Describe al sustantivo. & casa \textbf{grande}, flor \textbf{roja}. \\
\textbf{Verbo} & Expresa acción o estado. & Ana \textbf{salta}. / Yo \textbf{estoy} feliz. \\
\bottomrule
\end{tabular}
% ================== FIGURAS LITERARIAS ==================
\newpage
\section*{7. Figuras literarias}

Las figuras literarias son recursos del lenguaje que hacen que lo que decimos o escribimos sea más \textbf{bonito, expresivo o llamativo}. A continuación, se explican las más importantes con ejemplos sencillos:

\subsection*{Metáfora}
\textbf{¿Qué es?} Consiste en identificar directamente una cosa con otra, resaltando una semejanza.  
\ejemplo{Tus ojos son dos luceros.}  
(Los ojos no son luceros de verdad, pero brillan como ellos).

\subsection*{Símil o comparación}
\textbf{¿Qué es?} Compara dos cosas usando palabras como \emph{como, cual, semejante a, parecido a}.  
\ejemplo{Tus ojos brillan como luceros.}  

\subsection*{Prosopopeya o personificación}
\textbf{¿Qué es?} Dar cualidades humanas a objetos, animales o ideas.  
\ejemplo{El viento susurra entre los árboles.}  

\subsection*{Hipérbole}
\textbf{¿Qué es?} Exagerar mucho una cualidad o acción para dar énfasis.  
\ejemplo{Te llamé mil veces y no contestaste.}  

\subsection*{Anáfora}
\textbf{¿Qué es?} Repetir una o varias palabras al inicio de versos o frases.  
\ejemplo{Temprano levantó la muerte el vuelo,  
temprano madrugó la madrugada...}  

\subsection*{Onomatopeya}
\textbf{¿Qué es?} Usar palabras que imitan sonidos reales.  
\ejemplo{El gato hace miau. \quad El reloj tic-tac.}  

\subsection*{Aliteración}
\textbf{¿Qué es?} Repetir sonidos iguales o parecidos para dar musicalidad.  
\ejemplo{Mi mamá me mima mucho.}  

\subsection*{Paradoja}
\textbf{¿Qué es?} Expresar una idea que parece contradictoria, pero tiene sentido.  
\ejemplo{Vivo sin vivir en mí.}  

\subsection*{Ironía}
\textbf{¿Qué es?} Decir lo contrario de lo que en realidad se quiere dar a entender.  
\ejemplo{¡Qué puntual! (a alguien que siempre llega tarde).}  

\subsection*{Elipsis}
\textbf{¿Qué es?} Omitir palabras porque se sobreentienden.  
\ejemplo{Yo llevé flores; ella, (llevó) dulces.}  

% ================== TABLA RESUMEN ==================
\section*{8. Tabla–resumen de figuras literarias}
\renewcommand{\arraystretch}{1.3}
\begin{tabular}{@{}p{3.5cm}p{7cm}p{6cm}@{}}
\toprule
\textbf{Figura} & \textbf{Definición} & \textbf{Ejemplo} \\
\midrule
\textbf{Metáfora} & Identifica una cosa con otra. & Tus ojos son dos luceros. \\
\textbf{Símil} & Compara usando ``como'', ``cual''. & Corre como un rayo. \\
\textbf{Prosopopeya} & Da cualidades humanas a cosas. & El río cantaba. \\
\textbf{Hipérbole} & Exageración. & Lloré un mar de lágrimas. \\
\textbf{Anáfora} & Repetición al inicio. & Temprano madrugó la madrugada. \\
\textbf{Onomatopeya} & Imitación de sonidos. & Tic-tac, miau. \\
\textbf{Aliteración} & Repetición de sonidos. & Mi mamá me mima. \\
\textbf{Paradoja} & Contradicción aparente. & Vivo sin vivir en mí. \\
\textbf{Ironía} & Decir lo contrario. & ¡Qué bonito desastre hiciste! \\
\textbf{Elipsis} & Omitir palabras sobreentendidas. & Yo llevé flores; ella, dulces. \\
\bottomrule
\end{tabular}
% ================== ACTIVIDADES ==================
\newpage
\section*{9. Actividades (para el aula)}

\subsection*{Actividad 1: Adverbios}
Subrayá el \textbf{adverbio} en cada oración e indicá si es de lugar, tiempo, modo o cantidad.
\begin{enumerate}
  \item Hoy iremos al parque.
  \item El auto se mueve despacio.
  \item Poné la mochila aquí.
  \item Estudié mucho para la prueba.
\end{enumerate}

\subsection*{Actividad 2: Preposiciones}
Rodeá la \textbf{preposición} y señalá qué relación expresa (lugar, tiempo, compañía, causa).
\begin{enumerate}
  \item El cuaderno está sobre la mesa.
  \item Caminamos hacia la escuela.
  \item Fui al cine con mi hermano.
  \item Estudié durante dos horas.
\end{enumerate}

\subsection*{Actividad 3: Conjunciones}
Completá con la \textbf{conjunción} correcta: \emph{y / o / pero}.
\begin{enumerate}
  \item Quiero helado, \underline{\hspace{1.2cm}} no hay de chocolate.
  \item ¿Querés lápiz \underline{\hspace{1.2cm}} lapicero?
  \item Sofía \underline{\hspace{1.2cm}} Tomás juegan al ajedrez.
\end{enumerate}

\subsection*{Actividad 4: Clasificación}
Marcá qué clase de palabra es la destacada en cada caso (sustantivo, adjetivo, verbo, adverbio, preposición, conjunción).
\begin{enumerate}
  \item La \textbf{casa} es azul.
  \item Lucía canta \textbf{bien}.
  \item El lápiz está \textbf{en} la mesa.
  \item Pedro \textbf{y} Ana estudian.
  \item Juan \textbf{corre} rápido.
  \item Iremos \textbf{mañana}.
\end{enumerate}

% ================== SOLUCIONES ==================
\section*{Soluciones (para el docente)}
\textit{Sugerencia: ocultá esta sección para entregar la guía al estudiantado.}
\begin{itemize}
  \item \textbf{Act. 1} \\
  1) \emph{Hoy} (tiempo) \quad
  2) \emph{despacio} (modo) \quad
  3) \emph{aquí} (lugar) \quad
  4) \emph{mucho} (cantidad).
  \item \textbf{Act. 2} \\
  1) \emph{sobre} (lugar) \quad
  2) \emph{hacia} (dirección/lugar) \quad
  3) \emph{con} (compañía) \quad
  4) \emph{durante} (tiempo).
  \item \textbf{Act. 3} \\
  1) \emph{pero} \quad
  2) \emph{o} \quad
  3) \emph{y}.
  \item \textbf{Act. 4} \\
  1) \emph{casa} = sustantivo \quad
  2) \emph{bien} = adverbio \quad
  3) \emph{en} = preposición \quad
  4) \emph{y} = conjunción \quad
  5) \emph{corre} = verbo \quad
  6) \emph{mañana} = adverbio (tiempo).
\end{itemize}

\vfill
\hrule
\small
\textit{Material elaborado para nivel de primaria. Podés adaptar ejemplos según el contexto del grupo.}

\end{document}