% === Documento completo de Español ===
\documentclass[12pt]{article}
\usepackage[utf8]{inputenc}
\usepackage[spanish]{babel}
\usepackage{geometry}
\geometry{letterpaper, margin=2.5cm}
\setlength{\parskip}{1ex}
\setlength{\parindent}{0pt}
\usepackage{fancyhdr}
\pagestyle{fancy}
\fancyhf{} % limpia encabezados y pies
\rhead{Realizado por Julián Arias Meza} % esquina derecha
\cfoot{\thepage} % número de página en el centro del pie
% === Datos del documento ===
\title{\textbf{Guía Práctica – Examen de Admisión}\\
Colegio Técnico Profesional de Santa Lucía 2026}
\author{Área de Español}
\date{}

\begin{document}
\maketitle

\section*{Instrucciones}
Este documento contiene ejercicios de práctica de Español para el examen de admisión.  
Cada pregunta tiene tres opciones de respuesta. Seleccione la opción correcta.  

---

\section*{Ejercicios de Práctica – Español}

\begin{enumerate}
\item De acuerdo con las palabras subrayadas en una carta formal, ¿en qué casos se utiliza la mayúscula?
\begin{itemize}
  \item[a)] Al iniciar una oración.
  \item[b)] En el uso de abreviaturas.
  \item[c)] Al escribir sustantivos propios.
\end{itemize}

\item Según el fragmento: “Era de rigor que la mujer tuviera compañero...”, se deduce que:
\begin{itemize}
  \item[a)] Todas las jóvenes agradecidas eligieron vivir con sus padres y no casarse.
  \item[b)] El matrimonio siempre produce felicidad a pesar de las dificultades.
  \item[c)] Tradicionalmente la mujer debía casarse.
\end{itemize}

\item En la fábula “La liebre y la tortuga”, la característica principal del género es:
\begin{itemize}
  \item[a)] Narrativa lineal y estructura definida.
  \item[b)] Inclusión de una moraleja o enseñanza.
  \item[c)] Uso de descripciones detalladas de ambientes.
\end{itemize}
\newpage
\item Lea el siguiente texto: “El ágil gato persiguió al ratón veloz...”.  
Las palabras subrayadas se clasifican como:
\begin{itemize}
  \item[a)] Tres palabras agudas y tres palabras graves.
  \item[b)] Cuatro palabras graves y dos palabras agudas.
  \item[c)] Cuatro palabras graves y dos palabras esdrújulas.
\end{itemize}

\item En el texto: “El bosque encantado estaba lleno de árboles altos y sombras misteriosas...”, las palabras subrayadas corresponden a:
\begin{itemize}
  \item[a)] Tres adjetivos calificativos y tres adjetivos posesivos.
  \item[b)] Tres adjetivos posesivos y tres adjetivos calificativos.
  \item[c)] Cuatro adjetivos calificativos y dos adjetivos posesivos.
\end{itemize}
\end{enumerate}

---

\section*{Clave de Respuestas}
1-a, 2-c, 3-b, 4-b, 5-c  
\section*{Ejercicios Adicionales – Español}

\begin{enumerate}
\setcounter{enumi}{5}

\item En la oración: “Los estudiantes escribieron rápidamente sus respuestas”, la palabra \textit{rápidamente} es:
\begin{itemize}
  \item[a)] Un adjetivo calificativo.
  \item[b)] Un adverbio de modo.
  \item[c)] Un sustantivo abstracto.
\end{itemize}

\item En la oración: “María y Juan visitaron el Museo Nacional”, la palabra \textit{Museo Nacional} se escribe con mayúscula porque:
\begin{itemize}
  \item[a)] Es el inicio de una oración.
  \item[b)] Es un sustantivo propio.
  \item[c)] Es un título inventado.
\end{itemize}

\item ¿En cuál de las siguientes oraciones se usa un adjetivo posesivo?
\begin{itemize}
  \item[a)] “Ellos jugaron fútbol en la plaza.”
  \item[b)] “Mi cuaderno está en la mesa.”
  \item[c)] “El perro corre rápido.”
\end{itemize}
 \newpage
\item La palabra \textit{televisión} es:
\begin{itemize}
  \item[a)] Aguda.
  \item[b)] Grave.
  \item[c)] Esdrújula.
\end{itemize}

\item La palabra \textit{matemáticas} es:
\begin{itemize}
  \item[a)] Aguda.
  \item[b)] Grave.
  \item[c)] Esdrújula.
\end{itemize}

\item En el texto: “El día estaba soleado y las flores brillaban hermosas en el jardín”, los adjetivos calificativos son:
\begin{itemize}
  \item[a)] Día, flores, jardín.
  \item[b)] Soleado, hermosas.
  \item[c)] Estaba, brillaban.
\end{itemize}

\item ¿Cuál de las siguientes oraciones tiene un error de tilde?
\begin{itemize}
  \item[a)] “El arbol del parque es muy alto.”
  \item[b)] “La música alegró la fiesta.”
  \item[c)] “Los jóvenes participaron en el evento.”
\end{itemize}

\item ¿Cuál es un sinónimo de la palabra \textit{feliz}?
\begin{itemize}
  \item[a)] Contento.
  \item[b)] Triste.
  \item[c)] Cansado.
\end{itemize}

\item ¿Cuál es un antónimo de la palabra \textit{rápido}?
\begin{itemize}
  \item[a)] Ligero.
  \item[b)] Lento.
  \item[c)] Veloz.
\end{itemize}

\item En la oración: “La casa está al lado del río”, la palabra \textit{del} es:
\begin{itemize}
  \item[a)] Una contracción de “de + el”.
  \item[b)] Una preposición simple.
  \item[c)] Un adjetivo demostrativo.
\end{itemize}

\end{enumerate}

\section*{Clave de Respuestas – Ejercicios Adicionales}
6-b, 7-b, 8-b, 9-a, 10-c, 11-b, 12-a, 13-a, 14-b, 15-a
\end{document}