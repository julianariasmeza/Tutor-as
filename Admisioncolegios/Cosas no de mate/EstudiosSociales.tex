\documentclass[12pt]{article}
\usepackage[utf8]{inputenc}
\usepackage[spanish]{babel}
\usepackage{geometry}
\geometry{letterpaper, margin=2.5cm}
\usepackage{multicol}
\setlength{\parskip}{1ex}
\usepackage{fancyhdr}
\pagestyle{fancy}
\fancyhf{} % limpia encabezados y pies
\rhead{Realizado por Julián Arias Meza} % esquina derecha
\cfoot{\thepage} % número de página en el centro del pie
\title{\textbf{Guía Práctica – Examen de Admisión}\\
Colegio Técnico Profesional de Santa Lucía 2026}
\date{}

\begin{document}
\maketitle

\section*{Instrucciones}
Esta guía contiene explicaciones breves y ejercicios de práctica de Estudios Sociales y Educación Cívica. 
Marque la respuesta correcta en cada ítem. 
Al final encontrará la clave de respuestas.

\section*{Tema 1. Conquista y Colonización de Costa Rica}
\textbf{Resumen:} 
Durante la colonia, la población indígena fue explotada mediante encomiendas y repartimientos. Hubo abusos, enfermedades y concentración de la riqueza en los colonizadores. 

\textbf{Ejercicios:}
\begin{enumerate}
\item Una característica de la economía colonial en Costa Rica fue:
\begin{itemize}
  \item[a)] Igual distribución de riquezas entre españoles e indígenas.
  \item[b)] Explotación de la mano de obra indígena.
  \item[c)] Comercio internacional con tratados de libre comercio.
\end{itemize}

\item Costa Rica perteneció durante la colonia a:
\begin{itemize}
  \item[a)] El Virreinato de Nueva España.
  \item[b)] El Virreinato de Granada.
  \item[c)] El Virreinato del Río de la Plata.
\end{itemize}

\item Una consecuencia de la colonización en la población indígena fue:
\begin{itemize}
  \item[a)] Aumento de la población nativa.
  \item[b)] Disminución por enfermedades y explotación.
  \item[c)] Acceso igualitario a tierras y riqueza.
\end{itemize}
\end{enumerate}

\section*{Tema 2. Sociedad Colonial}
\textbf{Resumen:} 
La sociedad colonial tenía una jerarquía rígida: españoles en la cúspide, mestizos en el medio, indígenas y esclavos en la base.  

\textbf{Ejercicios:}
\begin{enumerate}
\setcounter{enumi}{3}
\item La sociedad colonial se caracterizó por:
\begin{itemize}
  \item[a)] Una jerarquía rígida donde los españoles ocupaban los principales cargos.
  \item[b)] Igualdad de derechos civiles y religiosos entre indígenas y españoles.
  \item[c)] Eliminación de diferencias sociales por la mezcla cultural.
\end{itemize}

\item La principal forma de organización económica durante la colonia fue:
\begin{itemize}
  \item[a)] El comercio internacional.
  \item[b)] La producción agrícola de subsistencia.
  \item[c)] La industrialización de las ciudades.
\end{itemize}

\item En la colonia, los cargos políticos y religiosos estaban reservados a:
\begin{itemize}
  \item[a)] Indígenas principales.
  \item[b)] Españoles peninsulares y criollos.
  \item[c)] Africanos esclavizados.
\end{itemize}
\end{enumerate}

\section*{Tema 3. Símbolos Patrios y Patrimonio Cultural}
\textbf{Resumen:} 
Los símbolos patrios y la marimba como instrumento nacional representan la identidad costarricense.  

\textbf{Ejercicios:}
\begin{enumerate}
\setcounter{enumi}{6}
\item La marimba fue declarada instrumento musical nacional en 1996 como respuesta a:
\begin{itemize}
  \item[a)] Su creciente uso en la música internacional.
  \item[b)] Una petición de las nuevas generaciones.
  \item[c)] El desplazamiento de la marimba por instrumentos electrónicos.
\end{itemize}

\item Un símbolo patrio que representa la biodiversidad de Costa Rica es:
\begin{itemize}
  \item[a)] El escudo nacional.
  \item[b)] La guaria morada.
  \item[c)] El himno nacional.
\end{itemize}
\newpage
\item La declaración de la marimba como instrumento nacional buscó principalmente:
\begin{itemize}
  \item[a)] Proteger un elemento cultural amenazado.
  \item[b)] Introducir un nuevo género musical.
  \item[c)] Sustituir al piano como instrumento popular.
\end{itemize}
\end{enumerate}

\section*{Tema 4. Campaña Nacional (1856–1857)}
\textbf{Resumen:} 
La Campaña Nacional contra William Walker defendió la soberanía costarricense y la identidad nacional.  

\textbf{Ejercicios:}
\begin{enumerate}
\setcounter{enumi}{9}
\item Una consecuencia de la Campaña Nacional fue:
\begin{itemize}
  \item[a)] La derrota de William Walker y su expulsión.
  \item[b)] La división de los países centroamericanos.
  \item[c)] El debilitamiento de la identidad costarricense.
\end{itemize}

\item La Campaña Nacional fortaleció principalmente:
\begin{itemize}
  \item[a)] La esclavitud.
  \item[b)] La soberanía nacional.
  \item[c)] El absolutismo monárquico.
\end{itemize}
\end{enumerate}

\section*{Tema 5. Reformas Sociales de la Década de 1940}
\textbf{Resumen:} 
Las reformas de los años 40 crearon instituciones como la CCSS y el Código de Trabajo, fundamentales en el Estado social costarricense.  

\textbf{Ejercicios:}
\begin{enumerate}
\setcounter{enumi}{11}
\item Una de las instituciones creadas en la década de 1940 fue:
\begin{itemize}
  \item[a)] El Tribunal Supremo de Elecciones.
  \item[b)] La Caja Costarricense de Seguro Social (CCSS).
  \item[c)] La Universidad Nacional (UNA).
\end{itemize}

\item El objetivo principal de las reformas sociales de los años 40 fue:
\begin{itemize}
  \item[a)] Garantizar armonía social y mejores condiciones de vida.
  \item[b)] Fortalecer el poder militar.
  \item[c)] Facilitar la migración hacia Estados Unidos.
\end{itemize}

\item Entre los logros de la década de 1940 se encuentran:
\begin{itemize}
  \item[a)] El Código de Trabajo y la creación de la CCSS.
  \item[b)] La abolición del ejército y el voto femenino.
  \item[c)] La independencia de Centroamérica.
\end{itemize}
\end{enumerate}

\section*{Tema 6. Pensamiento Crítico y Democracia}
\textbf{Resumen:} 
Se espera que los estudiantes comprendan cómo hechos como la corrupción afectan la democracia, la soberanía y el desarrollo.  

\textbf{Ejercicio:}
\begin{enumerate}
\setcounter{enumi}{14}
\item Según el material, la corrupción política en Costa Rica provoca:
\begin{itemize}
  \item[a)] Fortalecimiento de las relaciones internacionales.
  \item[b)] Estabilidad democrática y crecimiento económico.
  \item[c)] Inestabilidad política y retroceso democrático.
\end{itemize}
\end{enumerate}

\section*{Clave de respuestas}
1-b, 2-a, 3-b, 4-a, 5-b, 6-b, 7-c, 8-b, 9-a, 10-a, 11-b, 12-a, 13-a, 14-a, 15-c
\section*{Ejercicios Adicionales de Práctica}

\begin{enumerate}
\item Durante la colonia, una de las principales causas de la disminución de la población indígena fue:
\begin{itemize}
  \item[a)] La migración hacia Europa.  
  \item[b)] Las enfermedades y las duras condiciones de trabajo.  
  \item[c)] La industrialización en las ciudades.  
\end{itemize}

\item La Capitanía General de Guatemala, a la cual pertenecía Costa Rica, formaba parte del:
\begin{itemize}
  \item[a)] Virreinato del Río de la Plata.  
  \item[b)] Virreinato de Nueva España.  
  \item[c)] Virreinato del Perú.  
\end{itemize}
\newpage
\item Un símbolo patrio que representa la paz en Costa Rica es:
\begin{itemize}
  \item[a)] La bandera nacional.  
  \item[b)] El escudo nacional.  
  \item[c)] La paloma blanca.  
\end{itemize}

\item ¿Qué valor fomentó la Campaña Nacional de 1856–1857?
\begin{itemize}
  \item[a)] La sumisión frente a potencias extranjeras.  
  \item[b)] La defensa de la soberanía nacional.  
  \item[c)] El fortalecimiento del ejército permanente.  
\end{itemize}

\item Entre las reformas sociales impulsadas en los años 40 en Costa Rica se incluye:
\begin{itemize}
  \item[a)] La creación del Código de Trabajo.  
  \item[b)] La fundación del ICE.  
  \item[c)] La abolición del ejército.  
\end{itemize}

\item La corrupción política se considera una amenaza porque:
\begin{itemize}
  \item[a)] Fortalece la democracia.  
  \item[b)] Genera incertidumbre y debilita la confianza ciudadana.  
  \item[c)] Permite mayor participación de la sociedad civil.  
\end{itemize}

\item ¿Cuál de los siguientes NO es un símbolo patrio de Costa Rica?
\begin{itemize}
  \item[a)] El yigüirro.  
  \item[b)] La guaria morada.  
  \item[c)] El café.  
\end{itemize}

\item ¿Quién fue presidente de Costa Rica durante la Campaña Nacional?
\begin{itemize}
  \item[a)] Juan Rafael Mora Porras.  
  \item[b)] José Figueres Ferrer.  
  \item[c)] Braulio Carrillo Colina.  
\end{itemize}

\item El objetivo de declarar la marimba como instrumento nacional fue:
\begin{itemize}
  \item[a)] Sustituir los tambores y guitarras en las escuelas.  
  \item[b)] Rescatar y proteger un elemento cultural en riesgo de desaparecer.  
  \item[c)] Aumentar las exportaciones de instrumentos musicales.  
\end{itemize}

\item Una de las consecuencias de las reformas sociales de los años 40 fue:
\begin{itemize}
  \item[a)] El acceso universal a la salud.  
  \item[b)] La prohibición de sindicatos.  
  \item[c)] La eliminación de las universidades públicas.  
\end{itemize}
\end{enumerate}

\section*{Respuestas Ejercicios Adicionales}
1-b, 2-b, 3-b, 4-b, 5-a, 6-b, 7-c, 8-a, 9-b, 10-a
\end{document}