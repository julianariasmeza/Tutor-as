\documentclass[12pt]{article}
\usepackage[spanish]{babel}
\usepackage[utf8]{inputenc}
\usepackage[T1]{fontenc}
\usepackage{amsmath, amssymb}
\usepackage{geometry}
\geometry{letterpaper, margin=2.5cm}
\usepackage{fancyhdr}
\pagestyle{fancy}
\fancyhf{} % limpia encabezados y pies
\rhead{Realizado por Julián Arias Meza} % esquina derecha
\cfoot{\thepage} % número de página en el centro del pie
\begin{document}

\begin{center}
\Large \textbf{Práctica de Razonamiento Matemático}\\
\small Estilo examen de admisión CTP Santa Lucía
\end{center}

\vspace{1cm}

\textbf{Instrucciones:} Resuelva cada uno de los siguientes ejercicios sin calculadora. Escriba el procedimiento claramente.

\vspace{0.5cm}

\begin{enumerate}

\item En una tienda, un cuaderno cuesta ₡850. Si Daniel compra 15 cuadernos y paga con un billete de ₡20 000, ¿cuánto cambio recibe?

\item Laura desea repartir 245 caramelos en bolsas de 8 caramelos cada una. ¿Cuántos caramelos le faltan para que las bolsas queden completas sin que sobre ninguno?

\item Para construir un marco se requieren 2 piezas de 62,5 cm y 2 piezas de 84,3 cm. Si se dispone de reglas de 150 cm, 250 cm y 300 cm, ¿qué regla se debe escoger para que el desperdicio sea mínimo?

\item Un boleto de cine cuesta ₡3 200. Si un grupo de 7 amigos compra boletos y paga con ₡30 000, ¿cuánto dinero les sobra?

\item Un agricultor tiene 178 naranjas y quiere empacarlas en cajas de 12 naranjas cada una. ¿Cuántas naranjas le hacen falta para que todas las cajas estén completas?

\item Se desea construir una mesa con 2 tablas de 1,25 m y 2 tablas de 1,75 m. Si la ferretería vende tablas de 3,00 m y de 4,00 m, ¿cuál conviene comprar para tener el menor desperdicio?

\item Pedro compra 9 gaseosas a ₡1 450 cada una. Si paga con ₡20 000, ¿cuánto cambio recibe?

\item Una fábrica empaca chocolates en cajas de 24 unidades. Si produce 970 chocolates en un día, ¿cuántos chocolates le faltan para completar la última caja?

\item Para un proyecto se requieren 3 varillas de 1,20 m y 2 varillas de 0,85 m. Si en la ferretería hay varillas de 2,00 m, 3,00 m y 4,00 m, ¿qué medida debe escogerse para que el desperdicio sea el menor posible?

\item En una librería, una carpeta cuesta ₡2 350. Si Andrea compra 8 carpetas y paga con ₡20 000, ¿cuánto cambio recibe?

\item Luis tiene 215 canicas y desea repartirlas en bolsas de 9 canicas cada una. ¿Cuántas canicas le faltan para que todas las bolsas estén completas?

\item Un marco requiere 2 piezas de 42,7 cm y 2 piezas de 55,6 cm. ¿Cuál de las siguientes longitudes de madera se debe elegir: 120 cm, 220 cm o 280 cm?

\item El precio de un helado es de ₡950. Si un grupo de 18 estudiantes compra cada uno un helado y pagan en total con ₡20 000, ¿cuánto dinero falta o sobra?

\item Un distribuidor tiene 326 botellas y desea empacarlas en cajas de 15 botellas cada una. ¿Cuántas botellas le faltan para que todas las cajas estén completas?

\item Para hacer una cerca se necesitan 4 tablas de 0,90 m y 2 tablas de 1,35 m. ¿Qué medida de tabla conviene comprar entre 3,00 m, 3,60 m o 4,20 m para desperdiciar menos material?

\end{enumerate}
\section*{Soluciones}

\begin{enumerate}
% 1
\item \textbf{Cambio por compra de cuadernos}
\begin{align*}
\text{Datos:}&\ \ p=\text{₡}850,\quad n=15,\quad P=\text{₡}20\,000.\\
\text{Costo:}&\ \ C=n\cdot p=15\cdot 850=\text{₡}12\,750.\\
\text{Cambio:}&\ \ V=P-C=20\,000-12\,750=\boxed{\text{₡}7\,250}.
\end{align*}

% 2
\item \textbf{Bolsas de 8}
\begin{align*}
\text{Datos:}&\ \ N=245,\quad b=8.\\
245&=8\cdot 30+5\ \Rightarrow\ \text{sobran }5.\\
\text{Faltan:}&\ \ 8-5=\boxed{3}.
\end{align*}

% 3
\item \textbf{Marco (62,5 cm y 84,3 cm)}
\begin{align*}
L_\text{necesario}&=2\cdot 62{,}5+2\cdot 84{,}3=125{,}0+168{,}6= \boxed{293{,}6\ \text{cm}}.\\
\text{Reglas:}&\ 150\ (\text{insuf.}),\ 250\ (\text{insuf.}),\ 300\ (\text{sobra }6{,}4).\\
\Rightarrow&\ \boxed{300\ \text{cm}}.
\end{align*}

% 4
\item \textbf{Boletos de cine}
\begin{align*}
C&=7\cdot \text{₡}3\,200=\text{₡}22\,400,\\
\text{Cambio}&=30\,000-22\,400=\boxed{\text{₡}7\,600}.
\end{align*}

% 5
\item \textbf{Cajas de 12}
\begin{align*}
178&=12\cdot 14+10\ \Rightarrow\ \text{sobran }10.\\
\text{Faltan}&=12-10=\boxed{2}.
\end{align*}

% 6
\item \textbf{Tablas de 1,25 m y 1,75 m}
\begin{align*}
L_\text{total}&=2\cdot 1{,}25+2\cdot 1{,}75=6{,}00\ \text{m}.\\
\text{Con }3{,}00\ \text{m:}&\ \text{dos tablas: }(1{,}25+1{,}75)=3{,}00\ \Rightarrow\ \text{desperdicio }0.\\
\text{Con }4{,}00\ \text{m:}&\ \text{dos tablas suman }8{,}00\ \Rightarrow\ \text{desperdicio }2{,}00.\\
\Rightarrow&\ \boxed{3{,}00\ \text{m}}.
\end{align*}

% 7
\item \textbf{Gaseosas}
\begin{align*}
C&=9\cdot \text{₡}1\,450=\text{₡}13\,050,\\
\text{Cambio}&=20\,000-13\,050=\boxed{\text{₡}6\,950}.
\end{align*}

% 8
\item \textbf{Cajas de 24}
\begin{align*}
970&=24\cdot 40+10\ \Rightarrow\ \text{faltan }24-10=\boxed{14}.
\end{align*}

% 9
\item \textbf{Varillas (1,20 m y 0,85 m)}
\begin{align*}
L_\text{total}&=3\cdot 1{,}20+2\cdot 0{,}85=3{,}60+1{,}70=5{,}30\ \text{m}.\\
\text{Con }3{,}00\ \text{m:}&\ 2\ \text{varillas (6,00 m)}\Rightarrow \text{se puede cortar con }0{,}70\ \text{m de desperdicio}.\\
\text{Con }2{,}00\ \text{m:}&\ 4\ \text{varillas (8,00 m)}\Rightarrow 2{,}70\ \text{m de desperdicio}.\\
\text{Con }4{,}00\ \text{m:}&\ 2\ \text{varillas (8,00 m)}\Rightarrow 2{,}70\ \text{m de desperdicio}.\\
\Rightarrow&\ \boxed{3{,}00\ \text{m}}.
\end{align*}

% 10
\item \textbf{Carpetas}
\begin{align*}
C&=8\cdot \text{₡}2\,350=\text{₡}18\,800,\\
\text{Cambio}&=20\,000-18\,800=\boxed{\text{₡}1\,200}.
\end{align*}

% 11
\item \textbf{Bolsas de 9}
\begin{align*}
215&=9\cdot 23+8\ \Rightarrow\ \text{faltan }9-8=\boxed{1}.
\end{align*}

% 12
\item \textbf{Marco (42,7 cm y 55,6 cm)}
\begin{align*}
L_\text{necesario}&=2\cdot 42{,}7+2\cdot 55{,}6=85{,}4+111{,}2= \boxed{196{,}6\ \text{cm}}.\\
120\ \text{cm:}&\ \text{insuficiente};\quad 220\ \text{cm: sobra }23{,}4;\quad 280\ \text{cm: sobra }83{,}4.\\
\Rightarrow&\ \boxed{220\ \text{cm}}.
\end{align*}

% 13
\item \textbf{Helados}
\begin{align*}
C&=18\cdot \text{₡}950=\text{₡}17\,100,\\
\text{Saldo}&=20\,000-17\,100=\boxed{\text{₡}2\,900\ \text{(sobra)}}.
\end{align*}

% 14
\item \textbf{Cajas de 15}
\begin{align*}
326&=15\cdot 21+11\ \Rightarrow\ \text{faltan }15-11=\boxed{4}.
\end{align*}

% 15
\item \textbf{Cerca (0,90 m y 1,35 m)}
\begin{align*}
L_\text{total}&=4\cdot 0{,}90+2\cdot 1{,}35=3{,}60+2{,}70=6{,}30\ \text{m}.\\
3{,}00\ \text{m:}&\ 3\ \text{tablas (9,00 m)}\Rightarrow \text{desperdicio }2{,}70\ \text{m}.\\
3{,}60\ \text{m:}&\ 2\ \text{tablas (7,20 m)}\Rightarrow \text{se puede cortar con }0{,}90\ \text{m de desperdicio}.\\
4{,}20\ \text{m:}&\ 2\ \text{tablas (8,40 m)}\Rightarrow \text{desperdicio }2{,}10\ \text{m}.\\
\Rightarrow&\ \boxed{3{,}60\ \text{m}}.
\end{align*}

\end{enumerate}
\end{document}