\documentclass[12pt]{article}
\usepackage[spanish]{babel}
\usepackage[utf8]{inputenc}
\usepackage[T1]{fontenc}
\usepackage{geometry}
\geometry{letterpaper, margin=2.5cm}
% === Encabezados y pies de página ===
\usepackage{fancyhdr}
\pagestyle{fancy}
\fancyhf{} % limpia encabezados y pies
\rhead{Realizado por Julián Arias Meza} % esquina derecha
\cfoot{\thepage} % número de página en el centro del pie
\begin{document}

\begin{center}
\Large \textbf{Práctica de Razonamiento Matemático}\\
\small Estilo examen de admisión CTP Santa Lucía
\end{center}

\vspace{1cm}

\textbf{Instrucciones:} Resuelva cada uno de los siguientes ejercicios sin calculadora.  
Marque con una \textbf{X} la opción correcta.  

\vspace{0.5cm}

\begin{enumerate}

\item En una tienda, un cuaderno cuesta ₡850. Si Daniel compra 15 cuadernos y paga con un billete de ₡20 000, ¿cuánto cambio recibe?
\begin{itemize}
  \item[a)] ₡6 950
  \item[b)] ₡7 250
  \item[c)] ₡7 600
\end{itemize}

\item Laura desea repartir 245 caramelos en bolsas de 8 caramelos cada una. ¿Cuántos caramelos le faltan para que las bolsas queden completas?
\begin{itemize}
  \item[a)] 2
  \item[b)] 3
  \item[c)] 5
\end{itemize}

\item Para construir un marco se requieren 2 piezas de 62,5 cm y 2 piezas de 84,3 cm. Si se dispone de reglas de 150 cm, 250 cm y 300 cm, ¿qué regla se debe escoger para que el desperdicio sea mínimo?
\begin{itemize}
  \item[a)] 150 cm
  \item[b)] 250 cm
  \item[c)] 300 cm
\end{itemize}

\item Un boleto de cine cuesta ₡3 200. Si un grupo de 7 amigos compra boletos y paga con ₡30 000, ¿cuánto dinero les sobra?
\begin{itemize}
  \item[a)] ₡6 800
  \item[b)] ₡7 600
  \item[c)] ₡8 200
\end{itemize}

\item Un agricultor tiene 178 naranjas y quiere empacarlas en cajas de 12 naranjas cada una. ¿Cuántas naranjas le hacen falta?
\begin{itemize}
  \item[a)] 1
  \item[b)] 2
  \item[c)] 4
\end{itemize}

\item Se desea construir una mesa con 2 tablas de 1,25 m y 2 tablas de 1,75 m. Si la ferretería vende tablas de 3,00 m y de 4,00 m, ¿cuál conviene comprar para tener el menor desperdicio?
\begin{itemize}
  \item[a)] 3,00 m
  \item[b)] 4,00 m
  \item[c)] Ambas sirven por igual
\end{itemize}

\item Pedro compra 9 gaseosas a ₡1 450 cada una. Si paga con ₡20 000, ¿cuánto cambio recibe?
\begin{itemize}
  \item[a)] ₡6 850
  \item[b)] ₡6 950
  \item[c)] ₡7 050
\end{itemize}

\item Una fábrica empaca chocolates en cajas de 24 unidades. Si produce 970 chocolates en un día, ¿cuántos chocolates le faltan para completar la última caja?
\begin{itemize}
  \item[a)] 10
  \item[b)] 12
  \item[c)] 14
\end{itemize}

\item Para un proyecto se requieren 3 varillas de 1,20 m y 2 varillas de 0,85 m. Si en la ferretería hay varillas de 2,00 m, 3,00 m y 4,00 m, ¿qué medida debe escogerse para desperdiciar menos material?
\begin{itemize}
  \item[a)] 2,00 m
  \item[b)] 3,00 m
  \item[c)] 4,00 m
\end{itemize}

\item En una librería, una carpeta cuesta ₡2 350. Si Andrea compra 8 carpetas y paga con ₡20 000, ¿cuánto cambio recibe?
\begin{itemize}
  \item[a)] ₡1 000
  \item[b)] ₡1 200
  \item[c)] ₡1 500
\end{itemize}

\item Luis tiene 215 canicas y desea repartirlas en bolsas de 9 canicas cada una. ¿Cuántas canicas le faltan?
\begin{itemize}
  \item[a)] 1
  \item[b)] 2
  \item[c)] 3
\end{itemize}

\item Un marco requiere 2 piezas de 42,7 cm y 2 piezas de 55,6 cm. ¿Cuál longitud de madera se debe elegir?
\begin{itemize}
  \item[a)] 120 cm
  \item[b)] 220 cm
  \item[c)] 280 cm
\end{itemize}

\item El precio de un helado es de ₡950. Si un grupo de 18 estudiantes compra cada uno un helado y pagan en total con ₡20 000, ¿cuánto dinero falta o sobra?
\begin{itemize}
  \item[a)] ₡2 800 sobran
  \item[b)] ₡2 900 sobran
  \item[c)] ₡3 000 faltan
\end{itemize}

\item Un distribuidor tiene 326 botellas y desea empacarlas en cajas de 15 botellas cada una. ¿Cuántas botellas le faltan para completar la última caja?
\begin{itemize}
  \item[a)] 2
  \item[b)] 3
  \item[c)] 4
\end{itemize}

\item Para hacer una cerca se necesitan 4 tablas de 0,90 m y 2 tablas de 1,35 m. ¿Qué medida de tabla conviene comprar para desperdiciar menos?
\begin{itemize}
  \item[a)] 3,00 m
  \item[b)] 3,60 m
  \item[c)] 4,20 m
\end{itemize}

\end{enumerate}
\section*{Respuestas}
1-b, 2-b, 3-c, 4-b, 5-b, 6-a, 7-b, 8-c, 9-b, 10-b, 11-a, 12-b, 13-b, 14-c, 15-b


\end{document}