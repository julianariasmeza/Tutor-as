% ==========================
% PRÁCTICA GENERAL - EXAMEN DE ADMISIÓN (SECCIÓN: ESPAÑOL)
% ==========================
\documentclass[11pt,a4paper]{article}

% --- Idioma y codificación ---
\usepackage[spanish, es-nodecimaldot, shorthands=off]{babel}
\usepackage[utf8]{inputenc}
\usepackage[T1]{fontenc}

% --- Tipografía y microtipografía ---
\usepackage{lmodern}
\usepackage{microtype}

% --- Márgenes y formato de página ---
\usepackage{geometry}
\geometry{margin=2.2cm}

% --- Herramientas útiles ---
\usepackage{amsmath,amssymb}
\usepackage{enumitem}
\usepackage{multicol}
\usepackage{booktabs}
\usepackage{graphicx}
\usepackage{array}       % (dejamos una sola carga)
\usepackage{tcolorbox}
\usepackage{siunitx}     % Para unidades del SI
\tcbset{colback=gray!2, colframe=black!40, boxrule=0.4pt, arc=2mm}

% --- Cuadritos para marcar con X ---
\newcommand{\cuadro}{\fbox{\phantom{X}}}

% --- Formato de opciones (A–D) en 2 columnas compactas ---
\newlist{opciones}{itemize}{1}
\setlist[opciones]{label={}, left=0pt, itemsep=2pt, topsep=2pt, parsep=0pt}
\newcommand{\opcion}[2]{\item \cuadro\; \textbf{#1})\; #2}

% --- Entorno de pregunta con espacio para respuesta marcada ---
\newcommand{\pregunta}[1]{\noindent\textbf{#1}\par}

% --- Encabezado de sección y estilo general ---
\newcommand{\instrucciones}{
\begin{tcolorbox}
\textbf{Instrucciones generales.} Lea cada ítem y marque con una \textbf{X} el cuadro de la opción correcta.
Trabaje con calma y, cuando sea necesario, lea dos veces. No use corrector líquido sobre la hoja oficial.
\end{tcolorbox}
}

% --- Tabla de respuestas por número ---
\newenvironment{clave}{
\begin{tcolorbox}
\textbf{Respuestas (para revisión del encargado).}
\begin{center}
\begin{tabular}{@{}*{10}{c}@{}}
\toprule
\#1 & \#2 & \#3 & \#4 & \#5 & \#6 & \#7 & \#8 & \#9 & \#10\\
\midrule
}{\\\bottomrule
\end{tabular}

\vspace{4pt}
\begin{tabular}{@{}*{10}{c}@{}}
\toprule
\#11 & \#12 & \#13 & \#14 & \#15 & \#16 & \#17 & \#18 & \#19 & \#20\\
\midrule


\bottomrule
\end{tabular}
\end{center}
\end{tcolorbox}
}

% ==========================
% COMIENZO DEL DOCUMENTO
% ==========================
\begin{document}

% ==========================
% PORTADA (corregida y mejorada)
% ==========================
\begin{titlepage}
\centering

% --- Logo (opcional; si existe) ---
% \IfFileExists{logo_colegio.pdf}{\includegraphics[width=2.8cm]{logo_colegio}\par\vspace{1cm}}{}
% \IfFileExists{logo_colegio.png}{\includegraphics[width=2.8cm]{logo_colegio}\par\vspace{1cm}}{}

\vspace*{0.5cm} % margen superior extra

{\Large \textbf{Colegio / CTP:} \rule{9cm}{0.4pt}\par}
\vspace{1.2cm}

{\Huge \textbf{Práctica General — Examen de Admisión}}\par
\vspace{0.3cm}
{\large \textbf{Español • Estudios Sociales • Matemática • Ciencias}}\par

\vspace{1.5cm}
\begin{tcolorbox}
\centering
\textbf{Instrucciones generales}\\[2pt]
Lea con atención y \textbf{marque con X} la opción correcta en cada ítem.\\
Trabaje con calma. No use corrector líquido. Entregue la hoja completa.
\end{tcolorbox}

\vspace{1.3cm}

\raggedright
\begin{tabular}{p{4.2cm}p{10cm}}
\textbf{Nombre del estudiante:} & \rule{10cm}{0.4pt} \\[0.9cm]
\textbf{Sección / Grupo:} & \rule{10cm}{0.4pt} \\[0.9cm]
\textbf{Encargado (firma):} & \rule{10cm}{0.4pt} \\[0.9cm]
\textbf{Fecha:} & \rule{10cm}{0.4pt} \\
\end{tabular}

\vfill

% --- Tabla para anotación por sección ---
\centering
\begin{tabular}{|c|c|c|c|}
\hline
\textbf{Español} & \textbf{Sociales} & \textbf{Matemática} & \textbf{Ciencias} \\
\hline
\ \ \rule{2.2cm}{0.4pt}\ \  & \ \ \rule{2.2cm}{0.4pt}\ \  & \ \ \rule{2.2cm}{0.4pt}\ \  & \ \ \rule{2.2cm}{0.4pt}\ \  \\
\hline
\end{tabular}

\vspace{0.9cm}

% --- Calificación final ---
\begin{tabular}{|c|c|}
\hline
\textbf{Calificación final} & \rule{6cm}{0.4pt} \\
\hline
\end{tabular}

\vspace{0.9cm}
\textit{Realizado por: Julián Arias Meza}\par

\end{titlepage}

% ==========================
% ENCABEZADO DE LA SECCIÓN 1
% ==========================
\begin{center}
{\Large \textbf{Práctica general — Examen de admisión}}\\[2pt]
\textbf{Sección 1: Español}
\end{center}

\instrucciones


% ==========================
% ESPAÑOL
% ==========================

% --- 1. Acentuación ---
\pregunta{1) ¿Cuál forma está correctamente acentuada?}
\begin{opciones}
  \opcion{A}{Jovenes}
  \opcion{B}{Jóvenes}
  \opcion{C}{Jóvénes}
  \opcion{D}{Jovenés}
\end{opciones}

\pregunta{2) ¿Cuál palabra \emph{no} lleva tilde?}
\begin{opciones}
  \opcion{A}{Camión}
  \opcion{B}{Tibia}
  \opcion{C}{Canción}
  \opcion{D}{Sofá}
\end{opciones}

\pregunta{3) ¿Dónde lleva tilde la palabra \emph{diptongo}?}
\begin{opciones}
  \opcion{A}{Díptongo}
  \opcion{B}{Diptóngo}
  \opcion{C}{Diptongo}
  \opcion{D}{Díptóngo}
\end{opciones}

% --- 2. Ortografía (b/v, g/j, c/s/z, h, y/ll) ---
\pregunta{4) Complete: "El estudiante pro\_\_\_ó una solución creativa".}
\begin{opciones}
  \opcion{A}{v}
  \opcion{B}{b}
  \opcion{C}{\emph{h}}
  \opcion{D}{g}
\end{opciones}

\pregunta{5) ¿Cuál opción está bien escrita?}
\begin{opciones}
  \opcion{A}{Exámen}
  \opcion{B}{Examén}
  \opcion{C}{Examen}
  \opcion{D}{Examenes}
\end{opciones}

\pregunta{6) ¿Cuál es la forma correcta?}
\begin{opciones}
  \opcion{A}{Dirijir}
  \opcion{B}{Dirigir}
  \opcion{C}{Dirigirr}
  \opcion{D}{Dirigiréz}
\end{opciones}

% --- 3. Léxico: sinónimos y antónimos ---
\pregunta{7) Sinónimo de \emph{apresurar}.}
\begin{opciones}
  \opcion{A}{Acelerar}
  \opcion{B}{Detener}
  \opcion{C}{Ignorar}
  \opcion{D}{Olvidar}
\end{opciones}

\pregunta{8) Antónimo de \emph{escaso}.}
\begin{opciones}
  \opcion{A}{Pobre}
  \opcion{B}{Abundante}
  \opcion{C}{Débil}
  \opcion{D}{Insuficiente}
\end{opciones}

\pregunta{9) Sinónimo de \emph{objetivo} (como "meta").}
\begin{opciones}
  \opcion{A}{Finalidad}
  \opcion{B}{Opinión}
  \opcion{C}{Capricho}
  \opcion{D}{Fantasía}
\end{opciones}

% --- 4. Uso de tilde diacrítica, demostrativos y pronombres ---
\pregunta{10) Seleccione la opción correctamente escrita.}
\begin{opciones}
  \opcion{A}{\emph{Solo} quiero estudiar; no necesito más.}
  \opcion{B}{Sólo quiero estudiar; no necesito mas.}
  \opcion{C}{Solo quiero estudiar; no necesito más.}
  \opcion{D}{Sólo quiero estudiar; no necesito más.}
\end{opciones}

\pregunta{11) Señale la oración con tilde diacrítica correcta.}
\begin{opciones}
  \opcion{A}{\emph{El} dijo que \emph{mas} tarde llegaría.}
  \opcion{B}{\emph{Él} dijo que \emph{más} tarde llegaría.}
  \opcion{C}{\emph{El} dijo que \emph{más} tarde llegaría.}
  \opcion{D}{\emph{Él} dijo que \emph{mas} tarde llegaría.}
\end{opciones}

% --- 5. Redacción: concordancia y puntuación ---
\pregunta{12) ¿Cuál oración tiene concordancia correcta?}
\begin{opciones}
  \opcion{A}{La alumnas presentó su tarea.}
  \opcion{B}{Las alumna presentaron su tarea.}
  \opcion{C}{Las alumnas presentaron su tarea.}
  \opcion{D}{La alumna presentaron sus tarea.}
\end{opciones}

\pregunta{13) Seleccione la puntuación más adecuada.}
\begin{opciones}
  \opcion{A}{Trae lápiz, borrador cuaderno.}
  \opcion{B}{Trae: lápiz borrador, cuaderno.}
  \opcion{C}{Trae lápiz, borrador y cuaderno.}
  \opcion{D}{Trae, lápiz, borrador y cuaderno.}
\end{opciones}
\newpage
% --- 6. Comprensión de lectura (Texto breve + 5 ítems) ---
\begin{tcolorbox}
\textbf{Texto para las preguntas 14 a 18.} \\
En la escuela "Valle Verde", la profesora Catalina propuso un proyecto de lectura:
cada estudiante debía elegir un libro de su interés y, al finalizar,
presentar un comentario personal sobre el tema, los personajes y el mensaje
principal. Durante el mes, el aula se transformó; algunos formaron clubes de
lectura, otros crearon afiches y varios estudiantes compartieron fragmentos
que les habían conmovido. Al final, la docente destacó que el objetivo no era
"leer más rápido", sino \emph{comprender} y \emph{disfrutar} la lectura,
apreciando cómo las historias pueden cambiar nuestra manera de pensar.
\end{tcolorbox}

\pregunta{14) ¿Cuál fue el propósito del proyecto?}
\begin{opciones}
  \opcion{A}{Leer el mayor número de páginas.}
  \opcion{B}{Mejorar solo la ortografía.}
  \opcion{C}{Comprender y disfrutar la lectura.}
  \opcion{D}{Memorizar el nombre de los autores.}
\end{opciones}

\pregunta{15) ¿Qué actividad \emph{no} se menciona?}
\begin{opciones}
  \opcion{A}{Clubes de lectura}
  \opcion{B}{Creación de afiches}
  \opcion{C}{Compartir fragmentos}
  \opcion{D}{Concurso de deletreo}
\end{opciones}

\pregunta{16) Según el texto, el comentario personal debía incluir, entre otros:}
\begin{opciones}
  \opcion{A}{Precio del libro y editorial.}
  \opcion{B}{Tema, personajes y mensaje principal.}
  \opcion{C}{Biografía completa del autor.}
  \opcion{D}{Cantidad de capítulos y páginas.}
\end{opciones}

\pregunta{17) ¿Qué cambió en el aula durante el mes?}
\begin{opciones}
  \opcion{A}{El horario de todas las clases.}
  \opcion{B}{El mobiliario del aula.}
  \opcion{C}{Las dinámicas de trabajo, como clubes y afiches.}
  \opcion{D}{La cantidad de exámenes semanales.}
\end{opciones}

\pregunta{18) ¿Qué enfatizó la docente al final?}
\begin{opciones}
  \opcion{A}{Leer más rápido que los compañeros.}
  \opcion{B}{Comprender y disfrutar la lectura.}
  \opcion{C}{Memorizar fechas históricas.}
  \opcion{D}{Escribir resúmenes extensos.}
\end{opciones}

% --- 7. Varios: uso de mayúsculas y frases hechas ---
\pregunta{19) El uso correcto de mayúsculas es:}
\begin{opciones}
  \opcion{A}{visité el Museo Nacional de Costa Rica.}
  \opcion{B}{Visité el museo nacional de costa rica.}
  \opcion{C}{Visité el Museo Nacional de Costa Rica.}
  \opcion{D}{visité el museo Nacional de Costa Rica.}
\end{opciones}
\newpage
\pregunta{20) ¿Qué opción completa mejor la expresión? \\
"\ldots\ dejó \emph{entrever} que no asistiría al acto".}
\begin{opciones}
  \opcion{A}{Seguramente}
  \opcion{B}{Apenas}
  \opcion{C}{Sutilmente}
  \opcion{D}{Ruidosamente}
\end{opciones}

% --- CLAVE DE RESPUESTAS (ESPAÑOL) ---
% --- CLAVE DE RESPUESTAS (ESPAÑOL) ---
\begin{tcolorbox}
\textbf{Respuestas — Español (para revisión del encargado).}

\begin{center}
\begin{tabular}{|c|c|c|c|c|c|c|c|c|c|}
\hline
\#1 & \#2 & \#3 & \#4 & \#5 & \#6 & \#7 & \#8 & \#9 & \#10 \\
\hline
B & B & C & B & C & B & A & B & A & C \\
\hline
\#11 & \#12 & \#13 & \#14 & \#15 & \#16 & \#17 & \#18 & \#19 & \#20 \\
\hline
B & C & C & C & D & B & C & B & C & C \\
\hline
\end{tabular}
\end{center}
\end{tcolorbox}

% ==========================
% SECCIÓN 2: ESTUDIOS SOCIALES
% ==========================

\newpage
\begin{center}
{\Large \textbf{Sección 2: Estudios Sociales}}
\end{center}

\instrucciones

% --- 1. Historia de Costa Rica ---
\pregunta{1) ¿En qué año se abolió el ejército en Costa Rica?}
\begin{opciones}
  \opcion{A}{1940}
  \opcion{B}{1948}
  \opcion{C}{1950}
  \opcion{D}{1961}
\end{opciones}

\pregunta{2) ¿Quién fue el presidente que decretó la abolición del ejército?}
\begin{opciones}
  \opcion{A}{José María Castro Madriz}
  \opcion{B}{José Figueres Ferrer}
  \opcion{C}{Rafael Ángel Calderón Guardia}
  \opcion{D}{Otilio Ulate Blanco}
\end{opciones}

\pregunta{3) ¿Cuál fue una de las principales causas de la Guerra Civil de 1948?}
\begin{opciones}
  \opcion{A}{Fraude electoral}
  \opcion{B}{Invasión extranjera}
  \opcion{C}{Crisis cafetalera}
  \opcion{D}{Reformas educativas}
\end{opciones}

% --- 2. Geografía ---
\pregunta{4) ¿Cuál es la capital de la provincia de Guanacaste?}
\begin{opciones}
  \opcion{A}{Liberia}
  \opcion{B}{Nicoya}
  \opcion{C}{Santa Cruz}
  \opcion{D}{Cañas}
\end{opciones}

\pregunta{5) El río más largo de Costa Rica es:}
\begin{opciones}
  \opcion{A}{Reventazón}
  \opcion{B}{Tempisque}
  \opcion{C}{San Juan}
  \opcion{D}{Grande de Térraba}
\end{opciones}

\pregunta{6) ¿En qué provincia se ubica el volcán Irazú?}
\begin{opciones}
  \opcion{A}{Heredia}
  \opcion{B}{Cartago}
  \opcion{C}{San José}
  \opcion{D}{Alajuela}
\end{opciones}

% --- 3. Civismo ---
\pregunta{7) ¿Cuál es el nombre del himno nacional de Costa Rica?}
\begin{opciones}
  \opcion{A}{¡Oh gloria inmortal!}
  \opcion{B}{Noble patria, tu hermosa bandera}
  \opcion{C}{¡Patria querida!}
  \opcion{D}{Costa Rica siempre fiel}
\end{opciones}
\newpage
\pregunta{8) Según la Constitución Política, el Poder Judicial es ejercido por:}
\begin{opciones}
  \opcion{A}{Asamblea Legislativa}
  \opcion{B}{Tribunal Supremo de Elecciones}
  \opcion{C}{Corte Suprema de Justicia}
  \opcion{D}{Contraloría General}
\end{opciones}

\pregunta{9) ¿Qué artículo de la Constitución establece la abolición del ejército?}
\begin{opciones}
  \opcion{A}{Artículo 1}
  \opcion{B}{Artículo 12}
  \opcion{C}{Artículo 50}
  \opcion{D}{Artículo 90}
\end{opciones}

% --- 4. Cultura general ---
\pregunta{10) ¿Qué significa el acrónimo "ONU"?}
\begin{opciones}
  \opcion{A}{Organización Nacional Unida}
  \opcion{B}{Oficina de Naciones Unidas}
  \opcion{C}{Organización de las Naciones Unidas}
  \opcion{D}{Oficina Nacional Unificada}
\end{opciones}

\pregunta{11) ¿Cuál es el nombre del actual billete de ₡20\,000?}
\begin{opciones}
  \opcion{A}{María Isabel Carvajal (Carmen Lyra)}
  \opcion{B}{José Figueres Ferrer}
  \opcion{C}{Rafael Ángel Calderón Guardia}
  \opcion{D}{Justo A. Facio}
\end{opciones}

\pregunta{12) ¿Qué fecha se celebra la independencia de Costa Rica?}
\begin{opciones}
  \opcion{A}{15 de setiembre de 1821}
  \opcion{B}{12 de octubre de 1502}
  \opcion{C}{25 de julio de 1824}
  \opcion{D}{1 de diciembre de 1948}
\end{opciones}

% --- 5. Historia universal ---
\pregunta{13) ¿En qué año terminó la Segunda Guerra Mundial?}
\begin{opciones}
  \opcion{A}{1939}
  \opcion{B}{1942}
  \opcion{C}{1945}
  \opcion{D}{1950}
\end{opciones}

\pregunta{14) ¿Quién fue el líder del movimiento de independencia de la India?}
\begin{opciones}
  \opcion{A}{Nelson Mandela}
  \opcion{B}{Mahatma Gandhi}
  \opcion{C}{Simón Bolívar}
  \opcion{D}{Martin Luther King}
\end{opciones}

\pregunta{15) ¿Qué revolución inspiró los ideales de independencia en América Latina?}
\begin{opciones}
  \opcion{A}{Revolución Francesa}
  \opcion{B}{Revolución Rusa}
  \opcion{C}{Revolución Industrial}
  \opcion{D}{Revolución Cubana}
\end{opciones}

% --- 6. Nuevas preguntas ---
\pregunta{16) ¿En qué continente se encuentra Egipto?}
\begin{opciones}
  \opcion{A}{Asia}
  \opcion{B}{África}
  \opcion{C}{Europa}
  \opcion{D}{Oceanía}
\end{opciones}

\pregunta{17) ¿Cuál organismo internacional tiene como sede principal la ciudad de Nueva York?}
\begin{opciones}
  \opcion{A}{ONU}
  \opcion{B}{OEA}
  \opcion{C}{Unión Europea}
  \opcion{D}{OTAN}
\end{opciones}

\pregunta{18) ¿En qué año se firmaron los Acuerdos de Paz de Esquipulas II en Centroamérica?}
\begin{opciones}
  \opcion{A}{1987}
  \opcion{B}{1990}
  \opcion{C}{1994}
  \opcion{D}{2000}
\end{opciones}

\pregunta{19) ¿Cuál es la capital de Canadá?}
\begin{opciones}
  \opcion{A}{Toronto}
  \opcion{B}{Vancouver}
  \opcion{C}{Ottawa}
  \opcion{D}{Montreal}
\end{opciones}

\pregunta{20) ¿Qué país es conocido como "la cuna de la democracia"?}
\begin{opciones}
  \opcion{A}{Roma}
  \opcion{B}{Grecia}
  \opcion{C}{Inglaterra}
  \opcion{D}{Francia}
\end{opciones}

% --- CLAVE DE RESPUESTAS (ESTUDIOS SOCIALES) ---
\begin{tcolorbox}
\textbf{Respuestas — Estudios Sociales (para revisión del encargado).}

\begin{center}
\begin{tabular}{|c|c|c|c|c|c|c|c|c|c|}
\hline
\#1 & \#2 & \#3 & \#4 & \#5 & \#6 & \#7 & \#8 & \#9 & \#10 \\
\hline
B & B & A & A & D & B & B & C & B & C \\
\hline
\#11 & \#12 & \#13 & \#14 & \#15 & \#16 & \#17 & \#18 & \#19 & \#20 \\
\hline
A & A & C & B & A & B & A & A & C & B \\
\hline
\end{tabular}
\end{center}
\end{tcolorbox}

% ==========================
% SECCIÓN 3: MATEMÁTICA
% ==========================

\newpage
\begin{center}
{\Large \textbf{Sección 3: Matemática}}
\end{center}

\instrucciones

% --- Preguntas ---
\pregunta{1) En una tienda, un cuaderno cuesta ₡850. Si Daniel compra 15 cuadernos y paga con ₡20 000, ¿cuánto cambio recibe?}
\begin{opciones}
  \opcion{A}{₡6 950}
  \opcion{B}{₡7 250}
  \opcion{C}{₡7 600}
\end{opciones}

\pregunta{2) Laura desea repartir 245 caramelos en bolsas de 8 caramelos cada una. ¿Cuántos caramelos le faltan para que las bolsas queden completas?}
\begin{opciones}
  \opcion{A}{2}
  \opcion{B}{3}
  \opcion{C}{5}
\end{opciones}

\pregunta{3) Para construir un marco se requieren 2 piezas de 62,5 cm y 2 piezas de 84,3 cm. Si se dispone de reglas de 150 cm, 250 cm y 300 cm, ¿qué regla se debe escoger para que el desperdicio sea mínimo?}
\begin{opciones}
  \opcion{A}{150 cm}
  \opcion{B}{250 cm}
  \opcion{C}{300 cm}
\end{opciones}

\pregunta{4) Un boleto de cine cuesta ₡3 200. Si un grupo de 7 amigos paga con ₡30 000, ¿cuánto dinero les sobra?}
\begin{opciones}
  \opcion{A}{₡6 800}
  \opcion{B}{₡7 600}
  \opcion{C}{₡8 200}
\end{opciones}

\pregunta{5) Un agricultor tiene 178 naranjas y quiere empacarlas en cajas de 12 naranjas cada una. ¿Cuántas naranjas le hacen falta?}
\begin{opciones}
  \opcion{A}{1}
  \opcion{B}{2}
  \opcion{C}{4}
\end{opciones}

\pregunta{6) Pedro compra 9 gaseosas a ₡1 450 cada una. Si paga con ₡20 000, ¿cuánto cambio recibe?}
\begin{opciones}
  \opcion{A}{₡6 850}
  \opcion{B}{₡6 950}
  \opcion{C}{₡7 050}
\end{opciones}

\pregunta{7) Una fábrica empaca chocolates en cajas de 24 unidades. Si produce 970 chocolates en un día, ¿cuántos le faltan para completar la última caja?}
\begin{opciones}
  \opcion{A}{10}
  \opcion{B}{12}
  \opcion{C}{14}
\end{opciones}
\newpage
\pregunta{8) Luis tiene 215 canicas y desea repartirlas en bolsas de 9 canicas cada una. ¿Cuántas le faltan?}
\begin{opciones}
  \opcion{A}{1}
  \opcion{B}{2}
  \opcion{C}{3}
\end{opciones}

\pregunta{9) El precio de un helado es ₡950. Si un grupo de 18 estudiantes compra cada uno un helado y pagan en total con ₡20 000, ¿cuánto dinero sobra?}
\begin{opciones}
  \opcion{A}{₡2 800}
  \opcion{B}{₡2 900}
  \opcion{C}{₡3 000}
\end{opciones}

\pregunta{10) Un distribuidor tiene 326 botellas y desea empacarlas en cajas de 15 botellas cada una. ¿Cuántas botellas le faltan para completar la última caja?}
\begin{opciones}
  \opcion{A}{2}
  \opcion{B}{3}
  \opcion{C}{4}
\end{opciones}

\pregunta{11) Sofía compró 12 panes a ₡250 cada uno. Si paga con ₡5 000, ¿cuánto cambio recibe?}
\begin{opciones}
  \opcion{A}{₡1 000}
  \opcion{B}{₡1 500}
  \opcion{C}{₡2 000}
\end{opciones}

\pregunta{12) Una caja contiene 48 botellas. Si un camión transporta 17 cajas, ¿cuántas botellas lleva en total?}
\begin{opciones}
  \opcion{A}{786}
  \opcion{B}{816}
  \opcion{C}{864}
\end{opciones}

\pregunta{13) Un terreno rectangular mide 25 m de largo y 12 m de ancho. Su área es:}
\begin{opciones}
  \opcion{A}{250 m$^2$}
  \opcion{B}{300 m$^2$} % correcta
  \opcion{C}{275 m$^2$}
\end{opciones}


\pregunta{14) Una bicicleta cuesta ₡75 000 y tiene un 20\% de descuento. ¿Cuál es el precio con descuento?}
\begin{opciones}
  \opcion{A}{₡60 000}
  \opcion{B}{₡62 000}
  \opcion{C}{₡65 000}
\end{opciones}

\pregunta{15) Se lanza un dado. ¿Cuál es la probabilidad de obtener un número mayor que 4?}
\begin{opciones}
  \opcion{A}{1/6}
  \opcion{B}{1/3}
  \opcion{C}{1/2}
\end{opciones}

\pregunta{16) Un automóvil recorre 240 km en 4 horas. Su velocidad promedio es:}
\begin{opciones}
  \opcion{A}{50 km/h}
  \opcion{B}{55 km/h}
  \opcion{C}{60 km/h}
\end{opciones}

\pregunta{17) La edad de Pedro es el doble de la edad de Ana. Si entre ambos suman 36 años, ¿cuántos años tiene Pedro?}
\begin{opciones}
  \opcion{A}{18}
  \opcion{B}{24}
  \opcion{C}{30}
\end{opciones}

\pregunta{18) Un prisma rectangular tiene dimensiones $5 \times 4 \times 3$. Su volumen es:}
\begin{opciones}
  \opcion{A}{60}
  \opcion{B}{65}
  \opcion{C}{70}
\end{opciones}

\pregunta{19) Una persona compró 4 camisas a ₡6 750 cada una. Si paga con ₡30 000, ¿cuánto dinero sobra?}
\begin{opciones}
  \opcion{A}{₡2 500}
  \opcion{B}{₡2 800}
  \opcion{C}{₡3 000}
\end{opciones}

\pregunta{20) Una piscina se llena con 500 litros de agua por hora. ¿Cuántas horas se tardará en llenarse con 4 000 litros?}
\begin{opciones}
  \opcion{A}{6}
  \opcion{B}{7}
  \opcion{C}{8}
\end{opciones}

% --- CLAVE DE RESPUESTAS (MATEMÁTICA) --- (CORREGIDA)
\begin{tcolorbox}
\textbf{Respuestas — Matemática (para revisión del encargado).}

\begin{center}
\begin{tabular}{|c|c|c|c|c|c|c|c|c|c|}
\hline
\#1 & \#2 & \#3 & \#4 & \#5 & \#6 & \#7 & \#8 & \#9 & \#10 \\
\hline
B & B & C & B & B & B & C & A & B & C \\
\hline
\#11 & \#12 & \#13 & \#14 & \#15 & \#16 & \#17 & \#18 & \#19 & \#20 \\
\hline
C & B & B & A & B & C & B & A & C & C \\
\hline
\end{tabular}
\end{center}
\end{tcolorbox}


% ==========================
% SECCIÓN 4: CIENCIAS
% ==========================

\newpage
\begin{center}
{\Large \textbf{Sección 4: Ciencias}}
\end{center}

\instrucciones

% --- 1. Biología ---
\pregunta{1) ¿Cuál es la unidad básica de la vida?}
\begin{opciones}
  \opcion{A}{Tejido}
  \opcion{B}{Célula}
  \opcion{C}{Órgano}
  \opcion{D}{Molécula}
\end{opciones}

\pregunta{2) ¿Qué sistema del cuerpo humano transporta oxígeno y nutrientes?}
\begin{opciones}
  \opcion{A}{Nervioso}
  \opcion{B}{Digestivo}
  \opcion{C}{Circulatorio}
  \opcion{D}{Excretor}
\end{opciones}

\pregunta{3) ¿Cuál de los siguientes organismos realiza fotosíntesis?}
\begin{opciones}
  \opcion{A}{Hongo}
  \opcion{B}{Bacteria}
  \opcion{C}{Planta}
  \opcion{D}{Virus}
\end{opciones}

\pregunta{4) Anita se sintió enferma y su mamá le midió la temperatura, obteniendo 38\(^\circ\)C. ¿Qué instrumento utilizó?}
\begin{opciones}
  \opcion{A}{Calorímetro}
  \opcion{B}{Termómetro}
  \opcion{C}{Barómetro}
  \opcion{D}{Escala}
\end{opciones}

\pregunta{5) Instrumento utilizado para observar objetos muy pequeños, como células o bacterias.}
\begin{opciones}
  \opcion{A}{Microscopio}
  \opcion{B}{Telescopio}
  \opcion{C}{Láser}
  \opcion{D}{Prisma}
\end{opciones}

\pregunta{6) Relación donde un organismo se beneficia sin afectar al otro, como las aves que construyen nidos en árboles.}
\begin{opciones}
  \opcion{A}{Competencia}
  \opcion{B}{Comensalismo}
  \opcion{C}{Mutualismo}
  \opcion{D}{Parasitismo}
\end{opciones}
\newpage
% --- 2. Química ---
\pregunta{7) ¿Cuál es el símbolo químico del oxígeno?}
\begin{opciones}
  \opcion{A}{Ox}
  \opcion{B}{O}
  \opcion{C}{O\textsubscript{2}}
  \opcion{D}{Oxg}
\end{opciones}

\pregunta{8) El agua está formada por:}
\begin{opciones}
  \opcion{A}{Hidrógeno y carbono}
  \opcion{B}{Oxígeno y nitrógeno}
  \opcion{C}{Hidrógeno y oxígeno}
  \opcion{D}{Oxígeno y helio}
\end{opciones}

\pregunta{9) Claudia preparó un refresco mezclando agua y sirope; las partes no se distinguen. El refresco es:}
\begin{opciones}
  \opcion{A}{Mezcla homogénea}
  \opcion{B}{Mezcla heterogénea}
  \opcion{C}{Compuesto}
  \opcion{D}{Sustancia pura}
\end{opciones}

\pregunta{10) El aire que respiramos (nitrógeno, oxígeno y otros gases) es una:}
\begin{opciones}
  \opcion{A}{Mezcla homogénea}
  \opcion{B}{Mezcla heterogénea}
  \opcion{C}{Elemento}
  \opcion{D}{Compuesto}
\end{opciones}

\pregunta{11) El azúcar disuelta en agua se clasifica como:}
\begin{opciones}
  \opcion{A}{Mezcla homogénea}
  \opcion{B}{Mezcla heterogénea}
  \opcion{C}{Sustancia pura}
  \opcion{D}{Compuesto}
\end{opciones}

% --- 3. Física ---
\pregunta{12) ¿Cuál es la unidad de medida de la fuerza en el SI?}
\begin{opciones}
  \opcion{A}{Joule (J)}
  \opcion{B}{Newton (N)}
  \opcion{C}{Watt (W)}
  \opcion{D}{Pascal (Pa)}
\end{opciones}

\pregunta{13) Un cuerpo recorre 150 m en 30 s. Su rapidez promedio es:}
\begin{opciones}
  \opcion{A}{3 m/s}
  \opcion{B}{4 m/s}
  \opcion{C}{5 m/s}
  \opcion{D}{6 m/s}
\end{opciones}
\newpage
\pregunta{14) ¿Qué fenómeno explica la formación de un arcoíris?}
\begin{opciones}
  \opcion{A}{Reflexión del sonido}
  \opcion{B}{Refracción y dispersión de la luz}
  \opcion{C}{Difracción de electrones}
  \opcion{D}{Absorción de calor}
\end{opciones}

\pregunta{15) Cuando un lápiz en un vaso con agua parece “quebrado”, se debe a:}
\begin{opciones}
  \opcion{A}{Reflexión}
  \opcion{B}{Refracción}
  \opcion{C}{Absorción}
  \opcion{D}{Difracción}
\end{opciones}

\pregunta{16) Cuando la luz rebota en un espejo plano, se produce:}
\begin{opciones}
  \opcion{A}{Refracción}
  \opcion{B}{Reflexión}
  \opcion{C}{Dispersión}
  \opcion{D}{Absorción}
\end{opciones}

% --- 4. Ambiente y salud ---
\pregunta{17) El efecto invernadero se debe principalmente a la acumulación de:}
\begin{opciones}
  \opcion{A}{Ozono}
  \opcion{B}{Dióxido de carbono}
  \opcion{C}{Oxígeno}
  \opcion{D}{Nitrógeno}
\end{opciones}

\pregunta{18) Una medida para reducir la contaminación del aire en las ciudades es:}
\begin{opciones}
  \opcion{A}{Talar más árboles}
  \opcion{B}{Construir más carreteras}
  \opcion{C}{Aumentar el uso de vehículos eléctricos}
  \opcion{D}{Quemar más combustibles fósiles}
\end{opciones}

\pregunta{19) Los hongos y bacterias que descomponen restos de materia orgánica se clasifican como:}
\begin{opciones}
  \opcion{A}{Productores}
  \opcion{B}{Consumidores primarios}
  \opcion{C}{Descomponedores}
  \opcion{D}{Autótrofos}
\end{opciones}

\pregunta{20) La unidad básica de temperatura en el Sistema Internacional es:}
\begin{opciones}
  \opcion{A}{Grado Celsius (\(^\circ\)C)}
  \opcion{B}{Kelvin (K)}
  \opcion{C}{Grado Fahrenheit (\(^\circ\)F)}
  \opcion{D}{Joule (J)}
\end{opciones}

% --- CLAVE DE RESPUESTAS (CIENCIAS) ---
\begin{tcolorbox}
\textbf{Respuestas — Ciencias (para revisión del encargado).}

\begin{center}
\begin{tabular}{|c|c|c|c|c|c|c|c|c|c|}
\hline
\#1 & \#2 & \#3 & \#4 & \#5 & \#6 & \#7 & \#8 & \#9 & \#10 \\
\hline
B & C & C & B & A & B & B & C & A & A \\
\hline
\#11 & \#12 & \#13 & \#14 & \#15 & \#16 & \#17 & \#18 & \#19 & \#20 \\
\hline
A & B & C & B & B & B & B & C & C & B \\
\hline
\end{tabular}
\end{center}
\end{tcolorbox}
\noindent\fbox{%
\parbox{\linewidth}{%
%\textbf{Julián Arias Meza} es profesor de Matemática graduado de la Universidad de Costa Rica. 
%Cursó la carrera de Física en la misma universidad y actualmente cursa la licenciatura en 
%Ingeniería Física en el Tecnológico de Costa Rica (TEC). 
Integración  del uso de herramientas digitales y de inteligencia artificial (IA), 
enfocadas en la mejora de las capacidades pedagógicas: 
diseño de materiales adaptativos, retroalimentación automática y 
optimización de la evaluación formativa.

\medskip
\textbf{Áreas}
\begin{itemize}
  \item \textbf{Matemática:} aritmética, álgebra, trigonometría, geometría analítica, cálculo (diferencial e integral), probabilidad y estadística.
  \item \textbf{Física:} mecánica, ondas, electricidad y magnetismo, termodinámica y óptica; con resolución de problemas, interpretación física y uso riguroso del SI.
  \item \textbf{Química:} nomenclatura con sistema Stock y sistema estequiométrico, óxidos, hidruros, hidróxidos, hidrácidos, sales binarias, tipos de reacciones, balanceo y estequiometría.
  \item \textbf{Admisión (universidades y colegios):} preparación integral para exámenes de ingreso (UCR, TEC, UNA, Colegios Científicos, COVAO, entre otros): diagnóstico inicial, plan de estudio, técnicas de resolución, simulacros cronometrados y análisis de errores.
  \item \textbf{Formatos:} guías teóricas, bancos de ejercicios, prácticas con soluciones paso a paso, simulacros, rúbricas, presentaciones y resúmenes ejecutivos.
\end{itemize}

\text{Se entrega en:} formato PDF listo para imprimir, o \texttt{.docx} (Word).

\medskip
\textbf{Contacto (WhatsApp):} \texttt{7076-9371}

}}
\end{document}