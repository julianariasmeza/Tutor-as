\documentclass[11pt]{article}
\usepackage{caption}
\usepackage{graphicx}
\usepackage[spanish]{babel}
\usepackage[T1]{fontenc}
\usepackage[utf8]{inputenc}
\usepackage{siunitx}
\usepackage{amsmath, amssymb}
\usepackage{geometry}
\geometry{letterpaper, margin=2.3cm}
\sisetup{output-decimal-marker={,}}
\DeclareSIUnit{\inch}{in}
\DeclareSIUnit{\pound}{lb}
\DeclareSIUnit{\foot}{ft}

\begin{document}

\section*{Análisis de viga}

\subsection*{Datos}
Longitud: \(\SI{9.000}{\metre}\).

\begin{center}
\begin{tabular}{l c}
\multicolumn{2}{l}{\textbf{Apoyos}} \\\hline
Tipo & Posición \\\hline
pinned & \SI{0.000}{\metre} \\
roller & \SI{9.000}{\metre} \\
\end{tabular}
\end{center}

\medskip
\begin{center}
\begin{tabular}{l c}
\multicolumn{2}{l}{\textbf{Reacciones}} \\\hline
Posición & Valor \\\hline
\(\mathrm{R}(x=\SI{0.000}{\metre})\) & \SI{80.0000}{\kilo\newton} \\
\(\mathrm{R}(x=\SI{9.000}{\metre})\) & \SI{40.0000}{\kilo\newton} \\
\end{tabular}
\end{center}

\medskip

\subsection*{Modelo y ecuaciones}
\[
\frac{dV}{dx} = w(x), \qquad \frac{dM}{dx} = V(x).
\]
Saltos: una fuerza puntual \(P\) o reacción \(R\) produce un salto en \(V\); un momento aplicado \(M_0\) (o de empotramiento) produce un salto en \(M\).

\subsection*{Chequeo de equilibrio}
\[
\Sigma F_y = \SI{0.000000}{\kilo\newton}, \qquad
\Sigma M_{(x=0)} = \SI{0.000000}{\kilo\newton\metre}.
\]

\subsection*{Extremos numéricos}
\begin{tabular}{lccc}
 & \textbf{máx (+)} & \textbf{mín (–)} & \(\lvert\cdot\rvert\) \textbf{máx} \\\hline
\(V\) & \(\SI{80.0000}{\kilo\newton}\) en \(x=\SI{0.000}{\metre}\) &
\(\SI{-40.0752}{\kilo\newton}\) en \(x=\SI{6.008}{\metre}\) &
\(\SI{80.0000}{\kilo\newton}\) en \(x=\SI{0.000}{\metre}\) \\
\(M\) & \(\SI{160.1001}{\kilo\newton\metre}\) en \(x=\SI{4.003}{\metre}\) &
\(\SI{0.0000}{\kilo\newton\metre}\) en \(x=\SI{0.000}{\metre}\) &
\(\SI{160.1001}{\kilo\newton\metre}\) en \(x=\SI{4.003}{\metre}\) \\
\end{tabular}
\newpage
%\subsection*{Diagramas}
Los diagramas \(V(x)\) y \(M(x)\) se adjuntan a continuación.


\section*{Diagramas}
\begin{figure}[h]
  \centering
  \includegraphics[width=\linewidth]{\detokenize{V_f1311ebfa39841e6a12555f3d4aed9ea.png}}
  \caption{Diagrama de cortante \(V(x)\) en \(\SI{}{kN}\).}
\end{figure}

\begin{figure}[h]
  \centering
  \includegraphics[width=\linewidth]{\detokenize{M_c79425f0f27f4264b8377147cffd0e76.png}}
  \caption{Diagrama de momento \(M(x)\) en \(\SI{}{kN.m}\).}
\end{figure}

\end{document}
