
\documentclass[11pt]{article}
\usepackage{caption}
\usepackage{graphicx}
\usepackage[spanish]{babel}
\usepackage[T1]{fontenc}
\usepackage[utf8]{inputenc}
\usepackage{siunitx}
\usepackage{amsmath, amssymb}
\usepackage{geometry}
\geometry{letterpaper, margin=2.3cm}
\sisetup{output-decimal-marker={,}}
\DeclareSIUnit{\inch}{in}
\DeclareSIUnit{\pound}{lb}
\DeclareSIUnit{\foot}{ft}

\begin{document}

\section*{Análisis de viga}

\subsection*{Datos}
Longitud: \(\SI{32.000}{\foot}\).

\begin{center}
\begin{tabular}{l c}
\multicolumn{2}{l}{\textbf{Apoyos}} \\\hline
Tipo & Posición \\\hline
pinned & \SI{0.000}{\foot} \\
roller & \SI{24.000}{\foot} \\
\end{tabular}
\end{center}

\medskip
\begin{center}
\begin{tabular}{l c}
\multicolumn{2}{l}{\textbf{Reacciones}} \\\hline
Posición & Valor \\\hline
\(\mathrm{R}(x=\SI{0.000}{\foot})\) & \SI{18.0000}{\text{kips}} \\
\(\mathrm{R}(x=\SI{24.000}{\foot})\) & \SI{26.0000}{\text{kips}} \\
\end{tabular}
\end{center}

\medskip

\subsection*{Modelo y ecuaciones}
\[
\frac{dV}{dx} = w(x), \qquad \frac{dM}{dx} = V(x).
\]
Saltos: una fuerza puntual \(P\) o reacción \(R\) produce un salto en \(V\); un momento aplicado \(M_0\) (o de empotramiento) produce un salto en \(M\).

\subsection*{Chequeo de equilibrio}
\[
\Sigma F_y = \SI{0.000000}{\text{kips}}, \qquad
\Sigma M_{(x=0)} = \SI{0.000000}{\text{kips.ft}}.
\]

\subsection*{Extremos numéricos}
\begin{tabular}{lccc}
 & \textbf{máx (+)} & \textbf{mín (–)} & \(\lvert\cdot\rvert\) \textbf{máx} \\\hline
\(V\) & \(\SI{18.0000}{\text{kips}}\) en \(x=\SI{0.000}{\foot}\) &
\(\SI{-14.0201}{\text{kips}}\) en \(x=\SI{24.000}{\foot}\) &
\(\SI{18.0000}{\text{kips}}\) en \(x=\SI{0.000}{\foot}\) \\
\(M\) & \(\SI{108.0301}{\text{kips.ft}}\) en \(x=\SI{6.000}{\foot}\) &
\(\SI{-47.8797}{\text{kips.ft}}\) en \(x=\SI{24.000}{\foot}\) &
\(\SI{108.0301}{\text{kips.ft}}\) en \(x=\SI{6.000}{\foot}\) \\
\end{tabular}
\newpage
%\subsection*{Diagramas}
Los diagramas \(V(x)\) y \(M(x)\) se adjuntan a continuación.


\section*{Diagramas}
\begin{figure}[h]
  \centering
  \includegraphics[width=\linewidth]{\detokenize{V_82f3667b9d584671bf012b6b7e7d16f0.png}}
  \caption{Diagrama de cortante \(V(x)\) en \(\SI{}{kips}\).}
\end{figure}

\begin{figure}[h]
  \centering
  \includegraphics[width=\linewidth]{\detokenize{M_32937e20f399475697efbbb119a17f7b.png}}
  \caption{Diagrama de momento \(M(x)\) en \(\SI{}{kips.ft}\).}
\end{figure}

\end{document}
