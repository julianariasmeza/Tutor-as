\documentclass[11pt]{article}
\usepackage[spanish]{babel}
\usepackage[T1]{fontenc}
\usepackage[utf8]{inputenc}
\usepackage{amsmath, amssymb}
\usepackage{graphicx}
\usepackage{siunitx}
\usepackage{geometry}
\geometry{letterpaper, margin=2.3cm}
\sisetup{
  locale = DE,
  output-decimal-marker = {,},
  per-mode = symbol,
  group-minimum-digits = 4
}
\DeclareSIUnit{\kip}{kip}
\DeclareSIUnit{\foot}{ft}

\begin{document}

\section*{Análisis de viga AB}

\subsection*{Datos generales}
Longitud total: \( L = \SI{4,8}{\foot} \)

\begin{center}
\begin{tabular}{l c}
\multicolumn{2}{l}{\textbf{Cargas puntuales}} \\ \hline
\textbf{Carga} & \textbf{Posición} (\si{\foot}) \\ \hline
\( P_1 = \SI{1,5}{\kip} \) & \( x = 0,0 \) \\
\( P_2 = \SI{6,0}{\kip} \) & \( x = 2,4 \) \\
\( P_3 = \SI{1,5}{\kip} \) & \( x = 4,8 \) \\
\end{tabular}
\end{center}

\begin{center}
\begin{tabular}{l c}
\multicolumn{2}{l}{\textbf{Apoyos}} \\ \hline
\textbf{Tipo} & \textbf{Posición} (\si{\foot}) \\ \hline
Pasador (pinned) & \( x = 1,2 \) \\
Pasador (pinned) & \( x = 3,6 \) \\
\end{tabular}
\end{center}

---

\subsection*{Modelado de los apoyos}

En la figura original el apoyo situado en \(G\) aparece desplazado hacia abajo, lo que sugiere una reacción horizontal. Sin embargo, para poder trazar los diagramas de fuerza cortante \(V(x)\) y de momento flector \(M(x)\) de la viga horizontal \(AB\), se adopta el siguiente modelo equivalente:



---

\subsection*{Reacciones}

Por equilibrio global:

\[
\sum F_y = 0 \quad \Rightarrow \quad R_D + R_G = 1{,}5 + 6{,}0 + 1{,}5 = 9{,}0~\si{\kip}.
\]

Debido a la simetría de las cargas respecto al punto medio, las reacciones son iguales:

\[
\boxed{R_D = R_G = \SI{4,5}{\kip}}
\]

\begin{center}
\begin{tabular}{l c}
\multicolumn{2}{l}{\textbf{Reacciones verticales}} \\ \hline
\textbf{Posición} & \textbf{Magnitud (\si{\kip})} \\ \hline
\( x = 1,2~\si{\foot} \) & \( 4,5 \) \\
\( x = 3,6~\si{\foot} \) & \( 4,5 \) \\
\end{tabular}
\end{center}

---

\subsection*{Relaciones básicas}
\[
\frac{dV}{dx} = w(x), \qquad \frac{dM}{dx} = V(x)
\]
Los saltos en \(V(x)\) se producen por cargas puntuales o reacciones; los cambios de pendiente en \(M(x)\) corresponden a dichos saltos en el cortante.

---

\subsection*{Chequeo de equilibrio}
\[
\sum F_y = 4{,}5 + 4{,}5 - (1{,}5 + 6{,}0 + 1{,}5) = 0,
\qquad
\sum M_{(x=0)} = 0.
\]
El equilibrio global se cumple correctamente.

---

\subsection*{Valores extremos}

\begin{center}
\begin{tabular}{lccc}
\hline
 & \textbf{máx (+)} & \textbf{mín (–)} & \(|\cdot|\) \textbf{máx} \\
\hline
\(V(x)\) [\si{\kip}] & \(+3,0\) en \(x=1,2~\si{\foot}\) & \(-3,0\) en \(x=2,4~\si{\foot}\) & \(3,0\) \\
\(M(x)\) [\si{\kip.\foot}] & \(+1,8\) en \(x=2,4~\si{\foot}\) & \(-1,8\) en \(x=1,2~\si{\foot}\) & \(1,8\) \\
\hline
\end{tabular}
\end{center}

---

\subsection*{Conclusión}
Con las reacciones \(R_D = R_G = \SI{4,5}{\kip}\), los diagramas de fuerza cortante y momento flector presentan los valores esperados para una viga simplemente apoyada con cargas puntuales simétricas. Los resultados son consistentes con el equilibrio estático y justifican el valor máximo de cortante de \(\SI{3,0}{\kip}\) y de momento flector de \(\SI{1,8}{\kip.\foot}\).


\newpage
%\subsection*{Diagramas}
Los diagramas \(V(x)\) y \(M(x)\) se adjuntan a continuación.


\section*{Diagramas}
\begin{figure}[h]
  \centering
  \includegraphics[width=\linewidth]{\detokenize{V_1e526040453f4161967685c639c3cbfb.png}}
  \caption{Diagrama de cortante \(V(x)\) en \(\SI{}{kips }\).}
\end{figure}

\begin{figure}[h]
  \centering
  \includegraphics[width=\linewidth]{\detokenize{M_fa61f9fe1f164efd95e4ff2665c4081f.png}}
  \caption{Diagrama de momento \(M(x)\) en \(\SI{}{kips.ft}\).}
\end{figure}

\end{document}
