\documentclass[12pt]{article}
\usepackage[utf8]{inputenc}
\usepackage[spanish]{babel}
\usepackage{amsmath}
\usepackage{geometry}
\geometry{letterpaper, margin=2.5cm}
\usepackage{parskip}

\begin{document}

\section*{Ejercicios de razonamiento lógico con edades}

\begin{enumerate}
    \item Cuando Laura nació, su hermano mayor tenía 8 años. Actualmente, la edad de Laura representa los \(\dfrac{3}{5}\) de la edad de su hermano. ¿Qué edad tiene Laura?
    
    \begin{itemize}
        \item[a)] 12
        \item[b)] 15
        \item[c)] 18
        \item[d)] 20
    \end{itemize}

    \item Cuando Ana nació, su prima tenía 6 años. Hoy, la edad de Ana equivale a los \(\dfrac{4}{7}\) de la edad de su prima. ¿Cuál es la edad actual de la prima?
    
    \begin{itemize}
        \item[a)] 14
        \item[b)] 21
        \item[c)] 28
        \item[d)] 24
    \end{itemize}

    \item Cuando Tomás nació, su madre tenía 25 años. Actualmente, Tomás tiene la mitad de la edad de su madre. ¿Qué edad tiene Tomás?
    
    \begin{itemize}
        \item[a)] 25
        \item[b)] 20
        \item[c)] 30
        \item[d)] 35
    \end{itemize}
\end{enumerate}

\end{document}
