\documentclass[12pt]{article}
\usepackage[spanish]{babel}
\usepackage[utf8]{inputenc}
\usepackage[T1]{fontenc}
\usepackage{geometry}
\geometry{letterpaper, margin=2.5cm}
\usepackage{amsmath, amssymb}
\usepackage{siunitx}
\sisetup{output-decimal-marker = {,}}

\title{Práctica de Sucesiones Numéricas}
\author{}
\date{}

\begin{document}
\maketitle

\section*{Ejercicios}

\begin{enumerate}


    \item Considere la siguiente sucesión:  
    \( 1, 4, 9, 16, 25, t, u \)  
    ¿Cuál es el valor de \( t + u \)?  
    \begin{itemize}
        \item a) 101
        \item b) 113
        \item c) 85
        \item d) 145
    \end{itemize}

  

    \item Considere la siguiente sucesión:  
    \( 3, 6, 12, 24, 48, x, y \)  
    ¿Cuál es el valor de \( x + y \)?  
    \begin{itemize}
        \item a) 336
        \item b) 192
        \item c) 144
        \item d) 288
    \end{itemize}

    \item Considere la siguiente sucesión:  
    \( 7, 14, 21, 28, 35, m, n \)  
    ¿Cuál es el valor de \( m + n \)?  
    \begin{itemize}
        \item a) 98
        \item b) 91
        \item c) 112
        \item d) 126
    \end{itemize}

    \item Considere la siguiente sucesión:  
    \( 2, 4, 8, 16, 32, a, b \)  
    ¿Cuál es el valor de \( a + b \)?  
    \begin{itemize}
        \item a) 192
        \item b) 224
        \item c) 256
        \item d) 288
    \end{itemize}
\end{enumerate}

\newpage
\section*{Soluciones}

\begin{enumerate}
 

    \item Números cuadrados: \( 1^2, 2^2, 3^2, 4^2, 5^2 \)  
    \( t = 6^2 = 36,\quad u = 7^2 = 49 \)  
    \( t + u = 85 \) \(\Rightarrow\) opción c).

  

    \item Progresión geométrica: razón \(2\).  
    \( x = 48 \times 2 = 96,\quad y = 96 \times 2 = 192 \)  
    \( x + y = 288 \) \(\Rightarrow\) opción d).

    \item Progresión aritmética: diferencia \(7\).  
    \( m = 35 + 7 = 42,\quad n = 42 + 7 = 49 \)  
    \( m + n = 91 \).

    \item Progresión geométrica: razón \(2\).  
    \( a = 32 \times 2 = 64,\quad b = 64 \times 2 = 128 \)  
    \( a + b = 192 \) \(\Rightarrow\) opción a).
\end{enumerate}

\end{document}
