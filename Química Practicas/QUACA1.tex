\documentclass[12pt,a4paper]{article}
\usepackage{amsmath}
\usepackage[utf8]{inputenc}
\usepackage{array}
\usepackage{booktabs}

\begin{document}

\section*{Aniones Poliatómicos más comunes (todos poseen oxígeno)}

\begin{tabular}{>{\bfseries}l c c}
\toprule
Nombre & Fórmula & \# de Oxidación \\
\midrule
Hidróxido   & OH   & -1 \\
Nitrito     & NO$_2$  & -1 \\
Nitrato     & NO$_3$  & -1 \\
Hipoclorito & ClO     & -1 \\
Clorito     & ClO$_2$ & -1 \\
Clorato     & ClO$_3$ & -1 \\
Perclorato  & ClO$_4$ & -1 \\
Cianato     & CON     & -1 \\
Hipoyodito  & IO      & -1 \\
Yodito      & IO$_2$  & -1 \\
Yodato      & IO$_3$  & -1 \\
Peryodato   & IO$_4$  & -1 \\
Hipobromito & BrO     & -1 \\
Bromito     & BrO$_2$ & -1 \\
Bromato     & BrO$_3$ & -1 \\
Perbromato  & BrO$_4$ & -1 \\
Permanganato& MnO$_4$ & -1 \\
Manganato   & MnO$_4$ & -2 \\
Sulfito     & SO$_3$  & -2 \\
Sulfato     & SO$_4$  & -2 \\
Carbonato   & CO$_3$  & -2 \\
Cromato     & CrO$_4$ & -2 \\
Dicromato   & Cr$_2$O$_7$ & -2 \\
Zincato     & ZnO$_2$ & -2 \\
Fosfito     & PO$_3$  & -3 \\
Fosfato     & PO$_4$  & -3 \\
Arsenito    & AsO$_3$ & -3 \\
Arsenato    & AsO$_4$ & -3 \\
Borato      & BO$_3$  & -3 \\
Cromito     & CrO$_3$ & -3 \\
Aluminato   & AlO$_3$ & -3 \\
Silicato    & SiO$_3$ & -2 \\
Metasilicato& SiO$_4$ & -4 \\
Difosfato   & P$_2$O$_7$ & -4 \\
\bottomrule
\end{tabular}
\section*{Ejercicios de Nomenclatura Química}

\subsection*{Parte 1: Nombre correcto de los compuestos}

\begin{tabular}{ll}
CuO & Óxido de cobre (II) \\
Fe(OH)$_3$ & Hidróxido de hierro (III) \\
H$_2$S & Sulfuro de hidrógeno \\
HClO$_4$ & Ácido perclórico \\
Al$_2$O$_3$ & Óxido de aluminio (III) \\
H$_2$SiO$_3$ & Ácido silícico \\
SF$_6$ & Hexafluoruro de azufre \\
Co$_2$S$_3$ & Sulfuro de cobalto (III) \\
Mg(ClO$_2$)$_2$ & Clorito de magnesio \\
N$_2$ & Dinitrógeno \\
KHS & Hidrogenosulfuro de potasio \\
NaH$_2$PO$_4$ & Dihidrogenofosfato de sodio \\
N$_2$O$_5$ & Pentóxido de dinitrógeno \\
P$_2$O$_3$ & Trióxido de difósforo \\
CaI$_2$ & Yoduro de calcio \\
H$_3$AsO$_4$ & Ácido arsénico \\
HF(ac) & Ácido fluorhídrico \\
NaHCO$_3$ & Bicarbonato de sodio \\
Na$_2$S & Sulfuro de sodio \\
KClO$_3$ & Clorato de potasio \\
KNO$_2$ & Nitrito de potasio \\
CS$_2$ & Disulfuro de carbono \\
AlPO$_4$ & Fosfato de aluminio \\
SrO & Óxido de estroncio \\
NH$_4$NO$_3$ & Nitrato de amonio \\
Ca(ClO)$_2$ & Hipoclorito de calcio \\
Cl$_2$O$_3$ & Trióxido de dicloro \\
KI & Yoduro de potasio \\
Ca(HS)$_2$ & Hidrogenosulfuro de calcio \\
Al(HSO$_3$)$_3$ & Hidrogenosulfito de aluminio \\
Ba(OH)$_2$ & Hidróxido de bario \\
(NH$_4$)$_2$SO$_4$ & Sulfato de amonio \\
\end{tabular}

---

\subsection*{Parte 2: Fórmula de los compuestos dados}

\begin{tabular}{ll}
Óxido de rubidio & Rb$_2$O \\
Fosfina & PH$_3$ \\
Ácido sulfuroso & H$_2$SO$_3$ \\
Tetrafluoruro de azufre & SF$_4$ \\
Borato de sodio & NaBO$_2$ \\
Bromuro de cobre (II) & CuBr$_2$ \\
Sulfato de aluminio & Al$_2$(SO$_4$)$_3$ \\
Ácido clórico & HClO$_3$ \\
Cloruro de amonio & NH$_4$Cl \\
Dicromato de amonio & (NH$_4$)$_2$Cr$_2$O$_7$ \\
Nitrato de cromo (III) & Cr(NO$_3$)$_3$ \\
Arseniato de aluminio & AlAsO$_4$ \\
Ársina & AsH$_3$ \\
Cromato de sodio & Na$_2$CrO$_4$ \\
Hipoclorito de calcio & Ca(ClO)$_2$ \\
Óxido de azufre (IV) & SO$_2$ \\
Ácido silícico & H$_2$SiO$_3$ \\
Óxido de manganeso (IV) & MnO$_2$ \\
Perclorato de bario & Ba(ClO$_4$)$_2$ \\
Hidrógeno carbonato de amonio & NH$_4$HCO$_3$ \\
Cloruro de amonio & NH$_4$Cl \\
Hidrógeno sulfito de potasio & KHSO$_3$ \\
Monóxido de carbono & CO \\
Dihidrógeno fosfato de calcio & Ca(H$_2$PO$_4$)$_2$ \\
Arsenito de cobre (I) & Cu$_3$(AsO$_3$)$_2$ \\
Nitrato de mercurio (II) & Hg(NO$_3$)$_2$ \\
Sulfuro de sodio & Na$_2$S \\
Pentasulfuro de diarsénico & As$_2$S$_5$ \\
Ácido carbónico & H$_2$CO$_3$ \\
Arsenito de mercurio (II) & Hg$_3$(AsO$_3$)$_2$ \\
Nitrito de potasio & KNO$_2$ \\
Permanganato de potasio & KMnO$_4$ \\
Ácido clorhídrico & HCl(ac) \\
Carbonato de cobre (I) & Cu$_2$CO$_3$ \\
Amoniaco & NH$_3$ \\
\end{tabular}
\section*{Manganato vs. Permanganato}

En química inorgánica, los iones \textbf{manganato} y \textbf{permanganato} corresponden a oxoaniones del manganeso en diferentes estados de oxidación:

\subsection*{1. Manganato}
\begin{itemize}
    \item Fórmula: \(\text{MnO}_4^{2-}\)
    \item Estado de oxidación del manganeso: \(+6\)
    \item Color característico: verde.
    \item Se forma en condiciones alcalinas (básicas), por ejemplo en la disolución de manganeso en medio fuertemente básico con oxidantes.
    \item Ejemplo de compuesto: \(\text{Na}_2\text{MnO}_4\) (manganato de sodio).
\end{itemize}

\subsection*{2. Permanganato}
\begin{itemize}
    \item Fórmula: \(\text{MnO}_4^{-}\)
    \item Estado de oxidación del manganeso: \(+7\)
    \item Color característico: violeta intenso.
    \item Es un agente oxidante muy fuerte, estable en medio ácido y común en reacciones redox.
    \item Ejemplo de compuesto: \(\text{KMnO}_4\) (permanganato de potasio).
\end{itemize}

\subsection*{3. Relación entre ambos}
\begin{itemize}
    \item El \textbf{manganato} \((\text{MnO}_4^{2-})\) puede desproporcionarse en medio ácido para dar \textbf{permanganato} \((\text{MnO}_4^{-})\) y dióxido de manganeso \((\text{MnO}_2)\).
    \item Esta reacción explica por qué muchas veces, en soluciones, el manganato se transforma en permanganato.
\end{itemize}

\subsection*{4. Resumen}
\[
\text{Manganato: MnO}_4^{2-},\; \text{Mn en } +6,\; \text{verde}
\]
\[
\text{Permanganato: MnO}_4^{-},\; \text{Mn en } +7,\; \text{violeta}
\]
\section*{¿Cómo reconocer un manganato o un permanganato en compuestos químicos?}

Cuando nos dan un compuesto con \(\text{Mn}\) y oxígeno, debemos fijarnos en la \textbf{carga del anión} y en el \textbf{estado de oxidación del manganeso}. La diferencia está en el subíndice y la carga del ion:

\subsection*{1. Permanganato}
\begin{itemize}
    \item Fórmula general: \(\text{MnO}_4^{-}\).
    \item El manganeso tiene estado de oxidación \(+7\).
    \item Siempre aparece como \(\text{MnO}_4\) con \(-1\).
    \item Ejemplos:
    \begin{itemize}
        \item \(\text{KMnO}_4\): permanganato de potasio.
        \item \(\text{NaMnO}_4\): permanganato de sodio.
    \end{itemize}
\end{itemize}

\subsection*{2. Manganato}
\begin{itemize}
    \item Fórmula general: \(\text{MnO}_4^{2-}\).
    \item El manganeso tiene estado de oxidación \(+6\).
    \item Siempre aparece como \(\text{MnO}_4\) con \(-2\).
    \item Ejemplos:
    \begin{itemize}
        \item \(\text{Na}_2\text{MnO}_4\): manganato de sodio.
        \item \(\text{K}_2\text{MnO}_4\): manganato de potasio.
    \end{itemize}
\end{itemize}

\subsection*{3. Estrategia de reconocimiento}
\begin{enumerate}
    \item Identificar el ion \(\text{MnO}_4\).
    \item Revisar la carga:
    \begin{itemize}
        \item Si el compuesto tiene \(\text{M}^+\) (un catión con carga +1, como K$^+$ o Na$^+$), entonces el anión debe ser \(\text{MnO}_4^{-}\) y por lo tanto es \textbf{permanganato}.
        \item Si el compuesto tiene \(\text{M}^{2+}\) o dos cationes monovalentes, el anión debe ser \(\text{MnO}_4^{2-}\) y será \textbf{manganato}.
    \end{itemize}
\end{enumerate}

\subsection*{4. Ejemplos prácticos}
\begin{itemize}
    \item \(\text{KMnO}_4\): el potasio es +1, entonces el anión debe ser -1 → \textbf{permanganato}.
    \item \(\text{Na}_2\text{MnO}_4\): dos sodios (+2 en total), entonces el anión debe ser -2 → \textbf{manganato}.
\end{itemize}
\end{document}