\documentclass[12pt]{article}

% ===== Paquetes =====
\usepackage[spanish]{babel}
\usepackage[utf8]{inputenc}
\usepackage[T1]{fontenc}
\usepackage{amsmath, amssymb}
\usepackage{mhchem}      % Fórmulas químicas
\usepackage{booktabs}    % Tablas bonitas
\usepackage{geometry}
\usepackage{pdflscape} % o \usepackage{lscape}
\geometry{letterpaper, margin=2.5cm}

% ===== Encabezados y pies con numeración =====
\usepackage{fancyhdr}
% \usepackage{lastpage} % <-- Descomentar si querés "Página X de Y"

\pagestyle{fancy}
\fancyhf{} % Limpia encabezados y pies

% Encabezado
\fancyhead[L]{Julián Arias Meza}    % Esquina superior izquierda
\fancyhead[R]{Práctica de Química}  % Esquina superior derecha

% Pie de página con número
\fancyfoot[C]{\thepage}             % Número de página centrado en el pie
% --- Alternativa: "Página X de Y"
% \fancyfoot[C]{Página \thepage\ de \pageref{LastPage}}

\renewcommand{\headrulewidth}{0.4pt}
\renewcommand{\footrulewidth}{0pt}
\setlength{\headheight}{14pt}       % Evita warnings de fancyhdr
\pagenumbering{arabic}              % Asegura numeración arábiga


\begin{document}

\begin{center}
\Large \textbf{Práctica de Química}\\[0.3cm]

\end{center}

\section*{Formación de óxidos metálicos}

{\small
\begin{tabular}{@{}ll@{}}
\toprule
Elemento & Estados de oxidación comunes \\
\midrule
\(\mathrm{Ti}\) & \(+2,+3,+4\) \\
\(\mathrm{V}\) & \(+2,+3,+5\) \\
\(\mathrm{Cr}\) & \(+2,+3,+6\) \\
\(\mathrm{Mn}\) & \(+2,+4,+7\) \\
\(\mathrm{Fe}\) & \(+2,+3\) \\
\(\mathrm{Co}\) & \(+2,+3\) \\
\(\mathrm{Ni}\) & \(+2\) \\
\(\mathrm{Cu}\) & \(+1,+2\) \\
\(\mathrm{Zn}\) & \(+2\) \\
\(\mathrm{Ag}\) & \(+1\) \\
\(\mathrm{Cd}\) & \(+2\) \\
\(\mathrm{Sn}\) & \(+2,+4\) \\
\(\mathrm{Pb}\) & \(+2,+4\) \\
\(\mathrm{Pt}\) & \(+2,+4\) \\
\(\mathrm{Au}\) & \(+1,+3\) \\
\(\mathrm{Hg}\) & \(+1~(\mathrm{Hg}_2^{2+}),+2\) \\
\(\mathrm{Bi}\) & \(+3,+5\) \\
\(\mathrm{Al},\mathrm{Ga},\mathrm{In}\) & \(+3\) \\
\(\mathrm{Sc},\mathrm{Y},\mathrm{La}\) & \(+3\) \\
\(\mathrm{Li},\mathrm{Na},\mathrm{K},\mathrm{Rb},\mathrm{Cs},\mathrm{Fr}\) & \(+1\) \\
\(\mathrm{Be},\mathrm{Mg},\mathrm{Ca},\mathrm{Sr},\mathrm{Ba},\mathrm{Ra}\) & \(+2\) \\
\bottomrule
\end{tabular}
} \\ \\

Los \textbf{óxidos metálicos} son compuestos binarios formados por un elemento metálico y oxígeno. 
En general, su fórmula química es:
\[
\ce{M_x O_y}
\]
donde \(\ce{M}\) representa el metal y \(\ce{O}\) el oxígeno. 

\subsection*{Ejemplo de formación}
Si el sodio (Na) reacciona con oxígeno (\(\ce{O2}\)), se obtiene óxido de sodio:
\[
\ce{4Na (s) + O2 (g) -> 2Na2O (s)}
\]
\newpage
\section*{Sistema de Stock}
El \textbf{sistema de Stock} se utiliza para nombrar óxidos metálicos cuando el metal puede presentar más de un número de oxidación. 
Se coloca el nombre del compuesto seguido del número de oxidación del metal en números romanos y entre paréntesis.

\subsection*{Ejemplos}
\begin{itemize}
  \item \ce{FeO}: óxido de hierro (II).
  \item \ce{Fe2O3}: óxido de hierro (III).
  \item \ce{Cu2O}: óxido de cobre (I).
  \item \ce{CuO}: óxido de cobre (II).
\end{itemize}

\section*{¿Cuándo se usa el sistema de Stock?}
El \textbf{sistema de Stock} se emplea únicamente cuando un elemento puede presentar \textbf{más de un número de oxidación}.  
En esos casos, se indica con números romanos entre paréntesis después del nombre del metal.

\subsection*{Ejemplos donde se usa Stock}
\begin{itemize}
  \item \ce{FeO}: óxido de hierro (II), ya que el hierro puede ser \(+2\) o \(+3\).
  \item \ce{Fe2O3}: óxido de hierro (III).
  \item \ce{Cu2O}: óxido de cobre (I).
  \item \ce{CuO}: óxido de cobre (II).
  \item \ce{SnO2}: óxido de estaño (IV), porque el estaño puede ser \(+2\) o \(+4\).
\end{itemize}

\subsection*{Ejemplos donde \underline{no} se usa Stock}
Cuando el metal solo presenta un número de oxidación común, no es necesario aclararlo.  
\begin{itemize}
  \item \ce{Na2O}: óxido de sodio (\(Na\) siempre es \(+1\)).
  \item \ce{MgO}: óxido de magnesio (\(Mg\) siempre es \(+2\)).
  \item \ce{ZnO}: óxido de zinc (\(Zn\) siempre es \(+2\)).
  \item \ce{Ag2O}: óxido de plata (\(Ag\) casi siempre \(+1\)).
\end{itemize}
\section*{Hidruros}

Los \textbf{hidruros} son compuestos binarios entre el hidrógeno y otro elemento. En química inorgánica se distinguen, a grandes rasgos:
\begin{itemize}
  \item \textbf{Hidruros metálicos (iónicos)}: el hidrógeno actúa como anión \(\ce{H^-}\).
  \item \textbf{Hidruros covalentes (no metálicos)}: el enlace es compartido; muchos corresponden a \(\ce{EH_n}\) (por ejemplo, \(\ce{CH4}\), \(\ce{NH3}\), \(\ce{SiH4}\)).
\end{itemize}

\subsection*{Formación (reacciones típicas)}
\begin{align*}
\ce{2Na (s) + H2 (g) &-> 2NaH (s)}\\
\ce{Ca (s) + H2 (g) &-> CaH2 (s)}\\
\ce{2Al (s) + 3H2 (g) &-> 2AlH3 (s)} \quad &\text{(simplificada, en la práctica forma hidruros complejos)}
\end{align*}

\subsection*{Nombrado}
\textbf{Regla general (Stock):} \emph{hidruro de} \([\,\)metal\(\,]\) \((\mathrm{N\!O~en~romanos})\) cuando el metal presenta más de un estado de oxidación.\\
Si el metal \textbf{solo} presenta un estado de oxidación común, \textbf{no} se indica número romano.

\subsubsection*{Ejemplos (se usa Stock)}
\begin{itemize}
  \item \ce{FeH2}: hidruro de hierro (II).
  \item \ce{FeH3}: hidruro de hierro (III).
  \item \ce{CuH}: hidruro de cobre (I).
  \item \ce{CuH2}: hidruro de cobre (II).
\end{itemize}

\subsubsection*{Ejemplos (no se usa Stock)}
\begin{itemize}
  \item \ce{NaH}: hidruro de sodio (\(Na\) es \(+1\)).
  \item \ce{CaH2}: hidruro de calcio (\(Ca\) es \(+2\)).
  \item \ce{KH}: hidruro de potasio (\(K\) es \(+1\)).
\end{itemize}
\newpage
\section*{Hidróxidos}

Los \textbf{hidróxidos} son compuestos iónicos que contienen el anión hidróxido \(\ce{OH^-}\), con fórmula general \(\ce{M(OH)_{n}}\), donde \(n\) coincide con la \textbf{valencia (número de oxidación)} del metal en el compuesto. Son bases típicas en disolución acuosa.

\subsection*{Formación (rutas comunes)}
\begin{align*}
\ce{Na2O (s) + H2O (l) &-> 2NaOH (ac)} \quad &\text{(óxido básico + agua)}\\
\ce{CaO (s) + H2O (l) &-> Ca(OH)2 (ac)} \\
\ce{2Na (s) + 2H2O (l) &-> 2NaOH (ac) + H2 (g)} \quad &\text{(metal alcalino + agua)}
\end{align*}

\subsection*{Nombrado}
\textbf{Regla general:} \emph{hidróxido de} \([\,\)metal\(\,]\) \((\mathrm{N\!O~en~romanos})\) si el metal puede tener varios estados de oxidación.\\
Si el metal tiene \textbf{valencia única} (común), \textbf{no} se indica número romano.

\subsubsection*{Ejemplos (se usa Stock)}
\begin{itemize}
  \item \(\ce{Fe(OH)2}\): hidróxido de hierro (II).
  \item \(\ce{Fe(OH)3}\): hidróxido de hierro (III).
  \item \(\ce{CuOH}\): hidróxido de cobre (I).
  \item \(\ce{Cu(OH)2}\): hidróxido de cobre (II).
  \item \(\ce{Sn(OH)2}\) y \(\ce{Sn(OH)4}\): hidróxido de estaño (II) y (IV).
\end{itemize}

\subsubsection*{Ejemplos (no se usa Stock)}
\begin{itemize}
  \item \(\ce{NaOH}\): hidróxido de sodio (\(Na\) es \(+1\)).
  \item \(\ce{KOH}\): hidróxido de potasio (\(K\) es \(+1\)).
  \item \(\ce{Ca(OH)2}\): hidróxido de calcio (\(Ca\) es \(+2\)).
  \item \(\ce{Ba(OH)2}\): hidróxido de bario (\(Ba\) es \(+2\)).
\end{itemize}

\subsection*{Observaciones útiles}
\begin{itemize}
  \item El subíndice en \(\ce{(OH)_{n}}\) refleja la \textbf{carga} del catión metálico: p.ej. \(n=3\) en \(\ce{Al(OH)3}\) porque \(Al^{3+}\).
  \item Al calentar, muchos hidróxidos metálicos se deshidratan: 
  \(\;\ce{Fe(OH)3 (s) ->[\Delta] Fe2O3 (s) + 3H2O (g)}\) (global por 2 moles de \(\ce{Fe(OH)3}\)).
\end{itemize}
\section*{Radicales simples y formación de hidrácidos}
\section*{Radicales simples más comunes}

\small
\begin{tabular}{@{}lll@{}}
\toprule
Símbolo & Nombre del radical & Valencia (carga) \\
\midrule
\(\ce{H^-}\) & hidruro & \(-1\) \\
\(\ce{F^-}\) & fluoruro & \(-1\) \\
\(\ce{Cl^-}\) & cloruro & \(-1\) \\
\(\ce{Br^-}\) & bromuro & \(-1\) \\
\(\ce{I^-}\) & yoduro & \(-1\) \\
\(\ce{O^{2-}}\) & óxido & \(-2\) \\
\(\ce{S^{2-}}\) & sulfuro & \(-2\) \\
\(\ce{Se^{2-}}\) & seleniuro & \(-2\) \\
\(\ce{Te^{2-}}\) & telururo & \(-2\) \\
\(\ce{N^{3-}}\) & nitruro & \(-3\) \\
\(\ce{P^{3-}}\) & fosfuro & \(-3\) \\
\bottomrule
\end{tabular}

\subsection*{Radicales simples}
En química inorgánica se denomina \textbf{radical simple} al ion monoatómico con carga eléctrica, es decir, un solo átomo que ha ganado o perdido electrones. 

\begin{itemize}
  \item Los no metales de los grupos \(\ce{17}\) (halógenos), \(\ce{16}\) (calcógenos) y algunos del grupo \(\ce{15}\) forman con facilidad radicales aniónicos simples.
  \item Ejemplos comunes:
  \begin{itemize}
    \item \(\ce{Cl^-}\): cloruro.
    \item \(\ce{Br^-}\): bromuro.
    \item \(\ce{I^-}\): yoduro.
    \item \(\ce{S^{2-}}\): sulfuro.
  \end{itemize}
\end{itemize}

Estos radicales simples son la base para formar \textbf{hidrácidos} al combinarse con el hidrógeno.

\subsection*{Hidrácidos: definición}
Los \textbf{hidrácidos} son compuestos binarios formados por hidrógeno y un no metal (generalmente de los grupos 16 o 17). 

Su fórmula general es:
\[
\ce{H_n X}
\]
donde \(\ce{X}\) es un radical simple aniónico (halógeno, calcógeno, etc.).

\subsection*{Formación y nomenclatura}
Dependiendo del estado físico o la disolución, se nombran de forma distinta:

\begin{enumerate}
  \item \textbf{En fase gaseosa:} se nombran como \emph{halogenuros de hidrógeno} o \emph{sulfuros de hidrógeno}.  
  Ejemplos:
  \begin{itemize}
    \item \(\ce{HCl (g)}\): cloruro de hidrógeno.
    \item \(\ce{HBr (g)}\): bromuro de hidrógeno.
    \item \(\ce{H2S (g)}\): sulfuro de hidrógeno.
  \end{itemize}
  
  \item \textbf{En disolución acuosa (hidrácidos propiamente dichos):} se nombran como \emph{ácidos} + \emph{raíz del no metal} + \emph{-hídrico}.  
  Ejemplos:
  \begin{itemize}
    \item \(\ce{HCl (ac)}\): ácido clorhídrico.
    \item \(\ce{HBr (ac)}\): ácido bromhídrico.
    \item \(\ce{HI (ac)}\): ácido yodhídrico.
    \item \(\ce{H2S (ac)}\): ácido sulfhídrico.
  \end{itemize}
\end{enumerate}

\subsection*{Reacciones de formación}
\begin{align*}
\ce{H2 (g) + Cl2 (g) &-> 2HCl (g)} \quad &\text{(hidrácido en fase gaseosa)}\\
\ce{HCl (g) ->[agua] HCl (ac)} \quad &\text{(hidrácido en disolución acuosa)}
\end{align*}

\subsection*{Resumen}
\begin{itemize}
  \item En \textbf{fase gaseosa}: nombre de tipo \emph{halogenuro de hidrógeno}.  
  \item En \textbf{fase acuosa}: nombre de tipo \emph{ácido hídrico}.  
\end{itemize}
\section*{Sales binarias con metales}

Las \textbf{sales binarias} son compuestos iónicos formados por un metal y un no metal (radical simple). 
Su fórmula general es:

\[
\ce{M_x X_y}
\]

donde \(\ce{M}\) representa el metal con carga positiva (catión) y \(\ce{X}\) el radical simple con carga negativa (anión).

\subsection*{Formación}
Ejemplos típicos de reacciones de formación:

\begin{align*}
\ce{2Na (s) + Cl2 (g) &-> 2NaCl (s)} \\
\ce{Fe (s) + S (s) &-> FeS (s)} \\
\ce{Ca (s) + Cl2 (g) &-> CaCl2 (s)} 
\end{align*}

\subsection*{Nombrado (sistema de Stock)}
\begin{itemize}
  \item Si el metal presenta \textbf{varias valencias}, se debe indicar con números romanos.
  \item Si el metal presenta \textbf{una única valencia}, no se indica número romano.
\end{itemize}

\subsubsection*{Ejemplos donde se usa Stock}
\begin{itemize}
  \item \ce{FeCl2}: cloruro de hierro (II).
  \item \ce{FeCl3}: cloruro de hierro (III).
  \item \ce{CuCl}: cloruro de cobre (I).
  \item \ce{CuCl2}: cloruro de cobre (II).
  \item \ce{SnS2}: sulfuro de estaño (IV).
\end{itemize}

\subsubsection*{Ejemplos donde no se usa Stock}
\begin{itemize}
  \item \ce{NaCl}: cloruro de sodio (\(Na\) siempre \(+1\)).
  \item \ce{KBr}: bromuro de potasio (\(K\) siempre \(+1\)).
  \item \ce{CaCl2}: cloruro de calcio (\(Ca\) siempre \(+2\)).
  \item \ce{MgO}: óxido de magnesio (\(Mg\) siempre \(+2\)).
  \item \ce{AgCl}: cloruro de plata (\(Ag\) casi siempre \(+1\)).
\end{itemize}

\subsection*{Observaciones}
\begin{itemize}
  \item El subíndice en la fórmula se ajusta para que la suma de cargas positivas y negativas sea igual a cero.
  \item Ejemplo: en \(\ce{CaCl2}\), el \(Ca^{2+}\) requiere 2 aniones \(Cl^-\) para neutralizar la carga.
  \item Muchas sales binarias son \textbf{electrolitos} en disolución acuosa.
\end{itemize}
\newpage
\section*{Óxidos no metálicos y sistema estequiométrico}

Los \textbf{óxidos no metálicos}, también llamados \textbf{anhídridos}, son compuestos formados por un no metal y oxígeno.  
Su fórmula general es:

\[
\ce{X_x O_y}
\]

donde \(\ce{X}\) es un no metal y \(\ce{O}\) el oxígeno.

\subsection*{Formación (reacciones típicas)}
\begin{align*}
\ce{C (s) + O2 (g) &-> CO2 (g)} \\
\ce{2N2 (g) + O2 (g) &-> 2N2O (g)} \\
\ce{2SO2 (g) + O2 (g) &-> 2SO3 (g)}
\end{align*}

\subsection*{Nomenclatura: sistema estequiométrico}
En el caso de los óxidos no metálicos se utiliza el \textbf{sistema estequiométrico}, que emplea \textbf{prefijos griegos} para indicar la proporción de átomos en la fórmula.

\begin{itemize}
  \item \textbf{Mono-} (1)
  \item \textbf{Di-} (2)
  \item \textbf{Tri-} (3)
  \item \textbf{Tetra-} (4)
  \item \textbf{Penta-} (5)
  \item \textbf{Hexa-} (6)
  \item \textbf{Hepta-} (7)
  \item \textbf{Octa-} (8)
\end{itemize}

\textbf{Reglas:}
\begin{enumerate}
  \item Se nombran como \emph{óxido} seguido del prefijo y el nombre del no metal.
  \item El prefijo \textbf{mono-} solo se utiliza en el segundo elemento (ejemplo: monóxido de carbono, no “monoóxido”).
  \item Si el nombre resultante tiene dos vocales seguidas, se suele eliminar una (ejemplo: monóxido, no “monoóxido”).
\end{enumerate}

\subsection*{Ejemplos}
\begin{itemize}
  \item \(\ce{CO}\): monóxido de carbono.
  \item \(\ce{CO2}\): dióxido de carbono.
  \item \(\ce{SO2}\): dióxido de azufre.
  \item \(\ce{SO3}\): trióxido de azufre.
  \item \(\ce{N2O}\): monóxido de dinitrógeno (óxido nitroso).
  \item \(\ce{N2O5}\): pentóxido de dinitrógeno.
  \item \(\ce{P2O5}\): pentóxido de difósforo.
  \item \(\ce{Cl2O7}\): heptóxido de dicloro.
\end{itemize}

\subsection*{Observaciones}
\begin{itemize}
  \item Muchos óxidos no metálicos reaccionan con agua para formar \textbf{oxácidos}.  
  Ejemplo: \(\ce{SO3 + H2O -> H2SO4}\).
  \item Algunos óxidos no metálicos son contaminantes atmosféricos importantes (\(\ce{SO2}\), \(\ce{NO2}\)).
\end{itemize}
\section*{Sales binarias entre no metales y radicales simples}

Las \textbf{sales binarias entre no metales y radicales simples} son compuestos iónicos o covalentes en los cuales un \textbf{no metal} actúa como catión o anión, combinándose con otro no metal en forma de radical simple.

\subsection*{Formación}
\begin{align*}
\ce{NH3 (g) + HCl (g) &-> NH4Cl (s)} \quad &\text{(cloruro de amonio)} \\
\ce{N2 (g) + O2 (g) &-> 2NO (g)} \quad &\text{(monóxido de nitrógeno)} 
\end{align*}

\subsection*{Tipos principales}
\begin{itemize}
  \item \textbf{Sales de amonio:} el radical \(\ce{NH4^+}\) (amonio) se combina con haluros o sulfuros.  
    Ejemplos:
    \begin{itemize}
      \item \(\ce{NH4Cl}\): cloruro de amonio.
      \item \(\ce{(NH4)2S}\): sulfuro de amonio.
    \end{itemize}

  \item \textbf{Halogenuros de no metales:} un no metal se combina con radicales halógenos.  
    Ejemplos:
    \begin{itemize}
      \item \(\ce{PCl3}\): tricloruro de fósforo.
      \item \(\ce{PCl5}\): pentacloruro de fósforo.
      \item \(\ce{SF6}\): hexafluoruro de azufre.
    \end{itemize}

  \item \textbf{Nitruros y sulfuros no metálicos:} el nitrógeno o azufre se enlaza directamente con otros no metales.  
    Ejemplos:
    \begin{itemize}
      \item \(\ce{C3N4}\): tetranitruro de tricarbono.
      \item \(\ce{CS2}\): disulfuro de carbono.
    \end{itemize}
\end{itemize}

\subsection*{Nomenclatura}
En la mayoría de estos compuestos se aplica el \textbf{sistema estequiométrico}, indicando el número de átomos mediante prefijos griegos.

\subsubsection*{Ejemplos}
\begin{itemize}
  \item \(\ce{NCl3}\): tricloruro de nitrógeno.
  \item \(\ce{PBr3}\): tribromuro de fósforo.
  \item \(\ce{AsF5}\): pentafluoruro de arsénico.
\end{itemize}

\subsection*{Observaciones}
\begin{itemize}
  \item En compuestos como \(\ce{NH4Cl}\), aunque el amonio contiene un no metal, el radical \(\ce{NH4^+}\) se comporta como un “metal” por su carácter catiónico.
  \item Los halogenuros de no metales suelen ser volátiles y muchos son precursores de \textbf{oxácidos} al reaccionar con agua (ejemplo: \(\ce{PCl5 + H2O -> H3PO4 + HCl}\)).
\end{itemize}
\newpage
\section*{Obtención de números de oxidación}

El \textbf{número de oxidación} (N.O.) de un elemento en un compuesto indica la carga real o aparente que tendría si los electrones se asignaran siguiendo reglas de electronegatividad.  

\subsection*{Regla general}
La suma de todos los números de oxidación en una molécula o ion debe ser igual a:
\[
\text{Carga total del compuesto o ion}
\]

\begin{itemize}
  \item En moléculas neutras: la suma de N.O. = \(0\).
  \item En iones: la suma de N.O. = carga del ion.
\end{itemize}

\subsection*{Ejemplo 1: \(\ce{KMnO4}\)}
Se sabe que:
\begin{itemize}
  \item El potasio (\(\ce{K}\)) siempre es \(+1\).
  \item El oxígeno (\(\ce{O}\)) usualmente es \(-2\).
\end{itemize}

Planteamos la ecuación:
\[
(+1) + (x) + 4(-2) = 0
\]
\[
+1 + x - 8 = 0 \quad \Rightarrow \quad x = +7
\]

Por lo tanto:
\[
\ce{K^{+1}Mn^{+7}O4^{2-}}
\]

\subsection*{Ejemplo 2: \(\ce{Na2SO4}\)}
Se sabe que:
\begin{itemize}
  \item El sodio (\(\ce{Na}\)) es \(+1\).
  \item El oxígeno (\(\ce{O}\)) es \(-2\).
\end{itemize}

Planteamos:
\[
2(+1) + (x) + 4(-2) = 0
\]
\[
+2 + x - 8 = 0 \quad \Rightarrow \quad x = +6
\]

Por lo tanto:
\[
\ce{Na^{+1}2S^{+6}O4^{2-}}
\]

\subsection*{Ejemplo 3: \(\ce{SO4^{2-}}\)}
\[
(x) + 4(-2) = -2
\]
\[
x - 8 = -2 \quad \Rightarrow \quad x = +6
\]

\(\therefore\) El azufre tiene N.O. \(+6\).

\subsection*{Ejemplo 4: \(\ce{NH4^+}\)}
Se sabe que:
\begin{itemize}
  \item El hidrógeno (\(\ce{H}\)) es \(+1\).
\end{itemize}

Planteamos:
\[
(x) + 4(+1) = +1
\]
\[
x + 4 = +1 \quad \Rightarrow \quad x = -3
\]

Por lo tanto:
\[
\ce{N^{-3}H4^{+1}}
\]

\subsection*{Ejemplo 5: \(\ce{Cr2O7^{2-}}\)}
\[
2(x) + 7(-2) = -2
\]
\[
2x - 14 = -2 \quad \Rightarrow \quad 2x = +12 \quad \Rightarrow \quad x = +6
\]

El cromo tiene N.O. \(+6\).

\subsection*{Ejemplo 6: \(\ce{H2O2}\) (peróxido de hidrógeno)}
\[
2(+1) + 2(x) = 0
\]
\[
+2 + 2x = 0 \quad \Rightarrow \quad x = -1
\]

Aquí el oxígeno tiene N.O. \(-1\) (caso especial de los peróxidos).
\newpage
\section*{Tipos de reacciones químicas}

Las reacciones químicas se pueden clasificar de diversas formas. Una de las clasificaciones más comunes es según el \textbf{proceso que ocurre entre los reactivos y productos}.  

\subsection*{1. Reacciones de combinación o síntesis}
Dos o más sustancias se combinan para formar un solo producto.  
\[
\ce{A + B -> AB}
\]

\textbf{Ejemplos:}
\begin{align*}
\ce{2H2 (g) + O2 (g) &-> 2H2O (l)} \\
\ce{2Na (s) + Cl2 (g) &-> 2NaCl (s)}
\end{align*}

\subsection*{2. Reacciones de descomposición}
Una sola sustancia se descompone en dos o más productos. Generalmente requieren energía (calor, electricidad o luz).  
\[
\ce{AB -> A + B}
\]

\textbf{Ejemplos:}
\begin{align*}
\ce{2H2O2 (ac) ->[MnO2] 2H2O (l) + O2 (g)} \\
\ce{CaCO3 (s) ->[\Delta] CaO (s) + CO2 (g)}
\end{align*}

\subsection*{3. Reacciones de desplazamiento simple (sustitución)}
Un elemento libre reemplaza a otro en un compuesto.  
\[
\ce{A + BC -> AC + B}
\]

\textbf{Ejemplos:}
\begin{align*}
\ce{Zn (s) + 2HCl (ac) -> ZnCl2 (ac) + H2 (g)} \\
\ce{Cl2 (g) + 2NaBr (ac) -> 2NaCl (ac) + Br2 (l)}
\end{align*}

\subsection*{4. Reacciones de doble desplazamiento (metátesis)}
Ocurre un intercambio de iones entre dos compuestos en disolución acuosa.  
\[
\ce{AB + CD -> AD + CB}
\]

\textbf{Ejemplos:}
\begin{align*}
\ce{AgNO3 (ac) + NaCl (ac) &-> AgCl (s) + NaNO3 (ac)} \\
\ce{BaCl2 (ac) + Na2SO4 (ac) &-> BaSO4 (s) + 2NaCl (ac)}
\end{align*}

\subsection*{Observaciones}
\begin{itemize}
  \item En las reacciones de \textbf{doble desplazamiento}, muchas veces uno de los productos es insoluble y precipita.
  \item Las de \textbf{desplazamiento simple} dependen de la serie de reactividad del metal o no metal.
  \item Las de \textbf{descomposición} suelen requerir aporte de energía externa.
\end{itemize}
\section*{Clasificación energética y especial de reacciones químicas}

Además de su clasificación estructural, las reacciones químicas se pueden dividir de acuerdo al \textbf{intercambio de energía} y a las \textbf{características del proceso}.

\subsection*{1. Reacciones endotérmicas}
Son aquellas que \textbf{absorben energía} del medio, generalmente en forma de calor.  
\[
\ce{A + B +   [\Delta]  -> C}
\]

\textbf{Ejemplos:}
\begin{align*}
\ce{N2 (g) + O2 (g) ->[ \Delta ] 2NO (g)} \\
\ce{CaCO3 (s) ->[ \Delta ] CaO (s) + CO2 (g)}
\end{align*}

\subsection*{2. Reacciones exotérmicas}
Son aquellas que \textbf{liberan energía} al medio, en forma de calor, luz o sonido.  
\[
\ce{A + B -> C + \Delta}
\]

\textbf{Ejemplos:}
\begin{align*}
\ce{2H2 (g) + O2 (g) -> 2H2O (l) + \Delta} \\
\ce{CH4 (g) + 2O2 (g) -> CO2 (g) + 2H2O (g) + \Delta}
\end{align*}

\subsection*{3. Reacciones de combustión}
Son reacciones de \textbf{oxidación rápida} en las que una sustancia (generalmente un hidrocarburo) reacciona con oxígeno, liberando una gran cantidad de energía.  

\textbf{Ejemplos:}
\begin{align*}
\ce{CH4 (g) + 2O2 (g) -> CO2 (g) + 2H2O (g) + \Delta} \\
\ce{C3H8 (g) + 5O2 (g) -> 3CO2 (g) + 4H2O (g) + \Delta}
\end{align*}

\subsection*{4. Reacciones de precipitación}
Ocurren en disolución acuosa cuando la combinación de dos reactivos produce un \textbf{compuesto insoluble} que se separa como sólido (precipitado).  

\textbf{Ejemplos:}
\begin{align*}
\ce{AgNO3 (ac) + NaCl (ac) -> AgCl (s) v + NaNO3 (ac)} \\
\ce{BaCl2 (ac) + Na2SO4 (ac) -> BaSO4 (s) v + 2NaCl (ac)}
\end{align*}

\subsection*{Observaciones}
\begin{itemize}
  \item En las endotérmicas, el sistema \textbf{absorbe calor} y la temperatura del entorno puede disminuir.
  \item En las exotérmicas, el sistema \textbf{libera calor} y la temperatura del entorno aumenta.
  \item La combustión es un caso especial de reacción \textbf{altamente exotérmica}.
  \item Las reacciones de precipitación son muy utilizadas en química analítica para identificar iones en solución.
\end{itemize}
\section*{Clasificación de compuestos binarios, ternarios y cuaternarios}

Los compuestos químicos se pueden clasificar según el número de \textbf{elementos diferentes} que los forman.  
Un truco práctico es contar las \textbf{mayúsculas} en la fórmula química (cada mayúscula representa el símbolo de un elemento nuevo).  

\subsection*{1. Compuestos binarios}
Están formados por \textbf{dos elementos diferentes}.  
\[
\ce{AB}
\]

\textbf{Ejemplos:}
\begin{itemize}
  \item \(\ce{NaCl}\) → sodio (Na) y cloro (Cl).
  \item \(\ce{H2O}\) → hidrógeno (H) y oxígeno (O).
  \item \(\ce{CO2}\) → carbono (C) y oxígeno (O).
\end{itemize}

\subsection*{2. Compuestos ternarios}
Están formados por \textbf{tres elementos diferentes}.  
\[
\ce{ABC}
\]

\textbf{Ejemplos:}
\begin{itemize}
  \item \(\ce{Na2SO4}\) → sodio (Na), azufre (S) y oxígeno (O).
  \item \(\ce{CaCO3}\) → calcio (Ca), carbono (C) y oxígeno (O).
  \item \(\ce{H2SO4}\) → hidrógeno (H), azufre (S) y oxígeno (O).
\end{itemize}

\subsection*{3. Compuestos cuaternarios}
Están formados por \textbf{cuatro elementos diferentes}.  
\[
\ce{ABCD}
\]

\textbf{Ejemplos:}
\begin{itemize}
  \item \(\ce{KClO3}\) → potasio (K), cloro (Cl) y oxígeno (O).  
  Nota: aquí parecen tres, pero si la fórmula fuera \(\ce{NaHSO4}\), ya tendríamos cuatro: sodio (Na), hidrógeno (H), azufre (S) y oxígeno (O).
  \item \(\ce{NaHCO3}\) → sodio (Na), hidrógeno (H), carbono (C) y oxígeno (O).
\end{itemize}

\subsection*{Excepciones y observaciones}
\begin{itemize}
  \item \textbf{Poliatómicos con subíndices}: aunque el oxígeno aparezca varias veces (\(\ce{O3}\), \(\ce{O4}\)), se cuenta una sola vez porque es el mismo elemento.
  \item \textbf{Iones poliatómicos}: al ver un grupo como \(\ce{SO4^{2-}}\), ya hay 2 elementos: S y O.
  \item \textbf{Compuestos mayores}: existen compuestos con 5 o más elementos (pentaelementales, etc.), pero son menos comunes en cursos introductorios.
\end{itemize}
\section*{Práctica de nomenclatura (I): De la \emph{fórmula} al \emph{nombre}}
\textbf{Instrucciones.} Nombre cada compuesto. Use: (a) \textit{Stock} en compuestos con metales multivalentes, (b) \textit{estequiométrico} en óxidos no metálicos y halogenuros covalentes, (c) reglas de \textit{hidrácidos} según fase: \(\ce{(g)}\) = halogenuro de hidrógeno; \(\ce{(ac)}\) = ácido \dots hídrico.

\subsection*{A. Óxidos metálicos}
\begin{enumerate}
  \item \ce{FeO}
  \item \ce{Fe2O3}
  \item \ce{Cu2O}
  \item \ce{CuO}
  \item \ce{SnO2}
  \item \ce{PbO2}
\end{enumerate}

\subsection*{B. Óxidos no metálicos (anhídridos) — sistema estequiométrico}
\begin{enumerate}\setcounter{enumi}{6}
  \item \ce{CO}
  \item \ce{CO2}
  \item \ce{SO2}
  \item \ce{SO3}
  \item \ce{N2O5}
  \item \ce{Cl2O7}
\end{enumerate}

\subsection*{C. Hidruros}
\begin{enumerate}\setcounter{enumi}{12}
  \item \ce{NaH}
  \item \ce{CaH2}
  \item \ce{FeH2}
  \item \ce{CuH2}
  \item \ce{AlH3}
\end{enumerate}

\subsection*{D. Hidróxidos}
\begin{enumerate}\setcounter{enumi}{17}
  \item \ce{NaOH}
  \item \ce{Ca(OH)2}
  \item \ce{Fe(OH)3}
  \item \ce{Cu(OH)2}
  \item \ce{Al(OH)3}
\end{enumerate}

\subsection*{E. Hidrácidos}
\begin{enumerate}\setcounter{enumi}{22}
  \item \ce{HCl (g)}
  \item \ce{HCl (ac)}
  \item \ce{HBr (ac)}
  \item \ce{HF (ac)}
  \item \ce{HI (ac)}
  \item \ce{H2S (ac)}
\end{enumerate}

\subsection*{F. Sales binarias con metales}
\begin{enumerate}\setcounter{enumi}{28}
  \item \ce{FeCl2}
  \item \ce{FeCl3}
  \item \ce{CuCl}
  \item \ce{CuCl2}
  \item \ce{AgCl}
  \item \ce{ZnS}
  \item \ce{FeS}
  \item \ce{PbS}
  \item \ce{Ca3P2}
  \item \ce{Mg3N2}
  \item \ce{Hg2Cl2}
  \item \ce{HgCl2}
\end{enumerate}

\subsection*{G. Sales / compuestos entre no metales y radicales simples}
\begin{enumerate}\setcounter{enumi}{40}
  \item \ce{NH4Cl}
  \item \ce{(NH4)2S}
  \item \ce{NCl3}
  \item \ce{PCl5}
  \item \ce{SF6}
  \item \ce{CCl4}
\end{enumerate}
\section*{Práctica de nomenclatura (II): Del \emph{nombre} a la \emph{fórmula}}
\textbf{Instrucciones.} Escriba la fórmula correcta. Indique estados \((g)\) o \((ac)\) en hidrácidos cuando corresponda.

\subsection*{A. Óxidos metálicos}
\begin{enumerate}
  \item Óxido de hierro (II)
  \item Óxido de hierro (III)
  \item Óxido de cobre (I)
  \item Óxido de cobre (II)
  \item Óxido de estaño (IV)
  \item Dióxido de manganeso
\end{enumerate}

\subsection*{B. Óxidos no metálicos (anhídridos) — sistema estequiométrico}
\begin{enumerate}\setcounter{enumi}{6}
  \item Monóxido de carbono
  \item Dióxido de carbono
  \item Dióxido de azufre
  \item Trióxido de azufre
  \item Pentóxido de dinitrógeno
  \item Heptóxido de dicloro
\end{enumerate}

\subsection*{C. Hidruros}
\begin{enumerate}\setcounter{enumi}{12}
  \item Hidruro de sodio
  \item Hidruro de calcio
  \item Hidruro de hierro (II)
  \item Hidruro de cobre (II)
  \item Hidruro de aluminio
\end{enumerate}

\subsection*{D. Hidróxidos}
\begin{enumerate}\setcounter{enumi}{17}
  \item Hidróxido de sodio
  \item Hidróxido de calcio
  \item Hidróxido de hierro (III)
  \item Hidróxido de cobre (II)
  \item Hidróxido de aluminio
\end{enumerate}

\subsection*{E. Hidrácidos}
\begin{enumerate}\setcounter{enumi}{22}
  \item Cloruro de hidrógeno (g)
  \item Ácido clorhídrico (ac)
  \item Ácido fluorhídrico
  \item Ácido bromhídrico
  \item Ácido yodhídrico
  \item Ácido sulfhídrico
\end{enumerate}

\subsection*{F. Sales binarias con metales}
\begin{enumerate}\setcounter{enumi}{28}
  \item Cloruro de hierro (II)
  \item Cloruro de hierro (III)
  \item Cloruro de cobre (I)
  \item Cloruro de cobre (II)
  \item Cloruro de plata
  \item Sulfuro de zinc
  \item Sulfuro de hierro (II)
  \item Sulfuro de plomo (II)
  \item Fosfuro de calcio
  \item Nitruro de magnesio
  \item Cloruro de mercurio (I)
  \item Cloruro de mercurio (II)
\end{enumerate}

\subsection*{G. Sales / compuestos entre no metales y radicales simples}
\begin{enumerate}\setcounter{enumi}{40}
  \item Cloruro de amonio
  \item Sulfuro de amonio
  \item Tricloruro de nitrógeno
  \item Pentacloruro de fósforo
  \item Hexafluoruro de azufre
  \item Tetracloruro de carbono
\end{enumerate}
\section*{Soluciones de la práctica de nomenclatura}

\subsection*{Criterios rápidos}



\begin{itemize}
  \item \textbf{Stock} (número romano) $\Rightarrow$ metales con varias valencias: \ce{Fe, Cu, Sn, Pb, Hg}, \ldots
  \item \textbf{Estequiométrico} (prefijos) $\Rightarrow$ óxidos no metálicos y halogenuros (haluros) covalentes: mono-, di-, tri-, tetra-, penta-, hexa-, hepta-, octa-.
  \item \textbf{Hidrácidos:} \ce{HX(g)} = halogenuro de hidrógeno; \ce{HX(ac)} = ácido \textit{(raíz)}-hídrico \,(p.\,ej., \ce{HCl(ac)}: ácido clorhídrico).
\end{itemize}


% ============================
% PARTE I: Fórmula -> Nombre
% ============================
\section*{Soluciones (I): De la \emph{fórmula} al \emph{nombre}}

\subsection*{A. Óxidos metálicos}
\begin{enumerate}
  \item \ce{FeO} \quad \(\to\) \textbf{óxido de hierro (II)} \, [Stock]
  \item \ce{Fe2O3} \quad \(\to\) \textbf{óxido de hierro (III)} \, [Stock]
  \item \ce{Cu2O} \quad \(\to\) \textbf{óxido de cobre (I)} \, [Stock]
  \item \ce{CuO} \quad \(\to\) \textbf{óxido de cobre (II)} \, [Stock]
  \item \ce{SnO2} \quad \(\to\) \textbf{óxido de estaño (IV)} \, [Stock]
  \item \ce{PbO2} \quad \(\to\) \textbf{óxido de plomo (IV)} \, [Stock]
\end{enumerate}

\subsection*{B. Óxidos no metálicos (anhídridos) — sistema estequiométrico}
\begin{enumerate}\setcounter{enumi}{6}
  \item \ce{CO} \quad \(\to\) \textbf{monóxido de carbono}
  \item \ce{CO2} \quad \(\to\) \textbf{dióxido de carbono}
  \item \ce{SO2} \quad \(\to\) \textbf{dióxido de azufre}
  \item \ce{SO3} \quad \(\to\) \textbf{trióxido de azufre}
  \item \ce{N2O5} \quad \(\to\) \textbf{pentóxido de dinitrógeno}
  \item \ce{Cl2O7} \quad \(\to\) \textbf{heptóxido de dicloro}
\end{enumerate}

\subsection*{C. Hidruros}
\begin{enumerate}\setcounter{enumi}{12}
  \item \ce{NaH} \quad \(\to\) \textbf{hidruro de sodio} \, [única valencia]
  \item \ce{CaH2} \quad \(\to\) \textbf{hidruro de calcio} \, [única valencia]
  \item \ce{FeH2} \quad \(\to\) \textbf{hidruro de hierro (II)} \, [Stock]
  \item \ce{CuH2} \quad \(\to\) \textbf{hidruro de cobre (II)} \, [Stock]
  \item \ce{AlH3} \quad \(\to\) \textbf{hidruro de aluminio} \, [única valencia común]
\end{enumerate}

\subsection*{D. Hidróxidos}
\begin{enumerate}\setcounter{enumi}{17}
  \item \ce{NaOH} \quad \(\to\) \textbf{hidróxido de sodio}
  \item \ce{Ca(OH)2} \quad \(\to\) \textbf{hidróxido de calcio}
  \item \ce{Fe(OH)3} \quad \(\to\) \textbf{hidróxido de hierro (III)} \, [Stock]
  \item \ce{Cu(OH)2} \quad \(\to\) \textbf{hidróxido de cobre (II)} \, [Stock]
  \item \ce{Al(OH)3} \quad \(\to\) \textbf{hidróxido de aluminio}
\end{enumerate}

\subsection*{E. Hidrácidos}
\begin{enumerate}\setcounter{enumi}{22}
  \item \ce{HCl (g)} \quad \(\to\) \textbf{cloruro de hidrógeno} \, [fase gaseosa]
  \item \ce{HCl (ac)} \quad \(\to\) \textbf{ácido clorhídrico} \, [disolución acuosa]
  \item \ce{HBr (ac)} \quad \(\to\) \textbf{ácido bromhídrico}
  \item \ce{HF (ac)} \quad \(\to\) \textbf{ácido fluorhídrico}
  \item \ce{HI (ac)} \quad \(\to\) \textbf{ácido yodhídrico}
  \item \ce{H2S (ac)} \quad \(\to\) \textbf{ácido sulfhídrico}
\end{enumerate}

\subsection*{F. Sales binarias con metales}
\begin{enumerate}\setcounter{enumi}{28}
  \item \ce{FeCl2} \(\to\) \textbf{cloruro de hierro (II)} \, [Stock]
  \item \ce{FeCl3} \(\to\) \textbf{cloruro de hierro (III)} \, [Stock]
  \item \ce{CuCl} \(\to\) \textbf{cloruro de cobre (I)} \, [Stock]
  \item \ce{CuCl2} \(\to\) \textbf{cloruro de cobre (II)} \, [Stock]
  \item \ce{AgCl} \(\to\) \textbf{cloruro de plata}
  \item \ce{ZnS} \(\to\) \textbf{su\-lfuro de zinc}
  \item \ce{FeS} \(\to\) \textbf{su\-lfuro de hierro (II)} \, [Stock]
  \item \ce{PbS} \(\to\) \textbf{su\-lfuro de plomo (II)} \, [Stock]
  \item \ce{Ca3P2} \(\to\) \textbf{fosfuro de calcio}
  \item \ce{Mg3N2} \(\to\) \textbf{nitruro de magnesio}
  \item \ce{Hg2Cl2} \(\to\) \textbf{cloruro de mercurio (I)} \, [Stock]
  \item \ce{HgCl2} \(\to\) \textbf{cloruro de mercurio (II)} \, [Stock]
\end{enumerate}

\subsection*{G. Sales / compuestos entre no metales y radicales simples}
\begin{enumerate}\setcounter{enumi}{40}
  \item \ce{NH4Cl} \(\to\) \textbf{cloruro de amonio}
  \item \ce{(NH4)2S} \(\to\) \textbf{su\-lfuro de amonio}
  \item \ce{NCl3} \(\to\) \textbf{tricloruro de nitrógeno} \, [estequiométrico]
  \item \ce{PCl5} \(\to\) \textbf{pentacloruro de fósforo} \, [estequiométrico]
  \item \ce{SF6} \(\to\) \textbf{hexafluoruro de azufre} \, [estequiométrico]
  \item \ce{CCl4} \(\to\) \textbf{tetracloruro de carbono} \, [estequiométrico]
\end{enumerate}

% ============================
% PARTE II: Nombre -> Fórmula
% ============================
\section*{Soluciones (II): Del \emph{nombre} a la \emph{fórmula}}

\subsection*{A. Óxidos metálicos}
\begin{enumerate}
  \item Óxido de hierro (II) \(\to\) \ce{FeO}
  \item Óxido de hierro (III) \(\to\) \ce{Fe2O3}
  \item Óxido de cobre (I) \(\to\) \ce{Cu2O}
  \item Óxido de cobre (II) \(\to\) \ce{CuO}
  \item Óxido de estaño (IV) \(\to\) \ce{SnO2}
  \item Dióxido de manganeso \(\to\) \ce{MnO2}
\end{enumerate}

\subsection*{B. Óxidos no metálicos (anhídridos) — sistema estequiométrico}
\begin{enumerate}\setcounter{enumi}{6}
  \item Monóxido de carbono \(\to\) \ce{CO}
  \item Dióxido de carbono \(\to\) \ce{CO2}
  \item Dióxido de azufre \(\to\) \ce{SO2}
  \item Trióxido de azufre \(\to\) \ce{SO3}
  \item Pentóxido de dinitrógeno \(\to\) \ce{N2O5}
  \item Heptóxido de dicloro \(\to\) \ce{Cl2O7}
\end{enumerate}

\subsection*{C. Hidruros}
\begin{enumerate}\setcounter{enumi}{12}
  \item Hidruro de sodio \(\to\) \ce{NaH}
  \item Hidruro de calcio \(\to\) \ce{CaH2}
  \item Hidruro de hierro (II) \(\to\) \ce{FeH2}
  \item Hidruro de cobre (II) \(\to\) \ce{CuH2}
  \item Hidruro de aluminio \(\to\) \ce{AlH3}
\end{enumerate}

\subsection*{D. Hidróxidos}
\begin{enumerate}\setcounter{enumi}{17}
  \item Hidróxido de sodio \(\to\) \ce{NaOH}
  \item Hidróxido de calcio \(\to\) \ce{Ca(OH)2}
  \item Hidróxido de hierro (III) \(\to\) \ce{Fe(OH)3}
  \item Hidróxido de cobre (II) \(\to\) \ce{Cu(OH)2}
  \item Hidróxido de aluminio \(\to\) \ce{Al(OH)3}
\end{enumerate}

\subsection*{E. Hidrácidos}
\begin{enumerate}\setcounter{enumi}{22}
  \item Cloruro de hidrógeno (g) \(\to\) \ce{HCl (g)}
  \item Ácido clorhídrico (ac) \(\to\) \ce{HCl (ac)}
  \item Ácido fluorhídrico \(\to\) \ce{HF (ac)}
  \item Ácido bromhídrico \(\to\) \ce{HBr (ac)}
  \item Ácido yodhídrico \(\to\) \ce{HI (ac)}
  \item Ácido sulfhídrico \(\to\) \ce{H2S (ac)}
\end{enumerate}

\subsection*{F. Sales binarias con metales}
\begin{enumerate}\setcounter{enumi}{28}
  \item Cloruro de hierro (II) \(\to\) \ce{FeCl2}
  \item Cloruro de hierro (III) \(\to\) \ce{FeCl3}
  \item Cloruro de cobre (I) \(\to\) \ce{CuCl}
  \item Cloruro de cobre (II) \(\to\) \ce{CuCl2}
  \item Cloruro de plata \(\to\) \ce{AgCl}
  \item Sulfuro de zinc \(\to\) \ce{ZnS}
  \item Sulfuro de hierro (II) \(\to\) \ce{FeS}
  \item Sulfuro de plomo (II) \(\to\) \ce{PbS}
  \item Fosfuro de calcio \(\to\) \ce{Ca3P2}
  \item Nitruro de magnesio \(\to\) \ce{Mg3N2}
  \item Cloruro de mercurio (I) \(\to\) \ce{Hg2Cl2}
  \item Cloruro de mercurio (II) \(\to\) \ce{HgCl2}
\end{enumerate}

\subsection*{G. Sales / compuestos entre no metales y radicales simples}
\begin{enumerate}\setcounter{enumi}{40}
  \item Cloruro de amonio \(\to\) \ce{NH4Cl}
  \item Sulfuro de amonio \(\to\) \ce{(NH4)2S}
  \item Tricloruro de nitrógeno \(\to\) \ce{NCl3}
  \item Pentacloruro de fósforo \(\to\) \ce{PCl5}
  \item Hexafluoruro de azufre \(\to\) \ce{SF6}
  \item Tetracloruro de carbono \(\to\) \ce{CCl4}
\end{enumerate}

% ===== Mini-ejemplo con justificación detallada =====
\subsection*{Ejemplos con justificación (modelo de solución paso a paso)}
\textbf{Ejemplo: \ce{Fe2O3}.} El oxígeno es \(-2\) y hay 3 oxígenos: \(-6\). Para neutralidad total, el hierro suma \(+6\) en 2 átomos \(\Rightarrow\) cada Fe es \(+3\). Nombre: \emph{óxido de hierro (III)} [Stock].\\
\textbf{Ejemplo: \ce{Cl2O7}.} Hay 2 cloros y 7 oxígenos. En óxidos no metálicos se usa \emph{estequiométrico}: \emph{heptóxido de dicloro}.
\newpage
\begin{landscape}
\section*{Práctica: tipos de reacciones y clasificación \\ 
(binario/ternario/cuaternario)}

\textbf{Instrucciones.} Para cada ecuación:

\begin{enumerate}
  \item Identifique el \textbf{tipo de reacción}: \emph{combinación (síntesis), descomposición, desplazamiento simple, doble desplazamiento, combustión} o \emph{precipitación} (si corresponde).
  \item Clasifique el \textbf{compuesto destacado} (\(\boxed{\phantom{XX}}\)) como \textbf{binario}, \textbf{ternario} o \textbf{cuaternario}, según la cantidad de \emph{elementos distintos} que contiene.
\end{enumerate}

\footnotesize
\setlength{\tabcolsep}{6pt}
\begin{tabular}{@{}cllccl@{}}
\toprule
\# & Ecuación & Compuesto destacado & Tipo de reacción & Clasif. & Nota \\
\midrule
1  & \(\ce{2H2 + O2 -> 2H2O}\) & \(\boxed{\ce{H2O}}\) & \rule{2.8cm}{0.15mm} & \rule{2.5cm}{0.15mm} & \\
2  & \(\ce{CaCO3 -> CaO + CO2}\) & \(\boxed{\ce{CaCO3}}\) & \rule{2.8cm}{0.15mm} & \rule{2.5cm}{0.15mm} & \\
3  & \(\ce{Zn + 2HCl -> ZnCl2 + H2}\) & \(\boxed{\ce{ZnCl2}}\) & \rule{2.8cm}{0.15mm} & \rule{2.5cm}{0.15mm} & \\
4  & \(\ce{AgNO3 + NaCl -> AgCl v + NaNO3}\) & \(\boxed{\ce{AgCl}}\) & \rule{2.8cm}{0.15mm} & \rule{2.5cm}{0.15mm} & precipitado \\
5  & \(\ce{BaCl2 + Na2SO4 -> BaSO4 v + 2NaCl}\) & \(\boxed{\ce{BaSO4}}\) & \rule{2.8cm}{0.15mm} & \rule{2.5cm}{0.15mm} & precipitado \\
6  & \(\ce{2K + Cl2 -> 2KCl}\) & \(\boxed{\ce{KCl}}\) & \rule{2.8cm}{0.15mm} & \rule{2.5cm}{0.15mm} & \\
7  & \(\ce{2H2O2 -> 2H2O + O2}\) & \(\boxed{\ce{H2O2}}\) & \rule{2.8cm}{0.15mm} & \rule{2.5cm}{0.15mm} & peróxido \\
8  & \(\ce{C3H8 + 5O2 -> 3CO2 + 4H2O}\) & \(\boxed{\ce{CO2}}\) & \rule{2.8cm}{0.15mm} & \rule{2.5cm}{0.15mm} & combustión \\
9  & \(\ce{NH4Cl -> NH3 + HCl}\) & \(\boxed{\ce{NH4Cl}}\) & \rule{2.8cm}{0.15mm} & \rule{2.5cm}{0.15mm} & \\
10 & \(\ce{PCl3 + Cl2 -> PCl5}\) & \(\boxed{\ce{PCl5}}\) & \rule{2.8cm}{0.15mm} & \rule{2.5cm}{0.15mm} & \\
11 & \(\ce{Fe + S -> FeS}\) & \(\boxed{\ce{FeS}}\) & \rule{2.8cm}{0.15mm} & \rule{2.5cm}{0.15mm} & \\
12 & \(\ce{Cl2 + 2KI -> 2KCl + I2}\) & \(\boxed{\ce{KCl}}\) & \rule{2.8cm}{0.15mm} & \rule{2.5cm}{0.15mm} & halógeno desplaza \\
13 & \(\ce{CaO + H2O -> Ca(OH)2}\) & \(\boxed{\ce{Ca(OH)2}}\) & \rule{2.8cm}{0.15mm} & \rule{2.5cm}{0.15mm} & \\
14 & \(\ce{Na2O + CO2 -> Na2CO3}\) & \(\boxed{\ce{Na2CO3}}\) & \rule{2.8cm}{0.15mm} & \rule{2.5cm}{0.15mm} & \\
15 & \(\ce{NaHCO3 -> Na2CO3 + CO2 + H2O}\) & \(\boxed{\ce{NaHCO3}}\) & \rule{2.8cm}{0.15mm} & \rule{2.5cm}{0.15mm} & \\
16 & \(\ce{SO3 + H2O -> H2SO4}\) & \(\boxed{\ce{H2SO4}}\) & \rule{2.8cm}{0.15mm} & \rule{2.5cm}{0.15mm} & \\
17 & \(\ce{Ca + 2H2O -> Ca(OH)2 + H2}\) & \(\boxed{\ce{Ca(OH)2}}\) & \rule{2.8cm}{0.15mm} & \rule{2.5cm}{0.15mm} & \\
18 & \(\ce{CuSO4 + Zn -> ZnSO4 + Cu}\) & \(\boxed{\ce{ZnSO4}}\) & \rule{2.8cm}{0.15mm} & \rule{2.5cm}{0.15mm} & \\
19 & \(\ce{Na2S + Cd(NO3)2 -> CdS v + 2NaNO3}\) & \(\boxed{\ce{CdS}}\) & \rule{2.8cm}{0.15mm} & \rule{2.5cm}{0.15mm} & precipitado \\
20 & \(\ce{2KNO3 -> 2KNO2 + O2}\) & \(\boxed{\ce{KNO3}}\) & \rule{2.8cm}{0.15mm} & \rule{2.5cm}{0.15mm} & \\
\bottomrule
\end{tabular}
\normalsize
\end{landscape}
\newpage
\subsection*{Sugerencias rápidas (recordatorio)}
\begin{itemize}
  \item \textbf{Combinación (síntesis):} varios reactivos \(\rightarrow\) un solo producto global.
  \item \textbf{Descomposición:} un reactivo \(\rightarrow\) varios productos.
  \item \textbf{Desplazamiento simple:} elemento libre desplaza a otro en un compuesto (\(\ce{A + BC -> AC + B}\)).
  \item \textbf{Doble desplazamiento:} intercambio iónico en solución (\(\ce{AB + CD -> AD + CB}\)); muchas veces forma \textbf{precipitado}.
  \item \textbf{Combustión:} reacción con \(\ce{O2}\) (típicamente hidrocarburos), muy \textbf{exotérmica}.
  \item \textbf{Clasificación del compuesto:} cuente \textbf{elementos distintos} (mayúsculas). \textit{Ej.:} \(\ce{KCl}\) binario; \(\ce{H2SO4}\) ternario; \(\ce{NaHCO3}\) cuaternario.
\end{itemize}
\newpage
\begin{landscape}
\section*{Soluciones: tipos de reacciones y clasificación }

\small
\setlength{\tabcolsep}{6pt}
\begin{tabular}{@{}cllccl@{}}
\toprule
\# & Ecuación & Compuesto & Tipo de reacción & Clasif. & Nota \\
\midrule
1  & \ce{2H2 + O2 -> 2H2O}                 & \ce{H2O}       & Combinación (síntesis)         & Binario     & \\
2  & \ce{CaCO3 -> CaO + CO2}               & \ce{CaCO3}     & Descomposición                 & Ternario    & \\
3  & \ce{Zn + 2HCl -> ZnCl2 + H2}          & \ce{ZnCl2}     & Desplazamiento simple          & Binario     & \\
4  & \ce{AgNO3 + NaCl -> AgCl v + NaNO3}   & \ce{AgCl}      & Doble desplazamiento           & Binario     & Precipitación \\
5  & \ce{BaCl2 + Na2SO4 -> BaSO4 v + 2NaCl}& \ce{BaSO4}     & Doble desplazamiento           & Ternario    & Precipitación \\
6  & \ce{2K + Cl2 -> 2KCl}                 & \ce{KCl}       & Combinación (síntesis)         & Binario     & \\
7  & \ce{2H2O2 -> 2H2O + O2}               & \ce{H2O2}      & Descomposición                 & Binario     & Peróxido \\
8  & \ce{C3H8 + 5O2 -> 3CO2 + 4H2O}        & \ce{CO2}       & Combustión (exotérmica)        & Binario     & \\
9  & \ce{NH4Cl -> NH3 + HCl}               & \ce{NH4Cl}     & Descomposición                 & Ternario    & \\
10 & \ce{PCl3 + Cl2 -> PCl5}               & \ce{PCl5}      & Combinación (síntesis)         & Binario     & Halogenación \\
11 & \ce{Fe + S -> FeS}                    & \ce{FeS}       & Combinación (síntesis)         & Binario     & \\
12 & \ce{Cl2 + 2KI -> 2KCl + I2}           & \ce{KCl}       & Desplazamiento simple          & Binario     & Halógeno desplaza \\
13 & \ce{CaO + H2O -> Ca(OH)2}             & \ce{Ca(OH)2}   & Combinación (síntesis)         & Ternario    & Formación de base \\
14 & \ce{Na2O + CO2 -> Na2CO3}             & \ce{Na2CO3}    & Combinación (síntesis)         & Ternario    & \\
15 & \ce{NaHCO3 -> Na2CO3 + CO2 + H2O}     & \ce{NaHCO3}    & Descomposición                 & Cuaternario & \\
16 & \ce{SO3 + H2O -> H2SO4}               & \ce{H2SO4}     & Combinación (síntesis)         & Ternario    & Formación de oxácido \\
17 & \ce{Ca + 2H2O -> Ca(OH)2 + H2}        & \ce{Ca(OH)2}   & Desplazamiento simple          & Ternario    & Metal + agua \\
18 & \ce{CuSO4 + Zn -> ZnSO4 + Cu}         & \ce{ZnSO4}     & Desplazamiento simple          & Ternario    & Serie de reactividad \\
19 & \ce{Na2S + Cd(NO3)2 -> CdS v + 2NaNO3}& \ce{CdS}       & Doble desplazamiento           & Binario     & Precipitación \\
20 & \ce{2KNO3 -> 2KNO2 + O2}              & \ce{KNO3}      & Descomposición                 & Ternario    & \\
\bottomrule
\end{tabular}
\normalsize

\subsection*{Criterio de clasificación (recordatorio breve)}
\begin{itemize}
  \item \textbf{Binario}: 2 elementos distintos (contar mayúsculas diferentes en la fórmula).
  \item \textbf{Ternario}: 3 elementos distintos.
  \item \textbf{Cuaternario}: 4 elementos distintos.
\end{itemize}
\end{landscape}


\noindent\fbox{%
\parbox{\linewidth}{%
%\textbf{Julián Arias Meza} es profesor de Matemática graduado de la Universidad de Costa Rica. 
%Cursó la carrera de Física en la misma universidad y actualmente cursa la licenciatura en 
%Ingeniería Física en el Tecnológico de Costa Rica (TEC). 
%Integra el uso de herramientas digitales y de inteligencia artificial (IA), 
%enfocadas en la mejora de las capacidades pedagógicas: 
%diseño de materiales adaptativos, retroalimentación automática y 
%optimización de la evaluación formativa.


\medskip
\textbf{Áreas}
\begin{itemize}
  \item \textbf{Matemática:} aritmética, álgebra, trigonometría, geometría analítica, cálculo (diferencial e integral), probabilidad y estadística.
  \item \textbf{Física:} mecánica, ondas, electricidad y magnetismo, termodinámica y óptica; con resolución de problemas, interpretación física y uso riguroso del SI.
  \item \textbf{Química:} nomenclatura (Stock y sistemática), óxidos, hidruros, hidróxidos, hidrácidos, sales binarias, tipos de reacciones, balanceo y estequiometría.
  \item \textbf{Formatos:} guías teóricas, bancos de ejercicios, prácticas con soluciones paso a paso, simulacros, rúbricas, presentaciones y resúmenes ejecutivos.
\end{itemize}


\textbf{Entrego en:} PDF listo para imprimir,  y \texttt{.docx} (Word).


\medskip
\textbf{Contacto (WhatsApp):} \texttt{7076-9371}
}}

\end{document}


