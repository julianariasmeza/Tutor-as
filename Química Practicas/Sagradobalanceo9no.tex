% ==========================================================
% PRÁCTICA: REACCIONES QUÍMICAS — TRANSCRIPCIÓN EN LaTeX
% Compila con pdfLaTeX. Requiere el paquete mhchem.
% ==========================================================
\documentclass[11pt,letterpaper]{article}

\usepackage[T1]{fontenc}
\usepackage[utf8]{inputenc}
\usepackage[spanish]{babel}

\usepackage[a4paper,margin=2.2cm]{geometry}
\usepackage{booktabs,tabularx,array}
\usepackage{mhchem}   % para escribir fórmulas químicas
\usepackage{siunitx}  % por si se ocupan unidades
\sisetup{locale=DE,output-decimal-marker={,}} % coma decimal (preferencia del usuario)

\newcommand{\linea}{\rule{2.6cm}{0.4pt}}   % línea para completar a mano
\newcommand{\coef}{\rule{0.8cm}{0.4pt}}    % línea corta para coeficientes

\begin{document}

\begin{center}
  {\Large \textbf{Práctica Reacciones Químicas}}\\[2mm]
\end{center}

\noindent\textbf{Respuesta corta.}\\
\textbf{1}. Señale los \textbf{reactivos} y los \textbf{productos} en cada una de las siguientes ecuaciones químicas. Además, clasifíquelas en reacciones de \emph{combinación}, \emph{descomposición}, \emph{desplazamiento} y \emph{doble descomposición}, según corresponda.

\vspace{4mm}

\renewcommand{\arraystretch}{1.35}
\begin{tabularx}{\textwidth}{@{}>{\raggedright\arraybackslash}X c@{}}
\toprule
\textbf{Ecuación} & \textbf{Clasificación} \\
\midrule
\ce{2 LiOH (ac) + Ca(NO3)2 (ac) -> 2 LiNO3 (ac) + Ca(OH)2 (s)} & \linea \\ 
\ce{Sr(OH)2 + H2CO3 -> SrCO3 + 2 H2O} & \linea \\
\ce{P4 + 3 O2 -> 2 P2O3} & \linea \\
\ce{CH3CH2CH3 + 5 O2 -> 3 CO2 + 4 H2O} & \linea \\
\ce{3 NiBr2 + 2 Al -> 2 AlBr3 + 3 Ni} & \linea \\
\ce{2 KClO3 -> 2 KCl + 3 O2} & \linea \\
\ce{Cl2O5 + O2 -> Cl2O7} % revisar en enunciado: parecía heptóxido
& \linea \\
\ce{HCl (ac) + RbOH (ac) -> H2O (l) + RbCl (ac)} & \linea \\
\ce{GeH4 ->[\Delta] Ge + H2 ^} & \linea \\
\ce{CuSO4 (ac) + 2 NaOH (ac) -> Cu(OH)2 (s) + Na2SO4 (ac)} & \linea \\
\ce{3 HCl (ac) + Al(OH)3 (ac) -> AlCl3 (ac) + 3 H2O (l)} & \linea \\
\ce{HCl + RbOH (ac) -> H2O (l) + RbCl (ac)} & \linea \\
\ce{Mg + HCl -> MgCl2 + H2 ^} & \linea \\
\bottomrule
\end{tabularx}

\vspace{6mm}

\textbf{2}. Balancee las siguientes ecuaciones químicas utilizando los coeficientes mínimos enteros para cumplir la Ley de Conservación de la Masa. (Complete los espacios en blanco).

\begin{flushleft}
\begin{tabular}{@{}l@{}}
\coef\ \ce{Mn} \,+\, \coef\ \ce{O2} \,$\rightarrow$\, \coef\ \ce{Mn2O3}\\[2mm]
\coef\ \ce{As4} \,+\, \coef\ \ce{Cl2} \,$\rightarrow$\, \coef\ \ce{AsCl3}\\[2mm]
\coef\ \ce{Sn} \,+\, \coef\ \ce{P4} \,$\rightarrow$\, \coef\ \ce{SnP2}\\[2mm]
\coef\ \ce{As2O5} \,+\, \coef\ \ce{H2O} \,$\rightarrow$\, \coef\ \ce{H3AsO4}\\[2mm]
\coef\ \ce{Bi2O5} \,+\, \coef\ \ce{H2O} \,$\rightarrow$\, \coef\ \ce{Bi(OH)5}\\[2mm]
\coef\ \ce{U(OH)6} \,$\rightarrow$\, \coef\ \ce{UO3} \,+\, \coef\ \ce{H2O}\\
\end{tabular}
\end{flushleft}

\newpage
% ==========================================================
% SEGUNDA HOJA (ecuaciones con espacios para balancear)
% ==========================================================
\begin{flushleft}
\coef\ \ce{C5H12} \,+\, \coef\ \ce{O2} \,$\rightarrow$\, \coef\ \ce{CO2} \,+\, \coef\ \ce{H2O}\\[2mm]
\coef\ \ce{Fe(NO3)3} \,+\, \coef\ \ce{Ca} \,$\rightarrow$\, \coef\ \ce{Ca(NO3)2} \,+\, \coef\ \ce{Fe}\\[2mm]
\coef\ \ce{C6H18} \,+\, \coef\ \ce{O2} \,$\rightarrow$\, \coef\ \ce{CO2} \,+\, \coef\ \ce{H2O} \quad % revisar: quizá era C6H14
\\[2mm]
\coef\ \ce{C6H13OH} \,+\, \coef\ \ce{O2} \,$\rightarrow$\, \coef\ \ce{CO2} \,+\, \coef\ \ce{H2O}\\[2mm]
\coef\ \ce{Mn2(SO4)3} \,+\, \coef\ \ce{LiOH} \,$\rightarrow$\, \coef\ \ce{Mn(OH)3} \,+\, \coef\ \ce{Li2SO4}\\[2mm]
\coef\ \ce{Na3PO4} \,+\, \coef\ \ce{Li2SO4} \,$\rightarrow$\, \coef\ \ce{Na2SO4} \,+\, \coef\ \ce{Li3PO4}\\[2mm]
\coef\ \ce{C7H14} \,+\, \coef\ \ce{O2} \,$\rightarrow$\, \coef\ \ce{CO2} \,+\, \coef\ \ce{H2O}\\[2mm]
\coef\ \ce{C3H7OH} \,+\, \coef\ \ce{O2} \,$\rightarrow$\, \coef\ \ce{CO2} \,+\, \coef\ \ce{H2O}\\[2mm]
\coef\ \ce{Mn(OH)4} \,+\, \coef\ \ce{AlPO4} \,$\rightarrow$\, \coef\ \ce{Al(OH)3} \,+\, \coef\ \ce{Mn3(PO4)4} \quad % revisar especies
\\[2mm]
\coef\ \ce{NiSO4} \,+\, \coef\ \ce{CuNO3} \,$\rightarrow$\, \coef\ \ce{Cu2SO4} \,+\, \coef\ \ce{Ni(NO3)2} \quad % revisar: CuNO3/Cu2SO4
\\[2mm]
\coef\ \ce{Mn2S3} \,+\, \coef\ \ce{Cr(PO4)2} \,$\rightarrow$\, \coef\ \ce{MnPO4} \,+\, \coef\ \ce{SrS3} \quad % revisar último producto
\\[2mm]
\coef\ \ce{V2O5} \,+\, \coef\ \ce{HCl} \,$\rightarrow$\, \coef\ \ce{VOCl3} \,+\, \coef\ \ce{H2O}\\[2mm]
\coef\ \ce{As4} \,+\, \coef\ \ce{NaOH} \,$\rightarrow$\, \coef\ \ce{Na3AsO3} \,+\, \coef\ \ce{H2} \quad % revisar sal (NaAsO2/Na3AsO3?)
\\[2mm]
\coef\ \ce{KO2} \,+\, \coef\ \ce{CO2} \,$\rightarrow$\, \coef\ \ce{K2CO3} \,+\, \coef\ \ce{O2}\\[2mm]
\coef\ \ce{AlCl3} \,+\, \coef\ \ce{H2O} \,$\rightarrow$\, \coef\ \ce{CH4} \,+\, \coef\ \ce{Al(OH)3} \quad % revisar: parece ajeno; transcrito literal
\\[2mm]
\coef\ \ce{Hg(CN)2} \,+\, \coef\ \ce{H3PO4} \,$\rightarrow$\, \coef\ \ce{Hg3(PO4)2} \,+\, \coef\ \ce{HCN}\\[2mm]
\coef\ \ce{C4H8} \,+\, \coef\ \ce{O2} \,$\rightarrow$\, \coef\ \ce{CO2} \,+\, \coef\ \ce{H2O}\\[2mm]
\coef\ \ce{C7H14} \,+\, \coef\ \ce{O2} \,$\rightarrow$\, \coef\ \ce{CO2} \,+\, \coef\ \ce{H2O}
\end{flushleft}

\vspace{4mm}
\noindent\textit{Nota}: Las marcas ``\% revisar'' señalan lugares donde la imagen no fue totalmente nítida; por favor confirme la \emph{fórmula} exacta del enunciado físico antes de usar en evaluación.

\end{document}