\documentclass[12pt]{article}
\usepackage[utf8]{inputenc}
\usepackage[spanish]{babel}
\usepackage[margin=2.5cm]{geometry}
\usepackage{amsmath}
\usepackage{amssymb}
\usepackage{chemfig}
\usepackage[version=4]{mhchem}
\usepackage{fancyhdr}
\usepackage{enumitem}

% Configuración de encabezado
\pagestyle{fancy}
\fancyhf{}
\fancyhead[C]{Ejercicios de Balanceo de Ecuaciones Químicas}
\fancyfoot[C]{\thepage}

\title{\textbf{Ejercicios de Balanceo de Ecuaciones Químicas}}
\author{Práctica de Química General}
\date{\today}

\begin{document}

\maketitle

\section{Instrucciones}
Balancea las siguientes ecuaciones químicas utilizando coeficientes enteros mínimos. Recuerda que en una ecuación balanceada, el número de átomos de cada elemento debe ser igual en ambos lados de la ecuación.

\section{Nivel Básico}

\subsection{Ejercicios 1-10}
Balancea las siguientes ecuaciones:

\begin{enumerate}[label=\textbf{\arabic*.}]
    \item \ce{H2 + O2 -> H2O}
    
    \item \ce{Na + Cl2 -> NaCl}
    
    \item \ce{Al + O2 -> Al2O3}
    
    \item \ce{Ca + HCl -> CaCl2 + H2}
    
    \item \ce{Mg + N2 -> Mg3N2}
    
    \item \ce{Fe + O2 -> Fe2O3}
    
    \item \ce{C + O2 -> CO2}
    
    \item \ce{K + H2O -> KOH + H2}
    
    \item \ce{P + O2 -> P2O5}
    
    \item \ce{Li + N2 -> Li3N}
\end{enumerate}

\section{Nivel Intermedio}

\subsection{Ejercicios 11-20}
Balancea las siguientes ecuaciones más complejas:

\begin{enumerate}[label=\textbf{\arabic*.}, start=11]
    \item \ce{C2H6 + O2 -> CO2 + H2O}
    
    \item \ce{NH3 + O2 -> NO + H2O}
    
    \item \ce{C3H8 + O2 -> CO2 + H2O}
    
    \item \ce{Fe2O3 + H2 -> Fe + H2O}
    
    \item \ce{Al2(SO4)3 + Ca(OH)2 -> Al(OH)3 + CaSO4}
    
    \item \ce{C6H12O6 + O2 -> CO2 + H2O}
    
    \item \ce{CaCO3 + HCl -> CaCl2 + CO2 + H2O}
    
    \item \ce{KMnO4 + HCl -> KCl + MnCl2 + H2O + Cl2}
    
    \item \ce{C2H4 + O2 -> CO2 + H2O}
    
    \item \ce{PCl5 + H2O -> H3PO4 + HCl}
\end{enumerate}

\section{Nivel Avanzado}

\subsection{Ejercicios 21-30}
Balancea estas ecuaciones más desafiantes:

\begin{enumerate}[label=\textbf{\arabic*.}, start=21]
    \item \ce{C4H10 + O2 -> CO2 + H2O}
    
    \item \ce{Al + CuSO4 -> Al2(SO4)3 + Cu}
    
    \item \ce{C2H5OH + O2 -> CO2 + H2O}
    
    \item \ce{Na2CO3 + HCl -> NaCl + CO2 + H2O}
    
    \item \ce{Cr2O3 + Al -> Al2O3 + Cr}
    
    \item \ce{C6H6 + O2 -> CO2 + H2O}
    
    \item \ce{Fe + H2SO4 -> Fe2(SO4)3 + H2}
    
    \item \ce{Ca3(PO4)2 + H2SO4 -> CaSO4 + H3PO4}
    
    \item \ce{C8H18 + O2 -> CO2 + H2O}
    
    \item \ce{KClO3 -> KCl + O2}
\end{enumerate}

\section{Desafío Especial}

\subsection{Ejercicios 31-35}
Estos ejercicios incluyen reacciones más complejas:

\begin{enumerate}[label=\textbf{\arabic*.}, start=31]
    \item \ce{C12H22O11 + O2 -> CO2 + H2O} (combustión de sacarosa)
    
    \item \ce{Ca(OH)2 + H3PO4 -> Ca3(PO4)2 + H2O}
    
    \item \ce{Al(OH)3 + H2SO4 -> Al2(SO4)3 + H2O}
    
    \item \ce{C2H2 + O2 -> CO2 + H2O} (combustión de acetileno)
    
    \item \ce{Fe2(SO4)3 + KOH -> Fe(OH)3 + K2SO4}
\end{enumerate}

\newpage

\section{Respuestas}

\subsection{Nivel Básico (1-10)}
\begin{enumerate}[label=\textbf{\arabic*.}]
    \item \ce{2H2 + O2 -> 2H2O}
    \item \ce{2Na + Cl2 -> 2NaCl}
    \item \ce{4Al + 3O2 -> 2Al2O3}
    \item \ce{Ca + 2HCl -> CaCl2 + H2}
    \item \ce{3Mg + N2 -> Mg3N2}
    \item \ce{4Fe + 3O2 -> 2Fe2O3}
    \item \ce{C + O2 -> CO2}
    \item \ce{2K + 2H2O -> 2KOH + H2}
    \item \ce{4P + 5O2 -> 2P2O5}
    \item \ce{6Li + N2 -> 2Li3N}
\end{enumerate}

\subsection{Nivel Intermedio (11-20)}
\begin{enumerate}[label=\textbf{\arabic*.}, start=11]
    \item \ce{2C2H6 + 7O2 -> 4CO2 + 6H2O}
    \item \ce{4NH3 + 5O2 -> 4NO + 6H2O}
    \item \ce{C3H8 + 5O2 -> 3CO2 + 4H2O}
    \item \ce{Fe2O3 + 3H2 -> 2Fe + 3H2O}
    \item \ce{Al2(SO4)3 + 3Ca(OH)2 -> 2Al(OH)3 + 3CaSO4}
    \item \ce{C6H12O6 + 6O2 -> 6CO2 + 6H2O}
    \item \ce{CaCO3 + 2HCl -> CaCl2 + CO2 + H2O}
    \item \ce{2KMnO4 + 16HCl -> 2KCl + 2MnCl2 + 8H2O + 5Cl2}
    \item \ce{C2H4 + 3O2 -> 2CO2 + 2H2O}
    \item \ce{PCl5 + 4H2O -> H3PO4 + 5HCl}
\end{enumerate}

\subsection{Nivel Avanzado (21-30)}
\begin{enumerate}[label=\textbf{\arabic*.}, start=21]
    \item \ce{2C4H10 + 13O2 -> 8CO2 + 10H2O}
    \item \ce{2Al + 3CuSO4 -> Al2(SO4)3 + 3Cu}
    \item \ce{C2H5OH + 3O2 -> 2CO2 + 3H2O}
    \item \ce{Na2CO3 + 2HCl -> 2NaCl + CO2 + H2O}
    \item \ce{Cr2O3 + 2Al -> Al2O3 + 2Cr}
    \item \ce{2C6H6 + 15O2 -> 12CO2 + 6H2O}
    \item \ce{2Fe + 3H2SO4 -> Fe2(SO4)3 + 3H2}
    \item \ce{Ca3(PO4)2 + 3H2SO4 -> 3CaSO4 + 2H3PO4}
    \item \ce{2C8H18 + 25O2 -> 16CO2 + 18H2O}
    \item \ce{2KClO3 -> 2KCl + 3O2}
\end{enumerate}

\subsection{Desafío Especial (31-35)}
\begin{enumerate}[label=\textbf{\arabic*.}, start=31]
    \item \ce{C12H22O11 + 12O2 -> 12CO2 + 11H2O}
    \item \ce{3Ca(OH)2 + 2H3PO4 -> Ca3(PO4)2 + 6H2O}
    \item \ce{2Al(OH)3 + 3H2SO4 -> Al2(SO4)3 + 6H2O}
    \item \ce{2C2H2 + 5O2 -> 4CO2 + 2H2O}
    \item \ce{Fe2(SO4)3 + 6KOH -> 2Fe(OH)3 + 3K2SO4}
\end{enumerate}

\section{Consejos para el Balanceo}

\begin{itemize}
    \item Comienza balanceando los elementos más complejos primero
    \item Deja el hidrógeno y el oxígeno para el final
    \item En reacciones de combustión, balancea carbono, luego hidrógeno, y finalmente oxígeno
    \item Utiliza fracciones si es necesario, pero al final multiplica toda la ecuación para obtener coeficientes enteros
    \item Verifica siempre que el número de átomos de cada elemento sea igual en ambos lados
\end{itemize}

\end{document}