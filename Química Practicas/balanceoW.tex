\documentclass[12pt]{article}
\usepackage[utf8]{inputenc}
\usepackage[spanish]{babel}
\usepackage[margin=2.5cm]{geometry}
\usepackage{amsmath}
\usepackage{amssymb}

% Configuración básica de encabezado sin fancyhdr
\makeatletter
\def\@oddhead{\hfil Ejercicios de Balanceo de Ecuaciones Químicas \hfil}
\def\@evenhead{\hfil Ejercicios de Balanceo de Ecuaciones Químicas \hfil}
\def\@oddfoot{\hfil \thepage \hfil}
\def\@evenfoot{\hfil \thepage \hfil}
\makeatother

\title{\textbf{Ejercicios de Balanceo de Ecuaciones Químicas}}
\author{Práctica de Química General}
\date{\today}

\begin{document}

\maketitle

\section{Instrucciones}
Balancea las siguientes ecuaciones químicas utilizando coeficientes enteros mínimos. Recuerda que en una ecuación balanceada, el número de átomos de cada elemento debe ser igual en ambos lados de la ecuación.

\section{Nivel Básico}

\subsection{Ejercicios 1-10}
Balancea las siguientes ecuaciones:

\begin{enumerate}
    \item $\mathrm{H_2 + O_2 \rightarrow H_2O}$
    
    \item $\mathrm{Na + Cl_2 \rightarrow NaCl}$
    
    \item $\mathrm{Al + O_2 \rightarrow Al_2O_3}$
    
    \item $\mathrm{Ca + HCl \rightarrow CaCl_2 + H_2}$
    
    \item $\mathrm{Mg + N_2 \rightarrow Mg_3N_2}$
    
    \item $\mathrm{Fe + O_2 \rightarrow Fe_2O_3}$
    
    \item $\mathrm{C + O_2 \rightarrow CO_2}$
    
    \item $\mathrm{K + H_2O \rightarrow KOH + H_2}$
    
    \item $\mathrm{P + O_2 \rightarrow P_2O_5}$
    
    \item $\mathrm{Li + N_2 \rightarrow Li_3N}$
\end{enumerate}

\section{Nivel Intermedio}

\subsection{Ejercicios 11-20}
Balancea las siguientes ecuaciones más complejas:

\begin{enumerate}
\setcounter{enumi}{10}
    \item $\mathrm{C_2H_6 + O_2 \rightarrow CO_2 + H_2O}$
    
    \item $\mathrm{NH_3 + O_2 \rightarrow NO + H_2O}$
    
    \item $\mathrm{C_3H_8 + O_2 \rightarrow CO_2 + H_2O}$
    
    \item $\mathrm{Fe_2O_3 + H_2 \rightarrow Fe + H_2O}$
    
    \item $\mathrm{Al_2(SO_4)_3 + Ca(OH)_2 \rightarrow Al(OH)_3 + CaSO_4}$
    
    \item $\mathrm{C_6H_{12}O_6 + O_2 \rightarrow CO_2 + H_2O}$
    
    \item $\mathrm{CaCO_3 + HCl \rightarrow CaCl_2 + CO_2 + H_2O}$
    
    \item $\mathrm{KMnO_4 + HCl \rightarrow KCl + MnCl_2 + H_2O + Cl_2}$
    
    \item $\mathrm{C_2H_4 + O_2 \rightarrow CO_2 + H_2O}$
    
    \item $\mathrm{PCl_5 + H_2O \rightarrow H_3PO_4 + HCl}$
\end{enumerate}

\section{Nivel Avanzado}

\subsection{Ejercicios 21-30}
Balancea estas ecuaciones más desafiantes:

\begin{enumerate}
\setcounter{enumi}{20}
    \item $\mathrm{C_4H_{10} + O_2 \rightarrow CO_2 + H_2O}$
    
    \item $\mathrm{Al + CuSO_4 \rightarrow Al_2(SO_4)_3 + Cu}$
    
    \item $\mathrm{C_2H_5OH + O_2 \rightarrow CO_2 + H_2O}$
    
    \item $\mathrm{Na_2CO_3 + HCl \rightarrow NaCl + CO_2 + H_2O}$
    
    \item $\mathrm{Cr_2O_3 + Al \rightarrow Al_2O_3 + Cr}$
    
    \item $\mathrm{C_6H_6 + O_2 \rightarrow CO_2 + H_2O}$
    
    \item $\mathrm{Fe + H_2SO_4 \rightarrow Fe_2(SO_4)_3 + H_2}$
    
    \item $\mathrm{Ca_3(PO_4)_2 + H_2SO_4 \rightarrow CaSO_4 + H_3PO_4}$
    
    \item $\mathrm{C_8H_{18} + O_2 \rightarrow CO_2 + H_2O}$
    
    \item $\mathrm{KClO_3 \rightarrow KCl + O_2}$
\end{enumerate}

\section{Desafío Especial}

\subsection{Ejercicios 31-35}
Estos ejercicios incluyen reacciones más complejas:

\begin{enumerate}
\setcounter{enumi}{30}
    \item $\mathrm{C_{12}H_{22}O_{11} + O_2 \rightarrow CO_2 + H_2O}$ (combustión de sacarosa)
    
    \item $\mathrm{Ca(OH)_2 + H_3PO_4 \rightarrow Ca_3(PO_4)_2 + H_2O}$
    
    \item $\mathrm{Al(OH)_3 + H_2SO_4 \rightarrow Al_2(SO_4)_3 + H_2O}$
    
    \item $\mathrm{C_2H_2 + O_2 \rightarrow CO_2 + H_2O}$ (combustión de acetileno)
    
    \item $\mathrm{Fe_2(SO_4)_3 + KOH \rightarrow Fe(OH)_3 + K_2SO_4}$
\end{enumerate}

\newpage

\section{Respuestas}

\subsection{Nivel Básico (1-10)}
\begin{enumerate}
    \item $\mathrm{2H_2 + O_2 \rightarrow 2H_2O}$
    \item $\mathrm{2Na + Cl_2 \rightarrow 2NaCl}$
    \item $\mathrm{4Al + 3O_2 \rightarrow 2Al_2O_3}$
    \item $\mathrm{Ca + 2HCl \rightarrow CaCl_2 + H_2}$
    \item $\mathrm{3Mg + N_2 \rightarrow Mg_3N_2}$
    \item $\mathrm{4Fe + 3O_2 \rightarrow 2Fe_2O_3}$
    \item $\mathrm{C + O_2 \rightarrow CO_2}$
    \item $\mathrm{2K + 2H_2O \rightarrow 2KOH + H_2}$
    \item $\mathrm{4P + 5O_2 \rightarrow 2P_2O_5}$
    \item $\mathrm{6Li + N_2 \rightarrow 2Li_3N}$
\end{enumerate}

\subsection{Nivel Intermedio (11-20)}
\begin{enumerate}
\setcounter{enumi}{10}
    \item $\mathrm{2C_2H_6 + 7O_2 \rightarrow 4CO_2 + 6H_2O}$
    \item $\mathrm{4NH_3 + 5O_2 \rightarrow 4NO + 6H_2O}$
    \item $\mathrm{C_3H_8 + 5O_2 \rightarrow 3CO_2 + 4H_2O}$
    \item $\mathrm{Fe_2O_3 + 3H_2 \rightarrow 2Fe + 3H_2O}$
    \item $\mathrm{Al_2(SO_4)_3 + 3Ca(OH)_2 \rightarrow 2Al(OH)_3 + 3CaSO_4}$
    \item $\mathrm{C_6H_{12}O_6 + 6O_2 \rightarrow 6CO_2 + 6H_2O}$
    \item $\mathrm{CaCO_3 + 2HCl \rightarrow CaCl_2 + CO_2 + H_2O}$
    \item $\mathrm{2KMnO_4 + 16HCl \rightarrow 2KCl + 2MnCl_2 + 8H_2O + 5Cl_2}$
    \item $\mathrm{C_2H_4 + 3O_2 \rightarrow 2CO_2 + 2H_2O}$
    \item $\mathrm{PCl_5 + 4H_2O \rightarrow H_3PO_4 + 5HCl}$
\end{enumerate}

\subsection{Nivel Avanzado (21-30)}
\begin{enumerate}
\setcounter{enumi}{20}
    \item $\mathrm{2C_4H_{10} + 13O_2 \rightarrow 8CO_2 + 10H_2O}$
    \item $\mathrm{2Al + 3CuSO_4 \rightarrow Al_2(SO_4)_3 + 3Cu}$
    \item $\mathrm{C_2H_5OH + 3O_2 \rightarrow 2CO_2 + 3H_2O}$
    \item $\mathrm{Na_2CO_3 + 2HCl \rightarrow 2NaCl + CO_2 + H_2O}$
    \item $\mathrm{Cr_2O_3 + 2Al \rightarrow Al_2O_3 + 2Cr}$
    \item $\mathrm{2C_6H_6 + 15O_2 \rightarrow 12CO_2 + 6H_2O}$
    \item $\mathrm{2Fe + 3H_2SO_4 \rightarrow Fe_2(SO_4)_3 + 3H_2}$
    \item $\mathrm{Ca_3(PO_4)_2 + 3H_2SO_4 \rightarrow 3CaSO_4 + 2H_3PO_4}$
    \item $\mathrm{2C_8H_{18} + 25O_2 \rightarrow 16CO_2 + 18H_2O}$
    \item $\mathrm{2KClO_3 \rightarrow 2KCl + 3O_2}$
\end{enumerate}

\subsection{Desafío Especial (31-35)}
\begin{enumerate}
\setcounter{enumi}{30}
    \item $\mathrm{C_{12}H_{22}O_{11} + 12O_2 \rightarrow 12CO_2 + 11H_2O}$
    \item $\mathrm{3Ca(OH)_2 + 2H_3PO_4 \rightarrow Ca_3(PO_4)_2 + 6H_2O}$
    \item $\mathrm{2Al(OH)_3 + 3H_2SO_4 \rightarrow Al_2(SO_4)_3 + 6H_2O}$
    \item $\mathrm{2C_2H_2 + 5O_2 \rightarrow 4CO_2 + 2H_2O}$
    \item $\mathrm{Fe_2(SO_4)_3 + 6KOH \rightarrow 2Fe(OH)_3 + 3K_2SO_4}$
\end{enumerate}

\section{Consejos para el Balanceo}

\begin{itemize}
    \item Comienza balanceando los elementos más complejos primero
    \item Deja el hidrógeno y el oxígeno para el final
    \item En reacciones de combustión, balancea carbono, luego hidrógeno, y finalmente oxígeno
    \item Utiliza fracciones si es necesario, pero al final multiplica toda la ecuación para obtener coeficientes enteros
    \item Verifica siempre que el número de átomos de cada elemento sea igual en ambos lados
    \item Usa el método algebraico para ecuaciones complejas
    \item Practica el método de inspección para ecuaciones simples
\end{itemize}

\section{Métodos de Balanceo}

\subsection{Método por Inspección}
Adecuado para ecuaciones simples. Se ajustan los coeficientes observando directamente la ecuación.

\subsection{Método Algebraico}
Para ecuaciones más complejas:
\begin{enumerate}
    \item Asigna variables (a, b, c, d...) a cada coeficiente
    \item Plantea ecuaciones para cada elemento
    \item Resuelve el sistema de ecuaciones
    \item Convierte a números enteros mínimos
\end{enumerate}

\subsection{Método Redox}
Para reacciones de óxido-reducción:
\begin{enumerate}
    \item Identifica las especies que cambian su número de oxidación
    \item Balancea los electrones transferidos
    \item Balancea los átomos restantes
\end{enumerate}

\end{document}