% ==========================================================
% Tabla de números de oxidación — Metales y No Metales
% ==========================================================
\documentclass[11pt,letterpaper]{article}
\usepackage{amsmath}
\DeclareUnicodeCharacter{2212}{\textminus}

% --- Idioma y codificación ---
\usepackage[T1]{fontenc}
\usepackage[utf8]{inputenc}
\usepackage[spanish, es-nodecimaldot]{babel}

% --- Paquetes de tablas y color ---
\usepackage{booktabs, colortbl, array, xcolor}
\usepackage[a4paper,margin=2.4cm]{geometry}

% --- Configuración de color ---
\definecolor{azulTEC}{RGB}{0,70,140}
\definecolor{grisClaro}{RGB}{240,240,240}

% --- Estilo de tabla ---
\renewcommand{\arraystretch}{1.2}
\setlength{\tabcolsep}{10pt}

\begin{document}

\begin{center}
\Large \textbf{Símbolos y números de oxidación de los principales elementos químicos}
\end{center}

% ==========================================================
% METAL
% ==========================================================
\section*{Metales}

\rowcolors{2}{grisClaro}{white}
\begin{tabular}{|>{\bfseries}m{4cm}|c|m{6cm}|}
\hline
\rowcolor{azulTEC!30}
\centering \textbf{Elemento} & \textbf{Símbolo} & \textbf{Número de oxidación} \\
\hline
Hidrógeno & H & 1+, 1− \\
Potasio & K & 1+ \\
Sodio & Na & 1+ \\
Plata & Ag & 1+ \\
Litio & Li & 1+ \\
Cesio & Cs & 1+ \\
Rubidio & Rb & 1+ \\
Calcio & Ca & 2+ \\
Estroncio & Sr & 2+ \\
Escandio & Sc & 3+ \\
Bario & Ba & 2+ \\
Radio & Ra & 2+ \\
Magnesio & Mg & 2+ \\
Zinc & Zn & 2+ \\
Cadmio & Cd & 2+ \\
Aluminio & Al & 3+ \\
Cobre & Cu & 1+, 2+ \\
Mercurio & Hg & 1+, 2+ \\
Oro & Au & 1+, 3+ \\
Hierro & Fe & 2+, 3+ \\
Cobalto & Co & 2+, 3+ \\
Níquel & Ni & 2+, 3+ \\
Cromo & Cr & 2+, 3+, 6+ \\
Estaño & Sn & 2+, 4+ \\
Platino & Pt & 2+, 4+ \\
Bismuto & Bi & 3+, 5+ \\
Uranio & U & 6+ \\
Manganeso & Mn & 2+, 3+, 4+, 6+, 7+ \\
\hline
\end{tabular}

\vspace{1cm}

% ==========================================================
% NO METALES
% ==========================================================
\section*{No metales}

\rowcolors{2}{grisClaro}{white}
\begin{tabular}{|>{\bfseries}m{4cm}|c|m{6cm}|}
\hline
\rowcolor{azulTEC!30}
\centering \textbf{Elemento} & \textbf{Símbolo} & \textbf{Número de oxidación} \\
\hline
Flúor & F & 1− \\
Cloro & Cl & 1−, 1+, 3+, 5+, 7+ \\
Bromo & Br & 1−, 1+, 3+, 5+ \\
Yodo & I & 1−, 1+, 5+, 7+ \\
Oxígeno & O & 2−, 2− (peróxidos), 1− (superóxidos) \\
Azufre & S & 2−, 4+, 6+ \\
Boro & B & 3+ \\
Carbono & C & 4−, 4+ \\
Silicio & Si & 4+ \\
Nitrógeno & N & 3−, 3+, 5+ \\
Fósforo & P & 3−, 3+, 5+ \\
Arsénico & As & 3−, 3+, 5+ \\
Antimonio & Sb & 3−, 3+, 5+ \\
\hline
\end{tabular}

\vspace{0.5cm}
\begin{flushright}
\small Fuente: \textit{Química para secundaria, pág. 255}
\end{flushright}
\newpage
% ==========================================================
% RADICALES SIMPLES
% ==========================================================
\section*{Radicales simples}

\rowcolors{2}{grisClaro}{white}
\begin{tabular}{|>{\bfseries}m{4cm}|c|m{4cm}|}
\hline
\rowcolor{azulTEC!30}
\centering \textbf{Radical} & \textbf{Símbolo} & \textbf{Número de oxidación} \\
\hline
Fluoruro & F$^-$ & 1− \\
Cloruro & Cl$^-$ & 1− \\
Bromuro & Br$^-$ & 1− \\
Yoduro & I$^-$ & 1− \\
Cianuro & CN$^-$ & 1− \\
Tiocianato & SCN$^-$ & 1− \\
Sulfuro & S$^{2-}$ & 2− \\
Nitruro & N$^{3-}$ & 3− \\
\hline
\end{tabular}

\vspace{0.5cm}
\begin{flushright}
\small Fuente: \textit{Química para secundaria, pág. 256}
\end{flushright}
% ==========================================================
% RADICALES COMPUESTOS (u oxigenados)
% ==========================================================
\section*{Radicales compuestos (oxigenados)}

\rowcolors{2}{grisClaro}{white}
\begin{tabular}{|>{\bfseries}m{4cm}|c|m{4cm}|}
\hline
\rowcolor{azulTEC!30}
\centering \textbf{Radical} & \textbf{Símbolo} & \textbf{Número de oxidación} \\
\hline
Acetato & CH$_3$COO$^-$ & 1− \\
Hipoclorito & ClO$^-$ & 1− \\
Clorito & ClO$_2^-$ & 1− \\
Clorato & ClO$_3^-$ & 1− \\
Perclorato & ClO$_4^-$ & 1− \\
Hipobromito & BrO$^-$ & 1− \\
Bromito & BrO$_2^-$ & 1− \\
Bromato & BrO$_3^-$ & 1− \\
Hipoyodito & IO$^-$ & 1− \\
Yodato & IO$_3^-$ & 1− \\
Peryodato & IO$_4^-$ & 1− \\
Nitrito & NO$_2^-$ & 1− \\
Nitrato & NO$_3^-$ & 1− \\
Permanganato & MnO$_4^-$ & 1− \\
Manganato & MnO$_4^{2-}$ & 2− \\
Carbonato & CO$_3^{2-}$ & 2− \\
Hiposulfito & SO$_2^{2-}$ & 2− \\
Sulfito & SO$_3^{2-}$ & 2− \\
Sulfato & SO$_4^{2-}$ & 2− \\
Cromato & CrO$_4^{2-}$ & 2− \\
Dicromato & Cr$_2$O$_7^{2-}$ & 2− \\
Silicato & SiO$_3^{2-}$ & 2− \\
Arsenito & AsO$_3^{3-}$ & 3− \\
Arsenato & AsO$_4^{3-}$ & 3− \\
Aluminato & AlO$_3^{3-}$ & 3− \\
Borato & BO$_3^{3-}$ & 3− \\
Fosfito & PO$_3^{3-}$ & 3− \\
Fosfato & PO$_4^{3-}$ & 3− \\
Hidróxido & OH$^-$ & 1− \\
\hline
\end{tabular}

\vspace{0.5cm}
\begin{flushright}
\small Fuente: \textit{Química para secundaria, pág. 256}
\end{flushright}
\newpage
% ==========================================================
% NOMENCLATURAS QUÍMICAS
% ==========================================================
\section*{Nomenclaturas químicas de los compuestos inorgánicos}

\noindent
En la química inorgánica existen tres sistemas principales de nomenclatura para nombrar compuestos: \textbf{tradicional}, \textbf{Stock} y \textbf{sistemática}.  
Cada una sigue reglas distintas, pero todas buscan representar correctamente los elementos y sus números de oxidación.

\vspace{0.5cm}

% ----------------------------------------------------------
\subsection*{1. Nomenclatura tradicional}
Se basa en el uso de \textbf{prefijos y sufijos latinos} para indicar el número de oxidación del elemento metálico o no metálico.  
Cuando un elemento presenta dos valencias posibles:
\[
\text{Terminación ``-oso''} \rightarrow \text{valencia menor}, \quad
\text{Terminación ``-ico''} \rightarrow \text{valencia mayor}.
\]

\textbf{Ejemplos:}
\[
\begin{array}{ll}
\text{FeCl}_2 & \rightarrow \text{Cloruro ferroso (Fe}^{2+}\text{)} \\
\text{FeCl}_3 & \rightarrow \text{Cloruro férrico (Fe}^{3+}\text{)} \\
\text{Cu}_2\text{O} & \rightarrow \text{Óxido cuproso (Cu}^{+}\text{)} \\
\text{CuO} & \rightarrow \text{Óxido cúprico (Cu}^{2+}\text{)} \\
\text{H}_2\text{SO}_4 & \rightarrow \text{Ácido sulfúrico (S}^{6+}\text{)} \\
\text{H}_2\text{SO}_3 & \rightarrow \text{Ácido sulfuroso (S}^{4+}\text{)} \\
\end{array}
\]

\vspace{0.5cm}

% ----------------------------------------------------------
\subsection*{2. Nomenclatura de Stock}
Indica el \textbf{número de oxidación} del elemento en \textbf{números romanos entre paréntesis}, sin usar sufijos.  
Es el sistema más utilizado actualmente en química moderna.

\textbf{Ejemplos:}
\[
\begin{array}{ll}
\text{FeCl}_2 & \rightarrow \text{Cloruro de hierro (II)} \\
\text{FeCl}_3 & \rightarrow \text{Cloruro de hierro (III)} \\
\text{CuO} & \rightarrow \text{Óxido de cobre (II)} \\
\text{Cu}_2\text{O} & \rightarrow \text{Óxido de cobre (I)} \\
\text{SO}_2 & \rightarrow \text{Óxido de azufre (IV)} \\
\text{SO}_3 & \rightarrow \text{Óxido de azufre (VI)} \\
\end{array}
\]

\vspace{0.5cm}

% ----------------------------------------------------------
\subsection*{3. Nomenclatura sistemática (IUPAC)}
Emplea \textbf{prefijos numéricos griegos} (mono-, di-, tri-, tetra-, penta-, hexa-, etc.) para indicar la cantidad de átomos de cada elemento.  
No se utiliza número romano.

\textbf{Ejemplos:}
\[
\begin{array}{ll}
\text{CO} & \rightarrow \text{Monóxido de carbono} \\
\text{CO}_2 & \rightarrow \text{Dióxido de carbono} \\
\text{N}_2\text{O}_3 & \rightarrow \text{Trióxido de dinitrógeno} \\
\text{P}_2\text{O}_5 & \rightarrow \text{Pentóxido de difósforo} \\
\text{SO}_2 & \rightarrow \text{Dióxido de azufre} \\
\text{SO}_3 & \rightarrow \text{Trióxido de azufre} \\
\end{array}
\]

\vspace{0.6cm}

% ----------------------------------------------------------
\subsection*{4. Comparación entre nomenclaturas}

\begin{center}
\rowcolors{2}{grisClaro}{white}
\begin{tabular}{|>{\bfseries}m{3.5cm}|m{4cm}|m{4cm}|m{4cm}|}
\hline
\rowcolor{azulTEC!30}
\textbf{Fórmula} & \textbf{Tradicional} & \textbf{Stock} & \textbf{Sistemática (IUPAC)} \\
\hline
FeCl$_2$ & Cloruro ferroso & Cloruro de hierro (II) & Dicloruro de hierro \\
FeCl$_3$ & Cloruro férrico & Cloruro de hierro (III) & Tricloruro de hierro \\
CuO & Óxido cúprico & Óxido de cobre (II) & Monóxido de cobre \\
Cu$_2$O & Óxido cuproso & Óxido de cobre (I) & Monóxido de dicobre \\
SO$_3$ & Anhídrido sulfúrico & Óxido de azufre (VI) & Trióxido de azufre \\
CO$_2$ & Anhídrido carbónico & Óxido de carbono (IV) & Dióxido de carbono \\
\hline
\end{tabular}
\end{center}

\vspace{0.4cm}
\begin{flushright}
\small Elaborado según la nomenclatura IUPAC y textos de química general.
\end{flushright}
\newpage
% ==========================================================
% CLASIFICACIÓN DE LOS COMPUESTOS INORGÁNICOS
% ==========================================================
\section*{Clasificación de los compuestos inorgánicos}

\noindent
Los compuestos inorgánicos se clasifican según los elementos que los forman. A continuación se presentan las principales familias con sus fórmulas generales, reglas de formación y ejemplos en las nomenclaturas tradicional, Stock y sistemática (IUPAC).

% ----------------------------------------------------------
\subsection*{1. Óxidos metálicos}

\textbf{Fórmula general:} \quad M$_x$O$_y$

\textbf{Formación:} Metal + Oxígeno → Óxido metálico  
El oxígeno actúa con número de oxidación \(-2\).

\textbf{Ejemplos:}
\[
\begin{array}{llll}
\text{Fórmula} & \text{Tradicional} & \text{Stock} & \text{Sistemática} \\
\hline
\text{FeO} & \text{Óxido ferroso} & \text{Óxido de hierro (II)} & \text{Monóxido de hierro} \\
\text{Fe}_2\text{O}_3 & \text{Óxido férrico} & \text{Óxido de hierro (III)} & \text{Trióxido de dihierro} \\
\text{CuO} & \text{Óxido cúprico} & \text{Óxido de cobre (II)} & \text{Monóxido de cobre} \\
\end{array}
\]

% ----------------------------------------------------------
\subsection*{2. Óxidos no metálicos (anhídridos)}

\textbf{Fórmula general:} \quad X$_x$O$_y$

\textbf{Formación:} No metal + Oxígeno → Óxido no metálico (o anhídrido)

\textbf{Ejemplos:}
\[
\begin{array}{llll}
\text{SO}_2 & \text{Anhídrido sulfuroso} & \text{Óxido de azufre (IV)} & \text{Dióxido de azufre} \\
\text{SO}_3 & \text{Anhídrido sulfúrico} & \text{Óxido de azufre (VI)} & \text{Trióxido de azufre} \\
\text{CO}_2 & \text{Anhídrido carbónico} & \text{Óxido de carbono (IV)} & \text{Dióxido de carbono} \\
\end{array}
\]

% ----------------------------------------------------------
\subsection*{3. Hidróxidos (bases)}

\textbf{Fórmula general:} \quad M(OH)$_n$

\textbf{Formación:} Óxido metálico + Agua → Hidróxido

\textbf{Ejemplos:}
\[
\begin{array}{llll}
\text{NaOH} & \text{Hidróxido sódico} & \text{Hidróxido de sodio} & \text{Hidróxido de sodio} \\
\text{Ca(OH)}_2 & \text{Hidróxido cálcico} & \text{Hidróxido de calcio} & \text{Dihidróxido de calcio} \\
\text{Fe(OH)}_3 & \text{Hidróxido férrico} & \text{Hidróxido de hierro (III)} & \text{Trihidróxido de hierro} \\
\end{array}
\]

% ----------------------------------------------------------
\subsection*{4. Hidruros}

\textbf{Fórmula general:} \quad M$_x$H$_y$

\textbf{Formación:} Metal + Hidrógeno → Hidruro  
El hidrógeno actúa con número de oxidación \(-1\) cuando está con metales.

\textbf{Ejemplos:}
\[
\begin{array}{llll}
\text{NaH} & \text{Hidruro sódico} & \text{Hidruro de sodio} & \text{Monohidruro de sodio} \\
\text{CaH}_2 & \text{Hidruro cálcico} & \text{Hidruro de calcio} & \text{Dihidruro de calcio} \\
\text{AlH}_3 & \text{Hidruro alumínico} & \text{Hidruro de aluminio} & \text{Trihidruro de aluminio} \\
\end{array}
\]

% ----------------------------------------------------------
\subsection*{5. Hidrácidos}

\textbf{Fórmula general:} \quad H$_x$X  
(\(X\) es un no metal del grupo 16 o 17)

\textbf{Formación:} No metal + Hidrógeno → Hidrácido  
En disolución acuosa se nombran como “ácidos”.

\textbf{Ejemplos:}
\[
\begin{array}{llll}
\text{HCl(ac)} & \text{Ácido clorhídrico} & \text{Cloruro de hidrógeno} & \text{Hidruro de cloro} \\
\text{H}_2\text{S(ac)} & \text{Ácido sulfhídrico} & \text{Sulfuro de hidrógeno} & \text{Dihidruro de azufre} \\
\text{HF(ac)} & \text{Ácido fluorhídrico} & \text{Fluoruro de hidrógeno} & \text{Hidruro de flúor} \\
\end{array}
\]

% ----------------------------------------------------------
\subsection*{6. Sales binarias}

\textbf{Fórmula general:} \quad M$_x$X$_y$

\textbf{Formación:} Metal + No metal → Sal binaria

\textbf{Ejemplos:}
\[
\begin{array}{llll}
\text{NaCl} & \text{Cloruro sódico} & \text{Cloruro de sodio} & \text{Monocloruro de sodio} \\
\text{FeS} & \text{Sulfuro ferroso} & \text{Sulfuro de hierro (II)} & \text{Monosulfuro de hierro} \\
\text{CuBr}_2 & \text{Bromuro cúprico} & \text{Bromuro de cobre (II)} & \text{Dibromuro de cobre} \\
\end{array}
\]

% ----------------------------------------------------------
\subsection*{7. Oxácidos (oxiácidos)}

\textbf{Fórmula general:} \quad H$_x$X$_y$O$_z$

\textbf{Formación:} Anhídrido + Agua → Oxácido

\textbf{Ejemplos:}
\[
\begin{array}{llll}
\text{H}_2\text{SO}_4 & \text{Ácido sulfúrico} & \text{Ácido tetraoxosulfúrico (VI)} & \text{Tetraoxosulfato (VI) de hidrógeno} \\
\text{H}_2\text{SO}_3 & \text{Ácido sulfuroso} & \text{Ácido trioxosulfúrico (IV)} & \text{Trioxosulfato (IV) de hidrógeno} \\
\text{HNO}_3 & \text{Ácido nítrico} & \text{Ácido trioxonítrico (V)} & \text{Trioxonitrato (V) de hidrógeno} \\
\end{array}
\]

% ----------------------------------------------------------
\subsection*{8. Sales ternarias (oxisales)}

\textbf{Fórmula general:} \quad M$_x$X$_y$O$_z$

\textbf{Formación:} Oxácido - Hidrógeno + Metal → Sal ternaria

\textbf{Ejemplos:}
\[
\begin{array}{llll}
\text{Na}_2\text{SO}_4 & \text{Sulfato sódico} & \text{Sulfato de sodio} & \text{Tetraoxosulfato (VI) de disodio} \\
\text{KNO}_3 & \text{Nitrato potásico} & \text{Nitrato de potasio} & \text{Trioxonitrato (V) de potasio} \\
\text{CaCO}_3 & \text{Carbonato cálcico} & \text{Carbonato de calcio} & \text{Trioxocarbonato (IV) de calcio} \\
\end{array}
\]

\vspace{0.4cm}
\begin{flushright}
\small Elaborado según la clasificación inorgánica general y la nomenclatura IUPAC.
\end{flushright}
\end{document}