\documentclass[12pt]{article}

% -------------------- Paquetes y formato --------------------
\usepackage[spanish]{babel}
\usepackage[utf8]{inputenc}
\usepackage[T1]{fontenc}
\usepackage[a4paper, margin=2.2cm]{geometry}
\usepackage{enumitem}
\usepackage{multicol}
\usepackage{siunitx}
\sisetup{output-decimal-marker={,}}
\usepackage{amssymb, amsmath}
\usepackage{booktabs}
\usepackage{hyperref}
\hypersetup{colorlinks=true, linkcolor=blue!50!black, urlcolor=blue!50!black}

\setlist[itemize]{noitemsep, topsep=2pt}
\setlist[enumerate]{label=\textbf{\arabic*.}, leftmargin=*, itemsep=2pt, topsep=4pt}

\newcommand{\instrucciones}{
\textbf{Indicaciones.} En cada sección:
(i) escriba la \textbf{fórmula} a partir del \textbf{nombre} (nomenclatura de Stock) o
(ii) escriba el \textbf{nombre} a partir de la \textbf{fórmula}.
Asuma el estado de oxidación indicado en el nombre. Use las valencias comunes de los metales:
\(\mathrm{Li}^{+},\, \mathrm{Be}^{2+},\, \mathrm{Al}^{3+},\, \mathrm{Zn}^{2+},\, \mathrm{Ag}^{+},\, \mathrm{Cd}^{2+},\, \mathrm{Sn}^{2+/4+},\, \mathrm{Pb}^{2+/4+},\, \mathrm{Fe}^{2+/3+},\, \mathrm{Cu}^{+/2+},\, \mathrm{Ni}^{2+},\, \mathrm{Co}^{2+/3+},\, \mathrm{Cr}^{2+/3+/6+},\, \mathrm{Mn}^{2+/4+/7+},\, \mathrm{Ti}^{2+/3+/4+},\, \mathrm{V}^{2+/3+/5+},\, \mathrm{Au}^{+/3+},\, \mathrm{Pt}^{2+/4+},\, \mathrm{Hg}^{+}/\mathrm{Hg}_2^{2+},\, \mathrm{Bi}^{3+/5+},\, \mathrm{La}^{3+},\, \mathrm{Sc}^{3+},\, \mathrm{Y}^{3+}\).
Para hidruros: \(\mathrm{H}^{-}\). Para hidróxidos: \(\mathrm{OH}^{-}\). Para óxidos: \(\mathrm{O}^{2-}\).
}

\begin{document}
\begin{center}
  {\Large \textbf{Práctica: Óxidos metálicos, hidruros e hidróxidos}}\\[2pt]
  {\small (Selección aleatoria de metales con \(Z\in[1,38]\cup[46,57]\cup[78,89]\))}
\end{center}

\instrucciones

% ============================================================
\section*{A. Formular a partir del nombre}
\noindent\textit{Escriba la \textbf{fórmula} estequiométrica correcta.}

\subsection*{A1) Óxidos metálicos (10)}
\begin{enumerate}
  \item Óxido de titanio(IV)
  \item Óxido de vanadio(V)
  \item Óxido de hierro(III)
  \item Óxido de cobalto(II)
  \item Óxido de níquel(II)
  \item Óxido de cobre(I)
  \item Óxido de zinc(II)
  \item Óxido de estaño(IV)
  \item Óxido de plomo(II)
  \item Óxido de platino(IV)
\end{enumerate}

\subsection*{A2) Hidruros metálicos (10)}
\begin{enumerate}
  \item Hidruro de litio
  \item Hidruro de berilio
  \item Hidruro de aluminio
  \item Hidruro de hierro(II)
  \item Hidruro de cobre(II)
  \item Hidruro de cadmio
  \item Hidruro de indio(III)
  \item Hidruro de bario
  \item Hidruro de lantano(III)
  \item Hidruro de bismuto(III)
\end{enumerate}

\subsection*{A3) Hidróxidos (10)}
\begin{enumerate}
  \item Hidróxido de sodio
  \item Hidróxido de magnesio
  \item Hidróxido de aluminio
  \item Hidróxido de hierro(III)
  \item Hidróxido de níquel(II)
  \item Hidróxido de plata
  \item Hidróxido de estaño(II)
  \item Hidróxido de plomo(IV)
  \item Hidróxido de oro(III)
  \item Hidróxido de mercurio(II)
\end{enumerate}

% ============================================================
\section*{B. Nombrar a partir de la fórmula}
\noindent\textit{Escriba el \textbf{nombre de Stock} correspondiente.}

\subsection*{B1) Óxidos metálicos (8)}
\begin{multicols}{2}
\begin{enumerate}
  \item \(\mathrm{TiO}\)
  \item \(\mathrm{V_2O_5}\)
  \item \(\mathrm{FeO}\)
  \item \(\mathrm{Fe_2O_3}\)
  \item \(\mathrm{Cu_2O}\)
  \item \(\mathrm{CuO}\)
  \item \(\mathrm{SnO_2}\)
  \item \(\mathrm{PbO}\)
\end{enumerate}
\end{multicols}

\subsection*{B2) Hidruros metálicos (8)}
\begin{multicols}{2}
\begin{enumerate}
  \item \(\mathrm{LiH}\)
  \item \(\mathrm{MgH_2}\)
  \item \(\mathrm{AlH_3}\)
  \item \(\mathrm{FeH_2}\)
  \item \(\mathrm{AgH}\)
  \item \(\mathrm{CdH_2}\)
  \item \(\mathrm{AuH_3}\) \(\) % (asumiendo Au(III))
  \item \(\mathrm{BiH_3}\)
\end{enumerate}
\end{multicols}

\subsection*{B3) Hidróxidos (8)}
\begin{multicols}{2}
\begin{enumerate}
  \item \(\mathrm{KOH}\)
  \item \(\mathrm{Ca(OH)_2}\)
  \item \(\mathrm{Sc(OH)_3}\)
  \item \(\mathrm{Ni(OH)_2}\)
  \item \(\mathrm{AgOH}\)
  \item \(\mathrm{Hg(OH)_2}\)
  \item \(\mathrm{Pt(OH)_4}\)
  \item \(\mathrm{La(OH)_3}\)
\end{enumerate}
\end{multicols}

% ============================================================
\section*{C. Extra: completar reacciones de formación (6)}
\noindent\textit{Balancee y complete con los productos adecuados (estado físico opcional).}
\begin{enumerate}
  \item \(\mathrm{Fe} + \mathrm{O_2} \rightarrow \, ?\)
  \item \(\mathrm{Sn} + \mathrm{H_2} \rightarrow \, ?\)
  \item \(\mathrm{TiO_2} + \mathrm{H_2O} \rightarrow \, ?\) \quad (formación del hidróxido correspondiente)
  \item \(\mathrm{Cu_2O} + \mathrm{H_2O} \rightarrow \, ?\)
  \item \(\mathrm{BaO} + \mathrm{H_2O} \rightarrow \, ?\)
  \item \(\mathrm{PbO_2} + \mathrm{H_2O} \rightarrow \, ?\)
\end{enumerate}

% ============================================================
\section*{(Opcional) Banco de valencias útiles}
\small
\begin{tabular}{@{}ll@{}}
\toprule
Elemento & Estados de oxidación comunes \\
\midrule
\(\mathrm{Ti}\) & \(+2,+3,+4\) \\
\(\mathrm{V}\) & \(+2,+3,+5\) \\
\(\mathrm{Cr}\) & \(+2,+3,+6\) \\
\(\mathrm{Mn}\) & \(+2,+4,+7\) \\
\(\mathrm{Fe}\) & \(+2,+3\) \\
\(\mathrm{Co}\) & \(+2,+3\) \\
\(\mathrm{Ni}\) & \(+2\) \\
\(\mathrm{Cu}\) & \(+1,+2\) \\
\(\mathrm{Zn}\) & \(+2\) \\
\(\mathrm{Ag}\) & \(+1\) \\
\(\mathrm{Cd}\) & \(+2\) \\
\(\mathrm{Sn}\) & \(+2,+4\) \\
\(\mathrm{Pb}\) & \(+2,+4\) \\
\(\mathrm{Pt}\) & \(+2,+4\) \\
\(\mathrm{Au}\) & \(+1,+3\) \\
\(\mathrm{Hg}\) & \(+1~(\mathrm{Hg}_2^{2+}),+2\) \\
\(\mathrm{Bi}\) & \(+3,+5\) \\
\(\mathrm{Al},\mathrm{Ga},\mathrm{In}\) & \(+3\) \\
\(\mathrm{Sc},\mathrm{Y},\mathrm{La}\) & \(+3\) \\
\(\mathrm{Li},\mathrm{Na},\mathrm{K},\mathrm{Rb},\mathrm{Cs},\mathrm{Fr}\) & \(+1\) \\
\(\mathrm{Be},\mathrm{Mg},\mathrm{Ca},\mathrm{Sr},\mathrm{Ba},\mathrm{Ra}\) & \(+2\) \\
\bottomrule
\end{tabular}

% ============================================================
\section*{(Solucionario sugerido — puede ocultarse al imprimir)}
\small
\textbf{A1 (óxidos):}
1) \(\mathrm{TiO_2}\);
2) \(\mathrm{V_2O_5}\);
3) \(\mathrm{Fe_2O_3}\);
4) \(\mathrm{CoO}\);
5) \(\mathrm{NiO}\);
6) \(\mathrm{Cu_2O}\);
7) \(\mathrm{ZnO}\);
8) \(\mathrm{SnO_2}\);
9) \(\mathrm{PbO}\);
10) \(\mathrm{PtO_2}\).\\
\textbf{A2 (hidruros):}
1) \(\mathrm{LiH}\);
2) \(\mathrm{BeH_2}\);
3) \(\mathrm{AlH_3}\);
4) \(\mathrm{FeH_2}\);
5) \(\mathrm{CuH_2}\) (idealizado; en la práctica son complejos);
6) \(\mathrm{CdH_2}\);
7) \(\mathrm{InH_3}\);
8) \(\mathrm{BaH_2}\);
9) \(\mathrm{LaH_3}\);
10) \(\mathrm{BiH_3}\).\\
\textbf{A3 (hidróxidos):}
1) \(\mathrm{NaOH}\);
2) \(\mathrm{Mg(OH)_2}\);
3) \(\mathrm{Al(OH)_3}\);
4) \(\mathrm{Fe(OH)_3}\);
5) \(\mathrm{Ni(OH)_2}\);
6) \(\mathrm{AgOH}\);
7) \(\mathrm{Sn(OH)_2}\);
8) \(\mathrm{Pb(OH)_4}\);
9) \(\mathrm{Au(OH)_3}\);
10) \(\mathrm{Hg(OH)_2}\).\\
\textbf{B1 (óxidos):}
1) Óxido de titanio(II);
2) Óxido de vanadio(V);
3) Óxido de hierro(II);
4) Óxido de hierro(III);
5) Óxido de cobre(I);
6) Óxido de cobre(II);
7) Óxido de estaño(IV);
8) Óxido de plomo(II).\\
\textbf{B2 (hidruros):}
1) Hidruro de litio;
2) Hidruro de magnesio;
3) Hidruro de aluminio;
4) Hidruro de hierro(II);
5) Hidruro de plata;
6) Hidruro de cadmio;
7) Hidruro de oro(III);
8) Hidruro de bismuto(III).\\
\textbf{B3 (hidróxidos):}
1) Hidróxido de potasio;
2) Hidróxido de calcio;
3) Hidróxido de escandio;
4) Hidróxido de níquel(II);
5) Hidróxido de plata;
6) Hidróxido de mercurio(II);
7) Hidróxido de platino(IV);
8) Hidróxido de lantano(III).\\
\textbf{C (reacciones, productos esperados):}
1) \(\mathrm{4Fe + 3O_2 \rightarrow 2Fe_2O_3}\);
2) \(\mathrm{Sn + H_2 \rightarrow SnH_2}\) (idealizado);
3) \(\mathrm{TiO_2 + 2H_2O \rightarrow Ti(OH)_4}\);
4) \(\mathrm{Cu_2O + H_2O \rightarrow 2CuOH}\);
5) \(\mathrm{BaO + H_2O \rightarrow Ba(OH)_2}\);
6) \(\mathrm{PbO_2 + 2H_2O \rightarrow Pb(OH)_4}\).
\end{document}
