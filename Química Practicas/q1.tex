\documentclass[12pt]{article}

% Paquetes
\usepackage[spanish]{babel}
\usepackage[utf8]{inputenc}
\usepackage[T1]{fontenc}
\usepackage[a4paper, margin=2.2cm]{geometry}
\usepackage{multicol}
\usepackage{enumitem}
\usepackage{amsmath, amssymb}
\usepackage{siunitx}
\sisetup{output-decimal-marker={,}}

% Formato
\setlist[enumerate]{label=\textbf{\arabic*.}, leftmargin=*}

\begin{document}

\begin{center}
  {\Large \textbf{Práctica de Nomenclatura Química}} \\[4pt]
  \textbf{Óxidos, hidruros, hidrácidos y sales binarias}
\end{center}

\section*{Parte A. Determinar la fórmula a partir del nombre}
\begin{enumerate}
  \item Óxido de hierro(III)
  \item Óxido de cobre(I)
  \item Hidruro de calcio
  \item Hidruro de estaño(IV)
  \item Hidróxido de aluminio
  \item Hidróxido de plomo(II)
  \item Ácido clorhídrico
  \item Ácido sulfhídrico
  \item Cloruro de sodio
  \item Fluoruro de magnesio
\end{enumerate}

\section*{Parte B. Determinar el nombre a partir de la fórmula}
\begin{multicols}{2}
\begin{enumerate}
  \item $\mathrm{FeO}$
  \item $\mathrm{Fe_2O_3}$
  \item $\mathrm{NaH}$
  \item $\mathrm{BaH_2}$
  \item $\mathrm{HCl}$ (ac)
  \item $\mathrm{H_2S}$ (ac)
  \item $\mathrm{NaCl}$
  \item $\mathrm{MgF_2}$
\end{enumerate}
\end{multicols}

\section*{Parte C. Completar reacciones de formación}
Balancee y complete los productos:
\begin{enumerate}
  \item $\mathrm{2Fe + O_2 \;\rightarrow\; ?}$
  \item $\mathrm{Ca + H_2 \;\rightarrow\; ?}$
  \item $\mathrm{Na + Cl_2 \;\rightarrow\; ?}$
  \item $\mathrm{H_2 + S \;\rightarrow\; ?}$
\end{enumerate}



\end{document}
