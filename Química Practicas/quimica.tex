\documentclass[12pt]{article}
\usepackage[spanish]{babel}
\usepackage{amsmath}
\usepackage{siunitx}
\sisetup{output-decimal-marker = {,}}
\usepackage[margin=2cm]{geometry}

\begin{document}

\title{Ejercicios de Estequiometría Resueltos}
\author{}
\date{}
\maketitle

\section*{Ejercicio 1: Moléculas de etanol (\(C_2H_5OH\))}

\textbf{Datos:}
\[
m = \SI{4,765}{g}, \quad 
M_{C_2H_5OH} = \SI{46,06}{g/mol}, \quad 
N_A = 6,022 \times 10^{23} \, \text{moléculas/mol}
\]

\textbf{Fórmulas:}
\[
n = \frac{m}{M}, \quad N = n \, N_A
\]

\textbf{Desarrollo:}
\[
n = \frac{4,765}{46,06} = \SI{0,1035}{mol}
\]
\[
N = 0,1035 \times 6,022 \times 10^{23} = 6,23 \times 10^{22} \, \text{moléculas}
\]

\textbf{Respuesta:}
\[
\boxed{N = 6,23 \times 10^{22} \, \text{moléculas}}
\]

\section*{Ejercicio 2: Óxido de aluminio y ácido sulfúrico}

\textbf{Reacción balanceada:}
\[
Al_2O_3 (s) + 3 H_2SO_4 (ac) \rightarrow Al_2(SO_4)_3 (ac) + 3 H_2O (l)
\]

\textbf{Datos:}
\[
m_{Al_2(SO_4)_3} = \SI{75,400}{g}, \quad 
M_{Al_2(SO_4)_3} = \SI{342,150}{g/mol}, \quad 
M_{Al_2O_3} = \SI{101,961}{g/mol}
\]

\textbf{Fórmulas:}
\[
n = \frac{m}{M}
\]

\textbf{Desarrollo:}
\[
n_{Al_2(SO_4)_3} = \frac{75,400}{342,150} = \SI{0,2204}{mol}
\]
\[
m_{Al_2O_3} = 0,2204 \times 101,961 = \SI{22,47}{g}
\]

\textbf{Respuesta:}
\[
\boxed{m_{Al_2O_3} = \SI{22,47}{g}}
\]

\section*{Ejercicio 3: Reactivo limitante y rendimiento (Spiderweb)}

\textbf{Reacción:}
\[
C_4H_6Cl_2 + 2 NaOH \rightarrow Spiderweb + 2 NaCl + H_2O
\]

\textbf{Datos:}
\[
m_{C_4H_6Cl_2} = \SI{15,600}{g}, \quad M_{C_4H_6Cl_2} = \SI{125,002}{g/mol}
\]
\[
m_{NaOH} = \SI{10,000}{g}, \quad M_{NaOH} = \SI{39,997}{g/mol}
\]
\[
M_{Spiderweb} = \SI{109,084}{g/mol}, \quad m_{real} = \SI{17,400}{g}
\]

\textbf{1) Reactivo limitante:}
\[
n_{C_4H_6Cl_2} = \frac{15,600}{125,002} = \SI{0,125}{mol}
\]
\[
n_{NaOH} = \frac{10,000}{39,997} = \SI{0,250}{mol}
\]

Como la relación es \(1 : 2\), los reactivos están en la proporción exacta. 
\[
\boxed{\text{No hay exceso: la reacción es exacta.}}
\]

\textbf{2) Rendimiento teórico de Spiderweb:}
\[
m_{teo} = 0,125 \times 109,084 = \SI{13,64}{g}
\]

\textbf{3) Porcentaje de rendimiento:}
\[
\%R = \frac{17,400}{13,64} \times 100 = 127,6 \%
\]

\textbf{Respuesta:}
\[
\boxed{\%R = 127,6 \%}
\]
Este valor es mayor a 100\%, lo que indica posible error de medición o impurezas.

\end{document}

