\documentclass[12pt,a4paper]{article}

% Paquetes básicos
\usepackage[utf8]{inputenc}
\usepackage[T1]{fontenc}
\usepackage[spanish]{babel}
\usepackage{geometry}
\usepackage{multicol}
\usepackage{fancyhdr}
\usepackage{graphicx}
\usepackage{amsmath,amssymb}
\usepackage{array}

% Márgenes
\geometry{top=2.5cm, bottom=2.5cm, left=2.5cm, right=2.5cm}

% Encabezado y pie de página
\pagestyle{fancy}
\fancyhf{}
\fancyhead[L]{\small \textbf{Material de Décimo Año}\\Profesora: Gaudy Carpio Cordero}
\fancyhead[R]{\small Análisis de gráficas y Composición de funciones}
\fancyfoot[C]{\thepage}

\begin{document}

\section*{Ejercicios de Análisis de Gráficas}

\subsection*{Ejercicio 1}
Considere la siguiente información (empleo informal del 2014 al 2016).  

Proposiciones:  
\begin{enumerate}
  \item En el primer trimestre del año 2016 se registra el porcentaje más bajo de empleo informal. \textbf{(Verdadero)}
  \item En los últimos dos trimestres de 2014 el porcentaje estuvo entre 42\% y 46\%. \textbf{(Verdadero)}
  \item En la mayoría de los trimestres el porcentaje supera el 43\%. \textbf{(Verdadero)}
  \item En todo el periodo hubo una tendencia general a la baja. \textbf{(Falso)}
\end{enumerate}

---

\subsection*{Ejercicio 2}
Temperatura promedio por horas, un día de septiembre.  

\begin{itemize}
  \item Intervalo donde aumentó: $6$ a $10$ horas.  
  \item Temperatura a las 6 h: $18^\circ$C.  
  \item Temperatura a las 2 h: $20^\circ$C.  
  \item Intervalo donde disminuyó: $4$ a $6$ horas.  
  \item Hora de la máxima temperatura: $10$ h.  
  \item Temperatura a las 4 h: $20^\circ$C.
\end{itemize}

---

\subsection*{Ejercicio 3}
Gráfica de medicamento en la sangre de Pedro.  

\begin{enumerate}
  \item Cantidad inicial: $80$ mg. \textbf{(Verdadero)}
  \item Cantidad disminuye con los días. \textbf{(Verdadero)}
\end{enumerate}

---

\subsection*{Ejercicio 4}
Longitud de un pez con respecto a la edad.  

\begin{enumerate}
  \item Al nacer, longitud $<10$ cm. \textbf{(Verdadero)}  
  \item A los 4 años: longitud entre 45 y 60 cm. \textbf{(Verdadero)}  
  \item Un pez de 50 cm tiene a lo sumo 2 años. \textbf{(Falso)}  
  \item El crecimiento de 0--1 año es menor al de 1--3 años. \textbf{(Verdadero)}  
\end{enumerate}

---

\subsection*{Ejercicio 5}
Número de viajeros en una línea de autobuses.  

\begin{itemize}
  \item Decrecimiento: $8$--$12$ h y $16$--$18$ h.  
  \item Crecimiento: $6$--$8$ h y $14$--$16$ h.  
  \item Constante: $12$--$14$ h.  
\end{itemize}

---

\subsection*{Ejercicio 6}
Monto a pagar por bicicleta según horas:  

\[
M(x)=
\begin{cases}
1500 & 0 < x \leq 1 \\
3000 & 1 < x \leq 2 \\
4500 & 2 < x \leq 3
\end{cases}
\]

\[
M(3) = 4500 \text{ colones.}
\]
\newpage


\section*{Composición de Funciones}

En algunas situaciones, una variable depende de otra y esta última, a su vez, depende de una tercera variable. Esto produce la \textbf{composición de funciones}.  

La composición de dos funciones es el resultado de aplicar una función a la imagen de otra. Si $f$ y $g$ son funciones, se define la nueva función
\[
(f \circ g)(x) = f(g(x)),
\]
la cual se lee como ``$f$ compuesto con $g$''.  

Para que $f: A \to B$ y $g: B \to C$ puedan componerse, el \textit{ámbito} de $g$ debe coincidir con el \textit{dominio} de $f$.

---

% --- Ejercicios resueltos: Composición de funciones (formato vertical) ---

\subsection*{Ejercicios resueltos }

%========================
\textbf{Ejemplo 1.} \quad Sea \(f(x)=2x-1\), \(g(x)=3x+8\).

\[
\begin{aligned}
\textbf{Datos:}\quad & f(x)=2x-1,\quad g(x)=3x+8.\\[2mm]
\textbf{Definición:}\quad & (g\circ f)(x)=g(f(x)).\\[1mm]
\textbf{Sustitución:}\quad & g(f(x))=g(2x-1)=3(2x-1)+8.\\[1mm]
\textbf{Cálculo:}\quad & 3(2x-1)+8=6x-3+8=6x+5.\\[1mm]
\textbf{Resultado:}\quad & \boxed{(g\circ f)(x)=6x+5.}
\end{aligned}
\]

\[
\begin{aligned}
\textbf{Definición:}\quad & (f\circ g)(x)=f(g(x)).\\[1mm]
\textbf{Sustitución:}\quad & f(g(x))=f(3x+8)=2(3x+8)-1.\\[1mm]
\textbf{Cálculo:}\quad & 2(3x+8)-1=6x+16-1=6x+15.\\[1mm]
\textbf{Resultado:}\quad & \boxed{(f\circ g)(x)=6x+15.}
\end{aligned}
\]

\noindent\hrulefill

%========================
\textbf{Ejemplo 2.} \quad Sea \(f(x)=1-x\), \(g(x)=-x+6\).

\[
\begin{aligned}
\textbf{Datos:}\quad & f(x)=1-x,\quad g(x)=-x+6.\\[2mm]
\textbf{Definición:}\quad & (g\circ f)(x)=g(f(x)).\\[1mm]
\textbf{Sustitución:}\quad & g(f(x))=g(1-x)=-(1-x)+6.\\[1mm]
\textbf{Cálculo:}\quad & -(1-x)+6=-1+x+6=x+5.\\[1mm]
\textbf{Resultado:}\quad & \boxed{(g\circ f)(x)=x+5.}
\end{aligned}
\]

\[
\begin{aligned}
\textbf{Definición:}\quad & (f\circ g)(x)=f(g(x)).\\[1mm]
\textbf{Sustitución:}\quad & f(g(x))=f(-x+6)=1-(-x+6).\\[1mm]
\textbf{Cálculo:}\quad & 1-(-x+6)=1+x-6=x-5.\\[1mm]
\textbf{Resultado:}\quad & \boxed{(f\circ g)(x)=x-5.}
\end{aligned}
\]

\noindent\hrulefill

%========================
\newpage
\textbf{Ejemplo 3.} \quad Sea \(f(x)=x-2\), \(g(x)=4-3x\).

\[
\begin{aligned}
\textbf{Datos:}\quad & f(x)=x-2,\quad g(x)=4-3x.\\[2mm]
\textbf{Definición:}\quad & (g\circ f)(x)=g(f(x)).\\[1mm]
\textbf{Sustitución:}\quad & g(f(x))=g(x-2)=4-3(x-2).\\[1mm]
\textbf{Cálculo:}\quad & 4-3(x-2)=4-3x+6=-3x+10.\\[1mm]
\textbf{Resultado:}\quad & \boxed{(g\circ f)(x)=-3x+10.}
\end{aligned}
\]

\[
\begin{aligned}
\textbf{Definición:}\quad & (f\circ g)(x)=f(g(x)).\\[1mm]
\textbf{Sustitución:}\quad & f(g(x))=f(4-3x)=(4-3x)-2.\\[1mm]
\textbf{Cálculo:}\quad & (4-3x)-2=2-3x.\\[1mm]
\textbf{Resultado:}\quad & \boxed{(f\circ g)(x)=2-3x.}
\end{aligned}
\]

\noindent\hrulefill

%========================
\textbf{Ejemplo 4.} \quad Sea \(f(x)=5x-4\), \(g(x)=3x-7\).

\[
\begin{aligned}
\textbf{Datos:}\quad & f(x)=5x-4,\quad g(x)=3x-7.\\[2mm]
\textbf{Definición:}\quad & (g\circ f)(x)=g(f(x)).\\[1mm]
\textbf{Sustitución:}\quad & g(f(x))=g(5x-4)=3(5x-4)-7.\\[1mm]
\textbf{Cálculo:}\quad & 3(5x-4)-7=15x-12-7=15x-19.\\[1mm]
\textbf{Resultado:}\quad & \boxed{(g\circ f)(x)=15x-19.}
\end{aligned}
\]

\[
\begin{aligned}
\textbf{Definición:}\quad & (f\circ g)(x)=f(g(x)).\\[1mm]
\textbf{Sustitución:}\quad & f(g(x))=f(3x-7)=5(3x-7)-4.\\[1mm]
\textbf{Cálculo:}\quad & 5(3x-7)-4=15x-35-4=15x-39.\\[1mm]
\textbf{Resultado:}\quad & \boxed{(f\circ g)(x)=15x-39.}
\end{aligned}
\]
\newpage
\subsection*{Ejercicios extra}

Resuelva las siguientes composiciones de funciones. Recuerde expresar el procedimiento paso a paso:

\begin{enumerate}
    \item Sean \(f(x) = 4x + 1\), \(g(x) = x^2 - 3\).  
    Calcule \((g \circ f)(x)\) y \((f \circ g)(x)\).

    \item Sean \(f(x) = \dfrac{1}{x}\), \(g(x) = x + 5\).  
    Calcule \((g \circ f)(x)\) y \((f \circ g)(x)\).

    \item Sean \(f(x) = 2x^2\), \(g(x) = \sqrt{x}\).  
    Calcule \((g \circ f)(x)\) y \((f \circ g)(x)\).

    \item Sean \(f(x) = x - 7\), \(g(x) = 3x^2\).  
    Calcule \((g \circ f)(x)\) y \((f \circ g)(x)\).

    \item Sean \(f(x) = \dfrac{x+2}{x-1}\), \(g(x) = 5x\).  
    Calcule \((g \circ f)(x)\) y \((f \circ g)(x)\).
\end{enumerate}
\subsection*{Soluciones de los ejercicios extra }

% ========================
\textbf{1)} \; \(f(x)=4x+1\), \(\;g(x)=x^{2}-3\).

%--- (g ∘ f)
\[
\begin{aligned}
\textbf{Datos:}\;& f(x)=4x+1,\quad g(x)=x^{2}-3.\\[1mm]
\textbf{Definición:}\;& (g\circ f)(x)=g\big(f(x)\big).\\[1mm]
\textbf{Sustitución:}\;& g\big(f(x)\big)=g(4x+1)=(4x+1)^{2}-3.\\[1mm]
\textbf{Cálculo:}\;& (4x+1)^{2}-3=16x^{2}+8x+1-3=16x^{2}+8x-2.\\[1mm]
\textbf{Resultado:}\;& \boxed{(g\circ f)(x)=16x^{2}+8x-2.}
\end{aligned}
\]

%--- (f ∘ g)
\[
\begin{aligned}
\textbf{Definición:}\;& (f\circ g)(x)=f\big(g(x)\big).\\[1mm]
\textbf{Sustitución:}\;& f\big(g(x)\big)=f(x^{2}-3)=4(x^{2}-3)+1.\\[1mm]
\textbf{Cálculo:}\;& 4(x^{2}-3)+1=4x^{2}-12+1=4x^{2}-11.\\[1mm]
\textbf{Resultado:}\;& \boxed{(f\circ g)(x)=4x^{2}-11.}
\end{aligned}
\]

\textit{Dominios:} ambos polinomios, dominio \(\mathbb{R}\).

\noindent\hrulefill

% ========================
\textbf{2)} \; \(f(x)=\dfrac{1}{x}\) (con \(x\neq 0\)), \(\;g(x)=x+5\).

%--- (g ∘ f)
\[
\begin{aligned}
\textbf{Datos:}\;& f(x)=\dfrac{1}{x},\quad g(x)=x+5.\\[1mm]
\textbf{Definición:}\;& (g\circ f)(x)=g\big(f(x)\big).\\[1mm]
\textbf{Sustitución:}\;& g\big(f(x)\big)=g\!\left(\dfrac{1}{x}\right)=\dfrac{1}{x}+5.\\[1mm]
\textbf{Cálculo:}\;& \dfrac{1}{x}+5=\dfrac{1+5x}{x}.\\[1mm]
\textbf{Resultado:}\;& \boxed{(g\circ f)(x)=\dfrac{1+5x}{x}},\quad x\neq 0.
\end{aligned}
\]

%--- (f ∘ g)
\[
\begin{aligned}
\textbf{Definición:}\;& (f\circ g)(x)=f\big(g(x)\big).\\[1mm]
\textbf{Sustitución:}\;& f\big(g(x)\big)=f(x+5)=\dfrac{1}{x+5}.\\[1mm]
\textbf{Cálculo:}\;& \text{(ya está simplificado)}\\[1mm]
\textbf{Resultado:}\;& \boxed{(f\circ g)(x)=\dfrac{1}{x+5}},\quad x\neq -5.
\end{aligned}
\]

\textit{Dominios:} \((g\circ f)\): \(x\neq 0\). \((f\circ g)\): \(x\neq -5\).

\noindent\hrulefill

% ========================
\textbf{3)} \; \(f(x)=2x^{2}\), \(\;g(x)=\sqrt{x}\) (raíz principal).

%--- (g ∘ f)
\[
\begin{aligned}
\textbf{Datos:}\;& f(x)=2x^{2},\quad g(x)=\sqrt{x}.\\[1mm]
\textbf{Definición:}\;& (g\circ f)(x)=g\big(f(x)\big).\\[1mm]
\textbf{Sustitución:}\;& g\big(f(x)\big)=g(2x^{2})=\sqrt{2x^{2}}.\\[1mm]
\textbf{Cálculo:}\;& \sqrt{2x^{2}}=\sqrt{2}\,|x|.\\[1mm]
\textbf{Resultado:}\;& \boxed{(g\circ f)(x)=\sqrt{2}\,|x|},\quad \text{dominio: } \mathbb{R}.
\end{aligned}
\]

%--- (f ∘ g)
\[
\begin{aligned}
\textbf{Definición:}\;& (f\circ g)(x)=f\big(g(x)\big).\\[1mm]
\textbf{Sustitución:}\;& f\big(g(x)\big)=f(\sqrt{x})=2(\sqrt{x})^{2}.\\[1mm]
\textbf{Cálculo:}\;& 2(\sqrt{x})^{2}=2x.\\[1mm]
\textbf{Resultado:}\;& \boxed{(f\circ g)(x)=2x},\quad \text{dominio: } x\ge 0.
\end{aligned}
\]

\noindent\hrulefill

% ========================
\textbf{4)} \; \(f(x)=x-7\), \(\;g(x)=3x^{2}\).

%--- (g ∘ f)
\[
\begin{aligned}
\textbf{Datos:}\;& f(x)=x-7,\quad g(x)=3x^{2}.\\[1mm]
\textbf{Definición:}\;& (g\circ f)(x)=g\big(f(x)\big).\\[1mm]
\textbf{Sustitución:}\;& g\big(f(x)\big)=g(x-7)=3(x-7)^{2}.\\[1mm]
\textbf{Cálculo:}\;& 3(x-7)^{2}=3(x^{2}-14x+49)=3x^{2}-42x+147.\\[1mm]
\textbf{Resultado:}\;& \boxed{(g\circ f)(x)=3x^{2}-42x+147.}
\end{aligned}
\]

%--- (f ∘ g)
\[
\begin{aligned}
\textbf{Definición:}\;& (f\circ g)(x)=f\big(g(x)\big).\\[1mm]
\textbf{Sustitución:}\;& f\big(g(x)\big)=f(3x^{2})=3x^{2}-7.\\[1mm]
\textbf{Cálculo:}\;& \text{(ya está simplificado)}\\[1mm]
\textbf{Resultado:}\;& \boxed{(f\circ g)(x)=3x^{2}-7.}
\end{aligned}
\]

\textit{Dominios:} polinomiales, dominio \(\mathbb{R}\).

\noindent\hrulefill

% ========================
\newpage
\textbf{5)} \; \(f(x)=\dfrac{x+2}{\,x-1\,}\) (con \(x\neq 1\)), \(\;g(x)=5x\).

%--- (g ∘ f)
\[
\begin{aligned}
\textbf{Datos:}\;& f(x)=\dfrac{x+2}{x-1},\quad g(x)=5x.\\[1mm]
\textbf{Definición:}\;& (g\circ f)(x)=g\big(f(x)\big).\\[1mm]
\textbf{Sustitución:}\;& g\big(f(x)\big)=5\cdot \dfrac{x+2}{x-1}.\\[1mm]
\textbf{Cálculo:}\;& 5\cdot \dfrac{x+2}{x-1}= \dfrac{5(x+2)}{x-1}= \dfrac{5x+10}{x-1}.\\[1mm]
\textbf{Resultado:}\;& \boxed{(g\circ f)(x)=\dfrac{5x+10}{x-1}},\quad x\neq 1.
\end{aligned}
\]

%--- (f ∘ g)
\[
\begin{aligned}
\textbf{Definición:}\;& (f\circ g)(x)=f\big(g(x)\big).\\[1mm]
\textbf{Sustitución:}\;& f\big(g(x)\big)=f(5x)=\dfrac{5x+2}{5x-1}.\\[1mm]
\textbf{Cálculo:}\;& \text{(ya está simplificado)}\\[1mm]
\textbf{Resultado:}\;& \boxed{(f\circ g)(x)=\dfrac{5x+2}{5x-1}},\quad x\neq \dfrac{1}{5}.
\end{aligned}
\]

\textit{Dominios:} \((g\circ f)\): \(x\neq 1\). \((f\circ g)\): \(x\neq \dfrac{1}{5}\).

---


\end{document}