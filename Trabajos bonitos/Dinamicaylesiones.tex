% ==========================
% DOCUMENTO EN LATEX - DINÁMICA Y LESIONES DEPORTIVAS
% ==========================
\documentclass[12pt,a4paper]{article}

% --- Paquetes ---
\usepackage[spanish, es-nodecimaldot, shorthands=off]{babel}
\usepackage[utf8]{inputenc}
\usepackage[T1]{fontenc}
\usepackage{amsmath, amssymb}
\usepackage{geometry}
\geometry{margin=2.5cm}
\usepackage{siunitx}
\sisetup{locale=DE, output-decimal-marker={,}}

% --- Título ---
\title{Dinámica del movimiento de la pierna y su relación con lesiones deportivas}
\author{Trabajo académico de biomecánica aplicada}
\date{}

\begin{document}
\maketitle

\section{Fundamentos físicos y dinámicos}

El movimiento de la pierna durante actividades deportivas como correr, saltar o cambiar de dirección
puede modelarse mediante las leyes de la \textbf{dinámica de cuerpos rígidos}.
En particular, la rodilla actúa como una articulación tipo bisagra que conecta el fémur con la tibia,
y se encuentra sometida a fuerzas internas (músculos y ligamentos) y externas
(peso corporal, reacción del suelo, fricción).

\subsection{Segunda ley de Newton}
En forma vectorial, para un punto material de masa $m$:
\[
\sum \vec{F} = m \cdot \vec{a}
\]
donde $\vec{F}$ son las fuerzas aplicadas y $\vec{a}$ la aceleración lineal.
Este principio se aplica a cada segmento corporal (muslo, pierna, pie).

\subsection{Movimiento relativo y velocidad}
Si se considera la cadera como punto $A$ y el pie como punto $B$, la velocidad del pie se expresa como:
\[
\vec{v}_B = \vec{v}_A + \boldsymbol{\Omega}\times \vec{r}_{B/A} + \left(\vec{v}_{B/A}\right)_{xyz}
\]
donde:
\begin{itemize}
    \item $\vec{v}_A$: velocidad de la cadera,
    \item $\boldsymbol{\Omega}$: velocidad angular de la pierna,
    \item $\vec{r}_{B/A}$: vector desde la cadera al pie,
    \item $(\vec{v}_{B/A})_{xyz}$: velocidad relativa adicional (por ejemplo, en movimientos del tobillo).
\end{itemize}

\subsection{Aceleración}
La aceleración absoluta del pie $B$ se calcula como:
\[
\vec{a}_B = \vec{a}_A + \dot{\boldsymbol{\Omega}}\times \vec{r}_{B/A} + \boldsymbol{\Omega}\times(\boldsymbol{\Omega}\times \vec{r}_{B/A})
+ 2\boldsymbol{\Omega}\times(\vec{v}_{B/A})_{xyz} + (\vec{a}_{B/A})_{xyz}
\]
Los términos representan:
\begin{itemize}
    \item $\dot{\boldsymbol{\Omega}}\times \vec{r}_{B/A}$: aceleración tangencial (cambios rápidos de dirección).
    \item $\boldsymbol{\Omega}\times(\boldsymbol{\Omega}\times \vec{r}_{B/A})$: aceleración centrípeta (compresión sobre meniscos).
    \item $2\boldsymbol{\Omega}\times(\vec{v}_{B/A})$: aceleración de Coriolis (importante en movimientos combinados).
\end{itemize}

\subsection{Torque y palancas}
El movimiento articular está gobernado por los torques musculares:
\[
\tau = F \cdot d
\]
donde $F$ es la fuerza muscular y $d$ el brazo de palanca.
En la rodilla, la rótula actúa como polea que modifica el ángulo de aplicación de $F$,
mejorando la eficiencia del cuádriceps.

\subsection{Aspectos técnicos}
\begin{itemize}
    \item Un aumento excesivo en $\omega$ (velocidad angular de la pierna) o en $\alpha=\dot{\omega}$ (aceleración angular) puede generar fuerzas superiores a la resistencia de los ligamentos.
    \item La reacción del suelo mal alineada con el eje de la pierna produce torques de torsión que pueden llevar a lesiones.
    \item Los meniscos sufren cargas repetitivas por la componente centrípeta, y los ligamentos se sobrecargan en giros bruscos o aterrizajes mal ejecutados.
\end{itemize}
\section{Biomecánica y fisiología de las piernas en el deporte}

El sistema musculoesquelético de las piernas combina huesos, articulaciones,
músculos, tendones y ligamentos que actúan como un conjunto de palancas y ejes de rotación.
Su función principal es permitir el desplazamiento eficiente, absorber impactos,
y transmitir fuerzas entre el tronco y el suelo.

\subsection{Componentes principales}
\begin{itemize}
    \item \textbf{Huesos}: el fémur, la tibia, el peroné y la rótula constituyen las estructuras rígidas que soportan cargas.
    \item \textbf{Articulaciones}: la rodilla (bisagra modificada) y el tobillo (tipo tróclea) permiten el movimiento angular y la adaptación a diferentes superficies.
    \item \textbf{Ligamentos y meniscos}: proporcionan estabilidad mecánica, limitan movimientos extremos y distribuyen cargas.
    \item \textbf{Músculos}: 
    \begin{itemize}
        \item \emph{Cuádriceps}: responsables de la extensión de la rodilla.
        \item \emph{Isquiotibiales}: controlan la flexión y estabilizan la rodilla en cambios de dirección.
        \item \emph{Gemelos y sóleo}: participan en la propulsión durante la carrera o salto.
        \item \emph{Glúteos}: estabilizan la pelvis y generan potencia en la extensión de cadera.
    \end{itemize}
\end{itemize}

\subsection{Principios biomecánicos}
Desde el punto de vista mecánico:
\begin{itemize}
    \item Las piernas funcionan como \textbf{palancas de tercer género}, donde la fuerza muscular se aplica entre el eje articular y la resistencia externa.
    \item Cada paso o salto implica un \textbf{ciclo de apoyo y vuelo}, donde las fuerzas de reacción del suelo se transmiten a lo largo de la cadena cinética.
    \item La \textbf{absorción de impacto} se logra mediante la flexión de cadera, rodilla y tobillo, que convierten una fuerza externa en trabajo mecánico y calor.
    \item El \textbf{gasto energético} depende de la eficiencia en la técnica de carrera o salto, directamente relacionado con la longitud de zancada, cadencia y economía de movimiento.
\end{itemize}

\subsection{Aspectos fisiológicos}
La capacidad de un atleta de alto rendimiento depende de factores fisiológicos:
\begin{itemize}
    \item \textbf{Fibras musculares}: predominio de fibras rápidas (tipo II) favorece la explosividad, mientras que fibras lentas (tipo I) aumentan la resistencia aeróbica.
    \item \textbf{Sistemas energéticos}: 
    \begin{itemize}
        \item anaeróbico aláctico (ATP–fosfocreatina) para esfuerzos explosivos de pocos segundos,
        \item anaeróbico láctico (glucólisis) para esfuerzos de corta–media duración,
        \item aeróbico (oxidativo) para esfuerzos prolongados.
    \end{itemize}
    \item \textbf{Adaptaciones neuromusculares}: en atletas se observa mayor coordinación intermuscular, reclutamiento eficiente de unidades motoras y mejor respuesta refleja para estabilizar la rodilla y el tobillo.
    \item \textbf{Fatiga}: la disminución de la capacidad muscular o nerviosa compromete la estabilidad dinámica y aumenta el riesgo de lesiones.
\end{itemize}

\subsection{Importancia en el alto rendimiento}
En deportes de élite:
\begin{itemize}
    \item Se generan fuerzas equivalentes a varias veces el peso corporal en cada zancada o aterrizaje.
    \item La técnica eficiente minimiza la sobrecarga en ligamentos y meniscos.
    \item La prevención de lesiones requiere integrar el conocimiento dinámico con programas de fortalecimiento, propiocepción y control motor.
\end{itemize}

En cualquier persona que practica deportes de manera recreativa,
estos principios también aplican: aunque las intensidades sean menores,
la repetición de gestos técnicos (correr, saltar, girar) sin una correcta preparación
puede producir microlesiones acumulativas que llevan a problemas crónicos en rodilla, tobillo o cadera.
\section{Lesiones frecuentes en la rodilla desde la biomecánica}

La rodilla es una de las articulaciones más vulnerables en el deporte, pues combina
movilidad y carga de peso corporal. Desde el punto de vista de la dinámica,
las lesiones se producen cuando las fuerzas y torques que actúan en la articulación
superan la capacidad de resistencia de los tejidos.

\subsection{Ligamento cruzado anterior (LCA)}
\begin{itemize}
    \item \textbf{Función}: estabiliza la tibia evitando su desplazamiento anterior respecto al fémur.
    \item \textbf{Mecanismo de lesión}: giros bruscos con el pie fijo en el suelo o aterrizajes con la rodilla extendida.
    \item \textbf{Modelo dinámico}: el torque externo aplicado en la rodilla se puede expresar como:
    \[
    \tau = F \cdot d
    \]
    donde $F$ es la fuerza de reacción del suelo y $d$ el brazo de palanca lateral respecto al eje de la rodilla.
    Si $\tau$ excede la capacidad tensil del LCA, ocurre la rotura.
\end{itemize}

\subsection{Lesiones de meniscos}
\begin{itemize}
    \item \textbf{Función}: los meniscos distribuyen la carga de compresión y estabilizan el contacto entre fémur y tibia.
    \item \textbf{Mecanismo de lesión}: combinación de compresión y rotación rápida de la rodilla.
    \item \textbf{Modelo dinámico}: el término centrípeto de la aceleración
    \[
    a_c = \omega^2 \cdot r
    \]
    produce fuerzas de compresión $F_c = m\cdot a_c$ que se concentran en la superficie meniscal.
    En movimientos repetitivos o con alta velocidad angular $\omega$, aumenta el riesgo de desgarro.
\end{itemize}

\subsection{Lesiones de la rótula}
\begin{itemize}
    \item \textbf{Función}: la rótula actúa como polea, aumentando el brazo de palanca del cuádriceps.
    \item \textbf{Mecanismo de lesión}: impactos directos o desequilibrios de fuerzas entre cuádriceps e isquiotibiales.
    \item \textbf{Modelo dinámico}: el cuádriceps genera un torque:
    \[
    \tau_q = F_q \cdot d_r
    \]
    donde $F_q$ es la fuerza muscular y $d_r$ el brazo de palanca modificado por la rótula.
    Si $F_q$ es excesivo o se aplica de forma desbalanceada, puede generar luxación o fractura de la rótula.
\end{itemize}

\subsection{Factores de riesgo biomecánicos}
\begin{itemize}
    \item \textbf{Aceleraciones elevadas}: incrementan las fuerzas centrípetas y tangenciales sobre los tejidos.
    \item \textbf{Mala alineación articular}: aumenta el brazo de palanca y con ello el torque externo.
    \item \textbf{Fatiga muscular}: reduce la capacidad de absorción de impactos, trasladando la carga a ligamentos y meniscos.
    \item \textbf{Superficies de juego y calzado inadecuado}: alteran la fricción y la forma en que se transmite la reacción del suelo.
\end{itemize}

\subsection{Interpretación}
Desde la dinámica, las lesiones se entienden como el resultado de:
\begin{enumerate}
    \item Un exceso de fuerza ($F$) aplicada en direcciones no fisiológicas.
    \item Un torque ($\tau$) superior al que pueden resistir los ligamentos.
    \item Una repetición de cargas centrípetas que degrada progresivamente los meniscos.
\end{enumerate}

Por tanto, el estudio de las ecuaciones de movimiento y la medición de variables
como $\omega$ (velocidad angular), $\alpha$ (aceleración angular) y fuerzas de reacción,
son fundamentales para diseñar entrenamientos y protocolos de prevención.
\newpage
\section{Factores de riesgo y medidas preventivas}

El análisis dinámico y biomecánico permite identificar condiciones que incrementan
el riesgo de lesión, así como estrategias para reducir su incidencia.
La Tabla \ref{tabla:prevencion} resume los principales aspectos.

\begin{table}[h!]
\centering
\renewcommand{\arraystretch}{1.4}
\begin{tabular}{|p{6cm}|p{8cm}|}
\hline
\textbf{Factores de riesgo biomecánicos} & \textbf{Medidas preventivas y correctivas} \\ \hline

Aceleraciones angulares elevadas en rodilla y tobillo (\(\omega, \alpha\) altas) &
Fortalecimiento de cuádriceps, isquiotibiales y glúteos; entrenamiento excéntrico; control de técnica en aterrizajes. \\ \hline

Mala alineación articular (valgo de rodilla, pronación excesiva del pie) &
Entrenamiento de técnica de carrera y salto; uso de plantillas o calzado adecuado; corrección postural. \\ \hline

Fatiga muscular durante entrenamientos intensos o partidos prolongados &
Planificación de cargas, descanso suficiente, programas de recuperación y control de la fatiga. \\ \hline

Impactos repetitivos y superficies duras (saltos y carreras prolongadas) &
Uso de superficies con amortiguación adecuada, calzado deportivo específico, alternancia de tipos de entrenamiento. \\ \hline

Cambios bruscos de dirección sin control neuromuscular &
Ejercicios de propiocepción, agilidad y balance; programas de prevención (p. ej., FIFA 11+). \\ \hline

Desequilibrio muscular entre cuádriceps e isquiotibiales &
Rutinas de fortalecimiento equilibrado; pruebas periódicas de fuerza y flexibilidad. \\ \hline

\end{tabular}
\caption{Comparación entre factores de riesgo y medidas preventivas en lesiones de rodilla.}
\label{tabla:prevencion}
\end{table}

En conclusión, la integración del conocimiento físico, biomecánico y fisiológico
con programas de entrenamiento específicos permite reducir en forma significativa
el porcentaje de lesiones en atletas de alto rendimiento y en personas que practican
deporte recreativo.
\newpage
\section{Lesiones de tobillo y su análisis biomecánico}

El tobillo es una articulación fundamental para la locomoción,
pues transmite las fuerzas entre la pierna y el pie.
En deportes de alto rendimiento, concentra cargas equivalentes a varias veces el peso corporal,
especialmente en saltos, cambios de dirección y carreras de alta intensidad.

\subsection{Estructura y función}
\begin{itemize}
    \item Formado principalmente por la tibia, el peroné y el astrágalo.
    \item Ligamentos laterales (ligamento peroneoastragalino anterior y posterior, ligamento calcaneoperoneo) y ligamento deltoideo en la cara medial.
    \item Función de estabilizar y permitir la flexo-extensión y movimientos limitados de inversión y eversión.
\end{itemize}

\subsection{Lesiones más frecuentes}
\begin{itemize}
    \item \textbf{Esguince de tobillo} (inversión): ocurre cuando el pie rota hacia adentro con exceso de torque.
    \item \textbf{Roturas ligamentarias}: cuando el torque supera la resistencia del ligamento.
    \item \textbf{Lesiones osteocondrales}: daños en el cartílago por compresión repetitiva.
    \item \textbf{Fracturas por avulsión}: fragmentos óseos arrancados por tensión ligamentaria extrema.
\end{itemize}

\subsection{Modelo dinámico de la lesión}
Cuando el pie se apoya en el suelo y el cuerpo rota, se generan torques:
\[
\tau = F \cdot d
\]
donde:
\begin{itemize}
    \item $F$: fuerza de reacción del suelo,
    \item $d$: distancia lateral respecto al eje del tobillo.
\end{itemize}
Si $\tau$ es muy alto (por ejemplo en un mal aterrizaje), el ligamento peroneoastragalino puede sufrir rotura.

Además, las aceleraciones angulares se relacionan con cargas internas:
\[
a_c = \omega^2 \cdot r
\]
donde un $\omega$ elevado (giro rápido del pie) multiplica las fuerzas sobre los ligamentos.

\subsection{Factores de riesgo específicos}
\begin{itemize}
    \item Aterrizajes desbalanceados en un pie.
    \item Superficies irregulares o muy rígidas.
    \item Fatiga de músculos estabilizadores (peroneos).
    \item Uso de calzado inadecuado para el tipo de deporte.
\end{itemize}

\subsection{Prevención}
\begin{itemize}
    \item Ejercicios de propiocepción y equilibrio para mejorar la respuesta neuromuscular.
    \item Fortalecimiento de peroneos y músculos estabilizadores.
    \item Vendajes funcionales o tobilleras en atletas con antecedentes de esguinces.
    \item Técnica correcta de aterrizaje y cambios de dirección controlados.
\end{itemize}

En conclusión, el tobillo funciona como un eje de transmisión de fuerzas,
y las lesiones se producen cuando los torques y aceleraciones superan
la capacidad mecánica de ligamentos y estructuras óseas.
La prevención requiere un enfoque conjunto entre biomecánica, fisiología
y planificación deportiva.
\section{Prevención desde la fisioterapia y la preparación física}

La reducción del riesgo de lesión en rodilla y tobillo en atletas de alto rendimiento (y deportistas recreativos) requiere integrar fuerza, control neuromuscular, técnica, planificación de cargas y apoyo clínico. A continuación se organizan las líneas de intervención.

\subsection{Fortalecimiento muscular específico}
\begin{itemize}
    \item \textbf{Trabajo excéntrico}: cuádriceps (aterrizajes controlados, sentadillas con énfasis en la fase de descenso), isquiotibiales (``Nordic hamstring''), gemelos y sóleo (descensos excéntricos en escalón).
    \item \textbf{Estabilizadores clave}: 
    peroneos (disminuyen esguinces por inversión), 
    isquiotibiales y glúteos (reducen cizallamiento anterior tibial y protegen LCA), 
    \textit{core} y cadera (alineación pélvica y control de valgo dinámico de rodilla).
    \item \textbf{Progresión de cargas}: aumentar gradualmente volumen e intensidad; priorizar calidad técnica antes de sobrecarga.
\end{itemize}

\subsection{Entrenamiento neuromuscular y propioceptivo}
\begin{itemize}
    \item \textbf{Equilibrio dinámico}: tareas unipodales, cambios de apoyo, uso de superficies inestables (bosu) con progresión a perturbaciones externas.
    \item \textbf{Pliometría controlada}: saltos con foco en aterrizajes suaves (triple flexión: cadera–rodilla–tobillo), amortiguación activa y alineación de rodilla sobre el pie.
    \item \textbf{Ejercicios reactivos}: respuesta a estímulos impredecibles (visuales/sonoros) para mejorar reflejos estabilizadores.
\end{itemize}

\subsection{Mejora de la técnica deportiva}
\begin{itemize}
    \item \textbf{Aterrizajes}: enseñar flexión de cadera y rodilla al contactar el suelo; evitar rodilla en valgo y colapso pélvico.
    \item \textbf{Cambios de dirección y frenadas}: realizar pasos de ajuste con tronco estable y mirada al frente; reducir giros con el pie ``anclado''.
    \item \textbf{Carrera}: ajustar la longitud de la zancada y la cadencia para optimizar la economía de movimiento y disminuir los picos de carga en rodilla y tobillo.
\end{itemize}

\subsection{Planificación de cargas y recuperación}
\begin{itemize}
    \item \textbf{Monitoreo de carga externa e interna}: distancia, n.\ de saltos, aceleraciones (GPS/IMU) y esfuerzo percibido (RPE), frecuencia cardíaca.
    \item \textbf{Regla de progresión}: evitar incrementos súbitos $>$10--15\% semanal en volumen o intensidad.
    \item \textbf{Recuperación}: sueño, nutrición, sesiones de movilidad y flexibilidad; alternar sesiones de alta/baja demanda mecánica.
\end{itemize}

\subsection{Apoyo fisioterapéutico directo}
\begin{itemize}
    \item \textbf{Terapia manual y tejidos blandos}: liberación miofascial y manejo de puntos gatillo para reducir tensiones que alteran la mecánica.
    \item \textbf{Vendajes funcionales y ortesis}: uso de tobilleras o taping en casos de inestabilidad crónica o durante el retorno progresivo post-lesión.
    \item \textbf{Recuperación post-carga}: crioterapia, electroestimulación y terapia de contraste, aplicadas según protocolo y tolerancia individual.
    \item \textbf{Screening funcional}: pruebas como Y-Balance Test, FMS y saltos unipodales, que permiten detectar asimetrías de fuerza o rango articular y corregirlas.
\end{itemize}

\subsection{Programas de prevención basados en evidencia}
\begin{itemize}
    \item \textbf{FIFA 11+} (fútbol) y \textbf{PEP} (LCA): combinan fuerza, pliometría, propiocepción y técnica de aterrizaje/cambio de dirección; reducen de manera significativa la incidencia de lesiones de rodilla y tobillo.
    \item \textbf{Adaptación por deporte}: básquet/voleibol (alto volumen de saltos): énfasis en aterrizajes y control de valgo; atletismo: progresión de zancada/cadencia y superficies.
\end{itemize}

\subsection{Integración con medición y retroalimentación}
\begin{itemize}
    \item \textbf{Biofeedback y videoanálisis}: corrección de gestos en tiempo real (alineación rodilla–pie, control de tronco).
    \item \textbf{Sensores portátiles (IMU)}: seguimiento de velocidades/aceleraciones angulares ($\omega$, $\alpha$) y conteo de saltos/impactos para mantener la exposición dentro de umbrales seguros.
\end{itemize}

\subsection{Conclusión operativa}
La prevención efectiva no depende solo de ``más fuerza'', sino de enseñar al sistema neuromuscular a anticipar y responder a las fuerzas externas manteniendo alineación y control bajo alta demanda. La combinación de fortalecimiento dirigido, propiocepción, técnica eficiente, cargas bien planificadas y soporte fisioterapéutico reduce de forma significativa el riesgo de lesión en rodilla y tobillo.
\section{Modelo matemático con sensores y medidas fisiológicas para la prevención de lesiones}

\subsection{Señales, variables y preprocesamiento}
En un atleta de alto rendimiento (p.\,ej., futbolista), se pueden adquirir:
\begin{itemize}
    \item \textbf{IMU} (cintura, muslo, tibia, pie): $\omega$ (velocidad angular, \si{rad/s}), $\alpha$ (aceleración angular, \si{rad/s^2}), $a$ (aceleración lineal, \si{m/s^2}).
    \item \textbf{GPS/RTK}: distancia, velocidad, aceleraciones y cambios de dirección.
    \item \textbf{Plataformas de fuerza/plantillas}: fuerza de reacción del suelo $\mathrm{FRS}$ (\si{N}), tiempos de contacto.
    \item \textbf{Fisiología}: HRV (RMSSD), frecuencia cardíaca (FC), carga percibida (RPE), sueño (horas, calidad).
\end{itemize}
Se calculan \emph{features} por sesión: picos y promedios de $\omega$, $\alpha$, $a$, número de saltos, cambios de dirección $>30^{\circ}$, impulsos de fuerza, asimetrías (\%), y agregados fisiológicos (HRV normalizada, RPE).

\subsection{Carga mecánica con promedio móvil exponencial (EWMA)}
Definimos la \emph{carga externa} de la sesión $w_t$ como una combinación lineal de \emph{features} normalizadas (z-score):
\[
w_t \;=\; \sum_{k=1}^{K} \gamma_k\, z_{k,t},
\quad
z_{k,t} = \frac{x_{k,t}-\mu_k}{\sigma_k}.
\]
La \emph{carga elegida} (EWMA) se actualiza como:
\[
L_t \;=\; \lambda\, w_t \;+\; (1-\lambda)\, L_{t-1},
\quad 0<\lambda\leq 1.
\]
Valores típicos: $\lambda=0{,}3$--$0{,}5$ para dar más peso a las sesiones recientes.

\subsection{Índices biomecánicos y fisiológicos de riesgo}
Construimos un vector de estado de riesgo instantáneo:
\[
\mathbf{x}_t = 
\big[L_t,\; \omega_{p,\text{rod}},\; \alpha_{p,\text{rod}},\; \mathrm{FRS}_{p},\; \mathrm{Asim},\; \mathrm{HRV}_z,\; \mathrm{RPE}_z\big]^\top,
\]
donde los subíndices $p$ indican \emph{pico} o cuantil alto (p.\,ej., 95\,\%), 
$\mathrm{Asim}$ es la asimetría (\%), y las variables con $\_z$ están normalizadas.

\subsection{Modelo probabilístico de lesión (regresión logística)}
La probabilidad diaria de evento lesivo se modela como:
\[
P(\text{lesión}_t \mid \mathbf{x}_t) \;=\; 
\frac{1}{1+\exp\big(-(\beta_0 + \boldsymbol{\beta}^\top \mathbf{x}_t)\big)}.
\]
Interpretación: $\beta_i>0$ aumenta el riesgo; $\beta_i<0$ lo reduce (p.\,ej., mayor HRV suele ser protector).

\subsection{Detección temprana con control estadístico}
Se añade una regla de vigilancia:
\[
S_t \;=\; \alpha_c\, P(\text{lesión}_t) \;+\; (1-\alpha_c)\, S_{t-1}, 
\qquad
\text{alerta si } S_t > \tau,
\]
con $\alpha_c \in [0{,}2,0{,}5]$ y umbral $\tau \in [0{,}30,0{,}50]$ según la tolerancia al riesgo del cuerpo técnico.

\subsection{Ejemplo numérico (pasos verticales)}
%\datos
\begin{itemize}
\item Sesión actual: $w_t=1{,}8$ (adimensional, ya normalizado).
\item Estado previo: $L_{t-1}=1{,}2$, $\lambda=0{,}4$.
\item Picos: $\omega_{p,\text{rod}}=11{,}0\ \si{rad/s}$, $\alpha_{p,\text{rod}}=85{,}0\ \si{rad/s^2}$, $\mathrm{FRS}_p=2{,}8\,W$ (veces su peso).
\item Asimetría: $\mathrm{Asim}=8{,}0\ \%$.
\item Fisiología: $\mathrm{HRV}_z=+0{,}5$, $\mathrm{RPE}_z=+1{,}0$.
\item Coeficientes (ejemplo): 
$\beta_0=-2{,}0$;
$\boldsymbol{\beta}=[0{,}8,\; 0{,}06,\; 0{,}015,\; 0{,}40,\; 0{,}05,\; -0{,}30,\; 0{,}25]$.
\end{itemize}

%\formula
\[
\text{(1) } L_t=\lambda w_t+(1-\lambda)L_{t-1}, 
\quad
\text{(2) } \eta_t=\beta_0 + \boldsymbol{\beta}^\top \mathbf{x}_t, 
\quad
\text{(3) } P=\frac{1}{1+\mathrm{e}^{-\eta_t}}.
\]

%\sustitucion
\[
\text{(1) } L_t=0{,}4\cdot 1{,}8 + 0{,}6\cdot 1{,}2 = 0{,}72 + 0{,}72 = 1{,}44.
\]
Vector $\mathbf{x}_t=[L_t,\ \omega_p,\ \alpha_p,\ \mathrm{FRS}_p,\ \mathrm{Asim},\ \mathrm{HRV}_z,\ \mathrm{RPE}_z]$:
\[
\mathbf{x}_t=[1{,}44,\ 11{,}0,\ 85{,}0,\ 2{,}8,\ 8{,}0,\ 0{,}5,\ 1{,}0].
\]
\[
\text{(2) } \eta_t=-2{,}0 + (0{,}8)(1{,}44) + (0{,}06)(11{,}0) + (0{,}015)(85{,}0) + (0{,}40)(2{,}8) + (0{,}05)(8{,}0) + (-0{,}30)(0{,}5) + (0{,}25)(1{,}0).
\]

%\calculo
\[
\begin{aligned}
(0{,}8)(1{,}44)&=1{,}152,\\
(0{,}06)(11{,}0)&=0{,}66,\\
(0{,}015)(85{,}0)&=1{,}275,\\
(0{,}40)(2{,}8)&=1{,}12,\\
(0{,}05)(8{,}0)&=0{,}40,\\
(-0{,}30)(0{,}5)&=-0{,}15,\\
(0{,}25)(1{,}0)&=0{,}25.
\end{aligned}
\]
Suma parcial positiva: $1{,}152+0{,}66+1{,}275+1{,}12+0{,}40+0{,}25=4{,}857$.\\
Con el término negativo: $4{,}857-0{,}15=4{,}707$.\\
\[
\eta_t=-2{,}0 + 4{,}707 = 2{,}707.
\]
\[
\text{(3) } P=\frac{1}{1+\mathrm{e}^{-2{,}707}} \approx \frac{1}{1+0{,}0668} \approx 0{,}937.
\]

%\conclusion
La sesión arroja $L_t=1{,}44$ y una probabilidad instantánea de lesión $P\approx 0{,}94$ (alto riesgo \emph{hipotético} con dichos coeficientes). Operativamente: activar alerta, reducir carga explosiva, enfatizar propiocepción y técnica de aterrizaje, y revaluar en 24\,h con HRV y dolor percibido.

\subsection{Extensión: estado latente de fatiga (filtro de Kalman)}
Se puede modelar la \emph{fatiga} como un estado latente $f_t$:
\[
\underbrace{f_t}_{\text{estado}} = \phi\, f_{t-1} + \kappa\, w_t + \varepsilon_t,\qquad
\underbrace{y_t}_{\text{medida (p.\,ej., HRV}_z\text{)}} = c\, f_t + \nu_t,
\]
y estimarlo con un filtro de Kalman. Luego incluir $f_t$ en $\mathbf{x}_t$ para ajustar la probabilidad $P$ de forma más sensible a la recuperación.

\subsection{Notas de implementación}
\begin{itemize}
\item Normalizar variables (z-score) por posición y deporte; recalibrar $\mu_k$, $\sigma_k$ por mesociclo.
\item Ajustar $\lambda$ (EWMA) al perfil del calendario competitivo.
\item Reajustar $\boldsymbol{\beta}$ con datos históricos del equipo (aprendizaje supervisado).
\item Añadir reglas de negocio: si $\mathrm{Asim}>10\,\text{\%}$ o $\mathrm{FRS}_p>3{,}0\,W$, activar revisión técnica/fisioterapéutica aunque $P$ sea moderado.
\end{itemize}
% ==========================
% SECCIÓN: EWMA Y RIESGO LOGÍSTICO PARA PREVENCIÓN DE LESIONES
% ==========================
\section{Carga con EWMA e índice de riesgo logístico}

\subsection{Promedio móvil exponencial (EWMA) de la carga mecánica}
Sea $w_t$ la \emph{carga externa} de la sesión $t$ (combinación lineal de variables normalizadas:
picos de $\omega$ y $\alpha$ de rodilla, número de cambios de dirección, impulsos de fuerza, etc.).
Definimos el estimador suavizado:
\[
L_t \;=\; \lambda\, w_t + (1-\lambda)\, L_{t-1}, \qquad 0<\lambda\leq 1,\quad L_0 \text{ dado.}
\]
\textbf{Notas técnicas.}
\begin{itemize}
  \item $\lambda$ controla la memoria: valores típicos $0{,}3$–$0{,}5$ priorizan las sesiones recientes.
  \item Si se desea una ventana ``equivalente'' de $N$ sesiones: $\lambda \approx \dfrac{2}{N+1}$.
  \item Conviene estandarizar $w_t$ (z-score) por posición/deporte y recalibrar cada mesociclo.
\end{itemize}

\subsection{Modelo de probabilidad de lesión (regresión logística)}
Sea el vector de estado instantáneo
\[
\mathbf{x}_t=\big[L_t,\;\omega_{p,\text{rod}},\;\alpha_{p,\text{rod}},\;\mathrm{FRS}_{p},\;\mathrm{Asim},\;\mathrm{HRV}_z,\;\mathrm{RPE}_z\big]^\top.
\]
La probabilidad diaria de evento lesivo se modela como
\[
P(\text{lesión}_t\mid\mathbf{x}_t)\;=\;\frac{1}{1+\exp\!\big(-(\beta_0+\boldsymbol{\beta}^\top \mathbf{x}_t)\big)}.
\]
\textbf{Interpretación de coeficientes.} Si $\beta_i>0$ el predictor aumenta el riesgo; si $\beta_i<0$ lo reduce
(p.\,ej., HRV mayor suele ser protector).

\subsection{Regla operativa de alerta (suavizado de riesgo)}
Para disminuir falsos positivos día a día, se suaviza la señal de riesgo:
\[
S_t \;=\; \alpha_c\, P_t + (1-\alpha_c)\, S_{t-1}, \qquad \text{alerta si } S_t>\tau,
\]
con $\alpha_c\in[0{,}2,0{,}5]$ y umbral $\tau\in[0{,}30,0{,}50]$ según tolerancia al riesgo.

\subsection*{Ejemplo numérico (pasos verticales)}
%\datos
\[
w_t=1{,}8,\quad L_{t-1}=1{,}2,\quad \lambda=0{,}4.
\]
\[
\omega_{p,\text{rod}}=11{,}0\ \si{rad/s},\quad
\alpha_{p,\text{rod}}=85{,}0\ \si{rad/s^2},\quad
\mathrm{FRS}_p=2{,}8\ W,\quad
\mathrm{Asim}=8{,}0\ \%,\quad
\mathrm{HRV}_z=0{,}5,\quad
\mathrm{RPE}_z=1{,}0.
\]
Coeficientes (ilustrativos):
\[
\beta_0=-2{,}0,\quad
\boldsymbol{\beta}=[0{,}8,\;0{,}06,\;0{,}015,\;0{,}40,\;0{,}05,\;-0{,}30,\;0{,}25].
\]

%\formula
\[
\text{(1) } L_t=\lambda w_t+(1-\lambda)L_{t-1},\qquad
\text{(2) } \eta_t=\beta_0+\boldsymbol{\beta}^\top \mathbf{x}_t,\qquad
\text{(3) } P_t=\frac{1}{1+\mathrm{e}^{-\eta_t}}.
\]

%\sustitucion
\[
\text{(1) } L_t=0{,}4\cdot 1{,}8+0{,}6\cdot 1{,}2=0{,}72+0{,}72=1{,}44.
\]
\[
\mathbf{x}_t=[1{,}44,\ 11{,}0,\ 85{,}0,\ 2{,}8,\ 8{,}0,\ 0{,}5,\ 1{,}0].
\]
\[
\text{(2) } \eta_t=-2{,}0+(0{,}8)(1{,}44)+(0{,}06)(11{,}0)+(0{,}015)(85{,}0)
+(0{,}40)(2{,}8)+(0{,}05)(8{,}0)+(-0{,}30)(0{,}5)+(0{,}25)(1{,}0).
\]

%\calculo
\[
\begin{aligned}
(0{,}8)(1{,}44)&=1{,}152,& (0{,}06)(11{,}0)&=0{,}66,& (0{,}015)(85{,}0)&=1{,}275,\\
(0{,}40)(2{,}8)&=1{,}12,& (0{,}05)(8{,}0)&=0{,}40,& (-0{,}30)(0{,}5)&=-0{,}15,\\
(0{,}25)(1{,}0)&=0{,}25.
\end{aligned}
\]
Suma: $1{,}152+0{,}66+1{,}275+1{,}12+0{,}40+0{,}25-0{,}15=4{,}707$.\\
\[
\eta_t=-2{,}0+4{,}707=2{,}707,\qquad
P_t=\frac{1}{1+\mathrm{e}^{-2{,}707}}\approx 0{,}937.
\]

%\conclusion
La carga suavizada es $L_t=1{,}44$ y la probabilidad instantánea estimada de lesión es $P_t\approx 0{,}94$
(escenario de alto riesgo con estos coeficientes de ejemplo). Operativamente, se sugiere reducir
carga explosiva, enfatizar propiocepción y técnica de aterrizaje, y revaluar en 24\,h con HRV y dolor percibido.
\section*{Nota de recordatorio: Aplicación en Python para prevención de lesiones}

Este recordatorio resume la estructura y elementos clave para desarrollar una aplicación
en Python orientada a la prevención de lesiones deportivas mediante
promedios móviles exponenciales (EWMA) y un modelo logístico de riesgo.

\subsection*{Estructura del proyecto}
\begin{itemize}
    \item \textbf{app.py}: aplicación principal en \emph{Streamlit}.
    \item \textbf{core/features.py}: limpieza y construcción de variables (z-score, cargas).
    \item \textbf{core/ewma.py}: cálculo del promedio móvil exponencial (EWMA).
    \item \textbf{core/risk.py}: modelo logístico de riesgo, predicción y umbrales de alerta.
    \item \textbf{core/io\_utils.py}: lectura/escritura de datos en CSV/Excel.
    \item \textbf{models/logistic.pkl}: modelo entrenado (scikit-learn) con datos históricos.
    \item \textbf{data/ejemplo\_sesiones.csv}: conjunto de datos de ejemplo.
\end{itemize}

\subsection*{Datos de entrada esperados}
\begin{itemize}
    \item \textbf{Externos}: picos de $\omega$ y $\alpha$ en rodilla, fuerza de reacción (FRS), número de cambios de dirección, saltos.
    \item \textbf{Asimetrías}: porcentaje de diferencia entre miembros.
    \item \textbf{Fisiológicos}: HRV (RMSSD), RPE, horas de sueño.
    \item Opcional: columna \emph{injury} (0/1) para entrenar el modelo logístico.
\end{itemize}

\subsection*{Modelos y fórmulas}
\begin{itemize}
    \item \textbf{Carga suavizada (EWMA)}:
    \[
    L_t=\lambda w_t+(1-\lambda)L_{t-1}.
    \]
    \item \textbf{Modelo logístico de riesgo}:
    \[
    P(\text{lesión}_t)=\frac{1}{1+\exp\!\big(-(\beta_0+\boldsymbol{\beta}^\top \mathbf{x}_t)\big)}.
    \]
    \item \textbf{Suavizado de probabilidad (control)}:
    \[
    S_t=\alpha_c P_t+(1-\alpha_c)S_{t-1}, \qquad \text{alerta si } S_t>\tau.
    \]
\end{itemize}

\subsection*{Implementación práctica}
\begin{itemize}
    \item \textbf{Lenguaje y librerías}: Python, pandas, numpy, scikit-learn, streamlit, matplotlib, openpyxl.
    \item \textbf{Visualización}: gráficas interactivas en Streamlit; exportación a Excel o PDF/LaTeX.
    \item \textbf{Entrenamiento}: opcional con históricos (si se dispone de columna \emph{injury}).
    \item \textbf{Uso operativo}: carga de datos $\rightarrow$ cálculo de EWMA $\rightarrow$ probabilidad de lesión $\rightarrow$ alerta si $S_t>\tau$.
\end{itemize}

\subsection*{Notas finales}
\begin{itemize}
    \item Ajustar $\lambda$ (EWMA) y $\tau$ (umbral de alerta) según calendario competitivo.
    \item Los coeficientes $\beta$ deben entrenarse con datos reales del equipo/deporte.
    \item Este modelo se integra con informes en Excel o reportes técnicos en \LaTeX.
\end{itemize}
\section{Propuesta aplicada al contexto costarricense}

La predicción exacta de lesiones deportivas es sumamente compleja, 
pues influyen variables múltiples e impredecibles (condiciones del partido, contactos, estado mental, historial médico). 
Sin embargo, es posible desarrollar \textbf{modelos de monitoreo de carga y riesgo} 
que permitan reducir la probabilidad de lesiones, especialmente de rodilla, 
una de las más frecuentes en jugadores de fútbol en Costa Rica.

\subsection{Justificación}
\begin{itemize}
    \item Alta frecuencia de lesiones de ligamento cruzado anterior, meniscos y esguinces en el fútbol nacional.
    \item Uso extendido de canchas sintéticas, donde se asocia mayor incidencia de torsiones y lesiones articulares.
    \item Torneos con calendarios saturados, lo que favorece la acumulación de carga en cortos periodos.
    \item Limitados recursos en equipos pequeños, que carecen de tecnología avanzada de monitoreo.
\end{itemize}

\subsection{Enfoque metodológico}
\begin{enumerate}
    \item \textbf{Recolección de datos:} variables de carga externa (distancia, aceleraciones, saltos) mediante sensores GPS o IMU accesibles, más variables internas (HRV, RPE, horas de sueño).
    \item \textbf{Registro de lesiones:} documentación detallada de lesiones de rodilla y tobillo (tipo, contexto, tiempo de recuperación).
    \item \textbf{Cálculo de carga suavizada:} aplicación del modelo EWMA:
    \[
    L_t=\lambda w_t+(1-\lambda)L_{t-1},
    \]
    donde $w_t$ es la carga de la sesión y $L_t$ la carga suavizada.
    \item \textbf{Índice de riesgo logístico:} integración de variables fisiológicas y mecánicas en un modelo de regresión logística que estime la probabilidad de lesión.
    \item \textbf{Alertas visuales:} implementación de un sistema de semáforo (verde, amarillo, rojo) según umbrales definidos para riesgo acumulado.
\end{enumerate}

\subsection{Valor agregado local}
\begin{itemize}
    \item Comparación entre césped natural y sintético en equipos costarricenses.
    \item Diferencias por posición en el campo (defensas, mediocampistas, delanteros).
    \item Análisis de carga acumulada en semanas con partidos frecuentes.
    \item Desarrollo de una herramienta accesible para equipos con recursos limitados.
\end{itemize}

\subsection{Resultados esperados}
\begin{itemize}
    \item Identificación de patrones de riesgo asociados a picos de carga.
    \item Generación de recomendaciones prácticas para entrenadores y fisioterapeutas.
    \item Reducción de la incidencia de lesiones de rodilla mediante control de cargas.
    \item Potencial publicación de resultados en congresos y revistas regionales de ciencias del deporte.
\end{itemize}

\subsection{Conclusión}
Aunque no es viable predecir con exactitud cada lesión, 
el desarrollo de un sistema de monitoreo basado en EWMA y modelos de riesgo logístico 
puede aportar evidencia científica y práctica para reducir la probabilidad de lesiones en el fútbol costarricense. 
Esta propuesta constituye una oportunidad de innovación aplicada en biomecánica deportiva en el contexto local.
\section*{Propuesta de Trabajo Final de Graduación}

\subsection*{Título tentativo}
\textbf{Desarrollo de un modelo matemático y computacional basado en EWMA y riesgo logístico para el monitoreo de la carga y prevención de lesiones de rodilla en futbolistas costarricenses.}

\subsection*{Introducción}
Las lesiones de rodilla representan una de las principales causas de inactividad en futbolistas,
afectando tanto el rendimiento deportivo como la trayectoria profesional de los atletas.
En Costa Rica, la frecuencia de estas lesiones es elevada, especialmente en torneos con calendarios intensos
y en superficies sintéticas que incrementan la incidencia de torsiones articulares.
Si bien la predicción exacta de una lesión resulta inviable debido a la gran cantidad de variables implicadas,
sí es posible desarrollar modelos matemáticos que permitan \textbf{monitorear la carga física del atleta
e identificar momentos de mayor riesgo}.
En este contexto, el uso de promedios móviles exponenciales (EWMA) y modelos de riesgo logístico
ofrece un marco científico riguroso y aplicable al deporte nacional.

\subsection*{Planteamiento del problema}
La ausencia de sistemas accesibles y basados en evidencia para monitorear la carga de entrenamiento en equipos de fútbol costarricenses
ha contribuido a una alta incidencia de lesiones de rodilla.
Esto genera costos médicos elevados, disminución del rendimiento competitivo y menor desarrollo deportivo en la región.
Se plantea entonces la necesidad de un sistema matemático-computacional
que integre variables fisiológicas y de carga mecánica
para estimar de forma dinámica el riesgo de lesión y brindar alertas preventivas.

\subsection*{Objetivos}
\textbf{Objetivo general}
\begin{itemize}
    \item Desarrollar un modelo matemático y computacional basado en EWMA y regresión logística
    para monitorear la carga de entrenamiento y estimar el riesgo de lesiones de rodilla en futbolistas costarricenses.
\end{itemize}

\textbf{Objetivos específicos}
\begin{enumerate}
    \item Analizar los fundamentos físicos y biomecánicos relacionados con la dinámica de rodilla y tobillo en deportes de alto impacto.
    \item Implementar el modelo de carga acumulada mediante promedios móviles exponenciales (EWMA).
    \item Diseñar un modelo de regresión logística que integre variables fisiológicas y de carga externa.
    \item Desarrollar una aplicación en Python que permita la entrada de datos y genere indicadores de riesgo y reportes en tiempo real.
    \item Validar el modelo a partir de un conjunto de datos piloto de futbolistas costarricenses.
\end{enumerate}

\subsection*{Metodología}
\begin{enumerate}
    \item \textbf{Revisión bibliográfica}: estado del arte sobre modelos de carga, EWMA, ACWR y predicción de lesiones en fútbol.
    \item \textbf{Modelado matemático}: definición de ecuaciones de carga acumulada y probabilidad de lesión (regresión logística).
    \item \textbf{Desarrollo computacional}: implementación en Python (pandas, scikit-learn, streamlit) con interfaz interactiva.
    \item \textbf{Validación}: pruebas con datos simulados y, de ser posible, datos reales de un equipo de fútbol local.
    \item \textbf{Análisis}: evaluación del desempeño del modelo y comparación con hallazgos de literatura internacional.
\end{enumerate}

\subsection*{Resultados esperados}
\begin{itemize}
    \item Modelo matemático validado para el monitoreo de carga y riesgo de lesión.
    \item Aplicación computacional funcional con interfaz gráfica y reportes exportables.
    \item Identificación de patrones de carga y momentos críticos de riesgo en futbolistas.
    \item Recomendaciones prácticas para entrenadores y fisioterapeutas de equipos costarricenses.
\end{itemize}

\subsection*{Conclusiones preliminares}
El proyecto busca aportar una herramienta novedosa y de bajo costo,
que permita traducir fundamentos de la ingeniería física y la biomecánica
en beneficios prácticos para el deporte costarricense.
Con ello, se pretende reducir la incidencia de lesiones de rodilla,
mejorar la planificación de cargas y abrir un camino para investigaciones futuras
en biomecánica deportiva aplicada a nivel nacional.
\section{Bitácora de investigación: modelos para prevención de lesiones}

Este apartado recopila notas preliminares para una futura investigación orientada 
al desarrollo de modelos matemáticos y computacionales
que permitan monitorear la carga de entrenamiento y el riesgo de lesión en futbolistas.
El énfasis se encuentra en lesiones de rodilla, con aplicación en el contexto costarricense.

\subsection{Revisión bibliográfica inicial}
\begin{itemize}
    \item \textbf{EWMA y ACWR}: comparar promedios móviles exponenciales con promedios simples para estimar riesgo de lesión.
    \item \textbf{Modelos logísticos}: regresión logística aplicada a variables como carga, HRV, RPE y asimetrías.
    \item \textbf{Biomecánica de rodilla}: estudios sobre torques, aceleraciones angulares y fuerzas de reacción del suelo en fútbol.
\end{itemize}

\textbf{Tareas pendientes:}
\begin{enumerate}
    \item Buscar artículos en PubMed, Google Scholar y ResearchGate.
    \item Guardar resúmenes en Zotero/Mendeley.
    \item Crear una tabla con los hallazgos principales de cada referencia.
\end{enumerate}

\subsection{Fundamentos técnicos a repasar}
\begin{itemize}
    \item Estadística aplicada: regresión logística, métricas AUC, sensibilidad y especificidad.
    \item Series temporales: promedios móviles, EWMA, correlaciones cruzadas.
    \item Biomecánica: dinámica de rodilla y tobillo, cargas en césped natural vs sintético.
\end{itemize}

\subsection{Mini-proyectos en Python}
\begin{enumerate}
    \item Simular datos de jugadores: distancia, saltos, HRV, RPE.
    \item Calcular carga suavizada con EWMA:
    \[
    L_t = \lambda w_t + (1-\lambda) L_{t-1}.
    \]
    \item Construir un modelo logístico ficticio para estimar riesgo de lesión.
    \item Visualizar los resultados en gráficas simples con Python o Streamlit.
\end{enumerate}

\subsection{Contexto costarricense}
\begin{itemize}
    \item Factores locales: canchas sintéticas, climas variables, calendarios cargados.
    \item Problema: alta incidencia de lesiones de rodilla en equipos nacionales.
    \item Oportunidad: generar un modelo accesible y de bajo costo aplicable a ligas menores y equipos profesionales.
\end{itemize}

\subsection{Plan a mediano plazo}
\begin{enumerate}
    \item Crear un repositorio en GitHub/GitLab para código y apuntes.
    \item Redactar un marco teórico preliminar con lecturas revisadas.
    \item Contactar un profesor o club local para evaluar posibilidad de recolección de datos piloto.
\end{enumerate}

\subsection{Resultados esperados a largo plazo}
\begin{itemize}
    \item Estado del arte sobre prevención de lesiones basado en carga.
    \item Prototipo en Python (EWMA + riesgo logístico).
    \item Reportes exportables en Excel y PDF/\LaTeX.
    \item Posible propuesta formal de TFG en Ingeniería Física.
\end{itemize}
\section{Hipótesis y objetivo central de investigación}

\subsection{Hipótesis}
El monitoreo dinámico de variables biomecánicas (aceleraciones, impactos, torsiones, asimetrías)
y fisiológicas (variabilidad de la frecuencia cardiaca, percepción subjetiva de esfuerzo, horas de sueño),
integradas mediante un modelo matemático basado en promedios móviles exponenciales (EWMA)
y regresión logística,
permite estimar de forma probabilística el riesgo de lesión en futbolistas.
La implementación de este modelo contribuiría a reducir la exposición a cargas excesivas
y a mejorar las estrategias de prevención de lesiones de rodilla en el contexto costarricense.

\subsection{Objetivo central}
Desarrollar un sistema de monitoreo de riesgo de lesión para futbolistas costarricenses,
que combine el análisis de cargas de entrenamiento mediante EWMA
y un modelo logístico de predicción,
con el fin de generar indicadores de probabilidad de lesión
y proveer alertas preventivas útiles para entrenadores, fisioterapeutas y preparadores físicos.
\section{Limitaciones del proyecto}

Si bien la propuesta resulta innovadora y con gran potencial de aplicación en el contexto deportivo costarricense,
es importante reconocer las limitaciones que pueden surgir durante su desarrollo e implementación:

\begin{enumerate}
    \item \textbf{Disponibilidad de datos reales:} la recolección de información confiable sobre carga externa,
    variables fisiológicas y registros de lesiones en equipos de fútbol costarricenses
    puede resultar difícil por la falta de bases de datos centralizadas
    y la posible resistencia de clubes a compartir información sensible.

    \item \textbf{Tamaño de muestra reducido:} trabajar únicamente con un equipo o un número limitado de jugadores
    disminuye la robustez estadística del modelo,
    dificultando la generalización de los resultados a otras poblaciones deportivas.

    \item \textbf{Multifactorialidad de las lesiones:} las lesiones deportivas no dependen únicamente de la carga de entrenamiento,
    sino también de factores genéticos, ambientales, psicológicos y de contacto.
    En consecuencia, el modelo solo puede ofrecer una \emph{probabilidad relativa de riesgo},
    mas no una predicción exacta de la ocurrencia de una lesión.

    \item \textbf{Recursos limitados:} el acceso a sensores de alta precisión (GPS, IMU, plataformas de fuerza)
    puede verse restringido por su costo.
    Es posible que deba recurrirse a herramientas más accesibles
    o a variables autoinformadas como la percepción subjetiva de esfuerzo (RPE).

    \item \textbf{Aceptación en el ámbito deportivo:} entrenadores y jugadores pueden mostrar escepticismo
    frente a modelos matemáticos o algoritmos predictivos,
    por lo que será esencial presentar los resultados de manera clara y práctica
    (por ejemplo, mediante un sistema de alerta tipo semáforo).

    \item \textbf{Tiempo y alcance académico:} dado que se trata de un trabajo final de graduación,
    el proyecto debe limitar su alcance a un prototipo validado con datos simulados o un piloto reducido.
    Prometer una predicción exacta y exhaustiva de lesiones excedería el alcance realista del trabajo.
\end{enumerate}

En síntesis, el proyecto debe plantearse como un \textbf{sistema de monitoreo y estimación de riesgo},
con valor preventivo y de apoyo a la toma de decisiones,
más que como una herramienta de predicción absoluta de lesiones.
\end{document}