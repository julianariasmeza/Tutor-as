% ============================
% Documento LaTeX - Solución
% ============================
\documentclass[12pt,a4paper]{article}

% --- Idioma y codificación ---
\usepackage[spanish, es-nodecimaldot]{babel}
\usepackage[utf8]{inputenc}
\usepackage[T1]{fontenc}

% --- Paquetes matemáticos ---
\usepackage{amsmath, amssymb}
\usepackage{siunitx}
\sisetup{
  locale = DE,                   % coma decimal
  per-mode = symbol,             % unidades con "/"
  output-decimal-marker = {,},   % separador decimal coma
  exponent-product = \cdot,
  group-minimum-digits = 4
}

% --- Otros paquetes útiles ---
\usepackage{graphicx}
\usepackage{geometry}
\usepackage{float}
\geometry{margin=2.5cm}

% --- Inicio del documento ---
\begin{document}


\section*{Ejercicio 2.1}
Dos fuerzas \( \vec{P} \) y \( \vec{Q} \) se aplican en el punto \( A \) del gancho que se muestra en la Figura~\ref{fig:gancho}. Se sabe que:
\begin{itemize}
    \item \( P = \SI{75}{N} \)
    \item \( Q = \SI{125}{N} \)
    \item Ángulo entre fuerzas: \( 55^\circ \) (suma de \( 20^\circ \) y \( 35^\circ \))
\end{itemize}

Se pide determinar la magnitud y dirección de la fuerza resultante \( \vec{R} \) usando métodos gráficos y analíticos:

\subsection*{a) Ley del paralelogramo}
Este método consiste en construir un paralelogramo tomando los vectores \( \vec{P} \) y \( \vec{Q} \) como lados adyacentes. La diagonal que parte del punto de aplicación representa la fuerza resultante \( \vec{R} \).

\textbf{Paso 1: Calcular la magnitud de la resultante}
\[
R = \sqrt{P^2 + Q^2 + 2PQ\cos\theta}
\]
\[
R = \sqrt{75^2 + 125^2 + 2(75)(125)\cos(55^\circ)}
\]
\[
R = \sqrt{5625 + 15625 + 18750\cos(55^\circ)}
\]
\[
R = \sqrt{21250 + 10749,2} = \sqrt{31999,2} \approx \SI{179}{N}
\]

\textbf{Paso 2: Calcular la dirección de la resultante respecto a \( \vec{P} \)}
\[
\alpha = \tan^{-1}\left(\frac{Q\sin\theta}{P + Q\cos\theta}\right)
\]
\[
\alpha = \tan^{-1}\left(\frac{125\sin(55^\circ)}{75 + 125\cos(55^\circ)}\right)
\]
\[
\alpha = \tan^{-1}\left(\frac{102,5}{146,2}\right) \approx 75{,}1^\circ
\]

\begin{figure}[H]
    \centering
    \includegraphics[width=0.6\textwidth]{imagenes/Captura de Pantalla 2025-06-24 a la(s) 10.33.06.png}
    \caption{Ley del paralelogramo.}
    \label{fig:gancho}
\end{figure}

\begin{figure}[H]
    \centering
    \includegraphics[width=0.8\textwidth]{imagenes/Captura de Pantalla 2025-06-24 a la(s) 10.33.31.png}
    \caption{Solución gráfica usando Regla del triángulo}
\end{figure}

\subsection*{b) Regla del triángulo}
Este método consiste en colocar los vectores \( \vec{P} \) y \( \vec{Q} \) de manera consecutiva, es decir, el extremo de \( \vec{P} \) conecta con el inicio de \( \vec{Q} \). La resultante \( \vec{R} \) se representa como el lado que cierra el triángulo, desde el inicio de \( \vec{P} \) hasta el extremo de \( \vec{Q} \).

Se utiliza la ley del coseno para la magnitud:
\[
R = \sqrt{P^2 + Q^2 - 2PQ\cos(180^\circ - \theta)}
\]
\[
R = \sqrt{75^2 + 125^2 - 2(75)(125)\cos(125^\circ)}
\]
\[
\cos(125^\circ) = -\cos(55^\circ) \Rightarrow
\]
\[
R = \sqrt{5625 + 15625 + 2(75)(125)\cos(55^\circ)} = \SI{179}{N}
\]

Y la ley del seno para el ángulo \( \alpha \):
\[
\frac{\sin(\alpha)}{Q} = \frac{\sin(\theta)}{R} \Rightarrow
\]
\[
\alpha = \sin^{-1}\left(\frac{Q\sin(\theta)}{R}\right) = \sin^{-1}\left(\frac{125\sin(55^\circ)}{179}\right) \approx 75{,}1^\circ
\]

\newpage
\section*{Problema 2.4}

Se aplican dos fuerzas en el punto \(B\) de la viga \(AB\).  
Se desea determinar la magnitud y dirección de la resultante \(R\) mediante:  
\begin{enumerate}
  \item[a)] la ley del paralelogramo,  
  \item[b)] la regla del triángulo.  
\end{enumerate}

\subsection*{a) Ley del paralelogramo}
Se trazan las fuerzas de \(\SI{2}{kN}\) y \(\SI{3}{kN}\), aplicadas en el punto \(B\), con ángulos de \(40^\circ\) y \(60^\circ\) respecto a la vertical.  
Se construye el paralelogramo, y su diagonal representa la resultante \(R\).

\subsection*{b) Regla del triángulo}
Se colocan las fuerzas en forma consecutiva (la punta de la primera con la cola de la segunda).  
La tercera arista del triángulo representa la resultante \(R\).

\subsection*{Resultado}
De la construcción gráfica, se obtiene:
\[
R \approx \SI{3,30}{kN}, 
\qquad \alpha \approx 66,6^{\circ}
\]

\[
\boxed{R = \SI{3,30}{kN} \;\;\angle 66,6^{\circ}}
\]

\subsection*{Problema 2.4 — Procedimiento matemático}

\subsubsection*{Datos}
\[
P=\SI{2,00}{kN},\qquad Q=\SI{3,00}{kN},
\]
\[
\text{ángulos respecto a la vertical: } 
\widehat{(P,\ \downarrow)}=40^\circ\ \text{(hacia la izquierda)},\quad
\widehat{(Q,\ \downarrow)}=60^\circ\ \text{(hacia la derecha)}.
\]
El ángulo entre las fuerzas es 
\[
\theta=\;40^\circ+60^\circ=100^\circ.
\]

% =========================================================
\subsection*{A) Suma por componentes rectangulares}

\textbf{Sistema de ejes:} \(+x\) a la derecha, \(+y\) hacia arriba.  
Direcciones absolutas (medidas desde \(+x\), CCW):
\[
\theta_P=270^\circ-40^\circ=230^\circ,\qquad
\theta_Q=270^\circ+60^\circ=330^\circ.
\]

\paragraph{Fórmulas}
\[
P_x=P\cos\theta_P,\quad P_y=P\sin\theta_P,\qquad
Q_x=Q\cos\theta_Q,\quad Q_y=Q\sin\theta_Q,
\]
\[
R_x=P_x+Q_x,\quad R_y=P_y+Q_y,\qquad
R=\sqrt{R_x^{2}+R_y^{2}},\quad 
\alpha=\operatorname{atan2}(R_y,R_x).
\]

\paragraph{Sustitución y cálculo}
\[
\begin{aligned}
P_x&=2\cos 230^\circ=-1,2856\ \text{kN}, & 
P_y&=2\sin 230^\circ=-1,5321\ \text{kN},\\
Q_x&=3\cos 330^\circ=+2,5981\ \text{kN}, &
Q_y&=3\sin 330^\circ=-1,5000\ \text{kN}.
\end{aligned}
\]
\[
R_x= -1,2856+2,5981=+1,3125\ \text{kN},\qquad
R_y= -1,5321-1,5000=-3,0321\ \text{kN}.
\]
\[
R=\sqrt{(1,3125)^2+(-3,0321)^2}=3,304\ \text{kN}\approx\boxed{\SI{3,30}{kN}}.
\]
\[
\alpha=\operatorname{atan2}(-3,0321,\;1,3125)=-66,59^\circ.
\]

\paragraph{Interpretación de la dirección}
\(\alpha=-66,59^\circ\) indica que \(R\) forma un ángulo de 
\(\boxed{66,6^\circ}\) \textbf{por debajo de la horizontal hacia la derecha}
(sentido horario).  
Equivalente: \(\,R\) está a \(23,4^\circ\) a la derecha de la vertical hacia abajo.

\vspace{0.6em}
% =========================================================
\subsection*{B) Ley del paralelogramo (cosenos y senos)}

\paragraph{Fórmulas (paralelogramo / triángulo de fuerzas)}
Con \(\theta=100^\circ\) entre \(P\) y \(Q\):
\[
R=\sqrt{P^{2}+Q^{2}+2PQ\cos\theta},
\]
y, por ley de senos en el triángulo \(P\)-\(Q\)-\(R\),
\[
\frac{\sin\beta_P}{P}=\frac{\sin\theta}{R},\qquad
\frac{\sin\beta_Q}{Q}=\frac{\sin\theta}{R},
\]
donde \(\beta_P\) es el ángulo entre \(R\) y \(Q\), y \(\beta_Q\) el ángulo entre \(R\) y \(P\).

\paragraph{Sustitución y cálculo}
\[
R=\sqrt{2^{2}+3^{2}+2(2)(3)\cos 100^\circ}=3,304\ \text{kN}\approx\boxed{\SI{3,30}{kN}}.
\]
\[
\beta_P=\arcsin\!\left(\frac{P\sin 100^\circ}{R}\right)=36,59^\circ,
\quad
\beta_Q=\arcsin\!\left(\frac{Q\sin 100^\circ}{R}\right)=63,41^\circ.
\]

\paragraph{Dirección}
La resultante queda más \emph{cercana} a \(Q\) (porque \(Q>P\)).  
Ángulo de \(R\) respecto a la \textbf{vertical hacia abajo} del lado derecho:
\[
\theta_R=60^\circ-\beta_P=60^\circ-36,59^\circ=23,41^\circ.
\]
Respecto a la \textbf{horizontal}:
\[
\boxed{\alpha=90^\circ-\theta_R=66,59^\circ\approx 66,6^\circ}.
\]

\subsection*{Conclusión}
\[
\boxed{R=\SI{3,30}{kN}},\qquad 
\boxed{\alpha\approx 66,6^\circ\ \text{(debajo de la horizontal hacia la derecha)}}.
\]
Equivalente: \(R\) está a \(23,4^\circ\) a la derecha de la vertical descendente.
\begin{figure}[H]
    \centering
    \includegraphics[width=0.9\textwidth]{imagenes/Captura de Pantalla 2025-09-30 a la(s) 18.43.52.png}
    \caption{Ley del .}
    \label{fig:gancho}
\end{figure}
\newpage
\section*{Ejercicio 2.7}
Se aplican dos fuerzas en el gancho de apoyo que se muestra en la Figura~\ref{fig:gancho27}. Se sabe que:
\begin{itemize}
    \item \( P = \SI{35}{N} \)
    \item \( Q = \SI{50}{N} \)
    \item El ángulo entre las dos fuerzas es \( \alpha + 25^\circ \)
    \item La resultante \( \vec{R} \) debe ser horizontal.
\end{itemize}
\begin{figure}[H]
    \centering
    \includegraphics[width=0.8\textwidth]{imagenes/Captura de Pantalla 2025-06-24 a la(s) 10.58.48.png}
    \caption{Fuerzas aplicadas sobre el gancho en el Ejercicio 2.7.}
    \label{fig:gancho27}
\end{figure}
Se pide determinar por trigonometría:
\begin{enumerate}
    \item[a)] El ángulo \( \alpha \) requerido.
    \item[b)] La magnitud correspondiente de \( \vec{R} \).
\end{enumerate}

\subsection*{a) Cálculo del ángulo \( \alpha \)}
Aplicamos la ley de senos en el triángulo formado por las fuerzas:
\[
\frac{\sin(\alpha)}{50} = \frac{\sin(25^\circ)}{35}
\]
\[
\sin(\alpha) = \frac{50\sin(25^\circ)}{35} \approx 0{,}60374
\]
\[
\alpha = \sin^{-1}(0{,}60374) \approx 37{,}14^\circ
\]

\subsection*{b) Cálculo de la magnitud de \( \vec{R} \)}
Primero calculamos el ángulo opuesto a \( \vec{R} \) en el triángulo:
\[
\beta = 180^\circ - 25^\circ - \alpha = 180^\circ - 25^\circ - 37{,}14^\circ = 117{,}86^\circ
\]

Ahora aplicamos la ley de senos:
\[
\frac{R}{\sin(117{,}86^\circ)} = \frac{35}{\sin(25^\circ)}
\Rightarrow
R = \frac{35\sin(117{,}86^\circ)}{\sin(25^\circ)} \approx \SI{73{,}2}{N}
\]


\newpage
\section*{Problema 2.21 — Componentes en $x$ e $y$}

\subsection*{Geometría de referencia}
 Distancias desde \(O\) a los puntos de referencia:
\[
OA=\sqrt{600^{2}+800^{2}}=\SI{1000}{mm},\quad
OB=\sqrt{560^{2}+900^{2}}=\SI{1060}{mm},\quad
OC=\sqrt{480^{2}+900^{2}}=\SI{1020}{mm}.
\]

% ===========================
\subsection*{Fuerza de \(\SI{800}{N}\) (cuadrante I)}
 \(F=\SI{800}{N}\), proyecciones del vector-dirección: \((x,y)=(\SI{800}{mm},\SI{600}{mm})\), módulo del vector-dirección \(OA=\SI{1000}{mm}\).

 Componentes rectangulares por semejanza de triángulos:
\[
F_x = F\,\frac{\Delta x}{\lVert \mathbf{d}\rVert},\qquad
F_y = F\,\frac{\Delta y}{\lVert \mathbf{d}\rVert}.
\]

\(\Delta x=\SI{800}{mm}\), \(\Delta y=\SI{600}{mm}\), \(\lVert \mathbf{d}\rVert=\SI{1000}{mm}\).


\[
F_x=800\left(\frac{800}{1000}\right)=\SI{640}{N},\qquad
F_y=800\left(\frac{600}{1000}\right)=\SI{480}{N}.
\]


\[
\boxed{F_x=\SI{+640}{N}},\qquad
\boxed{F_y=\SI{+480}{N}}.
\]

% ===========================
\subsection*{Fuerza de \(\SI{424}{N}\) (cuadrante III)}
 \(F=\SI{424}{N}\), proyecciones: \((x,y)=(\SI{-560}{mm},\SI{-900}{mm})\), módulo \(OB=\SI{1060}{mm}\).


\[
F_x=F\,\frac{\Delta x}{\lVert \mathbf{d}\rVert},\qquad
F_y=F\,\frac{\Delta y}{\lVert \mathbf{d}\rVert}.
\]
 \(\Delta x=-\SI{560}{mm}\), \(\Delta y=-\SI{900}{mm}\), \(\lVert \mathbf{d}\rVert=\SI{1060}{mm}\).

\[
F_x=424\left(\frac{-560}{1060}\right)=424\left(-\frac{56}{106}\right)=-\SI{224}{N},
\]
\[
F_y=424\left(\frac{-900}{1060}\right)=424\left(-\frac{90}{106}\right)=-\SI{360}{N}.
\]


\[
\boxed{F_x=\SI{-224}{N}},\qquad
\boxed{F_y=\SI{-360}{N}}.
\]

% ===========================
\subsection*{Fuerza de \(\SI{408}{N}\) (cuadrante IV)}
\(F=\SI{408}{N}\), proyecciones: \((x,y)=(\SI{+480}{mm},\SI{-900}{mm})\), módulo \(OC=\SI{1020}{mm}\).


\[
F_x=F\,\frac{\Delta x}{\lVert \mathbf{d}\rVert},\qquad
F_y=F\,\frac{\Delta y}{\lVert \mathbf{d}\rVert}.
\]

 \(\Delta x=\SI{480}{mm}\), \(\Delta y=-\SI{900}{mm}\), \(\lVert \mathbf{d}\rVert=\SI{1020}{mm}\).


\[
F_x=408\left(\frac{480}{1020}\right)=408\left(\frac{48}{102}\right)=\SI{192}{N},
\]
\[
F_y=408\left(\frac{-900}{1020}\right)=408\left(-\frac{90}{102}\right)=-\SI{360}{N}.
\]


\[
\boxed{F_x=\SI{+192}{N}},\qquad
\boxed{F_y=\SI{-360}{N}}.
\]
\begin{figure}[H]
    \centering
    \includegraphics[width=0.8\textwidth]{imagenes/Captura de Pantalla 2025-09-30 a la(s) 18.59.29.png}
    \caption{Fuerzas aplicadas  Ejercicio 2.21.}
    \label{fig:gancho27}
\end{figure}
\newpage
\section*{Ejercicio 2.23}
Determine las componentes \( x \) y \( y \) de cada una de las fuerzas que se muestran en la Figura~\ref{fig:gancho223}.
\begin{figure}[H]
    \centering
    \includegraphics[width=0.55\textwidth]{imagenes/Captura de Pantalla 2025-06-24 a la(s) 11.21.42.png}
    \caption{Diagrama del ejercicio 2.23.}
    \label{fig:gancho223}
\end{figure}
\subsection*{Descomposición de fuerzas}
Se tienen tres fuerzas aplicadas en diferentes direcciones, y se desea conocer sus componentes en los ejes cartesianos.

\textbf{1) Fuerza de \( \SI{40}{lb} \)} con ángulo de \(60^\circ\) hacia abajo respecto al eje positivo \( x \):
\begin{align*}
F_x &= 40 \cos(60^\circ) = 20{,}0 \ \text{lb} \\
F_y &= -40 \sin(60^\circ) = -34{,}6 \ \text{lb}
\end{align*}

\textbf{2) Fuerza de \( \SI{50}{lb} \)} con ángulo de \(50^\circ\) hacia abajo respecto al eje negativo \( x \):
\begin{align*}
F_x &= -50 \sin(50^\circ) = -38{,}3 \ \text{lb} \\
F_y &= -50 \cos(50^\circ) = -32{,}1 \ \text{lb}
\end{align*}

\textbf{3) Fuerza de \( \SI{60}{lb} \)} con ángulo de \(25^\circ\) sobre el eje positivo \( x \):
\begin{align*}
F_x &= 60 \cos(25^\circ) = 54{,}4 \ \text{lb} \\
F_y &= 60 \sin(25^\circ) = 25{,}4 \ \text{lb}
\end{align*}
\newpage
\section*{Problema 2.37 — Resultante de tres fuerzas (\(\alpha=40^\circ\))}

\subsection*{Esquema angular}
El plano inclinado forma \(20^\circ\) con el eje \(x\). 
Las fuerzas y sus direcciones absolutas respecto de \(+x\) son:
\[
\begin{aligned}
&\text{Fuerza de }60\ \text{lb}: && \theta_{60}=20^\circ,\\
&\text{Fuerza de }80\ \text{lb}: && \theta_{80}=20^\circ+\alpha=60^\circ,\\
&\text{Fuerza de }120\ \text{lb}:&& \theta_{120}=-30^\circ \quad (\text{hacia la derecha y abajo}).
\end{aligned}
\]

\subsection*{Componentes rectangulares}


\[
F_1=60\ \text{lb},\quad F_2=80\ \text{lb},\quad F_3=120\ \text{lb}.
\]

 Para cada fuerza \(F\) con ángulo \(\theta\) medido desde \(+x\):
\[
F_x=F\cos\theta,\qquad F_y=F\sin\theta.
\]


\[
\begin{aligned}
&\textbf{60 lb:} &&F_{1x}=60\cos20^\circ=56{,}38\ \text{lb}, &&F_{1y}=60\sin20^\circ=20{,}52\ \text{lb},\\[2mm]
&\textbf{80 lb:} &&F_{2x}=80\cos60^\circ=40{,}00\ \text{lb}, &&F_{2y}=80\sin60^\circ=69{,}28\ \text{lb},\\[2mm]
&\textbf{120 lb:}&&F_{3x}=120\cos(-30^\circ)=103{,}92\ \text{lb}, &&F_{3y}=120\sin(-30^\circ)=-60{,}00\ \text{lb}.
\end{aligned}
\]

\subsection*{Suma vectorial}
\(R_x=\sum F_x,\; R_y=\sum F_y\).


\[
R_x=56{,}38+40{,}00+103{,}92=200{,}30\ \text{lb},\qquad
R_y=20{,}52+69{,}28-60{,}00=29{,}80\ \text{lb}.
\]

\subsection*{Magnitud y dirección}
 \(
R=\sqrt{R_x^2+R_y^2},\quad
\tan\alpha_R=\dfrac{R_y}{R_x}.
\)


\[
R=\sqrt{(200{,}30)^2+(29{,}80)^2}=202{,}50\ \text{lb}\approx \boxed{203\ \text{lb}},
\]
\[
\alpha_R=\arctan\!\left(\frac{29{,}80}{200{,}30}\right)=8{,}46^\circ.
\]


La resultante es
\[
\boxed{R\approx 203\ \text{lb}\;\; \angle\; 8{,}46^\circ}
\]
medida \emph{sobre la horizontal hacia la derecha} (ligeramente inclinada hacia arriba).\\
Como referencia en SI:
\[
R=202{,}50\ \text{lb}\times 4{,}448\,22\ \frac{\text{N}}{\text{lb}}
\approx \boxed{\SI{902}{N}},\qquad \angle\,\boxed{8{,}46^\circ}.
\]
\begin{figure}[H]
    \centering
    \includegraphics[width=0.55\textwidth]{imagenes/Captura de Pantalla 2025-09-30 a la(s) 19.09.13.png}
    \caption{Diagrama del ejercicio 2.23.}
    \label{fig:gancho223}
\end{figure}
\end{document}
