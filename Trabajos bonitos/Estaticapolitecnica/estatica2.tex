% ==========================================================
% Problemas 3.1 y 3.2 — Momento de una fuerza (desarrollo vertical)
% ==========================================================
\documentclass[11pt,letterpaper]{article}

% --- Idioma y formato ---
\usepackage[T1]{fontenc}
\usepackage[utf8]{inputenc}
\usepackage[spanish, es-nodecimaldot]{babel}

% --- Matemática y unidades ---
\usepackage{amsmath, amssymb, siunitx}
\sisetup{
  locale = DE,
  output-decimal-marker = {,},
  per-mode = symbol,
  exponent-product = \cdot,
  group-minimum-digits = 4
}

% --- Página ---
\usepackage[a4paper,margin=2.4cm]{geometry}
\setlength{\parskip}{4pt}
\setlength{\parindent}{0pt}

% ==========================================================
\begin{document}

\section*{Problema 3.1}
\[
\alpha = 28^\circ, \quad F = \SI{16}{N}, \quad AB = \SI{170}{mm} = \SI{0,17}{m}, \quad \beta = 20^\circ
\]

\subsection*{Determinación del ángulo efectivo}
\[
\theta = \alpha - \beta = 28^\circ - 20^\circ = 8^\circ
\]

\subsection*{Componentes de la fuerza}
\[
F_x = F \cos \theta = 16 \cos 8^\circ = \SI{15,8443}{N}
\]
\[
F_y = F \sin \theta = 16 \sin 8^\circ = \SI{2,2268}{N}
\]

\subsection*{Componentes de la posición}
\[
x = (0,17) \cos 20^\circ = \SI{0,159748}{m}
\]
\[
y = (0,17) \sin 20^\circ = \SI{0,058143}{m}
\]

\subsection*{Momento respecto a \( B \)}
\[
M_B = x F_y + y F_x
\]
\[
M_B = (0,159748)(2,2268) + (0,058143)(15,8443)
\]
\[
M_B = 0,3558 + 0,9213 = \SI{1,277}{N\cdot m}
\]

\[
\boxed{M_B = 1,277\ \si{N\cdot m}}
\]

% ==========================================================
\newpage
\section*{Problema 3.2}
\[
\alpha = 28^\circ, \quad F = \SI{16}{N}, \quad AB = \SI{170}{mm} = \SI{0,17}{m}
\]

\subsection*{Descomposición de la fuerza}
\[
Q = F \sin \alpha = 16 \sin 28^\circ = \SI{7,5115}{N}
\]
\[
P = F \cos \alpha = 16 \cos 28^\circ = \SI{14,147}{N}
\]

\subsection*{Momento respecto a \( B \)}
\[
M_B = r_{AB}\,Q = (0,17)(7,5115) = \SI{1,277}{N\cdot m}
\]

\[
\boxed{M_B = 1,277\ \si{N\cdot m}}
\]
\newpage
% ==========================================================
\section*{Problema 3.3}

\[
F = \SI{300}{N}, \quad \text{ángulo con la horizontal } = 25^\circ
\]
\[
DA_x = \SI{0,1}{m}, \quad DA_y = \SI{0,2}{m}
\]

\subsection*{(a) Momento de la fuerza de \SI{300}{N} alrededor de $D$}

Componentes de la fuerza:
\[
F_x = F \cos 25^\circ = 300 \cos 25^\circ = \SI{271,89}{N}
\]
\[
F_y = F \sin 25^\circ = 300 \sin 25^\circ = \SI{126,785}{N}
\]


\[
\vec{F} = (271,89)\,\mathbf{i} + (126,785)\,\mathbf{j}\ \si{N}
\]


\[
\vec{r} = \overrightarrow{DA} = (-0,1)\,\mathbf{i} + (-0,2)\,\mathbf{j}\ \si{m}
\]

Producto cruz:
\[
\vec{M}_D = \vec{r} \times \vec{F}
\]

Componente en \(\mathbf{k}\):
\[
M_D =
\begin{vmatrix}
\mathbf{i} & \mathbf{j} & \mathbf{k} \\
-0,1 & -0,2 & 0 \\
271,89 & 126,785 & 0
\end{vmatrix}
=
\left[ (-0,1)(126,785) - (-0,2)(271,89) \right]\mathbf{k}
\]


\[
(-0,1)(126,785) = -12,6785
\]
\[
(-0,2)(271,89) = -54,378
\]
\[
-(-0,2)(271,89) = +54,378
\]
\[
-12,6785 + 54,378 = 41,6995
\]

\[
M_D = 41,6995\ \mathbf{k}\ \si{N\cdot m}
\]

\[
\boxed{M_D = 41,7\ \si{N\cdot m}}
\]

\subsection*{(b) Fuerza mínima en \(B\) que produce el mismo momento alrededor de \(D\)}


\[
DB = \SI{0,28284}{m} \quad (\text{dirección }45^\circ)
\]

Sea \(Q\) la magnitud de la fuerza aplicada en \(B\), perpendicular a \(DB\).

\[
M_D = Q \cdot DB
\]

\[
41,700\ \si{N\cdot m} = Q \cdot 0,28284\ \si{m}
\]

\[
Q = \frac{41,700}{0,28284} = \SI{147,4}{N}
\]

\[
\boxed{Q = 147,4\ \si{N} \text{ a } 45^\circ}
\]

\newpage
% ==========================================================
\section*{Problema 3.4}

\[
F = \SI{300}{N}, \quad \text{ángulo con la horizontal } = 25^\circ
\]
\[
DC_x = \SI{0,2}{m}, \quad DC_y = \SI{0,125}{m}
\]

\subsection*{(a) Momento de la fuerza de \SI{300}{N} alrededor de $D$}


\[
\boxed{M_D = 41,7\ \si{N\cdot m}}
\]

\subsection*{(b) Fuerza horizontal en $C$ que produce el mismo momento alrededor de $D$}


\[
\vec{C} = C\,\mathbf{i} \quad (\text{fuerza horizontal pura})
\]


\[
\vec{r} = \overrightarrow{DC} = (0,2)\,\mathbf{i} - (0,125)\,\mathbf{j}\ \si{m}
\]

Momento:
\[
\vec{M}_D = \vec{r} \times \vec{C}
=
\begin{vmatrix}
\mathbf{i} & \mathbf{j} & \mathbf{k} \\
0,2 & -0,125 & 0 \\
C & 0 & 0
\end{vmatrix}
=
\left[(0,2)(0) - (-0,125)(C)\right]\mathbf{k}
=
(0,125\,C)\,\mathbf{k}
\]


\[
M_D = 0,125\,C
\]


\[
41,7 = 0,125\,C
\]

\[
C = \frac{41,7}{0,125} = \SI{333,60}{N}
\]


\[
\boxed{C \approx 334\ \si{N} \text{ hacia la derecha}}
\]

\subsection*{(c) Fuerza mínima en $C$ que produce el mismo momento alrededor de $D$}


Ángulo de la línea \(DC\) respecto a la horizontal:
\[
\tan \alpha = \frac{0,125}{0,2}
\quad\Longrightarrow\quad
\alpha = 32,0^\circ
\]

Longitud de \(DC\):
\[
DC = \sqrt{(0,2)^2 + (0,125)^2}
\]
\[
DC = \sqrt{0,04 + 0,015625}
= \sqrt{0,055625}
= \SI{0,23585}{m}
\]


\[
M_D = C \cdot DC
\]

\[
41,7\ \si{N\cdot m} = C \cdot 0,23585\ \si{m}
\]

\[
C = \frac{41,7}{0,23585} = \SI{176,8}{N}
\]

Orientación:
\[
\text{dirección de la fuerza } = \alpha + 90^\circ = 32^\circ + 90^\circ = 122^\circ
\]



\[
\boxed{C = 176,8\ \si{N} \text{ a } 58^\circ}
\]
\newpage
% ==========================================================
\section*{Problema 3.21}

\[
F = \SI{200}{N}
\]


\[
r_{CA} = \overrightarrow{AC} \quad\Rightarrow\quad
r_{CA} = (0,06)\,\mathbf{i} + (0,075)\,\mathbf{j} + 0\,\mathbf{k}\ \si{m}
\]


\[
F_C = F_x\,\mathbf{i} + F_y\,\mathbf{j} + F_z\,\mathbf{k}
\]


\[
F_x = 0
\]
\[
F_y = -(200)\cos 30^\circ = -200\cos 30^\circ\ \si{N}
\]
\[
F_z = (200)\sin 30^\circ = 200\sin 30^\circ\ \si{N}
\]


\[
F_C = 0\,\mathbf{i} + \bigl[-200\cos 30^\circ\bigr]\mathbf{j}
      + \bigl[200\sin 30^\circ\bigr]\mathbf{k}
\]


\[
\vec{M}_A = \vec{r}_{CA} \times \vec{F}_C
\]


\[
\vec{M}_A =
\begin{vmatrix}
\mathbf{i} & \mathbf{j} & \mathbf{k} \\
0,06 & 0,075 & 0 \\
0 & -200\cos 30^\circ & 200\sin 30^\circ
\end{vmatrix}
\]


Componente \(\mathbf{i}\):
\[
M_{A,x} =
\begin{vmatrix}
0,075 & 0 \\
-200\cos 30^\circ & 200\sin 30^\circ
\end{vmatrix}
=
0,075(200\sin 30^\circ) - 0(-200\cos 30^\circ)
\]
\[
M_{A,x} = 0,075 \cdot 200 \sin 30^\circ
= 15 \sin 30^\circ
= 15(0,5)
= 7,50\ \si{N\cdot m}
\]

Componente \(\mathbf{j}\) :
\[
M_{A,y} = -
\begin{vmatrix}
0,06 & 0 \\
0 & 200\sin 30^\circ
\end{vmatrix}
=
-
\left[
0,06(200\sin 30^\circ) - 0\cdot 0
\right]
\]
\[
M_{A,y} = -\left[0,06 \cdot 200 \cdot 0,5\right]
= -\left[0,06 \cdot 100\right]
= -6,00\ \si{N\cdot m}
\]

Componente \(\mathbf{k}\):
\[
M_{A,z} =
\begin{vmatrix}
0,06 & 0,075 \\
0 & -200\cos 30^\circ
\end{vmatrix}
=
0,06(-200\cos 30^\circ) - 0,075(0)
\]
\[
M_{A,z} = -12 \cos 30^\circ
= -12 (0,8660)
= -10,392 \ \si{N\cdot m}
\]


\[
\vec{M}_A =
(7,50)\,\mathbf{i}
+(-6,00)\,\mathbf{j}
+(-10,39)\,\mathbf{k}
\ \si{N\cdot m}
\]

\[
\boxed{\vec{M}_A =
7,50\,\mathbf{i}
-6,00\,\mathbf{j}
-10,39\,\mathbf{k}
\ \si{N\cdot m}}
\]
\newpage
% ==========================================================
\section*{Problema 3.22 — Momento respecto de \(O\)}

\[
\vec{r}_{BO} = (0)\,\mathbf{i} + (7)\,\mathbf{j} + (0)\,\mathbf{k}\ \si{m},
\qquad
\vec{F}_B = (295)\,\mathbf{i} + (980)\,\mathbf{j} + (440)\,\mathbf{k}\ \si{N}.
\]

\[
\vec{M}_O = \vec{r}_{BO} \times \vec{F}_B
=
\begin{vmatrix}
\mathbf{i} & \mathbf{j} & \mathbf{k} \\
0 & 7 & 0 \\
295 & 980 & 440
\end{vmatrix}
\]

\[
M_{O,x} = (7)(440) - (0)(980) = 3080
\]
\[
M_{O,y} = -\bigl[(0)(440) - (0)(295)\bigr] = 0
\]
\[
M_{O,z} = (0)(980) - (7)(295) = -2065 \approx -2070
\]

\[
\boxed{
\vec{M}_O =
3080\,\mathbf{i}
+
0\,\mathbf{j}
-
2070\,\mathbf{k}
\ \si{N\cdot m}
}
\]


\end{document}
