\documentclass[12pt]{article}
\usepackage[utf8]{inputenc}
\usepackage[spanish]{babel}
\usepackage{amsmath, amssymb, siunitx}
\usepackage{geometry}
\usepackage{graphicx}
\usepackage{float}
\usepackage{booktabs}
\geometry{margin=2.5cm}
\setlength{\parindent}{0pt}
\setlength{\parskip}{0.6em}
\begin{document}




\begin{center}
\Huge \textbf{Universidad Libre de Costa Rica (ULICORI)} \\[3 cm]

\huge \textbf{Análisis de Casos} \\[3cm]

\Large
\textbf{Curso:} Estadística \\[1 cm]
\textbf{Profesor:} Vladimir Araya Salazar \\[1 cm]
\textbf{Estudiante:} Saray Vega Solano \\[7cm]

\large 27 de abril de 2025
\end{center}








\newpage

\section*{Intervalo de Confianza para "Sin instrucción"}

\begin{itemize}
    \item Nivel de confianza: \( 98\% \)
    \item Promedio: \( \bar{x} = 46053,83 \)
    \item Desviación estándar: \( s = 42749,12 \)
    \item Número de datos: \( n = 6 \)
    \item Valor crítico: \( z = 2,33 \)
\end{itemize}

El error estándar se calcula como:

\[
\text{Error estándar} = \frac{42749,12}{\sqrt{6}} = 17452,26
\]

El margen de error es:

\[
\text{Margen} = 2,33 \times 17452,26 = 40663,77
\]

Así, el intervalo de confianza para la población de personas sin instrucción es:

\[
\left(Li= 46053,83 - 40663,77 \ y \ Ls= 46053,83 + 40663,77\right)
\]

es decir:

\[
\boxed{ \left(5390,06  - 86717,60\right) }
\]
\section*{Intervalo de Confianza para "Primaria Incompleta"}

\begin{itemize}
    \item Nivel de confianza: \( 95\% \)
    \item Promedio: \( \bar{x} = 137710,67 \)
    \item Desviación estándar: \( s = 144767,32 \)
    \item Número de datos: \( n = 6 \)
    \item Valor crítico: \( z = 1,96 \)
\end{itemize}

El error estándar se calcula como:

\[
\text{Error estándar} = \frac{144767,32}{\sqrt{6}} = 59100,76
\]

El margen de error es:

\[
\text{Margen} = 1,96 \times 59100,76 = 115837,49
\]

Así, el intervalo de confianza para la población de personas con primaria incompleta es:

\[
(Li = 137710,67 - 115837,49 \quad \text{y} \quad Ls = 137710,67 + 115837,49)
\]

es decir:

\[
\boxed{(21873,18\ - \,253548,16)}
\]
\section*{Intervalo de Confianza para "Primaria Completa"}

\begin{itemize}
    \item Nivel de confianza: \( 91\% \)
    \item Promedio: \( \bar{x} = 173589,00 \)
    \item Desviación estándar: \( s = 225575,32 \)
    \item Número de datos: \( n = 6 \)
    \item Valor crítico: \( z = 1,695 \)
\end{itemize}

El error estándar se calcula como:

\[
\text{Error estándar} = \frac{225575,32}{\sqrt{6}} = 92090,35
\]

El margen de error es:

\[
\text{Margen} = 1,695 \times 92090,35 = 156093,14
\]

Así, el intervalo de confianza para la población de personas con primaria completa es:

\[
(Li = 173589,00 - 156093,14 \quad \text{y} \quad Ls = 173589,00 + 156093,14)
\]

es decir:

\[
\boxed{(17495,86\ - \,329682,14)}
\]
\section*{Intervalo de Confianza para "Secundaria académica incompleta"}

\begin{itemize}
    \item Nivel de confianza: \( 89\% \)
    \item Promedio: \( \bar{x} = 116351,00 \)
    \item Desviación estándar: \( s = 120425,00 \)
    \item Número de datos: \( n = 6 \)
    \item Valor crítico: \( z = 1,645 \)
\end{itemize}

El error estándar se calcula como:

\[
\text{Error estándar} = \frac{120425,00}{\sqrt{6}} = 49163,09
\]

El margen de error es:

\[
\text{Margen} = 1,645 \times 49163,09 = 80873,28
\]

Así, el intervalo de confianza para la población de personas con secundaria académica incompleta es:

\[
(Li = 116351,00 - 80873,28 \quad \text{y} \quad Ls = 116351,00 + 80873,28)
\]

es decir:

\[
\boxed{(35477,72\ - \,197224,28)}
\]
\section*{Intervalo de Confianza para "Secundaria técnica incompleta"}

\begin{itemize}
    \item Nivel de confianza: \( 85\% \)
    \item Promedio: \( \bar{x} = 7312,00 \)
    \item Desviación estándar: \( s = 5513,62 \)
    \item Número de datos: \( n = 6 \)
    \item Valor crítico: \( z = 1,440 \)
\end{itemize}

El error estándar se calcula como:

\[
\text{Error estándar} = \frac{5513,62}{\sqrt{6}} = 2250,92
\]

El margen de error es:

\[
\text{Margen} = 1,440 \times 2250,92 = 3241,32
\]

Así, el intervalo de confianza para la población de personas con secundaria técnica incompleta es:

\[
(Li = 7312,00 - 3241,32 \quad \text{y} \quad Ls = 7312,00 + 3241,32)
\]

es decir:

\[
\boxed{(4070,68\ - \,10553,32)}
\]
\section*{Intervalo de Confianza para "Secundaria técnica completa"}

\begin{itemize}
    \item Nivel de confianza: \( 81\% \)
    \item Promedio: \( \bar{x} = 18773,17 \)
    \item Desviación estándar: \( s = 18247,43 \)
    \item Número de datos: \( n = 6 \)
    \item Valor crítico: \( z = 1,312 \)
\end{itemize}

El error estándar se calcula como:

\[
\text{Error estándar} = \frac{18247,43}{\sqrt{6}} = 7449,48
\]

El margen de error es:

\[
\text{Margen} = 1,312 \times 7449,48 = 9773,72
\]

Así, el intervalo de confianza para la población de personas con secundaria técnica completa es:

\[
(Li = 18773,17 - 9773,72 \quad \text{y} \quad Ls = 18773,17 + 9773,72)
\]

es decir:

\[
\boxed{(8999,45\ - \,28546,89)}
\]
\section*{Intervalo de Confianza para Educación superior Pregrado y grado}

\begin{itemize}
    \item Nivel de confianza: \( 73\% \)
    \item Promedio: \( \bar{x} = 141316,33 \)
    \item Desviación estándar: \( s = 240212,70 \)
    \item Número de datos: \( n = 6 \)
    \item Valor crítico: \( z = 1,067 \)
\end{itemize}

El error estándar se calcula como:

\[
\text{Error estándar} = \frac{240212,70}{\sqrt{6}} = 98066,01
\]

El margen de error es:

\[
\text{Margen} = 1,067 \times 98066,01 = 104636,43
\]

Así, el intervalo de confianza para la población de personas con educación superior pregrado y grado es:

\[
(Li = 141316,33 - 104636,43 \quad \text{y} \quad Ls = 141316,33 + 104636,43)
\]

es decir:

\[
\boxed{(36679,90\ - \,245952,76)}
\]
\section*{Intervalo de Confianza para "Educación superior Posgrado"}

\begin{itemize}
    \item Nivel de confianza: \( 70\% \)
    \item Promedio: \( \bar{x} = 15683,33 \)
    \item Desviación estándar: \( s = 29741,50 \)
    \item Número de datos: \( n = 6 \)
    \item Valor crítico: \( z = 1,036 \)
\end{itemize}

El error estándar se calcula como:

\[
\text{Error estándar} = \frac{29741,50}{\sqrt{6}} = 12141,86
\]

El margen de error es:

\[
\text{Margen} = 1,036 \times 12141,86 = 12578,97
\]

Así, el intervalo de confianza para la población de personas con educación superior posgrado es:

\[
(Li = 15683,33 - 12578,97 \quad \text{y} \quad Ls = 15683,33 + 12578,97)
\]

es decir:

\[
\boxed{(3104,36\ - \,28262,30)}
\]
\begin{table}[H]
\centering
\caption{Resumen de Intervalos de Confianza}
\tiny
\begin{tabular}{|l|c|c|c|c|}
\hline
\textbf{Nivel de instrucción} & \textbf{Promedio} & \textbf{Desv. Estándar} & \textbf{Límite Inferior} & \textbf{Límite Superior} \\
\hline
Sin instrucción & 46053,83 & 42749,12 & 5390,06 & 86717,60 \\
Primaria incompleta & 137710,67 & 144767,32 & 21873,18 & 253548,16 \\
Primaria completa & 173589,00 & 225575,32 & 17495,86 & 329682,14 \\
Secundaria académica incompleta & 116351,00 & 120425,00 & 35477,72 & 197224,28 \\
Secundaria académica completa & 7312,00 & 5513,62 & 4070,68 & 10553,32 \\
Secundaria técnica incompleta & 18773,17 & 18247,43 & 8999,45 & 28546,89 \\
Educación superior Pregrado y grado & 141316,33 & 240212,70 & 36679,90 & 245952,76 \\
Educación superior Posgrado & 15683,33 & 29741,50 & 3104,36 & 28262,30 \\
\hline
\end{tabular}
\end{table}
\newpage
\section*{Análisis de Ji-cuadrado (\( \chi^2 \))}

\textbf{Hipótesis:}
\begin{itemize}
    \item Hipótesis nula (\( H_0 \)): El nivel de instrucción y la región de planificación son independientes.
    \item Hipótesis alternativa (\( H_1 \)): El nivel de instrucción y la región de planificación no son independientes.
\end{itemize}
\begin{table}[H]
\centering
\caption{Tabla de Frecuencias Esperadas}
\tiny
\begin{tabular}{|l|c|c|c|c|c|c|c|}
\hline
\textbf{Nivel de instrucción} & \textbf{Central} & \textbf{Chorotega} & \textbf{Pacífico Central} & \textbf{Brunca} & \textbf{Huetar Caribe} & \textbf{Huetar Norte} & \textbf{Totales} \\
\hline
Sin instrucción & 151913,65 & 18775,17 & 14427,03 & 17706,14 & 21678,34 & 20171,23 & 202822 \\
Primaria Incompleta & 476149,71 & 58847,84 & 45219,30 & 55497,15 & 67947,40 & 63223,57 & 635714 \\
Primaria Completa & 639253,42 & 79006,00 & 60709,04 & 74507,55 & 91222,58 & 84880,62 & 853476 \\
Secundaria académica Incompleta & 574908,34 & 71053,52 & 54598,27 & 67007,87 & 82040,43 & 76336,83 & 767568 \\
Secundaria académica Completa & 438362,67 & 54177,70 & 41630,71 & 51092,92 & 62555,12 & 58206,17 & 585264 \\
Secundaria técnica Incompleta & 50995,66 & 6302,61 & 4842,99 & 5943,75 & 7277,17 & 6771,25 & 68085 \\
Secundaria técnica Completa & 66846,74 & 8261,66 & 6348,34 & 7791,26 & 9539,15 & 8875,97 & 89248 \\
Educación superior Pregrado y grado & 579008,37 & 71560,25 & 54987,64 & 67485,74 & 82625,51 & 76881,23 & 773042 \\
Educación superior Posgrado & 67450,44 & 8336,27 & 6405,68 & 7861,62 & 9625,30 & 8956,13 & 90054 \\
\hline
\textbf{Totales} & 3044889 & 376321 & 289169 & 354894 & 434511 & 404303 & 4065273 \\
\hline
\end{tabular}
\end{table}
\begin{table}[H]
\centering
\caption{Tabla de \(\frac{(O - E)^2}{E}\)}
\tiny
\begin{tabular}{|l|c|c|c|c|c|c|c|}
\hline
\textbf{Nivel de instrucción} & \textbf{Central} & \textbf{Chorotega} & \textbf{Pacífico Central} & \textbf{Brunca} & \textbf{Huetar Caribe} & \textbf{Huetar Norte} & \textbf{Total fila} \\
\hline
Sin instrucción & 2561,85 & 2929,90 & 2672,15 & 2102,81 & 7581,00 & 17581,27 & 35328,98 \\
Primaria Incompleta & 4189,54 & 2287,43 & 1841,66 & 10323,50 & 9090,88 & 18853,81 & 46186,82 \\
Primaria Completa & 53,41 & 130,18 & 75,79 & 640,01 & 205,66 & 684,68 & 1789,73 \\
Secundaria académica Incompleta & 353,79 & 713,97 & 77,18 & 1260,90 & 1458,46 & 224,18 & 4088,48 \\
Secundaria académica Completa & 479,52 & 31,55 & 2255,78 & 731,85 & 104,84 & 2815,78 & 6419,32 \\
Secundaria técnica Incompleta & 6312,36 & 2422,45 & 16769,88 & 4243,62 & 358,93 & 2966,38 & 32773,62 \\
Secundaria técnica Completa & 44,60 & 37,78 & 1079,36 & 14,17 & 885,98 & 11,39 & 2073,28 \\
Educación superior Pregrado y grado & 4718,25 & 1566,37 & 6424,65 & 7760,33 & 23087,75 & 21837,42 & 65494,77 \\
Educación superior Posgrado & 1167,90 & 912,64 & 1117,64 & 1506,65 & 5542,97 & 5838,37 & 16185,17 \\
\hline
\textbf{Totales} & 19881,23 & 11032,26 & 32314,08 & 28583,84 & 48316,47 & 70813,28 & 210941,16 \\
\hline
\end{tabular}
\end{table}

\vspace{0.5cm}

\textbf{Cálculo de grados de libertad:}

\[
g.l. = (r-1)(c-1) = (9-1)(6-1) = 8 \times 5 = 40
\]

\vspace{0.5cm}

\textbf{Valor crítico de Ji-cuadrado:}

Para un nivel de significancia de \( \alpha = 0,05 \) y \( 40 \) grados de libertad:

\[
\chi^2_{\text{teórico}} = 55,758
\]

\vspace{0.5cm}

\textbf{Valor calculado de Ji-cuadrado:}

\[
\chi^2_{\text{calculado}} = 210941,16
\]

\vspace{0.5cm}

\textbf{Comparación:}

\[
\chi^2_{\text{calculado}} = 210941,16 \gg \chi^2_{\text{teórico}} = 55,758
\]

\vspace{0.5cm}

\textbf{Decisión:}

Dado que el valor calculado de Ji-cuadrado es mucho mayor que el valor crítico, se \textbf{rechaza} la hipótesis nula \( H_0 \).

\vspace{0.5cm}

\textbf{Conclusión:}

Existe evidencia suficiente para afirmar que el nivel de instrucción y la región de planificación \textbf{no son independientes}.  
Por lo tanto, hay una relación significativa entre ambas variables en la población costarricense analizada.

\end{document}