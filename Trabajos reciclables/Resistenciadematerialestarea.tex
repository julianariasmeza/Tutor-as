\documentclass[11pt,a4paper]{article}

% Paquetes básicos y compatibles con conversores a Word
\usepackage[spanish]{babel}
\usepackage[utf8]{inputenc}
\usepackage[T1]{fontenc}
\usepackage{amsmath,amssymb}

\title{Tarea: Deformaciones en barras y alambres}
\author{}
\date{}

\begin{document}
\maketitle

\section*{Ejercicio 1}

La barra rígida \(AD\) está soportada por dos alambres de acero de
diámetro \(d = \tfrac{1}{16}\,\text{in}\) \((E = 29\times 10^{6}\,\text{psi})\),
además de un pasador y una ménsula en \(A\).
Los alambres estaban originalmente tensos.
Se desea determinar:

\begin{itemize}
  \item[(a)] La tensión adicional en cada alambre cuando se aplica una carga
    \(P = 220\,\text{lb}\) en \(D\).
  \item[(b)] La deflexión correspondiente del punto \(D\).
\end{itemize}

Las distancias horizontales son
\[
AB = BC = CD = 12\,\text{in},
\]
y las longitudes de los alambres son
\[
L_{BE} = 10\,\text{in}, \qquad L_{CF} = 18\,\text{in}.
\]

%-------------------------------------------------
\subsection*{Datos}

\begin{itemize}
  \item Módulo de elasticidad del acero:
    \[
      E = 29\times 10^{6}\,\text{psi}.
    \]
  \item Diámetro de cada alambre:
    \[
      d = \frac{1}{16}\,\text{in}.
    \]
  \item Área de la sección transversal de cada alambre:
    \[
      A = \frac{\pi d^{2}}{4}
        = \frac{\pi}{4}\left(\frac{1}{16}\right)^{2}\text{in}^{2}.
    \]
  \item Longitudes de los alambres:
    \[
      L_{BE} = 10\,\text{in}, \qquad L_{CF} = 18\,\text{in}.
    \]
  \item Carga aplicada en el extremo:
    \[
      P = 220\,\text{lb}.
    \]
  \item Distancias a lo largo de la barra rígida:
    \[
      AB = 12\,\text{in},\quad AC = 24\,\text{in},\quad AD = 36\,\text{in}.
    \]
\end{itemize}

%-------------------------------------------------
\subsection*{Desarrollo}

\subsubsection*{1. Relación geométrica: giro de la barra}

Sea \(\theta\) la rotación (en radianes) de la barra rígida \(ABCD\) respecto al
pivote en \(A\).

Las deflexiones verticales de los puntos \(B\), \(C\) y \(D\) son
proporcionales a sus distancias a \(A\):
\begin{align*}
  \delta_B &= AB\,\theta = 12\,\theta,\\
  \delta_C &= AC\,\theta = 24\,\theta,\\
  \delta_D &= AD\,\theta = 36\,\theta.
\end{align*}

Como los extremos superiores de los alambres están fijos en \(E\) y \(F\),
las elongaciones de los alambres coinciden con las deflexiones de los
puntos \(B\) y \(C\):
\[
  \delta_{BE} = \delta_B = 12\,\theta, 
  \qquad
  \delta_{CF} = \delta_C = 24\,\theta.
\]

\subsubsection*{2. Ley de Hooke en los alambres}

Para cada alambre se cumple
\[
  \delta = \frac{PL}{AE}
  \qquad\Longrightarrow\qquad
  P = \frac{EA}{L}\,\delta.
\]

\paragraph{Alambre \(BE\).}

\begin{align*}
  P_{BE}
    &= \frac{EA}{L_{BE}}\,\delta_{BE} \\
    &= \frac{E A}{L_{BE}}\,(12\,\theta) \\
    &= \left(29\times 10^{6}\right)
       \left(\frac{\pi}{4}\left(\frac{1}{16}\right)^{2}\right)
       \frac{12}{10}\,\theta \\
    &\approx 106{,}765\times 10^{3}\,\theta\ \text{lb}.
\end{align*}

\paragraph{Alambre \(CF\).}

\begin{align*}
  P_{CF}
    &= \frac{EA}{L_{CF}}\,\delta_{CF} \\
    &= \frac{E A}{L_{CF}}\,(24\,\theta) \\
    &= \left(29\times 10^{6}\right)
       \left(\frac{\pi}{4}\left(\frac{1}{16}\right)^{2}\right)
       \frac{24}{18}\,\theta \\
    &\approx 118{,}628\times 10^{3}\,\theta\ \text{lb}.
\end{align*}

\subsubsection*{3. Ecuación de equilibrio de momentos}

Se toma momento respecto al punto \(A\) sobre el cuerpo libre de la barra
\(ABCD\) (sentido antihorario positivo):

\begin{itemize}
  \item la fuerza \(P_{BE}\) actúa en \(B\), a \(12\,\text{in}\) de \(A\);
  \item la fuerza \(P_{CF}\) actúa en \(C\), a \(24\,\text{in}\) de \(A\);
  \item la carga \(P\) actúa en \(D\), a \(36\,\text{in}\) de \(A\) (sentido
        hacia abajo).
\end{itemize}

Entonces:
\begin{align*}
  \sum M_A &= 0 \\
  12\,P_{BE} + 24\,P_{CF} - 36\,P &= 0.
\end{align*}

Sustituyendo las expresiones de \(P_{BE}\) y \(P_{CF}\) en función de
\(\theta\):
\begin{align*}
  12\left(106{,}765\times 10^{3}\theta\right)
  + 24\left(118{,}628\times 10^{3}\theta\right)
  - 36(220) &= 0.
\end{align*}

Se agrupa el término en \(\theta\):
\begin{align*}
  \left[12(106{,}765\times 10^{3}) + 24(118{,}628\times 10^{3})\right]\theta
  &= 36(220),\\
  4{,}1283\times 10^{6}\,\theta &\approx 7{,}920.
\end{align*}

\subsubsection*{4. Cálculo de la rotación \(\theta\)}

\begin{align*}
  \theta
    &= \frac{36\cdot 220}{4{,}1283\times 10^{6}} \\
    &\approx 1{,}9185\times 10^{-3}\ \text{rad}.
\end{align*}

\subsubsection*{5. Tensiones adicionales en los alambres (inciso a)}

\paragraph{Alambre \(BE\).}

\begin{align*}
  P_{BE}
    &= 106{,}765\times 10^{3}\,\theta \\
    &\approx 106{,}765\times 10^{3}
      \left(1{,}9185\times 10^{-3}\right) \\
    &\approx 204{,}8\ \text{lb}
      \;\approx\; 205\ \text{lb}.
\end{align*}

\paragraph{Alambre \(CF\).}

\begin{align*}
  P_{CF}
    &= 118{,}628\times 10^{3}\,\theta \\
    &\approx 118{,}628\times 10^{3}
      \left(1{,}9185\times 10^{-3}\right) \\
    &\approx 227{,}6\ \text{lb}
      \;\approx\; 228\ \text{lb}.
\end{align*}

\subsubsection*{6. Deflexión del punto \(D\) (inciso b)}

\begin{align*}
  \delta_D &= AD\,\theta = 36\,\theta,\\
           &= 36\left(1{,}9185\times 10^{-3}\right)\ \text{in},\\
           &\approx 0{,}0691\ \text{in}.
\end{align*}

\bigskip
Por lo tanto:
\[
  P_{BE} \approx 205\ \text{lb},\qquad
  P_{CF} \approx 228\ \text{lb},\qquad
  \delta_D \approx 0{,}069\ \text{in}.
\]
\newpage
\section*{Ejercicio 2}

Sabiendo que cada uno de los ejes \(AB\), \(BC\) y \(CD\) consta de una varilla circular
sólida, determine:
\begin{itemize}
    \item[(a)] el eje en el que ocurre el máximo esfuerzo cortante,
    \item[(b)] la magnitud de dicho esfuerzo.
\end{itemize}

Los momentos aplicados y los diámetros son:
\[
T_1 = 400\ \text{lb·in}, \qquad
T_2 = 1200\ \text{lb·in}, \qquad
T_3 = 500\ \text{lb·in},
\]
\[
d_{AB} = 0.60\ \text{in}, \qquad
d_{BC} = 0.75\ \text{in}, \qquad
d_{CD} = 0.90\ \text{in}.
\]

%-------------------------------------------------------
\subsection*{Datos}

\[
T_{1} = 400\ \text{lb·in}, \qquad
T_{2} = 1200\ \text{lb·in}, \qquad
T_{3} = 500\ \text{lb·in}
\]

\[
d_{AB} = 0.60\ \text{in}, \qquad
d_{BC} = 0.75\ \text{in}, \qquad
d_{CD} = 0.90\ \text{in}
\]

Radio:
\[
c = \frac{d}{2}
\]

Momento polar para barra circular sólida:
\[
J = \frac{\pi d^{4}}{32} = \frac{\pi}{2}c^{4}
\]

Esfuerzo cortante máximo:
\[
\tau_{\max} = \frac{Tc}{J} = \frac{2T}{\pi c^{3}}
\]

%-------------------------------------------------------
\subsection*{Desarrollo}

\subsubsection*{1. Torques internos}

Tramo \(AB\):
\[
T_{AB} = 400\ \text{lb·in}
\]

Tramo \(BC\):
\[
T_{BC} = -400 + 1200 = 800\ \text{lb·in}
\]

Tramo \(CD\):
\[
T_{CD} = -400 + 1200 + 500 = 1300\ \text{lb·in}
\]

%-------------------------------------------------------
\subsubsection*{2. Esfuerzos cortantes}

\paragraph{Eje \(AB\).}

\[
d_{AB} = 0.60\ \text{in},
\qquad
c_{AB} = 0.30\ \text{in}
\]

\[
\tau_{\max,AB}
= \frac{2T_{AB}}{\pi c_{AB}^{3}}
= \frac{2(400)}{\pi(0.30)^{3}}
\]

\[
\tau_{\max,AB} \approx 9431\ \text{psi}
\]

\paragraph{Eje \(BC\).}

\[
d_{BC} = 0.75\ \text{in},
\qquad
c_{BC} = 0.375\ \text{in}
\]

\[
\tau_{\max,BC}
= \frac{2T_{BC}}{\pi c_{BC}^{3}}
= \frac{2(800)}{\pi(0.375)^{3}}
\]

\[
\tau_{\max,BC} \approx 9658\ \text{psi}
\]

\paragraph{Eje \(CD\).}

\[
d_{CD} = 0.90\ \text{in},
\qquad
c_{CD} = 0.45\ \text{in}
\]

\[
\tau_{\max,CD}
= \frac{2T_{CD}}{\pi c_{CD}^{3}}
= \frac{2(1300)}{\pi(0.45)^{3}}
\]

\[
\tau_{\max,CD} \approx 9082\ \text{psi}
\]

%-------------------------------------------------------
\subsection*{Conclusiones}

El esfuerzo cortante máximo ocurre en:
\[
\boxed{\text{Eje } BC}
\]

Valor del esfuerzo máximo:
\[
\boxed{\tau_{\max} \approx 9658\ \text{psi} \approx 9.66\ \text{ksi}}
\]
\newpage
\section*{Ejercicio 3}

Los pares mostrados en la figura se ejercen sobre las poleas A y B.
Sabiendo que los ejes son sólidos y de aluminio \((G = 77\times 10^{9}\ \text{Pa})\),
determine:
\begin{itemize}
    \item[(a)] el ángulo entre A y B,
    \item[(b)] el ángulo entre A y C.
\end{itemize}

Los datos geométricos y de carga son:
\[
T_A = 300\ \text{N·m}, \qquad
T_B = 400\ \text{N·m},
\]
\[
L_{AB} = 0.90\ \text{m}, \qquad
L_{BC} = 0.75\ \text{m},
\]
\[
d_{AB} = 30\ \text{mm} = 0.030\ \text{m}, \qquad
d_{BC} = 46\ \text{mm} = 0.046\ \text{m}.
\]

%-------------------------------------------------
\subsection*{Datos}

\[
G = 77\times 10^{9}\ \text{Pa}
\]

\[
T_A = 300\ \text{N·m}, \qquad
T_B = 400\ \text{N·m}
\]

\[
L_{AB} = 0.90\ \text{m}, \qquad
L_{BC} = 0.75\ \text{m}
\]

\[
d_{AB} = 0.030\ \text{m}, \qquad
d_{BC} = 0.046\ \text{m}
\]

Para un eje circular sólido:
\[
J = \frac{\pi d^{4}}{32}
\]

Ángulo de giro:
\[
\varphi = \frac{T L}{J G}
\]

%-------------------------------------------------
\subsection*{Desarrollo}

\subsubsection*{1. Torques internos}

Tramo \(AB\):
\[
T_{AB} = T_A = 300\ \text{N·m}
\]

Tramo \(BC\):
\[
T_{BC} = T_A + T_B = 300 + 400 = 700\ \text{N·m}
\]

%-------------------------------------------------
\subsubsection*{2. Momento polar de inercia}

\paragraph{Eje AB.}
\[
J_{AB} = \frac{\pi (0.030)^{4}}{32}
        = 7.96\times 10^{-8}\ \text{m}^{4}
\]

\paragraph{Eje BC.}
\[
J_{BC} = \frac{\pi (0.046)^{4}}{32}
        = 1.39\times 10^{-7}\ \text{m}^{4}
\]

%-------------------------------------------------
\subsubsection*{3. Ángulos de giro}

\paragraph{Ángulo entre A y B (tramo AB).}

\[
\varphi_{AB}
= \frac{T_{AB} L_{AB}}{J_{AB} G}
= \frac{300(0.90)}{(7.96\times 10^{-8})(77\times 10^{9})}
\]

\[
\varphi_{AB} = 4.41\times 10^{-2}\ \text{rad}
\]

\paragraph{Ángulo en el tramo BC.}

\[
\varphi_{BC}
= \frac{T_{BC} L_{BC}}{J_{BC} G}
= \frac{700(0.75)}{(1.39\times 10^{-7})(77\times 10^{9})}
\]

\[
\varphi_{BC} = 1.55\times 10^{-2}\ \text{rad}
\]

\paragraph{Ángulo entre A y C.}

\[
\varphi_{AC}
= \varphi_{AB} + \varphi_{BC}
= 0.0441 + 0.0155
= 5.96\times 10^{-2}\ \text{rad}
\]

%-------------------------------------------------


\[
\boxed{\varphi_{AB} \approx 4.41\times 10^{-2}\ \text{rad}
\quad (\approx 2.53^\circ)}
\]

\[
\boxed{\varphi_{AC} \approx 5.96\times 10^{-2}\ \text{rad}
\quad (\approx 3.42^\circ)}
\]
\newpage
\section*{Ejercicio 4}

Dos ejes sólidos de acero \((G = 77\times 10^{9}\ \text{Pa})\) están conectados por los
engranes que muestra la figura. Sabiendo que el radio del engrane B es
\(r_B = 20\ \text{mm}\), determine el ángulo que gira el extremo A cuando
\(T_A = 75\ \text{N·m}\).

Los datos geométricos son:
\[
r_B = 20\ \text{mm} = 0.020\ \text{m}, \qquad
r_C = 60\ \text{mm} = 0.060\ \text{m},
\]
\[
L_{AB} = 500\ \text{mm} = 0.50\ \text{m}, \qquad
L_{CD} = 400\ \text{mm} = 0.40\ \text{m},
\]
\[
d_{AB} = 20\ \text{mm} = 0.020\ \text{m}, \qquad
d_{CD} = 24\ \text{mm} = 0.024\ \text{m}.
\]

%-------------------------------------------------
\subsection*{Datos}

\[
G = 77\times 10^{9}\ \text{Pa}
\]

\[
T_A = 75\ \text{N·m}
\]

\[
r_B = 0.020\ \text{m}, \qquad
r_C = 0.060\ \text{m}
\]

\[
L_{AB} = 0.50\ \text{m}, \qquad
L_{CD} = 0.40\ \text{m}
\]

\[
d_{AB} = 0.020\ \text{m}, \qquad
d_{CD} = 0.024\ \text{m}
\]

Momento polar de inercia para un eje circular sólido:
\[
J = \frac{\pi d^{4}}{32}
\]

Ángulo de giro de un eje en torsión:
\[
\varphi = \frac{T L}{J G}
\]

%-------------------------------------------------
\subsection*{Desarrollo}

\subsubsection*{1. Relación de torques entre los engranes}

La fuerza tangencial de contacto entre los engranes B y C es la misma en ambos:
\[
F = \frac{T_{AB}}{r_B} = \frac{T_{CD}}{r_C}
\]

De aquí:
\[
T_{CD} = \frac{r_C}{r_B}\,T_{AB}
       = \frac{0.060}{0.020}(75)
       = 225\ \text{N·m}
\]

Por tanto:
\[
T_{AB} = 75\ \text{N·m}, \qquad
T_{CD} = 225\ \text{N·m}
\]

%-------------------------------------------------
\subsubsection*{2. Momento polar de inercia de cada eje}

\paragraph{Eje CD.}
\[
J_{CD} = \frac{\pi (0.024)^{4}}{32}
       = 3.26\times 10^{-8}\ \text{m}^{4}
\]

\paragraph{Eje AB.}
\[
J_{AB} = \frac{\pi (0.020)^{4}}{32}
       = 1.57\times 10^{-8}\ \text{m}^{4}
\]

%-------------------------------------------------
\subsubsection*{3. Giro en el eje CD y rotación en C y B}

\paragraph{Torsión en el eje CD.}
\[
\varphi_{CD}
= \frac{T_{CD} L_{CD}}{J_{CD} G}
= \frac{225(0.40)}{(3.26\times 10^{-8})(77\times 10^{9})}
\]

\[
\varphi_{CD} = 3.59\times 10^{-2}\ \text{rad}
\]

Como el extremo D está fijo, el ángulo en C es:
\[
\varphi_C = \varphi_{CD} = 3.59\times 10^{-2}\ \text{rad}
\]

\paragraph{Relación de giros entre los engranes.}

La deformación circunferencial en el punto de contacto es la misma:
\[
\delta = r_C \varphi_C = r_B \varphi_B
\]

De donde:
\[
\varphi_B = \frac{r_C}{r_B}\,\varphi_C
          = \frac{0.060}{0.020}(3.59\times 10^{-2})
          = 1.08\times 10^{-1}\ \text{rad}
\]

%-------------------------------------------------
\subsubsection*{4. Giro en el eje AB y rotación en A}

\paragraph{Torsión en el eje AB.}
\[
\varphi_{AB}
= \frac{T_{AB} L_{AB}}{J_{AB} G}
= \frac{75(0.50)}{(1.57\times 10^{-8})(77\times 10^{9})}
\]

\[
\varphi_{AB} = 3.10\times 10^{-2}\ \text{rad}
\]

\paragraph{Ángulo total en el extremo A.}

El extremo A gira la suma del giro propio del eje AB más la rotación que ya tiene B:
\[
\varphi_A = \varphi_B + \varphi_{AB}
          = 1.08\times 10^{-1} + 3.10\times 10^{-2}
          = 1.39\times 10^{-1}\ \text{rad}
\]

En grados:
\[
\varphi_A \approx 7.94^\circ
\]

%-------------------------------------------------
\subsection*{Conclusión}

\[
\boxed{\varphi_A \approx 1.39\times 10^{-1}\ \text{rad}
\quad (\approx 7.9^\circ)}
\]
\newpage
\section*{Ejercicio 5}

Un eje que consta de un tubo de acero de \(50\ \text{mm}\) de diámetro exterior
debe transmitir \(100\ \text{kW}\) de potencia mientras gira a una frecuencia de
\(20\ \text{Hz}\).
Determine el espesor del tubo que deberá utilizarse si el esfuerzo cortante
no debe exceder \(60\ \text{MPa}\).

%-------------------------------------------------
\subsection*{Datos}

\[
P = 100\ \text{kW} = 100\times 10^{3}\ \text{W}
\]

\[
f = 20\ \text{Hz}
\]

\[
d_{\text{ext}} = 50\ \text{mm} = 0.050\ \text{m},
\qquad
c_2 = \frac{d_{\text{ext}}}{2} = 25\ \text{mm} = 0.025\ \text{m}
\]

\[
\tau_{\text{adm}} = 60\ \text{MPa} = 60\times 10^{6}\ \text{Pa}
\]

Se desea hallar el espesor:
\[
t = c_2 - c_1,
\]
donde \(c_1\) es el radio interior del tubo.

Para un eje circular hueco:
\[
J = \frac{\pi}{2}\left(c_2^{4} - c_1^{4}\right)
\]

Esfuerzo cortante máximo en la superficie externa:
\[
\tau_{\max} = \frac{T c_2}{J}
\]

Potencia transmitida por un eje en rotación:
\[
P = T \omega, \qquad \omega = 2\pi f
\]

%-------------------------------------------------
\subsection*{Desarrollo}

\subsubsection*{1. Cálculo del torque transmitido}

\[
\omega = 2\pi f = 2\pi(20)\ \text{rad/s}
\]

\[
T = \frac{P}{\omega}
  = \frac{100\times 10^{3}}{2\pi(20)}
  = 7.96\times 10^{2}\ \text{N·m}
\]

\[
T \approx 7.96\times 10^{2}\ \text{N·m}
\]

%-------------------------------------------------
\subsubsection*{2. Relación entre el esfuerzo y los radios}

Se impone \(\tau_{\max} = \tau_{\text{adm}}\):

\[
\tau_{\text{adm}}
= \frac{T c_2}{J}
= \frac{T c_2}{\frac{\pi}{2}(c_2^{4} - c_1^{4})}
= \frac{2 T c_2}{\pi (c_2^{4} - c_1^{4})}
\]

Despejando \(c_1^{4}\):

\[
\tau_{\text{adm}}\,\pi\,(c_2^{4} - c_1^{4}) = 2 T c_2
\]

\[
c_2^{4} - c_1^{4}
= \frac{2 T c_2}{\pi \tau_{\text{adm}}}
\]

\[
c_1^{4} = c_2^{4} - \frac{2 T c_2}{\pi \tau_{\text{adm}}}
\]

Sustituyendo valores numéricos (en unidades SI):

\[
c_2^{4} = (0.025)^{4} = 3.90625\times 10^{-7}\ \text{m}^{4}
\]

\[
\frac{2 T c_2}{\pi \tau_{\text{adm}}}
= \frac{2(7.96\times 10^{2})(0.025)}{\pi(60\times 10^{6})}
= 2.11\times 10^{-7}\ \text{m}^{4}
\]

\[
c_1^{4}
= 3.90625\times 10^{-7} - 2.11\times 10^{-7}
= 1.80\times 10^{-7}\ \text{m}^{4}
\]

\[
c_1 = (1.80\times 10^{-7})^{1/4}
    = 2.06\times 10^{-2}\ \text{m}
    \approx 0.0206\ \text{m}
\]

En milímetros:

\[
c_1 \approx 20.6\ \text{mm}
\]

%-------------------------------------------------
\subsubsection*{3. Espesor del tubo}

\[
t = c_2 - c_1
  = 25\ \text{mm} - 20.6\ \text{mm}
  = 4.4\ \text{mm}\ (\text{aprox.})
\]

%-------------------------------------------------
\subsection*{Conclusión}

El espesor requerido del tubo es, aproximadamente,
\[
\boxed{t \approx 4.4\ \text{mm}}
\]
para que el esfuerzo cortante no exceda los \(60\ \text{MPa}\).
\newpage
\section*{Ejercicio 6}

Dos fuerzas verticales de \(15\ \text{kips}\) se aplican a una viga con la sección
transversal en T mostrada. Se pide determinar los esfuerzos máximos de tensión
y de compresión en la porción \(BC\) de la viga.

%-------------------------------------------------
\subsection*{Datos}

\begin{itemize}
  \item Cargas: \(P = 15\ \text{kips}\) aplicadas en los puntos \(B\) y \(C\).
  \item Geometría de la viga:
  \[
    AB = 40\ \text{in},\quad BC = 60\ \text{in},\quad CD = 40\ \text{in}.
  \]
  \item Sección transversal en T:
  \begin{itemize}
    \item Alma: \(3\ \text{in}\) de ancho por \(6\ \text{in}\) de altura.
    \item Ala: \(9\ \text{in}\) de ancho por \(2\ \text{in}\) de espesor.
  \end{itemize}
\end{itemize}

Tomamos como eje de referencia \(x_0\) la base de la sección (cara inferior del ala)
y medimos las coordenadas \(y_0\) hacia arriba.

%-------------------------------------------------
\subsection*{Centroide de la sección}

Se divide la sección en dos rectángulos:

\begin{itemize}
  \item[1)] Alma: \(b_1 = 3\ \text{in}\), \(h_1 = 6\ \text{in}\).
        \[
          A_1 = b_1 h_1 = 3\cdot 6 = 18\ \text{in}^2.
        \]
        El centroide del alma está a
        \[
          \bar y_{01} = 2 + \frac{6}{2} = 5\ \text{in}
        \]
        sobre la base (se suman los \(2\ \text{in}\) del ala más la mitad del alma).
  \item[2)] Ala: \(b_2 = 9\ \text{in}\), \(h_2 = 2\ \text{in}\).
        \[
          A_2 = b_2 h_2 = 9\cdot 2 = 18\ \text{in}^2.
        \]
        El centroide del ala está a
        \[
          \bar y_{02} = \frac{2}{2} = 1\ \text{in}
        \]
        sobre la base.
\end{itemize}

\noindent
La tabla de cálculo del centroide queda:

\begin{center}
\begin{tabular}{c c c c}
\hline
Elemento & \(A\) (\(\text{in}^2\)) & \(\bar y_0\) (\(\text{in}\)) & \(A\,\bar y_0\) (\(\text{in}^3\)) \\
\hline
1 (alma) & 18 & 5 & 90 \\
2 (ala)  & 18 & 1 & 18 \\
\hline
$\Sigma$ & 36 &     & 108 \\
\hline
\end{tabular}
\end{center}

\noindent
Área total:
\[
A = A_1 + A_2 = 36\ \text{in}^2.
\]

Coordenada del centroide global respecto a la base:
\[
\bar y_0 = \frac{\sum A_i \bar y_{0i}}{\sum A_i}
         = \frac{108}{36}
         = 3\ \text{in}.
\]

Por tanto, el **eje neutro** está a \(3\ \text{in}\) por encima de la base.

\[
y_{\text{top}} = 8 - 3 = 5\ \text{in}, \qquad
y_{\text{bot}} = 0 - 3 = -3\ \text{in}.
\]

%-------------------------------------------------
\subsection*{Momento de inercia respecto al eje neutro}

Se aplica el teorema de ejes paralelos para cada rectángulo.

\paragraph{Elemento 1 (alma).}

\[
d_1 = \bar y_{01} - \bar y_0 = 5 - 3 = 2\ \text{in}.
\]

\[
I_1 = \frac{1}{12} b_1 h_1^{3} + A_1 d_1^{2}
    = \frac{1}{12}(3)(6^{3}) + 18(2^{2})
    = \frac{1}{12}(3)(216) + 18(4)
    = 54 + 72
    = 126\ \text{in}^4.
\]

\paragraph{Elemento 2 (ala).}

\[
d_2 = \bar y_0 - \bar y_{02} = 3 - 1 = 2\ \text{in}.
\]

\[
I_2 = \frac{1}{12} b_2 h_2^{3} + A_2 d_2^{2}
    = \frac{1}{12}(9)(2^{3}) + 18(2^{2})
    = \frac{1}{12}(9)(8) + 18(4)
    = 6 + 72
    = 78\ \text{in}^4.
\]

\noindent
Momento de inercia total respecto al eje neutro:

\[
I = I_1 + I_2 = 126 + 78 = 204\ \text{in}^4.
\]

%-------------------------------------------------
\subsection*{Diagrama de momento en la parte \(BC\)}

Por simetría de cargas y geometría:

\[
R_A = R_D = 15\ \text{kips}.
\]

En el tramo \(BC\) (entre las cargas) el esfuerzo cortante es cero y, por tanto,
el momento flector es constante. Cortando entre \(B\) y \(C\) y haciendo suma de
momentos respecto al corte:

\[
M - P a = 0,
\qquad a = 40\ \text{in}.
\]

\[
M = P a = 15(40) = 600\ \text{kip·in}.
\]

Este es el momento que actúa en cualquier sección de la porción \(BC\).

%-------------------------------------------------
\subsection*{Esfuerzos de flexión}

La fórmula de flexión es:
\[
\sigma = -\,\frac{M y}{I},
\]
donde \(y\) se mide desde el eje neutro (positivo hacia arriba).

\paragraph{Fibra superior (compresión).}

\[
\sigma_{\text{top}}
= -\,\frac{M y_{\text{top}}}{I}
= -\,\frac{600(5)}{204}
= -14.71\ \text{ksi}.
\]

\paragraph{Fibra inferior (tensión).}

\[
\sigma_{\text{bot}}
= -\,\frac{M y_{\text{bot}}}{I}
= -\,\frac{600(-3)}{204}
= 8.82\ \text{ksi}.
\]

%-------------------------------------------------
\subsection*{Conclusiones}

\[
\boxed{\sigma_{\max,\ \text{compresión}} \approx -14.71\ \text{ksi}}
\]

\[
\boxed{\sigma_{\max,\ \text{tensión}} \approx 8.82\ \text{ksi}}
\]

En la porción \(BC\) de la viga, la fibra superior de la sección está sometida a
compresión y la fibra inferior a tensión.

\end{document}
