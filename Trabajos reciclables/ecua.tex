\documentclass[12pt]{article}
\usepackage[utf8]{inputenc}
\usepackage[spanish]{babel}
\usepackage{amsmath}
\usepackage{siunitx}
\usepackage{geometry}
\usepackage{fancyhdr}
\setlength{\headheight}{14.5pt}
\usepackage{graphicx}

\geometry{a4paper, margin=2.5cm}
\pagestyle{fancy}
\fancyhf{}
\rhead{Ecuaciones Diferenciales}
\lhead{Tarea 1}
\rfoot{Página \thepage}




\begin{document}



\section*{Ejercicio 1: Enfriamiento de un Líquido}

Una sustancia a \SI{90}{\degreeCelsius} se deja enfriar en una sala con temperatura constante de \SI{25}{\degreeCelsius}.  
Después de 5 minutos, su temperatura es de \SI{70}{\degreeCelsius}.

\begin{enumerate}
    \item Plantee el modelo diferencial utilizando la ley de enfriamiento de Newton.
    \item Resuelva la ecuación diferencial para encontrar la temperatura en función del tiempo.
    \item Estime la temperatura después de 10 minutos.
\end{enumerate}

\section*{1. Modelo diferencial}

Ley de enfriamiento de Newton:

\[
\frac{dT}{dt} = -k(T - T_a)
\]

Donde:
\begin{itemize}
    \item \( T(t) \): temperatura de la sustancia,
    \item \( T_a = \SI{25}{\degreeCelsius} \): temperatura ambiente,
    \item \( k \): constante de proporcionalidad.
\end{itemize}

Sustituimos:

\[
\frac{dT}{dt} = -k(T - 25)
\]

\section*{2. Resolución de la ecuación diferencial}

Separación de variables:

\[
\frac{dT}{T - 25} = -k\,dt
\]

Integramos ambos lados:

\[
\int \frac{1}{T - 25} \, dT = \int -k \, dt
\]

\[
\ln|T - 25| = -kt + C
\]

Aplicamos exponencial:

\[
|T - 25| = e^{-kt + C} = Ce^{-kt}
\]

Entonces:

\[
T(t) = 25 + Ce^{-kt}
\]

\subsection*{Condición inicial: \( T(0) = 90 \)}

\[
T(0) = 25 + Ce^{0} = 90 \Rightarrow C = 65
\]

Por lo tanto:

\[
T(t) = 25 + 65e^{-kt}
\]

\subsection*{Dato adicional: \( T(5) = 70 \)}

\[
70 = 25 + 65e^{-5k}
\Rightarrow 45 = 65e^{-5k}
\]

\[
e^{-5k} = \frac{45}{65} = \frac{9}{13}
\]

\[
-5k = \ln\left(\frac{9}{13}\right)
\Rightarrow k = -\frac{1}{5} \ln\left(\frac{9}{13}\right)
\]

\[
\ln\left(\frac{9}{13}\right) \approx -0{,}3677
\Rightarrow k \approx -\frac{1}{5}(-0{,}3677) = 0{,}0735~\si{min^{-1}}
\]

\subsection*{Ecuación final}

\[
T(t) = 25 + 65e^{-0{,}0735 t}
\]

\section*{3. Estimación a los 10 minutos}

Sustituimos \( t = 10 \):

\[
T(10) = 25 + 65e^{-0{,}0735 \cdot 10} = 25 + 65e^{-0{,}735}
\]

\[
e^{-0{,}735} \approx 0{,}4795
\Rightarrow T(10) = 25 + 65 \cdot 0{,}4795 = 25 + 31{,}1679
\]

\[
T(10) \approx \boxed{56{,}1679~\si{\degreeCelsius}}
\]

\section*{Conclusión}

Después de 10 minutos, la temperatura de la sustancia ha descendido hasta aproximadamente \textbf{56,17~\si{\degreeCelsius}}.  
Este resultado está acorde con la ley de enfriamiento de Newton, que modela la disminución exponencial de temperatura.
\section*{Ejercicio 2 – Crecimiento poblacional (10 puntos)}

Un cultivo bacteriano tiene inicialmente 500 bacterias. Después de 3 horas, se ha duplicado.

\begin{enumerate}
  \item \textbf{Plantee y resuelva la ecuación diferencial que modela el crecimiento.}

  Se asume que el crecimiento es proporcional a la cantidad actual de bacterias. Por lo tanto, el modelo es:

  \[
  \frac{dP}{dt} = kP
  \]

  Se separan variables:

  \[
  \frac{1}{P} \, dP = k \, dt
  \]

  Integramos ambos lados:

  \[
  \int \frac{1}{P} \, dP = \int k \, dt
  \Rightarrow \ln|P| = kt + C
  \Rightarrow P(t) = Ce^{kt}
  \]

  Usamos la condición inicial \( P(0) = 500 \):

  \[
  P(0) = Ce^0 = 500 \Rightarrow C = 500
  \]

  Entonces, la solución es:

  \[
  P(t) = 500e^{kt}
  \]

  Ahora usamos el dato \( P(3) = 1000 \) para encontrar \( k \):

  \[
  1000 = 500e^{3k}
  \Rightarrow 2 = e^{3k}
  \Rightarrow \ln(2) = 3k
  \Rightarrow k = \frac{\ln(2)}{3} \approx \frac{0{,}6931}{3} = 0{,}2310~\si{h^{-1}}
  \]

  Ecuación final:

  \[
  P(t) = 500e^{0{,}2310t}
  \]

  \item \textbf{¿Cuántas bacterias habrá después de 5 horas?}

  Sustituimos \( t = 5 \) en la ecuación:

  \[
  P(5) = 500e^{0{,}2310 \cdot 5} = 500e^{1{,}155}
  \]

  \[
  e^{1{,}155} \approx 3{,}17402
  \Rightarrow P(5) = 500 \cdot 3{,}17402 = 1587{,}01
  \]

  Por lo tanto:

  \[
  \boxed{P(5) \approx 1588~\text{bacterias}}
  \]

  \item \textbf{¿Cuánto tiempo pasará para que se triplique la población inicial?}

  Deseamos saber el tiempo \( t \) tal que \( P(t) = 3 \cdot 500 = 1500 \):

  \[
  1500 = 500e^{0{,}2310t}
  \Rightarrow 3 = e^{0{,}2310t}
  \Rightarrow \ln(3) = 0{,}2310t
  \Rightarrow t = \frac{\ln(3)}{0{,}2310}
  \]

  \[
  \ln(3) \approx 1{,}0986
  \Rightarrow t = \frac{1{,}0986}{0{,}2310} = \boxed{4{,}756~\text{horas}}
  \]

\end{enumerate}
\newpage
\section*{Ejercicio 3 – Mezcla de sal (10 puntos)}

Un tanque contiene \SI{100}{L} de agua pura. Se introduce una solución salina con una concentración de \( 0{,}5~\si{kg/L} \) a razón de \( 3~\si{L/min} \), mientras se retira la mezcla homogénea a la misma velocidad.

\begin{enumerate}
  \item \textbf{Plantee la ecuación diferencial que describe la cantidad de sal en el tanque en función del tiempo.}

  Sea \( S(t) \) la cantidad de sal (en kilogramos) en el tanque al tiempo \( t \) (en minutos).  
  El volumen total permanece constante: \( V = \SI{100}{L} \).

  Aplicamos el principio de entrada menos salida:

  \[
  \frac{dS}{dt} = \text{entrada} - \text{salida}
  \]

  \textbf{Entrada:}

  \[
  0{,}5~\si{kg/L} \cdot 3~\si{L/min} = 1{,}5~\si{kg/min}
  \]

  \textbf{Salida:}

  \[
  \frac{S(t)}{100} \cdot 3 = \frac{3}{100} S(t)
  \]

  Por lo tanto, la ecuación diferencial es:

  \[
  \frac{dS}{dt} = 1{,}5 - \frac{3}{100} S(t)
  \]

  \item \textbf{Determine la función que describe la cantidad de sal en el tanque.}

  Reescribimos:

  \[
  \frac{dS}{dt} + \frac{3}{100} S = 1{,}5
  \]

  Es una ecuación lineal de primer orden. Calculamos el factor integrante:

  \[
  \mu(t) = e^{\int \frac{3}{100} dt} = e^{\frac{3}{100} t}
  \]

  Multiplicamos ambos lados de la ecuación:

  \[
  e^{\frac{3}{100} t} \cdot \frac{dS}{dt} + \frac{3}{100} e^{\frac{3}{100} t} \cdot S = 1{,}5 e^{\frac{3}{100} t}
  \]

  Reconocemos la derivada del producto:

  \[
  \frac{d}{dt} \left(S \cdot e^{\frac{3}{100} t} \right) = 1{,}5 e^{\frac{3}{100} t}
  \]

  Integramos:

  \[
  S(t) \cdot e^{\frac{3}{100} t} = \int 1{,}5 e^{\frac{3}{100} t} dt = \frac{1{,}5}{\frac{3}{100}} e^{\frac{3}{100} t} + C = 50 e^{\frac{3}{100} t} + C
  \]

  \[
  S(t) = 50 + C e^{-\frac{3}{100} t}
  \]

  Condición inicial: \( S(0) = 0 \)

  \[
  0 = 50 + C \Rightarrow C = -50
  \]

  Solución final:

  \[
  S(t) = 50 \left(1 - e^{-\frac{3}{100} t} \right)
  \]

  \item \textbf{¿Cuánta sal habrá después de 10 minutos?}

  Sustituimos \( t = 10 \):

  \[
  S(10) = 50 \left(1 - e^{-0{,}3} \right)
  \]

  \[
  e^{-0{,}3} \approx 0{,}7408
  \Rightarrow S(10) = 50 \cdot (1 - 0{,}7408) = 50 \cdot 0{,}2592 = \boxed{12{,}96~\si{kg}}
  \]

\end{enumerate}
\newpage
\section*{Ejercicio 4 – Circuito eléctrico RL (Aplicación Industrial) (8 puntos)}

En un circuito RL en serie con resistencia de \( R = \SI{20}{\ohm} \) y una inductancia de \( L = \SI{0,5}{H} \), se aplica una fuente de voltaje constante de \( V = \SI{100}{V} \).  
Determine la corriente \( i(t) \) en el circuito si inicialmente no hay corriente.

\begin{enumerate}
  \item \textbf{Plantee y resuelva la ecuación diferencial que modela la corriente en el circuito.}

  Usamos la ley de voltajes de Kirchhoff:

  \[
  V = L \frac{di}{dt} + R i(t)
  \]

  Sustituimos los valores:

  \[
  100 = 0{,}5 \frac{di}{dt} + 20 i(t)
  \]

  Multiplicamos por 2 para simplificar:

  \[
  200 = \frac{di}{dt} + 40i(t)
  \]

  Reescribimos:

  \[
  \frac{di}{dt} + 40i = 200
  \]

  Esta es una ecuación diferencial lineal de primer orden.

  Calculamos el factor integrante:

  \[
  \mu(t) = e^{\int 40 dt} = e^{40t}
  \]

  Multiplicamos toda la ecuación por \( \mu(t) \):

  \[
  e^{40t} \frac{di}{dt} + 40e^{40t} i = 200e^{40t}
  \]

  Reconocemos la derivada del producto:

  \[
  \frac{d}{dt} \left( i(t) \cdot e^{40t} \right) = 200e^{40t}
  \]

  Integramos ambos lados:

  \[
  i(t) \cdot e^{40t} = \int 200e^{40t} dt = \frac{200}{40} e^{40t} + C = 5e^{40t} + C
  \]

  Despejamos \( i(t) \):

  \[
  i(t) = 5 + C e^{-40t}
  \]

  Aplicamos la condición inicial \( i(0) = 0 \):

  \[
  0 = 5 + C \Rightarrow C = -5
  \]

  Por lo tanto, la solución es:

  \[
  i(t) = 5 \left(1 - e^{-40t}\right)
  \]

  \item \textbf{Determine la corriente después de 0,1 segundos.}

  Sustituimos \( t = 0{,}1~\text{s} \):

  \[
  i(0{,}1) = 5 \left(1 - e^{-40 \cdot 0{,}1} \right) = 5 \left(1 - e^{-4} \right)
  \]

  \[
  e^{-4} \approx 0{,}0183
  \Rightarrow i(0{,}1) = 5 \cdot (1 - 0{,}0183) = 5 \cdot 0{,}9817 = \boxed{4{,}9085~\si{A}}
  \]

\end{enumerate}
\newpage
\section*{Ejercicio 5 – Movimiento de una partícula en un medio viscoso (7 puntos)}

Una partícula de masa \( m = \SI{2}{kg} \) se mueve en línea recta. Su velocidad \( v(t) \) satisface la ecuación:

\[
m \frac{dv}{dt} = -kv
\quad \text{con} \quad k = \SI{3}{kg/s}
\quad \text{y} \quad v(0) = \SI{10}{m/s}
\]

\begin{enumerate}
  \item \textbf{Resuelva la ecuación para \( v(t) \).}

  Dividimos ambos lados por \( m \):

  \[
  \frac{dv}{dt} = -\frac{k}{m} v = -\frac{3}{2}v
  \]

  Separación de variables:

  \[
  \frac{1}{v} dv = -\frac{3}{2} dt
  \]

  Integramos ambos lados:

  \[
  \int \frac{1}{v} dv = \int -\frac{3}{2} dt
  \Rightarrow \ln|v| = -\frac{3}{2}t + C
  \Rightarrow v(t) = Ce^{-\frac{3}{2}t}
  \]

  Usamos la condición inicial \( v(0) = 10 \):

  \[
  v(0) = C \Rightarrow C = 10
  \Rightarrow \boxed{v(t) = 10 e^{-\frac{3}{2}t}}
  \]

  \item \textbf{¿Cuál es la velocidad después de 2 segundos?}

  \[
  v(2) = 10 e^{-\frac{3}{2} \cdot 2} = 10 e^{-3}
  \]

  \[
  e^{-3} \approx 0{,}0498
  \Rightarrow v(2) = 10 \cdot 0{,}0498 = \boxed{0{,}498~\si{m/s}}
  \]

\end{enumerate}
\newpage
\section*{Planteamiento de problemas (20 puntos)}

A continuación, se plantean y resuelven dos problemas típicos de la Ingeniería Civil que se modelan mediante ecuaciones diferenciales ordinarias de primer orden.

\subsection*{Problema 1: Drenaje de un tanque cilíndrico}

Un tanque cilíndrico vertical con base circular contiene agua. El radio del tanque es \( r = \SI{0,5}{m} \) y se drena a través de un pequeño orificio en la base. La velocidad de salida del agua está dada por la ley de Torricelli:

\[
v = \sqrt{2gh}
\]

Donde:
- \( h(t) \) es la altura del agua en el tiempo \( t \),
- \( g = \SI{9,81}{m/s^2} \) es la aceleración debido a la gravedad.

La ecuación diferencial que modela la variación de altura en el tiempo es:

\[
\frac{dh}{dt} = -k \sqrt{h}
\]

Para este caso, suponemos que \( k = 0{,}4 \) y la altura inicial del agua es \( h(0) = \SI{1,0}{m} \).

\subsubsection*{Solución:}

\[
\frac{dh}{\sqrt{h}} = -k\,dt
\Rightarrow \int h^{-1/2} \, dh = \int -k \, dt
\]

\[
2\sqrt{h} = -kt + C
\]

Usamos la condición inicial \( h(0) = 1 \):

\[
2\sqrt{1} = -k(0) + C \Rightarrow C = 2
\]

Entonces:

\[
2\sqrt{h} = -kt + 2
\Rightarrow \sqrt{h} = 1 - \frac{k}{2}t
\Rightarrow h(t) = \left(1 - \frac{0{,}4}{2}t\right)^2 = (1 - 0{,}2t)^2
\]

\subsubsection*{Interpretación:}

La altura del agua disminuye cuadráticamente hasta vaciarse completamente en \( t = \SI{5}{min} \).

---
\newpage
\subsection*{Problema 2: Enfriamiento del concreto}

Durante el curado del concreto, es importante controlar la temperatura para evitar agrietamientos. Supongamos que una losa de concreto recién vaciada tiene una temperatura de \( \SI{70}{\degreeCelsius} \), y se encuentra en una cámara de curado a \( \SI{25}{\degreeCelsius} \). Aplicamos la ley de enfriamiento de Newton para modelar la temperatura del concreto:

\[
\frac{dT}{dt} = -k(T - T_a)
\]

Donde:
- \( T(t) \) es la temperatura del concreto,
- \( T_a = \SI{25}{\degreeCelsius} \),
- \( k = 0{,}1~\si{min^{-1}} \),
- \( T(0) = \SI{70}{\degreeCelsius} \).

\subsubsection*{Solución:}

\[
\frac{dT}{dt} = -0{,}1(T - 25)
\Rightarrow \frac{dT}{T - 25} = -0{,}1 dt
\Rightarrow \int \frac{1}{T - 25} dT = \int -0{,}1 dt
\]

\[
\ln|T - 25| = -0{,}1 t + C
\Rightarrow T(t) = 25 + Ce^{-0{,}1t}
\]

Usamos \( T(0) = 70 \Rightarrow C = 45 \)

\[
T(t) = 25 + 45e^{-0{,}1t}
\]

\subsubsection*{Respuesta:}

Esta ecuación permite estimar la temperatura del concreto a cualquier instante \( t \), lo cual es útil para definir el tiempo óptimo de curado o aplicar medidas de enfriamiento.

\end{document}
