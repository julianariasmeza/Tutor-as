\documentclass[12pt]{article}
\usepackage[utf8]{inputenc}
\usepackage[spanish]{babel}
\usepackage{amsmath, amssymb}
\usepackage{geometry}
\usepackage{fancyhdr}
\usepackage{siunitx}
\geometry{a4paper, margin=2.5cm}
\pagestyle{fancy}
\fancyhf{}
\rhead{Ecuaciones Diferenciales}
\lhead{Tarea 2}
\rfoot{Página \thepage}

\title{Tarea 2 — Ecuaciones Diferenciales}
\author{Nombre del estudiante \\ Curso: Ecuaciones Diferenciales}
\date{}

\begin{document}

\maketitle

\section*{Ejercicio 1 (5 puntos)}

Resolver la siguiente ecuación diferencial:

\[
\frac{dy}{dx} = \frac{1}{x + y^2}
\]

\subsection*{Paso 1: Identificación del tipo de ecuación}

Observamos que la ecuación no es de variables separables, ni es lineal.  
Tampoco puede clasificarse como homogénea ni exacta.  
Intentamos simplificarla mediante un cambio de variable.

\subsection*{Paso 2: Intento de sustitución \( u = y^2 \)}

\[
u = y^2 \Rightarrow \frac{du}{dx} = 2y \frac{dy}{dx}
\Rightarrow \frac{dy}{dx} = \frac{1}{2y} \frac{du}{dx}
\]

Sustituimos en la ecuación original:

\[
\frac{1}{2y} \frac{du}{dx} = \frac{1}{x + u}
\Rightarrow \frac{du}{dx} = \frac{2y}{x + u}
\]

Esta forma aún depende de \( y \), por lo que no resulta útil.  
Descartamos esta sustitución.

\subsection*{Paso 3: Otra forma — Reescribir la ecuación}

Multiplicamos ambos lados por el denominador:

\[
\left(x + y^2\right) \frac{dy}{dx} = 1
\]

Sigue siendo una ecuación implícita que mezcla variables.

\subsection*{Paso 4: Conclusión}

La ecuación:

\[
\frac{dy}{dx} = \frac{1}{x + y^2}
\]

no se puede resolver mediante funciones elementales.  
No admite solución cerrada con los métodos básicos de primer orden (separación, lineales, exactas, homogéneas o sustituciones clásicas).  
Su solución solo puede expresarse mediante funciones especiales o integrales no elementales.

\subsection*{Resultado final}

La solución general no puede escribirse en forma cerrada usando funciones elementales.  
Se concluye que esta ecuación **no es integrable mediante métodos estándar**.


\section*{Ejercicio 2 (5 puntos)}

Resolver la siguiente ecuación diferencial:

\[
\frac{dy}{dx} = \frac{xy + 2y - x - 2}{xy - 3y + x - 3}
\]

\subsection*{Paso 1: Factorización del numerador y denominador}

\textbf{Numerador:} \( xy + 2y - x - 2 \)

Agrupamos los términos:

\[
xy + 2y - x - 2 = (xy + 2y) - (x + 2)
\]

Sacamos factor común en cada grupo:

\[
= y(x + 2) - 1(x + 2)
\Rightarrow (y - 1)(x + 2)
\]

\vspace{1em}

\textbf{Denominador:} \( xy - 3y + x - 3 \)

Agrupamos:

\[
xy - 3y + x - 3 = (xy - 3y) + (x - 3)
\]

Factor común en cada grupo:

\[
= y(x - 3) + 1(x - 3)
\Rightarrow (y + 1)(x - 3)
\]

\vspace{1em}

\textbf{Sustituyendo en la ecuación:}

\[
\frac{dy}{dx} = \frac{(y - 1)(x + 2)}{(y + 1)(x - 3)}
\]


\subsection*{Paso 2: Separación de variables}

\[
\frac{dy}{dx} = \frac{y - 1}{y + 1} \cdot \frac{x + 2}{x - 3}
\Rightarrow \frac{y + 1}{y - 1} \, dy = \frac{x + 2}{x - 3} \, dx
\]

\subsection*{Paso 3: Simplificar integrando}

\[
\frac{y + 1}{y - 1} = 1 + \frac{2}{y - 1}
\Rightarrow \int \left(1 + \frac{2}{y - 1} \right) dy = y + 2 \ln|y - 1|
\]

\[
\frac{x + 2}{x - 3} = 1 + \frac{5}{x - 3}
\Rightarrow \int \left(1 + \frac{5}{x - 3} \right) dx = x + 5 \ln|x - 3|
\]

\subsection*{Paso 4: Solución general}

\[
\boxed{
y + 2 \ln|y - 1| = x + 5 \ln|x - 3| + C
}
\]

\section*{Ejercicio 3}

Resolver la ecuación:

\[
2x^3 y\,dx + (x^4 + y^4)\,dy = 0
\]

\subsection*{Paso 1: Reorganizar y simplificar}

Partimos de la ecuación:

\[
2x^3 y\,dx + (x^4 + y^4)\,dy = 0
\]

Restamos el primer término a ambos lados:

\[
(x^4 + y^4)\,dy = -2x^3 y\,dx
\]

Dividimos ambos lados entre \( dx \) para obtener la ecuación en forma explícita:

\[
\frac{dy}{dx} = \frac{-2x^3 y}{x^4 + y^4}
\]

Ahora analizamos la estructura del lado derecho. Observamos que tanto el numerador como el denominador son expresiones **homogéneas de grado 4**, ya que:

- En el numerador: \( -2x^3 y \) tiene grado \( 3 + 1 = 4 \)
- En el denominador: \( x^4 + y^4 \) tiene grado máximo \( 4 \)

Esto nos permite aplicar un cambio de variable apropiado para ecuaciones homogéneas. Para confirmar, expresamos ambos términos en función de \( \frac{y}{x} \):

Factorizamos \( x^4 \) del denominador:

\[
\frac{dy}{dx} = \frac{-2x^3 y}{x^4 (1 + \left(\frac{y}{x}\right)^4)} = \frac{-2y}{x \left(1 + \left(\frac{y}{x}\right)^4\right)}
\]

Esta forma confirma que es una ecuación homogénea, y está lista para aplicar el cambio de variable:

\[
v = \frac{y}{x}
\]


\subsection*{Paso 2: Homogeneidad}

Reescribimos como:

\[
\frac{dy}{dx} = -\frac{2y}{x} \cdot \frac{1}{1 + \left( \frac{y}{x} \right)^4}
\]

La ecuación es homogénea, entonces hacemos el cambio:

\[
v = \frac{y}{x} \Rightarrow y = vx \Rightarrow \frac{dy}{dx} = v + x \frac{dv}{dx}
\]

Sustituimos en la ecuación:

\[
v + x \frac{dv}{dx} = -\frac{2v}{1 + v^4}
\]

\subsection*{Paso 3: Resolver para \( \frac{dv}{dx} \)}

\[
x \frac{dv}{dx} = -\frac{2v}{1 + v^4} - v = -v \left( \frac{2 + (1 + v^4)}{1 + v^4} \right)
= -v \left( \frac{3 + v^4}{1 + v^4} \right)
\]

\[
\frac{dv}{dx} = -\frac{v (3 + v^4)}{x (1 + v^4)}
\]

\subsection*{Paso 4: Separación de variables}

\[
\frac{1 + v^4}{v (3 + v^4)}\,dv = -\frac{dx}{x}
\]

\subsection*{Paso 5: Integración}

\[
\int \frac{1 + v^4}{v (3 + v^4)}\,dv = \int -\frac{1}{x} \, dx
\]

\subsection*{ Fracciones parciales}

Observamos que:

\[
\frac{1 + v^4}{v (3 + v^4)} = \frac{1}{v} \cdot \frac{1 + v^4}{3 + v^4}
\]

Planteamos:

\[
\frac{1 + v^4}{v (3 + v^4)} = \frac{A}{v} + \frac{Bv^3 + Cv^2 + Dv + E}{3 + v^4}
\]

Multiplicamos por \( v(3 + v^4) \):

\[
1 + v^4 = A(3 + v^4) + v(Bv^3 + Cv^2 + Dv + E)
\]
\newpage
Desarrollamos:

\[
1 + v^4 = 3A + Av^4 + Bv^4 + Cv^3 + Dv^2 + Ev
\]

Agrupamos por potencias de \( v \):

\[
1 + v^4 = 3A + (A + B)v^4 + Cv^3 + Dv^2 + Ev
\]

Igualamos coeficientes:

\[
\begin{cases}
A + B = 1 \\
C = 0 \\
D = 0 \\
E = 0 \\
3A = 1 \Rightarrow A = \frac{1}{3}
\Rightarrow B = \frac{2}{3}
\end{cases}
\]

\subsection*{Paso 5: Integral}

\[
\int \frac{1 + v^4}{v(3 + v^4)}\,dv = \int \left( \frac{1}{3v} + \frac{2v^3}{3(3 + v^4)} \right) dv
\]

La segunda integral se resuelve con el cambio \( u = 3 + v^4 \Rightarrow du = 4v^3\,dv \)

\[
\int \frac{2v^3}{3(3 + v^4)}\,dv = \frac{1}{6} \int \frac{4v^3}{3 + v^4} dv = \frac{1}{6} \ln|3 + v^4|
\]

Entonces:

\[
\int \frac{1 + v^4}{v (3 + v^4)}\,dv = \frac{1}{3} \ln |v| + \frac{1}{6} \ln |3 + v^4| + C
\]

\subsection*{Paso 6: Sustituir \( v = \frac{y}{x} \)}

\[
\frac{1}{3} \ln \left| \frac{y}{x} \right| + \frac{1}{6} \ln \left| 3 + \left( \frac{y}{x} \right)^4 \right| = \ln|x| + C
\]

\subsection*{Paso 7: Solución general}

\[
\boxed{
\frac{1}{3} \ln \left| \frac{y}{x} \right| + \frac{1}{6} \ln \left| \frac{y^4}{x^4} + 3 \right| = \ln |x| + C
}
\]



\newpage
\section*{Ejercicio 4}

Resolver la ecuación diferencial:

\[
y\,dx + (xy + 2x - y e^y)\,dy = 0
\]

\subsection*{Paso 1: Verificar si es exacta}

Sean:
\[
M(x,y) = y, \quad N(x,y) = xy + 2x - y e^y
\]

Derivamos:
\[
\frac{\partial M}{\partial y} = 1, \qquad \frac{\partial N}{\partial x} = y + 2
\Rightarrow \frac{\partial M}{\partial y} \ne \frac{\partial N}{\partial x}
\]

Por tanto, no es exacta.

\subsection*{Paso 2: Calcular factor integrante \( \mu(y) \)}

Buscamos \( \mu(y) \) tal que:
\[
g(y) = \frac{N_x - M_y}{M} = \frac{y + 2 - 1}{y} = \frac{y + 1}{y} = 1 + \frac{1}{y}
\]

Entonces:
\[
\mu(y) = \exp\left(\int \left(1 + \frac{1}{y} \right) dy \right) = \exp(y + \ln|y|) = y e^y
\]

\subsection*{Paso 3: Multiplicar por el factor integrante}

Multiplicamos toda la ecuación por \( \mu(y) = y e^y \):

\[
y^2 e^y\,dx + (x y^2 e^y + 2 x y e^y - y^2 e^{2y})\,dy = 0
\]

\[
M(x,y) = y^2 e^y, \quad N(x,y) = x y^2 e^y + 2 x y e^y - y^2 e^{2y}
\]

Calculamos derivadas:

\[
\frac{\partial M}{\partial y} = 2 y e^y + y^2 e^y = e^y (2 y + y^2)
\]
\[
\frac{\partial N}{\partial x} = y^2 e^y + 2 y e^y = e^y (2 y + y^2)
\]

\[
\Rightarrow \frac{\partial M}{\partial y} = \frac{\partial N}{\partial x} \Rightarrow \text{¡Ahora sí es exacta!}
\]

\subsection*{Paso 4: Hallar la función potencial \( U(x,y) \)}

Sabemos que:
\[
\frac{\partial U}{\partial x} = M = y^2 e^y
\Rightarrow U(x,y) = \int y^2 e^y\,dx = x y^2 e^y + f(y)
\]

Derivamos con respecto a \( y \):

\[
\frac{\partial U}{\partial y} = 2 x y e^y + x y^2 e^y + f'(y)
\]

Queremos que:
\[
\frac{\partial U}{\partial y} = N = x y^2 e^y + 2 x y e^y - y^2 e^{2y}
\]

Comparando:
\[
2 x y e^y + x y^2 e^y + f'(y) = x y^2 e^y + 2 x y e^y - y^2 e^{2y}
\Rightarrow f'(y) = - y^2 e^{2y}
\]

Entonces:
\[
f(y) = \int - y^2 e^{2y}\,dy
\]


\subsection*{ Identificación para integración por partes}

Aplicamos la fórmula de integración por partes:

\[
\int u \, dv = uv - \int v \, du
\]

Sea:
\[
u = y^2 \quad \Rightarrow \quad du = 2y\,dy
\]
\[
dv = e^{2y} dy \quad \Rightarrow \quad v = \frac{1}{2} e^{2y}
\]

\subsection*{Paso 2: Primera aplicación}

\[
\int y^2 e^{2y} \, dy = y^2 \cdot \frac{1}{2} e^{2y} - \int \frac{1}{2} \cdot 2y e^{2y} \, dy
\]
\[
= \frac{1}{2} y^2 e^{2y} - \int y e^{2y} \, dy
\]

\subsection*{Paso 3: Segunda integración por partes}

Ahora integramos:
\[
\int y e^{2y} \, dy
\]

Sea:
\[
u = y \quad \Rightarrow \quad du = dy
\]
\[
dv = e^{2y} dy \quad \Rightarrow \quad v = \frac{1}{2} e^{2y}
\]

\[
\int y e^{2y} \, dy = y \cdot \frac{1}{2} e^{2y} - \int \frac{1}{2} e^{2y} dy
= \frac{1}{2} y e^{2y} - \frac{1}{4} e^{2y}
\]

\subsection*{Paso 4: Sustituimos en el resultado anterior}

\[
\int y^2 e^{2y} \, dy = \frac{1}{2} y^2 e^{2y} - \left( \frac{1}{2} y e^{2y} - \frac{1}{4} e^{2y} \right)
\]
\[
= \frac{1}{2} y^2 e^{2y} - \frac{1}{2} y e^{2y} + \frac{1}{4} e^{2y}
\]

\subsection*{Paso 5: Consideramos el signo y la constante de integración}

Como:
\[
f(y) = \int -y^2 e^{2y} dy
\]

Entonces:

\[
f(y) = -\left( \frac{1}{2} y^2 e^{2y} - \frac{1}{2} y e^{2y} + \frac{1}{4} e^{2y} \right) + C
\]

\[
\boxed{
f(y) = -\frac{1}{2} e^{2y} y^2 + \frac{1}{2} e^{2y} y - \frac{1}{4} e^{2y} + C
}
\]
\subsection*{Paso 5: Solución general}

\[
U(x,y) = x y^2 e^y - \frac{1}{2} e^{2y} y^2 + \frac{1}{2} e^{2y} y - \frac{1}{4} e^{2y}
\]

Por tanto, la solución general es:

\[
\boxed{
x y^2 e^y - \frac{1}{2} e^{2y} y^2 + \frac{1}{2} e^{2y} y - \frac{1}{4} e^{2y} = C
}
\]



\newpage
\section*{Ejercicio 5}

\[
\frac{dy}{dx} = a(b^2 - y^2)
\]

\subsection*{Paso 1: Separación de variables}

\[
\frac{1}{b^2 - y^2} \, dy = a \, dx
\]

Recordamos que:
\[
\frac{1}{b^2 - y^2} = \frac{1}{2b} \left( \frac{1}{b - y} + \frac{1}{b + y} \right)
\]

Entonces:

\[
\int \frac{1}{b^2 - y^2} \, dy = \int a \, dx
\Rightarrow
\int \frac{1}{2b} \left( \frac{1}{b - y} + \frac{1}{b + y} \right) dy = \int a \, dx
\]

\subsection*{Paso 2: Integración}

\[
\frac{1}{2b} \int \left( \frac{1}{b - y} + \frac{1}{b + y} \right) dy = ax + C
\]

Sustituimos:
\[
\int \frac{1}{b - y} \, dy = -\ln |b - y|, \quad
\int \frac{1}{b + y} \, dy = \ln |b + y|
\]

Entonces:

\[
\frac{1}{2b} \left[ -\ln |b - y| + \ln |b + y| \right] = ax + C
\]

\subsection*{Paso 3: Simplificación}

\[
\frac{1}{2b} \ln \left| \frac{b + y}{b - y} \right| = ax + C
\]

Multiplicamos ambos lados por \( 2b \):

\[
\ln \left| \frac{b + y}{b - y} \right| = 2abx + C_1
\quad \text{(donde } C_1 = 2bC \text{)}
\]

\subsection*{Paso 4: Solución general}

Elevamos ambos lados como potencias de \( e \):

\[
\left| \frac{b + y}{b - y} \right| = e^{2abx + C_1} = C_2 e^{2abx}
\]

donde \( C_2 = e^{C_1} \) es una constante arbitraria positiva.

\[
\frac{b + y}{b - y} = C_2 e^{2abx}
\]

\subsection*{Paso 5: Despeje de \( y \)}

Multiplicamos en cruz:

\[
b + y = C_2 e^{2abx} (b - y)
\]

\[
b + y = b C_2 e^{2abx} - y C_2 e^{2abx}
\]

Llevamos los términos con \( y \) a un lado:

\[
y + y C_2 e^{2abx} = b C_2 e^{2abx} - b
\]

\[
y (1 + C_2 e^{2abx}) = b (C_2 e^{2abx} - 1)
\]

\subsection*{Solución explícita}

\[
\boxed{
y(x) = b \cdot \frac{C_2 e^{2abx} - 1}{C_2 e^{2abx} + 1}
}
\]
\newpage
\section*{Problema}
Resolver la siguiente ecuación diferencial con condición inicial:

\[
xy(1 + x y^2) \frac{dy}{dx} = 1, \quad y(1) = 0
\]

\section*{Paso 1: Aislar la derivada}
\[
\frac{dy}{dx} = \frac{1}{xy(1 + x y^2)}
\]

Invertimos la derivada para obtener una forma más integrable:
\[
\frac{dx}{dy} = xy(1 + x y^2)
\]

\section*{Paso 2: Sustitución}
Sea \( z = \frac{1}{x} \Rightarrow x = \frac{1}{z} \)

Entonces:
\[
\frac{dx}{dy} = -\frac{1}{z^2} \cdot \frac{dz}{dy}
\]

Sustituyendo en la ecuación:
\[
- \frac{1}{z^2} \cdot \frac{dz}{dy} = \frac{1}{z} y \left( 1 + \frac{1}{z} y^2 \right)
\]

Multiplicamos ambos lados por \( z^2 \):
\[
- \frac{dz}{dy} = z y + y^3
\]

\section*{Paso 3: Forma estándar de ecuación lineal}
\[
\frac{dz}{dy} + y z = - y^3
\]

Esta es una ecuación lineal de primer orden. Usamos un factor integrante:

\[
\mu(y) = e^{\int y\, dy} = e^{\frac{y^2}{2}}
\]

Multiplicamos toda la ecuación por \( \mu(y) \):

\[
e^{\frac{y^2}{2}} \frac{dz}{dy} + y e^{\frac{y^2}{2}} z = - y^3 e^{\frac{y^2}{2}}
\]

Esto equivale a:
\[
\frac{d}{dy} \left( e^{\frac{y^2}{2}} z \right) = - y^3 e^{\frac{y^2}{2}}
\]

\section*{Paso 4: Integración}

\[
e^{\frac{y^2}{2}} z = \int - y^3 e^{\frac{y^2}{2}} \, dy
\]
\subsection*{Desarrollo de la integral}

Queremos resolver:

\[
\int - y^3 e^{\frac{y^2}{2}} \, dy
\]

\noindent Hacemos el cambio de variable:

\[
u = \frac{y^2}{2} \quad \Rightarrow \quad du = y \, dy \quad \Rightarrow \quad y^3 \, dy = y^2 (y \, dy) = 2u \, du
\]

\noindent Entonces la integral se convierte en:

\[
\int - y^3 e^{\frac{y^2}{2}} \, dy = \int -2u e^u \, du
\]

\noindent Usamos integración por partes:

\[
\int u e^u \, du = u e^u - \int e^u \, du = u e^u - e^u + C
\]

\[
\Rightarrow \int -2u e^u \, du = -2 (u e^u - e^u) + C = -2u e^u + 2e^u + C
\]

\noindent Volviendo a la variable \( y \):

\[
u = \frac{y^2}{2} \Rightarrow
\int - y^3 e^{\frac{y^2}{2}} \, dy = -y^2 e^{\frac{y^2}{2}} + 2 e^{\frac{y^2}{2}} + C
\]

\[
\boxed{
\int - y^3 e^{\frac{y^2}{2}} \, dy = -y^2 e^{\frac{y^2}{2}} + 2 e^{\frac{y^2}{2}} + C
}
\]

La integral resuelta da:
\[
e^{\frac{y^2}{2}} z = - e^{\frac{y^2}{2}} y^2 + 2 e^{\frac{y^2}{2}} + C
\]

Dividiendo entre \( e^{\frac{y^2}{2}} \):

\[
z = - y^2 + 2 + C e^{- \frac{y^2}{2}}
\]

Recordando que \( z = \frac{1}{x} \), entonces:

\[
\frac{1}{x} = - y^2 + 2 + C e^{- \frac{y^2}{2}}
\quad \Rightarrow \quad
x = \frac{1}{- y^2 + 2 + C e^{- \frac{y^2}{2}}}
\]

\section*{Paso 5: Aplicar condición inicial}

Usamos la condición \( y(1) = 0 \Rightarrow x = 1 \)

\[
\frac{1}{1} = -0 + 2 + C \cdot e^{0}
\quad \Rightarrow \quad 1 = 2 + C \Rightarrow C = -1
\]

\section*{Solución final}

\[
x = \frac{1}{- y^2 + 2 - e^{- \frac{y^2}{2}}}
\]




\newpage

\section*{Ejercicio}

Resolver la ecuación diferencial con condición inicial:

\[
x \frac{dy}{dx} + y = e^x, \qquad y(1) = 2
\]

\section*{Paso 1: Forma estándar}

Dividimos toda la ecuación entre \( x \) (asumiendo \( x \neq 0 \)):

\[
\frac{dy}{dx} + \frac{1}{x} y = \frac{e^x}{x}
\]

Identificamos:

\[
P(x) = \frac{1}{x}, \qquad Q(x) = \frac{e^x}{x}
\]

\section*{Paso 2: Factor integrante}

Calculamos el factor integrante:

\[
\mu(x) = e^{\int \frac{1}{x} \, dx} = e^{\ln |x|} = |x|
\]

Como trabajamos en \( x > 0 \), entonces:

\[
\mu(x) = x
\]

\section*{Paso 3: Multiplicamos por el factor integrante}

\[
x \cdot \frac{dy}{dx} + y = e^x
\Rightarrow
\frac{d}{dx} (x y) = e^x
\]

\section*{Paso 4: Integración}

Integramos ambos lados:

\[
\int \frac{d}{dx} (x y) \, dx = \int e^x \, dx
\Rightarrow
x y = e^x + C
\]

Despejamos:

\[
y = \frac{e^x + C}{x}
\]

\section*{Paso 5: Condición inicial}

Usamos \( y(1) = 2 \):

\[
2 = \frac{e^1 + C}{1}
\Rightarrow
2 = e + C
\Rightarrow
C = 2 - e
\]

\section*{Solución final}

\[
\boxed{
y(x) = \frac{e^x + 2 - e}{x}
}
\]



\end{document}
