\documentclass[12pt]{article}
\usepackage[utf8]{inputenc}
\usepackage[spanish]{babel}
\usepackage{amsmath}
\usepackage{geometry}
\geometry{a4paper, margin=2.5cm}

\title{Solución de Ecuación de Bernoulli}
\date{}
\begin{document}

\maketitle

\section*{Problema}

Determinar la solución general de la ecuación diferencial no lineal mediante el método de Bernoulli:

\[
\frac{dy}{dx} + 0{,}6y = 0{,}15y^2
\]

\section*{ Forma estándar de Bernoulli}

La ecuación tiene la forma general de Bernoulli:

\[
\frac{dy}{dx} + P(x)y = Q(x)y^n
\]

En este caso:

\begin{itemize}
    \item \( P(x) = 0{,}6 \)
    \item \( Q(x) = 0{,}15 \)
    \item \( n = 2 \)
\end{itemize}

\section*{ Sustitución}

Aplicamos el cambio de variable:

\[
v = y^{1-n} = y^{-1} \Rightarrow y = \frac{1}{v}
\]

Derivamos:

\[
\frac{dy}{dx} = -\frac{1}{v^2} \cdot \frac{dv}{dx}
\]

\section*{Sustituimos en la ecuación original}

\[
-\frac{1}{v^2} \cdot \frac{dv}{dx} + 0{,}6 \cdot \frac{1}{v} = 0{,}15 \cdot \frac{1}{v^2}
\]

Multiplicamos por \( v^2 \) para eliminar denominadores:

\[
- \frac{dv}{dx} + 0{,}6v = 0{,}15
\]

\section*{ Ecuación lineal en \( v \)}

Reordenamos:

\[
\frac{dv}{dx} - 0{,}6v = -0{,}15
\]

Esta es una ecuación lineal de primer orden.

\section*{ Factor integrante}

Calculamos el factor integrante:

\[
\mu(x) = e^{\int -0{,}6\,dx} = e^{-0{,}6x}
\]

Multiplicamos toda la ecuación por \( \mu(x) \):

\[
e^{-0{,}6x} \cdot \frac{dv}{dx} - 0{,}6e^{-0{,}6x} v = -0{,}15e^{-0{,}6x}
\]

La parte izquierda es la derivada del producto:

\[
\frac{d}{dx} \left( v \cdot e^{-0{,}6x} \right) = -0{,}15e^{-0{,}6x}
\]

\section*{ Integramos ambos lados}

\[
\int \frac{d}{dx} \left( v \cdot e^{-0{,}6x} \right) dx = \int -0{,}15e^{-0{,}6x} dx
\]

\[
v \cdot e^{-0{,}6x} = \frac{-0{,}15}{-0{,}6} e^{-0{,}6x} + C = 0{,}25 e^{-0{,}6x} + C
\]

\section*{ Despejamos \( v \) y luego \( y \)}

\[
v = 0{,}25 + Ce^{0{,}6x}
\]

Recordando que \( y = \frac{1}{v} \), obtenemos la solución general:

\[
\boxed{y(x) = \frac{1}{0{,}25 + Ce^{0{,}6x}}}
\]



\newpage
\section*{Problema de Mezcla - Solución Paso a Paso}

Un tanque contiene \(150~\text{L}\) de agua pura. Se bombea agua salada con una concentración de \(3~\text{g/L}\) a razón de \(6~\text{L/min}\), y se extrae al mismo ritmo. ¿Cuánta sal habrá después de 15 minutos?

\section*{ Definición de variables}

Sea \(A(t)\) la cantidad de sal (en gramos) en el tanque en el tiempo \(t\), en minutos.

\begin{itemize}
  \item Volumen del tanque: constante, \(150~\text{L}\).
  \item Entrada: \(6~\text{L/min} \times 3~\text{g/L} = 18~\text{g/min}\).
  \item Salida: concentración instantánea: \( \dfrac{A(t)}{150}~\text{g/L} \), multiplicado por \(6~\text{L/min}\): \( \dfrac{6A(t)}{150} = \dfrac{A(t)}{25}~\text{g/min} \).
\end{itemize}

\section*{ Ecuación diferencial}

\[
\frac{dA}{dt} = \text{entrada} - \text{salida} = 18 - \frac{A}{25}
\]



Reescribimos:

\[
\frac{dA}{dt} + \frac{1}{25}A = 18
\]

Factor integrante:

\[
\mu(t) = e^{\int \frac{1}{25} dt} = e^{t/25}
\]

Multiplicamos ambos lados por \( \mu(t) \):

\[
e^{t/25} \cdot \frac{dA}{dt} + \frac{1}{25}e^{t/25} A = 18e^{t/25}
\]

Reconocemos el lado izquierdo como derivada de un producto:

\[
\frac{d}{dt} \left( A \cdot e^{t/25} \right) = 18e^{t/25}
\]



\[
\int \frac{d}{dt} \left( A \cdot e^{t/25} \right) dt = \int 18e^{t/25} dt
\]

\[
A \cdot e^{t/25} = 18 \cdot 25 \cdot e^{t/25} + C = 450e^{t/25} + C
\]

Despejamos:

\[
A(t) = 450 + Ce^{-t/25}
\]

\section*{ Condición inicial}

Dado que inicialmente el tanque contiene agua pura:

\[
A(0) = 0 = 450 + C \Rightarrow C = -450
\]

Entonces, la solución particular es:

\[
A(t) = 450 - 450e^{-t/25}
\]

\section*{ Evaluación en \(t = 15\) minutos}

\[
A(15) = 450 \left(1 - e^{-15/25} \right) = 450 \left(1 - e^{-0{,}6} \right)
\]

Aproximando:

\[
e^{-0{,}6} \approx 0{,}5488 \Rightarrow A(15) \approx 450 \cdot (1 - 0{,}5488) = 450 \cdot 0{,}4512 \approx \boxed{203{,}03~\text{g}}
\]





\newpage
\section*{Cálculo del Wronskiano y Verificación de Independencia Lineal}

Sean las funciones:

\begin{align*}
    y_1(x) &= -3x \\
    y_2(x) &= \sen(2x) \\
    y_3(x) &= \cos(2x)
\end{align*}

Calcular el Wronskiano \( W(x) \) y determinar si las funciones son linealmente independientes.

\section*{ Derivadas de cada función}

\begin{align*}
    y_1(x) &= -3x & \Rightarrow & \quad y_1'(x) = -3 & \Rightarrow & \quad y_1''(x) = 0 \\
    y_2(x) &= \sen(2x) & \Rightarrow & \quad y_2'(x) = 2\cos(2x) & \Rightarrow & \quad y_2''(x) = -4\sen(2x) \\
    y_3(x) &= \cos(2x) & \Rightarrow & \quad y_3'(x) = -2\sen(2x) & \Rightarrow & \quad y_3''(x) = -4\cos(2x)
\end{align*}

\section*{ Construcción del Wronskiano}

\[
W(x) =
\begin{vmatrix}
y_1(x) & y_2(x) & y_3(x) \\
y_1'(x) & y_2'(x) & y_3'(x) \\
y_1''(x) & y_2''(x) & y_3''(x)
\end{vmatrix}
=
\begin{vmatrix}
-3x & \sen(2x) & \cos(2x) \\
-3 & 2\cos(2x) & -2\sen(2x) \\
0 & -4\sen(2x) & -4\cos(2x)
\end{vmatrix}
\]

\section*{ Cálculo del determinante}



\begin{align*}
W(x) &= -3x \cdot 
\begin{vmatrix}
2\cos(2x) & -2\sen(2x) \\
-4\sen(2x) & -4\cos(2x)
\end{vmatrix}
- \sen(2x) \cdot
\begin{vmatrix}
-3 & -2\sen(2x) \\
0 & -4\cos(2x)
\end{vmatrix} \\
&\quad + \cos(2x) \cdot 
\begin{vmatrix}
-3 & 2\cos(2x) \\
0 & -4\sen(2x)
\end{vmatrix}
\end{align*}



\subsection*{Matriz 1}
\begin{align*}
\begin{vmatrix}
2\cos(2x) & -2\sen(2x) \\
-4\sen(2x) & -4\cos(2x)
\end{vmatrix}
&= (2\cos(2x))(-4\cos(2x)) - (-2\sen(2x))(-4\sen(2x)) \\
&= -8\cos^2(2x) - 8\sen^2(2x) \\
&= -8(\cos^2(2x) + \sen^2(2x)) = -8
\end{align*}

\subsection*{Matriz 2}
\begin{align*}
\begin{vmatrix}
-3 & -2\sen(2x) \\
0 & -4\cos(2x)
\end{vmatrix}
= (-3)(-4\cos(2x)) = 12\cos(2x)
\end{align*}

\subsection*{Matriz 3}
\begin{align*}
\begin{vmatrix}
-3 & 2\cos(2x) \\
0 & -4\sen(2x)
\end{vmatrix}
= (-3)(-4\sen(2x)) = 12\sen(2x)
\end{align*}



\begin{align*}
W(x) &= -3x(-8) - \sen(2x)(12\cos(2x)) + \cos(2x)(12\sen(2x)) \\
&= 24x - 12\sen(2x)\cos(2x) + 12\sen(2x)\cos(2x) \\
&= \boxed{24x}
\end{align*}

\section*{Conclusión}

Como \( W(x) = 24x \neq 0 \) para \( x \neq 0 \), entonces las funciones \( y_1, y_2, y_3 \) son linealmente independientes siempre y cuando \( x = 0 \).






\newpage
\section*{Problema a)}

Resolver la ecuación diferencial:

\[
y'' + 8y' + 25y = 0
\]

\section*{ Ecuación característica}

Asumimos una solución de la forma \( y = e^{rx} \), y sustituimos:

\[
r^2 + 8r + 25 = 0
\]

\section*{Paso 2: Solución de la ecuación cuadrática}

\begin{align*}
r &= \frac{-8 \pm \sqrt{8^2 - 4 \cdot 1 \cdot 25}}{2 \cdot 1} \\
  &= \frac{-8 \pm \sqrt{64 - 100}}{2} \\
  &= \frac{-8 \pm \sqrt{-36}}{2} \\
  &= \frac{-8 \pm 6i}{2} \\
  &= -4 \pm 3i
\end{align*}

\section*{Paso 3: Solución general}

Como las raíces son complejas conjugadas, la solución general es:

\[
\boxed{
y(x) = e^{-4x} \left( C_1 \cos(3x) + C_2 \sin(3x) \right)
}
\]

\newpage
\section*{Problema b)}

Resolver la ecuación diferencial de tercer orden:

\[
y''' + 3y'' - 10y' = 0
\]

\section*{Paso 1: Ecuación característica}

Asumimos \( y = e^{rx} \) y sustituimos:

\[
r^3 + 3r^2 - 10r = 0
\]

Sacamos factor común:

\[
r(r^2 + 3r - 10) = 0
\]

\section*{Paso 2: Solución de la ecuación cuadrática}

\begin{align*}
r &= \frac{-3 \pm \sqrt{3^2 - 4(1)(-10)}}{2(1)} \\
  &= \frac{-3 \pm \sqrt{9 + 40}}{2} \\
  &= \frac{-3 \pm \sqrt{49}}{2} \\
  &= \frac{-3 \pm 7}{2}
\end{align*}

Entonces, las raíces son:

\[
r_1 = 0, \quad r_2 = 2, \quad r_3 = -5
\]

\section*{Paso 3: Solución general}

Como las raíces son reales y distintas, la solución general es:

\[
\boxed{
y(x) = C_1 + C_2 e^{2x} + C_3 e^{-5x}
}
\]

\newpage
\section*{Problema a}

Resolver la ecuación no homogénea:

\[
y'' + 3y' + 4y = 3x + 2
\]

\section*{Paso 1: Solución de la homogénea asociada}

Ecuación homogénea:

\[
y'' + 3y' + 4y = 0
\]

Ecuación característica:

\[
r^2 + 3r + 4 = 0
\]

\[
r = \frac{-3 \pm \sqrt{9 - 16}}{2} = \frac{-3 \pm \sqrt{-7}}{2} = -\frac{3}{2} \pm \frac{\sqrt{7}}{2}i
\]

Solución homogénea:

\[
y_h(x) = e^{-\frac{3}{2}x} \left( C_1 \cos\left( \frac{\sqrt{7}}{2}x \right) + C_2 \sin\left( \frac{\sqrt{7}}{2}x \right) \right)
\]

\section*{Paso 2: Propuesta para la solución particular \( y_p(x) \)}

Proponemos:

\[
y_p(x) = Ax + B
\]

\[
y_p' = A, \quad y_p'' = 0
\]

Sustituimos:

\[
0 + 3A + 4(Ax + B) = 3x + 2
\Rightarrow 4A x + (3A + 4B) = 3x + 2
\]

Igualamos coeficientes:

\[
4A = 3 \Rightarrow A = \frac{3}{4}
\]

\[
3A + 4B = 2 \Rightarrow 3 \cdot \frac{3}{4} + 4B = 2 \Rightarrow \frac{9}{4} + 4B = 2
\Rightarrow 4B = -\frac{1}{4} \Rightarrow B = -\frac{1}{16}
\]

\section*{Paso 3: Solución general}

\[
y(x) = y_h(x) + y_p(x)
\]

\[
\boxed{
y(x) = e^{-\frac{3}{2}x} \left( C_1 \cos\left( \frac{\sqrt{7}}{2}x \right) + C_2 \sin\left( \frac{\sqrt{7}}{2}x \right) \right)
+ \frac{3}{4}x - \frac{1}{16}
}
\]


\newpage

\section*{Problema b)}

Resolver la ecuación diferencial:

\[
y'' - 4y = 3e^{4x}
\]

\section*{Paso 1: Resolver la homogénea asociada}

La ecuación homogénea es:

\[
y'' - 4y = 0
\]

Ecuación característica:

\[
r^2 - 4 = 0 \Rightarrow r = \pm 2
\]

Solución homogénea:

\[
y_h(x) = C_1 e^{2x} + C_2 e^{-2x}
\]

\section*{Paso 2: Propuesta para la solución particular}

Proponemos:

\[
y_p(x) = Ae^{4x}
\]

\[
y_p' = 4Ae^{4x}, \quad y_p'' = 16Ae^{4x}
\]

Sustituimos en la ecuación:

\[
y_p'' - 4y_p = 16Ae^{4x} - 4Ae^{4x} = 12Ae^{4x}
\]

Igualamos:

\[
12Ae^{4x} = 3e^{4x} \Rightarrow A = \frac{3}{12} = \frac{1}{4}
\]

\section*{Paso 3: Solución general}

\[
\boxed{
y(x) = C_1 e^{2x} + C_2 e^{-2x} + \frac{1}{4} e^{4x}
}
\]


\end{document}

