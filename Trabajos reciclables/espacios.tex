\documentclass[12pt,a4paper]{article}
\usepackage[spanish]{babel}
\usepackage[utf8]{inputenc}
\usepackage[T1]{fontenc}
\usepackage{amsmath, amssymb}
\usepackage{geometry}
\usepackage{xcolor}
\usepackage{tikz}
\usepackage{multicol}
\geometry{margin=2cm}

\title{\textbf{Propiedades de los Espacios Vectoriales y Combinaciones Lineales}}
\author{Curso de Álgebra Lineal}
\date{}

\begin{document}
\maketitle
\tableofcontents
\newpage

%-----------------------------------------------
\section{Definición de espacio vectorial}
Un conjunto $V$ es un \textbf{espacio vectorial} sobre el cuerpo $\mathbb{R}$ si cumple:
\begin{itemize}
    \item Existe una \textbf{suma de vectores} $(\vec{u},\vec{v}) \mapsto \vec{u} + \vec{v}$.
    \item Existe un \textbf{producto por un escalar} $(k, \vec{u}) \mapsto k \vec{u}$.
\end{itemize}

\bigskip
Debe cumplir las propiedades de cerradura, conmutatividad, asociatividad, existencia de elemento neutro e inverso.

%-----------------------------------------------
\section{Propiedades de la suma de vectores}

Sean $\vec{u},\vec{v},\vec{w} \in V$:

\begin{enumerate}
    \item \textbf{Cerradura:} $\vec{u} + \vec{v} \in V$.
    \item \textbf{Conmutativa:} $\vec{u} + \vec{v} = \vec{v} + \vec{u}$.
    \item \textbf{Asociativa:} $(\vec{u} + \vec{v}) + \vec{w} = \vec{u} + (\vec{v} + \vec{w})$.
    \item \textbf{Neutro:} existe $\vec{0}$ tal que $\vec{u} + \vec{0} = \vec{u}$.
    \item \textbf{Inverso:} existe $-\vec{u}$ tal que $\vec{u} + (-\vec{u}) = \vec{0}$.
\end{enumerate}

\textbf{Ejemplo:} 
\[
(2,-3) + (-2,3) = (0,0)
\]



%-----------------------------------------------
\section{Propiedades del producto por un escalar}

Sean $\vec{u} \in V$ y $k,r \in \mathbb{R}$:

\begin{enumerate}
    \item \textbf{Cerradura:} $k\vec{u} \in V$.
    \item \textbf{Asociativa:} $(r k)\vec{u} = r (k\vec{u})$.
    \item \textbf{Neutro:} $1 \cdot \vec{u} = \vec{u}$.
    \item \textbf{Inverso:} $(-1) \vec{u} = -\vec{u}$.
    \item \textbf{Conmutativa:} $k \vec{u} = \vec{u} k$.
\end{enumerate}

\textbf{Ejemplo:}
\[
k(4,0,1,5) = (4k,0,k,5k) \in \mathbb{R}^4
\]

%-----------------------------------------------
\section{Combinaciones lineales}

Un vector $\vec{v}$ es combinación lineal de $\vec{v}_1, \vec{v}_2, \dots, \vec{v}_n$ si:
\[
\vec{v} = k_1 \vec{v}_1 + k_2 \vec{v}_2 + \dots + k_n \vec{v}_n
\]

\subsection{Ejemplo con matrices}
Determinar $a,b,c,d$ tal que:
\[
\begin{pmatrix} a & b \\ c & d \end{pmatrix}
= k 
\begin{pmatrix} 1 & 1 \\ 1 & 0 \end{pmatrix}
+ \ell 
\begin{pmatrix} 0 & 0 \\ 0 & 1 \end{pmatrix}
+ m
\begin{pmatrix} 0 & 0 \\ 0 & 1 \end{pmatrix}
\]

De la igualdad se obtiene:
\[
a = k + \ell, \quad b = k + \ell, \quad c = k, \quad d = m
\]

%-----------------------------------------------
\section{Subespacios vectoriales}

Un subconjunto $S \subset V$ es un \textbf{subespacio vectorial} si:
\begin{enumerate}
    \item Contiene al vector nulo.
    \item Es cerrado bajo suma y producto por escalar.
\end{enumerate}

\subsection{Ejemplo}
Sea 
\[
S = \{ (x,y,z) \in \mathbb{R}^3 \;|\; x - y + 3z = 0 \}
\]

Tomamos $\alpha = y$ y $\beta = z$:
\[
(x,y,z) = (\alpha - 3\beta, \alpha, \beta)
= \alpha (1,1,0) + \beta (-3,0,1)
\]

Esto muestra que $S$ está generado por $\{(1,1,0), (-3,0,1)\}$.

%-----------------------------------------------
\section{Ejemplo de comprobación}

Probar que $(1,-1,2)$ es combinación lineal de
$(1,1,1)$, $(1,1,0)$ y $(1,0,0)$:
\[
(1,-1,2) = a(1,1,1) + b(1,1,0) + c(1,0,0)
\]

\textbf{Sistema:}
\[
\begin{cases}
1 = a+b+c \\
-1 = a+b \\
2 = a
\end{cases}
\quad \Rightarrow \quad
a = 2, \; b = -3, \; c = 2
\]

\bigskip
\[
(1,-1,2) = 2(1,1,1) -3(1,1,0) + 2(1,0,0)
\]

\end{document}
