\documentclass[11pt]{article}

% ============================================================
% PAQUETES BÁSICOS
% ============================================================
\usepackage[utf8]{inputenc}
\usepackage[T1]{fontenc}
\usepackage[spanish]{babel}
\usepackage{amsmath, amssymb}
\usepackage[a4paper, margin=2.5cm]{geometry}
\usepackage{siunitx}
\usepackage{enumitem}

\sisetup{
  locale = DE,                  % coma decimal
  output-decimal-marker = {,},  % separador decimal coma
  per-mode = symbol,
  exponent-product = \cdot,
  group-minimum-digits = 4
}

% ============================================================
% DOCUMENTO
% ============================================================
\begin{document}

\begin{center}
  {\Large \textbf{Práctica de Álgebra — Factorización, Fracciones y Ecuaciones Cuadráticas}}\\[4pt]
  
\end{center}

\vspace{1em}

% ============================================================
\section*{I. Factorización}

\subsection*{1. Casos notables}

\begin{enumerate}[label=\arabic*)]
  \item Factorice: \(x^2 - 12x + 36\)
  \item Factorice: \(m^2 + 18m + 81\)
  \item Factorice: \(y^2 - 49\)
  \item Factorice: \(9a^2 - 25b^2\)
  \item Factorice: \(25x^2 + 30x + 9\)
\end{enumerate}

\subsection*{2. Factor común y agrupación}

\begin{enumerate}[label=\arabic*)]
  \item Factorice: \(7m^3 + 10m^4 - 5m^2\)
  \item Factorice: \(ax + bx + aw + bw\)
  \item Factorice: \(a(b+c) + x(b+c)\)
  \item Factorice: \(6x^2y - 9xy^2\)
  \item Factorice: \(8a^3b^2 - 4a^2b + 2ab\)
\end{enumerate}

\subsection*{3. Expresiones mixtas}

\begin{enumerate}[label=\arabic*)]
  \item Factorice: \(x^3 - 3x^2 - 4x\)
  \item Factorice: \(4x^2 - 20x + 25\)
  \item Factorice: \(9p^2 - 6p + 1\)
\end{enumerate}

% ============================================================
\section*{II. Simplificación de expresiones algebraicas}

\begin{enumerate}[label=\arabic*)]
  \item Simplifique: \(\dfrac{2x - 8}{x^2 - 16}\)
  \item Simplifique: \(\dfrac{x + 2}{x^2 - 4}\)
  \item Simplifique: \(\dfrac{x^2 - 1}{x + 1}\)
  \item Simplifique: \(\dfrac{x^2 - 9x + 18}{x^2 - 6x + 9}\)
  \item Simplifique: \(\dfrac{a^2 - 9b^2}{a^2 - 6ab + 9b^2}\)
\end{enumerate}

% ============================================================
\section*{III. Operaciones con fracciones algebraicas}

\subsection*{1. Suma y resta con denominadores iguales}

\begin{enumerate}[label=\arabic*)]
  \item \(\dfrac{3x}{5y} + \dfrac{2x}{5y}\)
  \item \(\dfrac{4a}{7b} - \dfrac{a}{7b}\)
  \item \(\dfrac{5m}{2n} + \dfrac{3m}{2n}\)
\end{enumerate}

\subsection*{2. Suma y resta con denominadores distintos}

\begin{enumerate}[label=\arabic*)]
  \item \(\dfrac{x}{x+2} + \dfrac{3}{x+2}\)
  \item \(\dfrac{2}{x} - \dfrac{1}{x+1}\)
  \item \(\dfrac{3}{y+1} + \dfrac{5}{y-1}\)
  \item \(\dfrac{x}{x^2 - 9} + \dfrac{3}{x+3}\)
\end{enumerate}

\subsection*{3. Multiplicación y división de fracciones algebraicas}

\begin{enumerate}[label=\arabic*)]
  \item \(\dfrac{2x}{3y} \cdot \dfrac{9y}{4x}\)
  \item \(\dfrac{3a^2}{4b} \div \dfrac{6a}{b^2}\)
  \item \(\dfrac{x^2 - 9}{x^2 - x - 6} \cdot \dfrac{x - 3}{x + 2}\)
\end{enumerate}

% ============================================================
\section*{IV. Ecuaciones cuadráticas}

\subsection*{1. Identificación de coeficientes}

\begin{enumerate}[label=\arabic*)]
  \item En la ecuación \( -15 - x^2 - 9x = 0 \), indique los valores de \(a\), \(b\) y \(c\).
  \item En la ecuación \( 3x^2 + 5x - 2 = 0 \), indique los valores de \(a\), \(b\) y \(c\).
\end{enumerate}

\subsection*{2. Cálculo e interpretación del discriminante}

\begin{enumerate}[label=\arabic*)]
  \item Calcule el discriminante de \( x^2 - 6x + 9 = 0 \) y describa el tipo de soluciones.
  \item Calcule el discriminante de \( 2x^2 + 3x + 5 = 0 \).
  \item Complete: Si el discriminante es menor que cero, el conjunto solución es \underline{\hspace{3cm}}.
\end{enumerate}

\subsection*{3. Aplicación de la fórmula general}

\begin{enumerate}[label=\arabic*)]
  \item Resuelva: \( x^2 - 7x + 12 = 0 \)
  \item Resuelva: \( 2x^2 + 5x - 3 = 0 \)
  \item Resuelva: \( 3x^2 - 4x - 1 = 0 \)
\end{enumerate}

\subsection*{4. Análisis del conjunto solución}

\begin{enumerate}[label=\arabic*)]
  \item Si el discriminante es positivo, ¿qué se puede afirmar sobre las raíces?
  \item Si el discriminante es cero, ¿qué tipo de raíces tiene la ecuación?
  \item Si el discriminante es negativo, ¿qué se puede afirmar del conjunto solución?
\end{enumerate}

% ============================================================
\end{document}