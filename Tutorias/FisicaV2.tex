\documentclass[12pt]{article}
\usepackage[utf8]{inputenc}
\usepackage[spanish]{babel}
\usepackage{amsmath}
\usepackage{siunitx}

\begin{document}

\section*{Ejercicios de Movimiento Rectilíneo Uniformemente Acelerado}

\subsection*{Ejercicio 1}
Un auto parte del reposo y alcanza una velocidad de \(\SI{20}{m/s}\) en \(\SI{8}{s}\).  
¿Cuál es la aceleración del auto?  
\begin{itemize}
  \item Datos: \(v_i = 0\), \(v_f = 20\,\si{m/s}\), \(t = 8\,\si{s}\)
  \item Fórmula: \(a = \dfrac{v_f - v_i}{t}\)
  \item Procedimiento:
  \item Respuesta:
\end{itemize}
\newpage
\subsection*{Ejercicio 2}
Un automóvil parte del reposo y alcanza una velocidad de \(\SI{20}{m/s}\) en \(\SI{5}{s}\) con una aceleración constante.  
¿Cuál es su desplazamiento?  
\begin{itemize}
  \item Datos: \(v_i = 0\), \(v_f = 20\,\si{m/s}\), \(t = 5\,\si{s}\)
  \item Fórmula: \(d = v_i \cdot t + \tfrac{1}{2} a t^2\)
  \item Procedimiento:
  \item Respuesta:
\end{itemize}
\newpage
\subsection*{Ejercicio 3}
Un carro parte del reposo y acelera uniformemente a razón de \(3\,\si{m/s^2}\).  
¿Cuánto tiempo le toma alcanzar una velocidad de \(18\,\si{m/s}\)?  
\begin{itemize}
  \item Datos: \(v_i = 0\), \(a = 3\,\si{m/s^2}\), \(v_f = 18\,\si{m/s}\)
  \item Fórmula: \(t = \dfrac{v_f - v_i}{a}\)
  \item Procedimiento:
  \item Respuesta:
\end{itemize}
\newpage
\subsection*{Ejercicio 4}
Un tren alcanza una velocidad final de \(\SI{50}{m/s}\) después de acelerar uniformemente a \(\SI{4}{m/s^2}\) durante \(\SI{10}{s}\).  
¿Cuál fue su velocidad inicial?  
\begin{itemize}
  \item Datos: \(v_f = 50\,\si{m/s}\), \(a = 4\,\si{m/s^2}\), \(t = 10\,\si{s}\)
  \item Fórmula: \(v_i = v_f - a \cdot t\)
  \item Procedimiento:
  \item Respuesta:
\end{itemize}
\newpage
\subsection*{Ejercicio 5}
Un tren con velocidad inicial de \(\SI{15}{m/s}\) desacelera uniformemente a \(-2\,\si{m/s^2}\) durante \(\SI{6}{s}\).  
¿Cuál es su velocidad final?  
\begin{itemize}
  \item Datos: \(v_i = 15\,\si{m/s}\), \(a = -2\,\si{m/s^2}\), \(t = 6\,\si{s}\)
  \item Fórmula: \(v_f = v_i + a \cdot t\)
  \item Procedimiento:
  \item Respuesta:
\end{itemize}
\newpage
\section*{Fórmulas del MRUA}
\begin{align}
v_f &= v_i + a \cdot t \\[6pt]
d &= v_i \cdot t + \tfrac{1}{2} a t^2 \\[6pt]
v_f^2 - v_i^2 &= 2 a d \\[6pt]
d &= \tfrac{(v_f + v_i)}{2} \cdot t \\[6pt]
a &= \dfrac{v_f - v_i}{t}
\end{align}

\section*{Análisis dimensional en el SI}
\begin{align}
a &= \dfrac{v_f - v_i}{t} \quad \Rightarrow \quad \left[ \dfrac{m}{s^2} \right] \\[6pt]
d &= v_i \cdot t + \tfrac{1}{2} a t^2 \quad \Rightarrow \quad [m] \\[6pt]
v_f^2 - v_i^2 &= 2 a d \quad \Rightarrow \quad \left[ \dfrac{m^2}{s^2} \right]
\end{align}

\section*{Características del MRUA}
\begin{itemize}
  \item Movimiento en línea recta.
  \item Velocidad variable.
  \item Aceleración constante.
  \item Si la velocidad aumenta, la aceleración es positiva.
  \item Si la velocidad disminuye, la aceleración es negativa.
  \item Si el móvil parte del reposo: \(v_i = 0\).
  \item Si el móvil se detiene: \(v_f = 0\).
\end{itemize}

\end{document}