\documentclass[11pt]{article}

% ============================================================
% PAQUETES BÁSICOS
% ============================================================
\usepackage[utf8]{inputenc}
\usepackage[T1]{fontenc}
\usepackage[spanish]{babel}
\usepackage{amsmath, amssymb}
\usepackage{graphicx}
\usepackage{siunitx}
\usepackage[a4paper, margin=2.5cm]{geometry}

% ============================================================
% CONFIGURACIÓN DE UNIDADES Y DECIMALES
% ============================================================
\sisetup{
  locale = DE,
  output-decimal-marker = {,}, % coma decimal
  per-mode = symbol,
  exponent-product = \cdot,
  group-minimum-digits = 4
}

% ============================================================
% DOCUMENTO
% ============================================================
\begin{document}

\begin{center}
{\Large \textbf{Ejemplos de equilibrio térmico}}\\[4pt]
\textit{Transferencia de calor y temperatura de equilibrio}\\[6pt]
\end{center}

% ============================================================
\section*{Ejemplo 1}

\textbf{Enunciado:}  
Determinar la temperatura final de equilibrio cuando \SI{10}{g} de leche a \SI{10}{\celsius} se mezclan con \SI{160}{g} de café a \SI{90}{\celsius}, suponiendo que ambos líquidos tienen el mismo calor específico del agua 
(\(c = \SI{4186}{J/(kg·°C)}\)) y se desprecia el calor del recipiente.

% ------------------------------------------------------------
\subsection*{Datos}
\begin{align*}
m_\text{leche} &= \SI{10}{g} = \SI{0,010}{kg},\\
m_\text{café} &= \SI{160}{g} = \SI{0,160}{kg},\\
T_{i,\text{leche}} &= \SI{10}{\celsius}, &
T_{i,\text{café}} &= \SI{90}{\celsius},\\
c_\text{agua} &= \SI{4186}{\joule\per\kilogram\per\celsius}.
\end{align*}

% ------------------------------------------------------------
\subsection*{Fórmula}
Como el calor cedido por el café es igual al calor absorbido por la leche:
\[
m_\text{café}\,c\,(T_{i,\text{café}} - T_f) = m_\text{leche}\,c\,(T_f - T_{i,\text{leche}}).
\]

Simplificando (se cancela \(c\)):
\[
m_\text{café}(T_{i,\text{café}} - T_f) = m_\text{leche}(T_f - T_{i,\text{leche}}).
\]

% ------------------------------------------------------------
\subsection*{Sustitución}
\[
0,160(90 - T_f) = 0,010(T_f - 10).
\]

% ------------------------------------------------------------
\subsection*{Cálculo}
\[
14,4 - 0,160T_f = 0,010T_f - 0,1,
\]
\[
14,5 = 0,170T_f,
\]
\[
T_f = \frac{14,5}{0,170} = \SI{85,3}{\celsius}.
\]

% ------------------------------------------------------------
\subsection*{Conclusión}
La temperatura final de equilibrio del sistema leche–café es:
\[
\boxed{T_f = \SI{85,3}{\celsius}}.
\]
\newpage
% ============================================================
\section*{Ejemplo 2}

\textbf{Enunciado:}  
Si \SI{100}{g} de agua a \SI{100}{\celsius} se vierten dentro de una taza de aluminio de \SI{20}{g} que contiene \SI{50}{g} de agua a \SI{20}{\celsius}, determinar la temperatura de equilibrio del sistema.

% ------------------------------------------------------------
\subsection*{Datos}
\begin{align*}
m_1 &= \SI{100}{g} = \SI{0,100}{kg} \quad \text{(agua caliente)},\\
m_2 &= \SI{50}{g} = \SI{0,050}{kg} \quad \text{(agua fría)},\\
m_\text{taza} &= \SI{20}{g} = \SI{0,020}{kg},\\
T_{i,1} &= \SI{100}{\celsius}, &
T_{i,2} &= \SI{20}{\celsius}, &
T_{i,\text{taza}} &= \SI{20}{\celsius},\\
c_\text{agua} &= \SI{4186}{\joule\per\kilogram\per\celsius}, &
c_\text{Al} &= \SI{900}{\joule\per\kilogram\per\celsius}.
\end{align*}

% ------------------------------------------------------------
\subsection*{Fórmula}
El calor perdido por el agua caliente es ganado por el agua fría y la taza:
\[
m_1c_\text{agua}(T_{i,1} - T_f)
= m_2c_\text{agua}(T_f - T_{i,2}) + m_\text{taza}c_\text{Al}(T_f - T_{i,\text{taza}}).
\]

% ------------------------------------------------------------
\subsection*{Sustitución}
\[
(0,100)(4186)(100 - T_f) =
(0,050)(4186)(T_f - 20) + (0,020)(900)(T_f - 20).
\]

% ------------------------------------------------------------
\subsection*{Cálculo}
\[
418,6(100 - T_f) = 209,3(T_f - 20) + 18(T_f - 20),
\]
\[
41860 - 418,6T_f = 227,3T_f - 4546,
\]
\[
41860 + 4546 = 645,9T_f,
\]
\[
T_f = \frac{46406}{645,9} = \SI{71,9}{\celsius}.
\]

% ------------------------------------------------------------
\subsection*{Conclusión}
La temperatura final de equilibrio del sistema agua–taza de aluminio es:
\[
\boxed{T_f = \SI{71,9}{\celsius}}.
\]
\newpage
% ============================================================
\section*{Ejemplo 3}

\textbf{Enunciado:}  
Un técnico de laboratorio pone una muestra de \SI{0,0850}{kg} de un material desconocido, que está a \SI{100,0}{\celsius}, en un calorímetro cuyo recipiente, inicialmente a \SI{19,0}{\celsius}, está hecho con \SI{0,150}{kg} de cobre y contiene \SI{0,200}{kg} de agua.  
La temperatura final del calorímetro y su contenido es de \SI{26,1}{\celsius}.  
Calcular el calor específico de la muestra.

\subsection*{Datos}
\begin{align*}
m_\text{muestra} &= \SI{0,0850}{kg}, &
T_{i,\text{muestra}} &= \SI{100,0}{\celsius},\\
m_\text{agua} &= \SI{0,200}{kg}, &
T_{i,\text{agua}} &= \SI{19,0}{\celsius},\\
m_\text{Cu} &= \SI{0,150}{kg}, &
T_{i,\text{Cu}} &= \SI{19,0}{\celsius},\\
T_f &= \SI{26,1}{\celsius},\\
c_\text{agua} &= \SI{4186}{\joule\per\kilogram\per\celsius},\\
c_\text{Cu} &= \SI{387}{\joule\per\kilogram\per\celsius}.
\end{align*}

\subsection*{Fórmula}
El calor perdido por la muestra es igual al calor ganado por el agua y por el cobre:
\[
m_\text{muestra}\,c_\text{muestra}(T_{i,\text{muestra}} - T_f)
= m_\text{agua}\,c_\text{agua}(T_f - T_{i,\text{agua}})
+ m_\text{Cu}\,c_\text{Cu}(T_f - T_{i,\text{Cu}}).
\]
Despejando:
\[
c_\text{muestra} =
\frac{m_\text{agua}\,c_\text{agua}(T_f - T_{i,\text{agua}})
+ m_\text{Cu}\,c_\text{Cu}(T_f - T_{i,\text{Cu}})}
{m_\text{muestra}(T_{i,\text{muestra}} - T_f)}.
\]

\subsection*{Sustitución}
\[
c_\text{muestra} =
\frac{(0,200)(4186)(26,1 - 19,0)
+ (0,150)(387)(26,1 - 19,0)}
{(0,0850)(100,0 - 26,1)}.
\]

\subsection*{Cálculo}
\[
\begin{aligned}
Q_\text{agua} &= (0,200)(4186)(7,1) = \SI{5944,1}{\joule},\\[2pt]
Q_\text{Cu}   &= (0,150)(387)(7,1) = \SI{412,2}{\joule},\\[2pt]
Q_\text{ganado} &= Q_\text{agua} + Q_\text{Cu}
= \SI{6356,3}{\joule}.
\end{aligned}
\]
Entonces:
\[
c_\text{muestra} =
\frac{6356,3}{(0,0850)(73,9)}
= \SI{1,01e3}{\joule\per\kilogram\per\celsius}.
\]

\subsection*{Conclusión}
El calor específico de la muestra es, aproximadamente,
\[
\boxed{c_\text{muestra} \approx \SI{1,0e3}{\joule\per\kilogram\per\celsius}}.
\]

% ============================================================
\section*{Ejercicio 4}

\textbf{Enunciado:}  
Un vaso aislado con masa despreciable contiene \SI{0,250}{kg} de agua a \SI{75,0}{\celsius}.  
¿Cuántos kilogramos de hielo a \SI{-20,0}{\celsius} deben ponerse en el agua para que la temperatura final del sistema sea \SI{40,0}{\celsius}?

\subsection*{Datos}
\begin{align*}
m_\text{agua} &= \SI{0,250}{kg}, &
T_{i,\text{agua}} &= \SI{75,0}{\celsius},\\
m_\text{hielo} &= m \quad \text{(incógnita)}, &
T_{i,\text{hielo}} &= \SI{-20,0}{\celsius},\\
T_f &= \SI{40,0}{\celsius},\\
c_\text{agua} &= \SI{4186}{\joule\per\kilogram\per\celsius},\\
c_\text{hielo} &= \SI{2090}{\joule\per\kilogram\per\celsius},\\
L_f &= \SI{3,34e5}{\joule\per\kilogram}
\quad \text{(calor latente de fusión del agua)}.
\end{align*}

\subsection*{Fórmula}
El agua caliente pierde calor y el hielo lo gana en tres etapas:
\begin{itemize}
  \item Calentamiento del hielo de \SI{-20}{\celsius} a \SI{0}{\celsius}.
  \item Fusión del hielo a \SI{0}{\celsius}.
  \item Calentamiento del agua procedente del hielo de \SI{0}{\celsius} a \SI{40}{\celsius}.
\end{itemize}
Balance de energía:
\[
m_\text{agua}c_\text{agua}(T_{i,\text{agua}} - T_f)
= m\big[c_\text{hielo}(0 - T_{i,\text{hielo}})
+ L_f
+ c_\text{agua}(T_f - 0)\big].
\]

\subsection*{Sustitución}
\[
(0,250)(4186)(75,0 - 40,0)
= m\big[2090(20) + 3,34\cdot10^{5} + 4186(40)\big].
\]

\subsection*{Cálculo}
\[
\begin{aligned}
Q_\text{perdido} &= (0,250)(4186)(35)
= \SI{36627,5}{\joule},\\[4pt]
\text{término entre corchetes} &= 2090(20) + 3,34\cdot10^{5} + 4186(40)\\
&= 41800 + 334000 + 167440\\
&= \SI{5,4324e5}{\joule\per\kilogram}.
\end{aligned}
\]
Entonces,
\[
m = \frac{36627,5}{5,4324\cdot10^{5}}
= \SI{6,74e-2}{\kilogram}.
\]

\subsection*{Conclusión}
Se deben agregar aproximadamente
\[
\boxed{m_\text{hielo} \approx \SI{6,7e-2}{\kilogram} \approx \SI{0,067}{\kilogram}}
\]
de hielo, es decir, alrededor de \SI{67}{g}.
\end{document}