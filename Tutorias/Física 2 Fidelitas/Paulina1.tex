% ==========================================================
% Ondas mecánicas y sonoras - Ejemplos resueltos
% Formato vertical, con explicación
% ==========================================================
\documentclass[11pt,letterpaper]{article}

% --- Idioma y codificación ---
\usepackage[T1]{fontenc}
\usepackage[utf8]{inputenc}
\usepackage[spanish, es-nodecimaldot]{babel}

% --- Matemática y SI ---
\usepackage{amsmath, amssymb, bm}
\usepackage{siunitx}
\sisetup{
  locale = DE,                 % coma decimal
  output-decimal-marker = {,}, % coma decimal
  per-mode = symbol,
  exponent-product = \cdot,
  group-minimum-digits = 4
}

% --- Página ---
\usepackage[a4paper,margin=2.4cm]{geometry}
\setlength{\parindent}{0pt}
\setlength{\parskip}{6pt}

% --- Inicio del documento ---
\begin{document}

\begin{center}
{\Large \textbf{Ejemplos resueltos de ondas y sonido}}\\[2mm]
Formato paso a paso, con interpretación física.
\end{center}

% ==========================================================
\section*{Ejemplo 15.6 \quad Onda estacionaria en una cuerda de guitarra}
% ==========================================================

Una cuerda de una guitarra está en el eje \(x\) cuando se encuentra en equilibrio.
El extremo en \(x = 0\) (el puente de la guitarra) está fijo.

Una onda sinusoidal de amplitud \(A = \SI{0,750}{mm} = 7{,}50 \times 10^{-4}\ \si{m}\),
frecuencia \(f = \SI{440}{Hz}\) y rapidez de propagación \(v = \SI{143}{m/s}\)
viaja hacia la izquierda. La onda se refleja en el extremo fijo y la superposición entre
la onda incidente y la reflejada forma una onda estacionaria.

\begin{itemize}
  \item[a)] Determine la ecuación del desplazamiento transversal \(y(x,t)\) en función de la posición y el tiempo.
  \item[b)] Ubique los nodos.
  \item[c)] Calcule la amplitud máxima en los vientres (antinodos), así como la velocidad y aceleración transversales máximas.
\end{itemize}

\subsection*{Solución}

\textbf{Datos:}
\[
A = 7{,}50\times10^{-4}\ \si{m}, \quad
f = \SI{440}{Hz}, \quad
v = \SI{143}{m/s}
\]

La onda estacionaria que resulta de la interferencia de dos ondas de igual amplitud que viajan en sentidos opuestos se puede escribir como
\[
y(x,t) = A_0 \sin(kx)\,\sin(\omega t),
\]
donde:
\[
A_0 = 2A \quad (\text{la amplitud de la onda estacionaria es el doble}),
\]
\[
\omega = 2\pi f, \qquad
k = \frac{\omega}{v} = \frac{2\pi f}{v}.
\]

\textbf{1) Cálculo de \(\omega\):}
\[
\omega = 2\pi f
       = 2\pi (440\ \si{s^{-1}})
       \approx 2{,}760 \times 10^{0}\ \si{rad/s}
       = \SI{2760}{rad/s}.
\]

\textbf{2) Cálculo de \(k\):}
\[
k = \frac{\omega}{v}
  = \frac{2760\ \si{rad/s}}{143\ \si{m/s}}
  = 19{,}3\ \si{rad/m}.
\]

\textbf{3) Amplitud de la onda estacionaria:}
\[
A_0 = 2A = 2(7{,}50\times10^{-4}\ \si{m})
        = 1{,}50\times10^{-3}\ \si{m}
        = \SI{1,50}{mm}.
\]

Entonces, la ecuación queda:

\[
\boxed{
y(x,t)
= \left(1{,}50\times10^{-3}\ \si{m}\right)
\sin\!\bigl(19{,}3\ \si{rad/m}\, x\bigr)\,
\sin\!\bigl(2760\ \si{rad/s}\, t\bigr)
}
\]

Comentario didáctico:
\begin{itemize}
  \item \(\sin(kx)\) controla la forma espacial (nodos y vientres).
  \item \(\sin(\omega t)\) describe la oscilación temporal de cada punto.
\end{itemize}

\vspace{4mm}
\textbf{b) Posición de los nodos}

Un nodo es un punto que no se mueve nunca, es decir:
\[
y(x,t)=0 \quad \forall t
\quad\Longrightarrow\quad
\sin(kx)=0.
\]

Esto ocurre cuando
\[
kx = n\pi
\quad\Longrightarrow\quad
x = \frac{n\pi}{k}
= \frac{n\pi}{19{,}3\ \si{rad/m}},
\quad n = 0,1,2,3,\dots
\]

Numéricamente:
\[
x_0 = 0,
\quad
x_1 \approx \frac{\pi}{19{,}3} = 0{,}163\ \si{m},
\quad
x_2 \approx 0{,}325\ \si{m},
\quad
x_3 \approx 0{,}488\ \si{m},
\dots
\]

Estos son los \textbf{nodos}.

\vspace{4mm}
\textbf{c) Amplitud máxima, velocidad transversal máxima y aceleración transversal máxima}

En un vientre (antinodo) la cuerda alcanza su máxima oscilación.
Allí \(\sin(kx)=\pm1\), por lo que el desplazamiento máximo respecto al equilibrio es
\[
y_{\text{max}} = A_0 = 1{,}50\times10^{-3}\ \si{m}
= \SI{1,50}{mm}.
\]

Para un punto fijo \(x\), la velocidad transversal es
\[
v_y(x,t) = \frac{\partial y(x,t)}{\partial t}
= A_0 \sin(kx)\,\omega \cos(\omega t).
\]

La \textbf{velocidad máxima} (en magnitud) en un antinodo (\(\sin(kx)=\pm1\)) es:
\[
v_{y,\text{max}}
= A_0\,\omega
= \left(1{,}50\times10^{-3}\ \si{m}\right)
  (2760\ \si{rad/s})
\approx \SI{4,15}{m/s}.
\]

Interpretación: el punto de la cuerda en el antinodo puede moverse arriba/abajo tan rápido como \(\pm 4{,}15\ \si{m/s}\).

Ahora derivamos otra vez para la aceleración transversal:
\[
a_y(x,t)
= \frac{\partial^2 y(x,t)}{\partial t^2}
= -A_0 \sin(kx)\,\omega^2 \sin(\omega t).
\]

En un antinodo (\(\sin(kx)=\pm1\)), el valor máximo (en magnitud) es
\[
a_{y,\text{max}}
= A_0 \omega^2
= \left(1{,}50\times10^{-3}\ \si{m}\right)
  (2760\ \si{rad/s})^2
\approx 1{,}15\times10^{4}\ \si{m/s^2}.
\]

Nota al estudiante: esta aceleración es enorme, muchísimo mayor que \(g \approx 9{,}8\ \si{m/s^2}\).
Por eso las cuerdas vibran tan ``rápido'' aunque el desplazamiento sea pequeño.

% ==========================================================
\section*{Ejemplo 15.7 \quad Contrabajo gigante}
% ==========================================================

Se quiere construir un contrabajo con cuerdas de \SI{5,00}{m} de longitud entre dos puntos fijos.
Cada cuerda tiene densidad lineal de masa \(\mu = \SI{40,0}{g/m} = 40,0 \times 10^{-3}\ \si{kg/m}\)
y su frecuencia fundamental medida es \(f_1 = \SI{20,0}{Hz}\).

\begin{itemize}
  \item[a)] Calcular la tensión \(F\) en la cuerda.
  \item[b)] Calcular la frecuencia y la longitud de onda del segundo armónico (\(n=2\)).
  \item[c)] Calcular la frecuencia y la longitud de onda del segundo sobretono (que corresponde a \(n=3\)).
\end{itemize}

\subsection*{Solución}

\textbf{a) Tensión en la cuerda}

Para una cuerda fija en ambos extremos, la frecuencia fundamental es
\[
f_1 = \frac{1}{2L}\sqrt{\frac{F}{\mu}}.
\]

Despejando \(F\):
\[
F = 4 \mu L^2 f_1^2.
\]

Sustitución numérica:
\[
F = 4 \left(40{,}0\times10^{-3}\ \si{kg/m}\right)
      \left(5{,}00\ \si{m}\right)^2
      \left(20{,}0\ \si{s^{-1}}\right)^2.
\]

\[
F = 4 \,(40{,}0\times10^{-3}) \,(25{,}0)\,(400)
\ \si{kg\ m/s^2}.
\]

\[
F \approx 1600\ \si{N}.
\]

Comentario: esto es una tensión bastante grande (del orden de centenares de newton).

\vspace{4mm}
\textbf{b) Segundo armónico \((n=2)\)}

Para una cuerda fija en ambos extremos:
\[
f_n = n f_1,
\qquad
\lambda_n = \frac{2L}{n}.
\]

Entonces, para \(n=2\):
\[
f_2 = 2 f_1 = 2(20{,}0\ \si{Hz}) = \SI{40,0}{Hz},
\]
\[
\lambda_2 = \frac{2L}{2}
= L
= \SI{5,00}{m}.
\]

\vspace{4mm}
\textbf{c) Segundo sobretono}

Ojo didáctico: el “primer sobretono’’ es \(n=2\), el “segundo sobretono’’ es \(n=3\).

Para \(n=3\):
\[
f_3 = 3 f_1 = 3(20{,}0\ \si{Hz}) = \SI{60,0}{Hz},
\]
\[
\lambda_3 = \frac{2L}{3}
= \frac{2(5{,}00\ \si{m})}{3}
= \SI{3,33}{m}\ (\text{aprox.}).
\]

Interpretación: al aumentar \(n\), sube la frecuencia y baja la longitud de onda. Eso da notas más agudas.

% ==========================================================
\section*{Ejemplo 16.3 \quad Longitud de onda de las ondas del sonar}
% ==========================================================

Un barco usa un sistema de sonar para detectar objetos bajo el agua.
Se pide:

\begin{itemize}
  \item Determinar la rapidez \(v\) de las ondas sonoras en el agua usando
        el módulo volumétrico \(B\) y la densidad \(\rho\) del agua.
  \item Calcular la longitud de onda \(\lambda\) para una frecuencia
        \(f = \SI{262}{Hz}\).
\end{itemize}

Datos:
\[
\rho = 1{,}00\times10^{3}\ \si{kg/m^3}, \qquad
B = 2{,}18\times10^{9}\ \si{Pa}, \qquad
f = \SI{262}{Hz}.
\]

\subsection*{Solución}

\textbf{a) Rapidez del sonido en el agua}

Para una onda sonora en un fluido:
\[
v = \sqrt{\frac{B}{\rho}},
\]
donde \(B\) es el módulo volumétrico (qué tan ``incompresible'' es el medio).

Sustitución:
\[
v = \sqrt{\frac{2{,}18\times10^{9}\ \si{Pa}}
                {1{,}00\times10^{3}\ \si{kg/m^3}} }.
\]

\[
v \approx \SI{1480}{m/s}.
\]

Comentario: el sonido viaja más rápido en agua que en aire porque el agua es mucho menos compresible.

\vspace{4mm}
\textbf{b) Longitud de onda en el agua}

Relación onda mecánica:
\[
v = \lambda f
\quad\Longrightarrow\quad
\lambda = \frac{v}{f}.
\]

Sustitución:
\[
\lambda
= \frac{1480\ \si{m/s}}
        {262\ \si{s^{-1}}}
\approx \SI{5,65}{m}.
\]

Interpretación: para la misma frecuencia, la onda en agua tiene longitud de onda más grande que en aire.

% ==========================================================
\section*{Ejemplo 16.4 \quad Rapidez del sonido en el aire}
% ==========================================================

Calcule la rapidez del sonido en el aire a \(T = \SI{20}{^\circ C}\) y determine
el intervalo de longitudes de onda audible para el oído humano
(en el aire a esa temperatura), sabiendo que el oído típico escucha
entre \(\SI{20}{Hz}\) y \(\SI{20000}{Hz}\).

Datos:
\[
\gamma = 1{,}40, \quad
R = \SI{8,314}{J/(mol\cdot K)}, \quad
T = \SI{293}{K} \ (\text{equivale a } 20^\circ\text{C}), \quad
M = 28{,}8\times10^{-3}\ \si{kg/mol}.
\]

\subsection*{Solución}

Rapidez del sonido en un gas ideal (aprox):
\[
v = \sqrt{\frac{\gamma R T}{M}}.
\]

Sustitución:
\[
v
= \sqrt{
\frac{(1{,}40)(8{,}314\ \si{J/(mol\cdot K)})(293\ \si{K})}
     {28{,}8\times10^{-3}\ \si{kg/mol}}
}
\approx \SI{344}{m/s}.
\]

Comentario: este valor coincide con la rapidez medida del sonido en el aire a temperatura ambiente.

Ahora usamos \(v = \lambda f \Rightarrow \lambda = v/f\).

\textbf{Frecuencia baja (20 Hz):}
\[
\lambda_{\text{baja}}
= \frac{344\ \si{m/s}}{20\ \si{s^{-1}}}
= \SI{17,2}{m}
\approx \SI{17}{m}.
\]

\textbf{Frecuencia alta (20000 Hz):}
\[
\lambda_{\text{alta}}
= \frac{344\ \si{m/s}}{20000\ \si{s^{-1}}}
= \SI{1,72e-2}{m}
\approx \SI{1,7}{cm}.
\]

Interpretación: sonidos graves tienen longitudes de onda muy grandes, sonidos agudos muy pequeñas.

% ==========================================================
\section*{Ejemplo 16.8 \quad Sordera temporal o permanente}
% ==========================================================

Una exposición de \SI{10}{min} a un sonido de \SI{120}{dB}
desplaza temporalmente el umbral auditivo de \SI{0}{dB} a \SI{28}{dB}
(en \SI{1000}{Hz}).

Diez años de exposición al sonido de \SI{92}{dB}
causan un desplazamiento permanente del umbral a \SI{28}{dB}.

Pregunta: ¿qué intensidades \(I\) corresponden a \(\beta = \SI{28}{dB}\)
y a \(\beta = \SI{92}{dB}\)?

Sabemos que el nivel sonoro en decibelios es
\[
\beta = (10\ \si{dB}) \log_{10}\!\left(\frac{I}{I_0}\right),
\quad
I_0 = 1\times10^{-12}\ \si{W/m^2}.
\]

\subsection*{Solución}

Despejamos la intensidad:
\[
I = I_0 \, 10^{\beta/(10\ \si{dB})}.
\]

\textbf{Para \(\beta = 28\ \si{dB}\):}
\[
I_{28\ \mathrm{dB}}
= \left(1\times10^{-12}\ \si{W/m^2}\right)
  10^{28/10}
= 10^{-12}\cdot10^{2{,}8}\ \si{W/m^2}.
\]

\[
10^{2{,}8} \approx 6{,}3\times10^{2}.
\]

\[
I_{28\ \mathrm{dB}}
\approx 6{,}3\times10^{-10}\ \si{W/m^2}.
\]

\textbf{Para \(\beta = 92\ \si{dB}\):}
\[
I_{92\ \mathrm{dB}}
= \left(1\times10^{-12}\ \si{W/m^2}\right)
  10^{92/10}
= 10^{-12}\cdot10^{9{,}2}\ \si{W/m^2}.
\]

\[
10^{9{,}2} \approx 1{,}6\times10^{9}.
\]

\[
I_{92\ \mathrm{dB}}
\approx 1{,}6\times10^{-3}\ \si{W/m^2}.
\]

Comentario para el estudiante:
un aumento de pocos decibelios representa \textbf{un cambio enorme} en intensidad física (potencia por unidad de área).

% ==========================================================
\section*{Ejemplo 16.14 \quad Efecto Doppler I: longitudes de onda}
% ==========================================================

La sirena de una patrulla emite una onda sinusoidal con frecuencia
\(f_s = \SI{300}{Hz}\).
La rapidez del sonido en el aire es \(v = \SI{340}{m/s}\).
El aire está en reposo.

\begin{itemize}
  \item[a)] Calcule la longitud de onda de la sirena si la fuente está en reposo.
  \item[b)] Si la patrulla se mueve a \(v_s = \SI{30}{m/s}\),
            calcule la longitud de onda delante de la patrulla
            y la longitud de onda detrás de la patrulla.
\end{itemize}

\subsection*{Solución}

\textbf{a) Fuente en reposo}

Cuando fuente y aire están en reposo relativo, usamos
\[
\lambda = \frac{v}{f_s}.
\]

\[
\lambda
= \frac{340\ \si{m/s}}{300\ \si{s^{-1}}}
= \SI{1,13}{m}.
\]

\textbf{b) Fuente moviéndose a \(v_s = \SI{30}{m/s}\)}

Ahora hay \textbf{efecto Doppler en la onda misma}: las crestas se “aprietan” delante y se “estiran” detrás.

\underline{Delante de la fuente (acercándose):}
la velocidad efectiva de acercamiento entre crestas y aire es \(v - v_s\), de modo que
\[
\lambda_{\text{enfrente}}
= \frac{v - v_s}{f_s}
= \frac{340\ \si{m/s} - 30\ \si{m/s}}{300\ \si{s^{-1}}}
= \SI{1,03}{m}.
\]

\underline{Detrás de la fuente (alejándose):}
\[
\lambda_{\text{detrás}}
= \frac{v + v_s}{f_s}
= \frac{340\ \si{m/s} + 30\ \si{m/s}}{300\ \si{s^{-1}}}
= \SI{1,23}{m}.
\]

Interpretación: delante (hacia donde avanza la patrulla) la longitud de onda disminuye y por eso el sonido se oye más agudo; detrás aumenta y el sonido se oye más grave.

% ==========================================================
\section*{Ejemplo 16.15 \quad Efecto Doppler II: frecuencias}
% ==========================================================

Si un receptor \(L\) está en reposo y la sirena del ejemplo anterior
se aleja de \(L\) con rapidez \(\SI{30}{m/s}\),
¿qué frecuencia oye el receptor?

Tenemos:
\[
v = \SI{340}{m/s}, \quad
v_s = \SI{30}{m/s} \ (\text{fuente alejándose}), \quad
v_L = 0, \quad
f_s = \SI{300}{Hz}.
\]

\subsection*{Solución}

Para una fuente que se aleja de un observador estacionario,
la frecuencia observada es
\[
f_L
= \frac{v}{v+v_s} f_s.
\]

Sustitución:
\[
f_L
= \frac{340\ \si{m/s}}
       {340\ \si{m/s} + 30\ \si{m/s}}
  (300\ \si{s^{-1}}).
\]

\[
f_L
= \frac{340}{370} (300\ \si{Hz})
\approx \SI{276}{Hz}.
\]

Comprobación alternativa:
detrás de la patrulla (donde está el receptor) la onda tiene
\(\lambda_{\text{detrás}} \approx \SI{1,23}{m}\) (del Ejemplo 16.14).
Como el aire está en reposo, la onda viaja a \(v = \SI{340}{m/s}\), así:
\[
f_L = \frac{v}{\lambda_{\text{detrás}}}
    = \frac{340\ \si{m/s}}{1{,}23\ \si{m}}
    \approx \SI{276}{Hz}.
\]

Interpretación didáctica:
Cuando la fuente se aleja, cada cresta tarda más en llegar $\Rightarrow$ frecuencia percibida disminuye.

% ==========================================================
\end{document}
