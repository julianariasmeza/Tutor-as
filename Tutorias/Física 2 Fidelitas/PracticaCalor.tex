\documentclass[11pt]{article}

% ============================================================
% PAQUETES Y CONFIGURACIÓN
% ============================================================
\usepackage[utf8]{inputenc}
\usepackage[T1]{fontenc}
\usepackage[spanish]{babel}
\usepackage{amsmath, amssymb}
\usepackage{graphicx}
\usepackage{siunitx}
\usepackage[a4paper, margin=2.5cm]{geometry}

\sisetup{
  locale = DE,
  output-decimal-marker = {,},  % coma decimal
  per-mode = symbol,
  exponent-product = \cdot,
  group-minimum-digits = 4
}

% ============================================================
% DOCUMENTO
% ============================================================
\begin{document}

\begin{center}
{\Large \textbf{Práctica de ejercicios sobre equilibrio térmico y calorimetría}}\\[4pt]
\textit{Física — Transferencia de calor y mezclas térmicas}\\[6pt]
\end{center}

% ============================================================
\section*{Ejercicio 1}
Una barra metálica de \SI{250}{g} a \SI{95,0}{\celsius} se introduce en un calorímetro con \SI{180}{g} de agua a \SI{22,0}{\celsius}.  
Si la temperatura final del sistema es de \SI{27,0}{\celsius} y se desprecia el calor del calorímetro, determine el calor específico del metal.

% ============================================================
\section*{Ejercicio 2}
Se mezclan \SI{20}{g} de leche a \SI{15,0}{\celsius} con \SI{180}{g} de café a \SI{85,0}{\celsius}.  
Suponiendo que ambos líquidos tienen el mismo calor específico del agua (\(c = \SI{4186}{\joule\per\kilogram\per\celsius}\)) y que no hay pérdida de calor, determine la temperatura final de equilibrio.

% ============================================================
\section*{Ejercicio 3}
Un recipiente de aluminio de \SI{25}{g} contiene \SI{60}{g} de agua a \SI{20,0}{\celsius}.  
Se introduce un cuerpo de cobre de \SI{40}{g} a \SI{90,0}{\celsius}.  
Determine la temperatura de equilibrio final del sistema.  
Use \(c_\text{agua} = \SI{4186}{\joule\per\kilogram\per\celsius}\) y \(c_\text{Cu} = \SI{387}{\joule\per\kilogram\per\celsius}\), \(c_\text{Al} = \SI{900}{\joule\per\kilogram\per\celsius}\).

% ============================================================
\section*{Ejercicio 4}
En un calorímetro de cobre de \SI{0,200}{kg} se vierten \SI{0,150}{kg} de agua a \SI{18,0}{\celsius}.  
Se coloca una muestra desconocida de \SI{0,080}{kg} a \SI{100,0}{\celsius}.  
Si la temperatura final es de \SI{24,0}{\celsius}, determine el calor específico del material.  
Use \(c_\text{agua} = \SI{4186}{\joule\per\kilogram\per\celsius}\) y \(c_\text{Cu} = \SI{387}{\joule\per\kilogram\per\celsius}\).

% ============================================================
\section*{Ejercicio 5}
Una taza contiene \SI{0,300}{kg} de agua a \SI{80,0}{\celsius}.  
Se agregan cubos de hielo a \SI{-10,0}{\celsius}.  
¿Cuántos kilogramos de hielo deben colocarse para que la temperatura final del sistema sea \SI{30,0}{\celsius}?  
Use \(c_\text{hielo} = \SI{2090}{\joule\per\kilogram\per\celsius}\), \(c_\text{agua} = \SI{4186}{\joule\per\kilogram\per\celsius}\) y \(L_f = \SI{3,34e5}{\joule\per\kilogram}\).

% ============================================================
\section*{Ejercicio 6}
Un bloque de aluminio de \SI{0,050}{kg} a \SI{150,0}{\celsius} se sumerge en \SI{0,200}{kg} de aceite a \SI{25,0}{\celsius}.  
La temperatura final del sistema es de \SI{40,0}{\celsius}.  
Determine el calor específico del aceite.  
Use \(c_\text{Al} = \SI{900}{\joule\per\kilogram\per\celsius}\).

% ============================================================
\section*{Ejercicio 7}
Se mezcla \SI{0,100}{kg} de vapor de agua a \SI{100,0}{\celsius} con \SI{0,400}{kg} de agua líquida a \SI{20,0}{\celsius}.  
Determine la temperatura de equilibrio final y si todo el vapor se condensa.  
Use \(c_\text{agua} = \SI{4186}{\joule\per\kilogram\per\celsius}\), \(L_v = \SI{2,26e6}{\joule\per\kilogram}\).

% ============================================================
\section*{Ejercicio 8}
Una pieza de hierro de \SI{0,300}{kg} a \SI{180,0}{\celsius} se introduce en \SI{0,250}{kg} de agua a \SI{25,0}{\celsius}.  
Determine la temperatura final del equilibrio térmico.  
Use \(c_\text{Fe} = \SI{450}{\joule\per\kilogram\per\celsius}\) y \(c_\text{agua} = \SI{4186}{\joule\per\kilogram\per\celsius}\).

% ============================================================
\section*{Ejercicio 9}
Una muestra desconocida de masa \SI{0,120}{kg} se calienta hasta \SI{95,0}{\celsius} y se introduce en \SI{0,200}{kg} de agua a \SI{20,0}{\celsius}.  
Si la temperatura de equilibrio es \SI{25,4}{\celsius}, determine el calor específico del material.  
Use \(c_\text{agua} = \SI{4186}{\joule\per\kilogram\per\celsius}\).

% ============================================================
\section*{Ejercicio 10}
Se colocan \SI{0,150}{kg} de hielo a \SI{0,0}{\celsius} dentro de \SI{0,300}{kg} de agua a \SI{60,0}{\celsius}.  
Determine la temperatura final del sistema y si todo el hielo se derrite.  
Use \(c_\text{agua} = \SI{4186}{\joule\per\kilogram\per\celsius}\), \(L_f = \SI{3,34e5}{\joule\per\kilogram}\).
\newpage
\newpage
\section*{Soluciones}

% ============================================================
\subsection*{Ejercicio 1}

\textbf{Datos}
\begin{align*}
m_\text{metal} &= \SI{0,250}{\kilogram},\\
m_\text{agua} &= \SI{0,180}{\kilogram},\\
T_{i,\text{metal}} &= \SI{95,0}{\celsius},\\
T_{i,\text{agua}} &= \SI{22,0}{\celsius},\\
T_f &= \SI{27,0}{\celsius},\\
c_\text{agua} &= \SI{4186}{\joule\per\kilogram\per\celsius}.
\end{align*}

\textbf{Fórmulas}
\[
m_\text{metal}c_\text{metal}(T_{i,\text{metal}} - T_f)
= m_\text{agua}c_\text{agua}(T_f - T_{i,\text{agua}}).
\]
\[
c_\text{metal} =
\frac{m_\text{agua}c_\text{agua}(T_f - T_{i,\text{agua}})}
{m_\text{metal}(T_{i,\text{metal}} - T_f)}.
\]

\textbf{Sustitución}
\[
c_\text{metal} =
\frac{(0,180)(4186)(27,0 - 22,0)}
{(0,250)(95,0 - 27,0)}.
\]

\textbf{Cálculo}
\[
\begin{aligned}
c_\text{metal}
&= \frac{(0,180)(4186)(5,0)}{(0,250)(68,0)}\\[4pt]
&= \frac{(0,180)(20\,930)}{17,0}\\[4pt]
&= \frac{3767,4}{17,0}\\[4pt]
&\approx \SI{2,22e2}{\joule\per\kilogram\per\celsius}.
\end{aligned}
\]

\textbf{Conclusión}
\[
\boxed{c_\text{metal} \approx \SI{2,22e2}{\joule\per\kilogram\per\celsius}}.
\]


% ============================================================
\subsection*{Ejercicio 2}

\textbf{Datos}
\begin{align*}
m_\text{leche} &= \SI{0,020}{\kilogram}, &
T_{i,\text{leche}} &= \SI{15,0}{\celsius},\\
m_\text{café} &= \SI{0,180}{\kilogram}, &
T_{i,\text{café}} &= \SI{85,0}{\celsius},\\
c &= \SI{4186}{\joule\per\kilogram\per\celsius}.
\end{align*}

\textbf{Fórmulas}
\[
m_\text{café}c(T_{i,\text{café}} - T_f)
= m_\text{leche}c(T_f - T_{i,\text{leche}}).
\]
Se simplifica \(c\) y se despeja:
\[
m_\text{café}(T_{i,\text{café}} - T_f)
= m_\text{leche}(T_f - T_{i,\text{leche}}).
\]
\[
T_f = \frac{m_\text{café}T_{i,\text{café}} + m_\text{leche}T_{i,\text{leche}}}
{m_\text{café} + m_\text{leche}}.
\]

\textbf{Sustitución}
\[
T_f = \frac{(0,180)(85,0) + (0,020)(15,0)}{0,180 + 0,020}.
\]

\textbf{Cálculo}
\[
\begin{aligned}
T_f &= \frac{15,3 + 0,3}{0,200}\\[4pt]
&= \frac{15,6}{0,200}\\[4pt]
&= \SI{78,0}{\celsius}.
\end{aligned}
\]

\textbf{Conclusión}
\[
\boxed{T_f = \SI{78,0}{\celsius}}.
\]


% ============================================================
\subsection*{Ejercicio 3}

\textbf{Datos}
\begin{align*}
m_\text{Al} &= \SI{0,025}{\kilogram}, &
T_{i,\text{Al}} &= \SI{20,0}{\celsius},\\
m_\text{agua} &= \SI{0,060}{\kilogram}, &
T_{i,\text{agua}} &= \SI{20,0}{\celsius},\\
m_\text{Cu} &= \SI{0,040}{\kilogram}, &
T_{i,\text{Cu}} &= \SI{90,0}{\celsius},\\
c_\text{Al} &= \SI{900}{\joule\per\kilogram\per\celsius},\\
c_\text{Cu} &= \SI{387}{\joule\per\kilogram\per\celsius},\\
c_\text{agua} &= \SI{4186}{\joule\per\kilogram\per\celsius}.
\end{align*}

\textbf{Fórmulas}
\[
m_\text{Cu}c_\text{Cu}(T_{i,\text{Cu}} - T_f)
= m_\text{agua}c_\text{agua}(T_f - T_{i,\text{agua}})
+ m_\text{Al}c_\text{Al}(T_f - T_{i,\text{Al}}).
\]

\textbf{Sustitución}
\[
(0,040)(387)(90,0 - T_f)
= (0,060)(4186)(T_f - 20,0)
+ (0,025)(900)(T_f - 20,0).
\]

\textbf{Cálculo}
\[
\begin{aligned}
15,48(90,0 - T_f)
&= 251,16(T_f - 20,0) + 22,50(T_f - 20,0)\\[4pt]
15,48(90,0 - T_f)
&= 273,66(T_f - 20,0).
\end{aligned}
\]

Desarrollando:
\[
\begin{aligned}
1393,2 - 15,48T_f &= 273,66T_f - 5473,2,\\[4pt]
1393,2 + 5473,2 &= 273,66T_f + 15,48T_f,\\[4pt]
6866,4 &= 289,14T_f,\\[4pt]
T_f &= \frac{6866,4}{289,14}
\approx \SI{23,7}{\celsius}.
\end{aligned}
\]

\textbf{Conclusión}
\[
\boxed{T_f \approx \SI{23,7}{\celsius}}.
\]


% ============================================================
\subsection*{Ejercicio 4}

\textbf{Datos}
\begin{align*}
m_\text{muestra} &= \SI{0,080}{\kilogram}, &
T_{i,\text{muestra}} &= \SI{100,0}{\celsius},\\
m_\text{agua} &= \SI{0,150}{\kilogram}, &
T_{i,\text{agua}} &= \SI{18,0}{\celsius},\\
m_\text{Cu} &= \SI{0,200}{\kilogram}, &
T_{i,\text{Cu}} &= \SI{18,0}{\celsius},\\
T_f &= \SI{24,0}{\celsius},\\
c_\text{agua} &= \SI{4186}{\joule\per\kilogram\per\celsius},\\
c_\text{Cu} &= \SI{387}{\joule\per\kilogram\per\celsius}.
\end{align*}

\textbf{Fórmulas}
\[
m_\text{muestra}c_\text{muestra}(T_{i,\text{muestra}} - T_f)
= m_\text{agua}c_\text{agua}(T_f - T_{i,\text{agua}})
+ m_\text{Cu}c_\text{Cu}(T_f - T_{i,\text{Cu}}).
\]
\[
c_\text{muestra} =
\frac{m_\text{agua}c_\text{agua}(T_f - T_{i,\text{agua}})
+ m_\text{Cu}c_\text{Cu}(T_f - T_{i,\text{Cu}})}
{m_\text{muestra}(T_{i,\text{muestra}} - T_f)}.
\]

\textbf{Sustitución}
\[
c_\text{muestra} =
\frac{(0,150)(4186)(24,0 - 18,0)
+ (0,200)(387)(24,0 - 18,0)}
{(0,080)(100,0 - 24,0)}.
\]

\textbf{Cálculo}
\[
\begin{aligned}
Q_\text{agua} &= (0,150)(4186)(6,0) = \SI{3770,4}{\joule},\\[2pt]
Q_\text{Cu}   &= (0,200)(387)(6,0) = \SI{464,4}{\joule},\\[2pt]
Q_\text{ganado} &= 3770,4 + 464,4 = \SI{4234,8}{\joule}.
\end{aligned}
\]
Denominador:
\[
(0,080)(100,0 - 24,0) = (0,080)(76,0) = 6,08.
\]
Entonces:
\[
c_\text{muestra} = \frac{4234,8}{6,08}
\approx \SI{6,96e2}{\joule\per\kilogram\per\celsius}.
\]

\textbf{Conclusión}
\[
\boxed{c_\text{muestra} \approx \SI{6,96e2}{\joule\per\kilogram\per\celsius}}.
\]


% ============================================================
\subsection*{Ejercicio 5}

\textbf{Datos}
\begin{align*}
m_\text{agua} &= \SI{0,300}{\kilogram}, &
T_{i,\text{agua}} &= \SI{80,0}{\celsius},\\
m_\text{hielo} &= m \quad (\text{incógnita}), &
T_{i,\text{hielo}} &= \SI{-10,0}{\celsius},\\
T_f &= \SI{30,0}{\celsius},\\
c_\text{agua} &= \SI{4186}{\joule\per\kilogram\per\celsius},\\
c_\text{hielo} &= \SI{2090}{\joule\per\kilogram\per\celsius},\\
L_f &= \SI{3,34e5}{\joule\per\kilogram}.
\end{align*}

\textbf{Fórmulas}
El agua pierde:
\[
Q_\text{perdido} = m_\text{agua}c_\text{agua}(T_{i,\text{agua}} - T_f).
\]
El hielo gana (tres etapas):
\[
Q_\text{ganado} = m\Big[c_\text{hielo}(0 - T_{i,\text{hielo}})
+ L_f + c_\text{agua}(T_f - 0)\Big].
\]
Balance:
\[
m_\text{agua}c_\text{agua}(T_{i,\text{agua}} - T_f)
= m\Big[c_\text{hielo}(0 - T_{i,\text{hielo}})
+ L_f + c_\text{agua}(T_f - 0)\Big].
\]

\textbf{Sustitución}
\[
(0,300)(4186)(80,0 - 30,0)
= m\Big[2090(10) + 3,34\cdot10^{5} + 4186(30)\Big].
\]

\textbf{Cálculo}
\[
\begin{aligned}
Q_\text{perdido} &= (0,300)(4186)(50)
= \SI{62790}{\joule},\\[4pt]
\text{corchetes} &= 2090(10) + 3,34\cdot10^{5} + 4186(30)\\
&= 20\,900 + 334\,000 + 125\,580\\
&= \SI{4,8048e5}{\joule\per\kilogram}.
\end{aligned}
\]
Entonces,
\[
m = \frac{62790}{4,8048\cdot10^{5}}
\approx \SI{1,31e-1}{\kilogram}.
\]

\textbf{Conclusión}
\[
\boxed{m_\text{hielo} \approx \SI{1,31e-1}{\kilogram} \approx \SI{0,131}{\kilogram}}
\]
(es decir, aproximadamente \(\SI{131}{\gram}\) de hielo).


% ============================================================
\subsection*{Ejercicio 6}

\textbf{Datos}
\begin{align*}
m_\text{Al} &= \SI{0,050}{\kilogram}, &
T_{i,\text{Al}} &= \SI{150,0}{\celsius},\\
m_\text{aceite} &= \SI{0,200}{\kilogram}, &
T_{i,\text{aceite}} &= \SI{25,0}{\celsius},\\
T_f &= \SI{40,0}{\celsius},\\
c_\text{Al} &= \SI{900}{\joule\per\kilogram\per\celsius},\\
c_\text{aceite} &= ? 
\end{align*}

\textbf{Fórmulas}
\[
m_\text{Al}c_\text{Al}(T_{i,\text{Al}} - T_f)
= m_\text{aceite}c_\text{aceite}(T_f - T_{i,\text{aceite}}).
\]
\[
c_\text{aceite} =
\frac{m_\text{Al}c_\text{Al}(T_{i,\text{Al}} - T_f)}
{m_\text{aceite}(T_f - T_{i,\text{aceite}})}.
\]

\textbf{Sustitución}
\[
c_\text{aceite} =
\frac{(0,050)(900)(150,0 - 40,0)}
{(0,200)(40,0 - 25,0)}.
\]

\textbf{Cálculo}
\[
\begin{aligned}
\text{Numerador} &= (0,050)(900)(110)
= 45(110) = 4950,\\[2pt]
\text{Denominador} &= (0,200)(15) = 3,0,\\[2pt]
c_\text{aceite} &= \frac{4950}{3,0}
= \SI{1,65e3}{\joule\per\kilogram\per\celsius}.
\end{aligned}
\]

\textbf{Conclusión}
\[
\boxed{c_\text{aceite} \approx \SI{1,65e3}{\joule\per\kilogram\per\celsius}}.
\]


% ============================================================
\subsection*{Ejercicio 7}

\textbf{Datos}
\begin{align*}
m_\text{vapor} &= \SI{0,100}{\kilogram}, &
T_{i,\text{vapor}} &= \SI{100,0}{\celsius},\\
m_\text{agua} &= \SI{0,400}{\kilogram}, &
T_{i,\text{agua}} &= \SI{20,0}{\celsius},\\
c_\text{agua} &= \SI{4186}{\joule\per\kilogram\per\celsius},\\
L_v &= \SI{2,26e6}{\joule\per\kilogram}.
\end{align*}

\textbf{Análisis}

1. \emph{Suposición:} todo el vapor se condensa y la temperatura final \(T_f < \SI{100}{\celsius}\).  
   El vapor liberaría:
   \[
   Q_\text{vapor} = m_\text{vapor}L_v 
   + m_\text{vapor}c_\text{agua}(100 - T_f).
   \]
   El agua absorbería:
   \[
   Q_\text{agua} = m_\text{agua}c_\text{agua}(T_f - 20).
   \]
   El balance con \(T_f\) como incógnita lleva a una solución
   \(T_f \approx \SI{144}{\celsius}\), lo cual es imposible
   (no puede superar \(\SI{100}{\celsius}\) en equilibrio con agua y vapor).

2. \emph{Conclusión:} no todo el vapor se condensa.  
   La temperatura final será \(T_f = \SI{100}{\celsius}\), con agua líquida
   y vapor en equilibrio.

\textbf{Cálculo de la masa que se condensa (opcional)}
Calentar el agua de \(\SI{20}{\celsius}\) a \(\SI{100}{\celsius}\) requiere:
\[
Q_\text{agua} = (0,400)(4186)(100 - 20)
= \SI{133952}{\joule}.
\]
Cada kilogramo de vapor que se condensa suministra \(L_v\), por lo que
la masa condensada es:
\[
m_\text{cond} = \frac{133952}{2,26\cdot10^{6}}
\approx \SI{5,93e-2}{\kilogram}.
\]

\textbf{Conclusión}
\[
\boxed{T_f = \SI{100}{\celsius}, \quad
m_\text{cond} \approx \SI{5,9e-2}{\kilogram}\ \text{y no todo el vapor se condensa}.}
\]


% ============================================================
\subsection*{Ejercicio 8}

\textbf{Datos}
\begin{align*}
m_\text{Fe} &= \SI{0,300}{\kilogram}, &
T_{i,\text{Fe}} &= \SI{180,0}{\celsius},\\
m_\text{agua} &= \SI{0,250}{\kilogram}, &
T_{i,\text{agua}} &= \SI{25,0}{\celsius},\\
c_\text{Fe} &= \SI{450}{\joule\per\kilogram\per\celsius},\\
c_\text{agua} &= \SI{4186}{\joule\per\kilogram\per\celsius}.
\end{align*}

\textbf{Fórmulas}
\[
m_\text{Fe}c_\text{Fe}(T_{i,\text{Fe}} - T_f)
= m_\text{agua}c_\text{agua}(T_f - T_{i,\text{agua}}).
\]

\textbf{Sustitución}
\[
(0,300)(450)(180,0 - T_f)
= (0,250)(4186)(T_f - 25,0).
\]

\textbf{Cálculo}
\[
\begin{aligned}
135(180,0 - T_f)
&= 1046,5(T_f - 25,0).
\end{aligned}
\]
Desarrollando:
\[
\begin{aligned}
24\,300 - 135T_f &= 1046,5T_f - 26\,162,5,\\[4pt]
24\,300 + 26\,162,5 &= 1046,5T_f + 135T_f,\\[4pt]
50\,462,5 &= 1181,5T_f,\\[4pt]
T_f &= \frac{50\,462,5}{1181,5}
\approx \SI{42,7}{\celsius}.
\end{aligned}
\]

\textbf{Conclusión}
\[
\boxed{T_f \approx \SI{42,7}{\celsius}}.
\]


% ============================================================
\subsection*{Ejercicio 9}

\textbf{Datos}
\begin{align*}
m_\text{muestra} &= \SI{0,120}{\kilogram}, &
T_{i,\text{muestra}} &= \SI{95,0}{\celsius},\\
m_\text{agua} &= \SI{0,200}{\kilogram}, &
T_{i,\text{agua}} &= \SI{20,0}{\celsius},\\
T_f &= \SI{25,4}{\celsius},\\
c_\text{agua} &= \SI{4186}{\joule\per\kilogram\per\celsius}.
\end{align*}

\textbf{Fórmulas}
\[
m_\text{muestra}c_\text{muestra}(T_{i,\text{muestra}} - T_f)
= m_\text{agua}c_\text{agua}(T_f - T_{i,\text{agua}}).
\]
\[
c_\text{muestra} =
\frac{m_\text{agua}c_\text{agua}(T_f - T_{i,\text{agua}})}
{m_\text{muestra}(T_{i,\text{muestra}} - T_f)}.
\]

\textbf{Sustitución}
\[
c_\text{muestra} =
\frac{(0,200)(4186)(25,4 - 20,0)}
{(0,120)(95,0 - 25,4)}.
\]

\textbf{Cálculo}
\[
\begin{aligned}
\text{Numerador} &= (0,200)(4186)(5,4)
= \SI{4520,88}{\joule},\\[2pt]
\text{Denominador} &= (0,120)(69,6)
= 8,352,\\[2pt]
c_\text{muestra} &= \frac{4520,88}{8,352}
\approx \SI{5,41e2}{\joule\per\kilogram\per\celsius}.
\end{aligned}
\]

\textbf{Conclusión}
\[
\boxed{c_\text{muestra} \approx \SI{5,41e2}{\joule\per\kilogram\per\celsius}}.
\]


% ============================================================
\subsection*{Ejercicio 10}

\textbf{Datos}
\begin{align*}
m_\text{hielo} &= \SI{0,150}{\kilogram}, &
T_{i,\text{hielo}} &= \SI{0,0}{\celsius},\\
m_\text{agua} &= \SI{0,300}{\kilogram}, &
T_{i,\text{agua}} &= \SI{60,0}{\celsius},\\
c_\text{agua} &= \SI{4186}{\joule\per\kilogram\per\celsius},\\
L_f &= \SI{3,34e5}{\joule\per\kilogram}.
\end{align*}

\textbf{Paso 1: Verificar si se derrite todo el hielo}

Calor máximo que puede ceder el agua al enfriarse hasta \(\SI{0}{\celsius}\):
\[
Q_\text{agua,max} = m_\text{agua}c_\text{agua}(60,0 - 0,0)
= (0,300)(4186)(60)
= \SI{75348}{\joule}.
\]

Calor necesario para derretir todo el hielo:
\[
Q_\text{fusión} = m_\text{hielo}L_f
= (0,150)(3,34\cdot 10^{5})
= \SI{50100}{\joule}.
\]

Como \(Q_\text{agua,max} > Q_\text{fusión}\), el agua puede derretir todo el hielo y aún sobra calor para calentar el agua resultante por encima de \(\SI{0}{\celsius}\).

\textbf{Paso 2: Temperatura final}

Después de la fusión, la masa total de agua es:
\[
m_\text{total} = 0,300 + 0,150 = \SI{0,450}{\kilogram}.
\]

Calor sobrante luego de derretir el hielo:
\[
Q_\text{sobrante} = Q_\text{agua,max} - Q_\text{fusión}
= 75348 - 50100
= \SI{25248}{\joule}.
\]

Este calor eleva la temperatura de \(\SI{0,450}{\kilogram}\) de agua desde
\(\SI{0}{\celsius}\) hasta \(T_f\):
\[
Q_\text{sobrante} = m_\text{total}c_\text{agua}T_f.
\]

\textbf{Sustitución}
\[
25248 = (0,450)(4186)T_f.
\]

\textbf{Cálculo}
\[
\begin{aligned}
(0,450)(4186) &= 1883,7,\\[2pt]
T_f &= \frac{25248}{1883,7}
\approx \SI{13,4}{\celsius}.
\end{aligned}
\]

\textbf{Conclusión}
\[
\boxed{T_f \approx \SI{13,4}{\celsius},\quad
\text{todo el hielo se derrite}.}
\]


\end{document}
