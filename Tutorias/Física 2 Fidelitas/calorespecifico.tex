\documentclass[11pt]{article}

% ============================================================
% PAQUETES BÁSICOS
% ============================================================
\usepackage[utf8]{inputenc}
\usepackage[T1]{fontenc}
\usepackage[spanish]{babel}
\usepackage{amsmath, amssymb}
\usepackage{graphicx}
\usepackage{siunitx}
\usepackage[a4paper, margin=2.5cm]{geometry}

% ============================================================
% CONFIGURACIÓN DE UNIDADES Y DECIMALES
% ============================================================
\sisetup{
  locale = DE,                  % para usar coma decimal
  output-decimal-marker = {,},  % separador decimal coma
  per-mode = symbol,            % estilo de unidades por símbolo
  exponent-product = \cdot,     % usa el punto centrado en potencias
  group-minimum-digits = 4
}

% ============================================================
% DOCUMENTO
% ============================================================
\begin{document}

\begin{center}
{\Large \textbf{Ejercicio 4. Determinación del calor específico de un cilindro}}\\[4pt]
\textit{Laboratorio de Física — Mezcla térmica}\\[6pt]
\end{center}

Se desea determinar el calor específico de un cilindro metálico mediante un experimento de mezcla térmica. 
Se desprecia el calor absorbido por el calorímetro.

% ------------------------------------------------------------
\subsection*{Datos}
\begin{align*}
m_\text{calorímetro} &= \SI{50}{g},\\
m_\text{cilindro} &= \SI{318}{g} = \SI{0,318}{kg},\\
m_\text{calorímetro + agua} &= \SI{222}{g} \Rightarrow 
m_\text{agua} = 222 - 50 = \SI{172}{g} = \SI{0,172}{kg},\\
T_{i,\text{cil}} &= \SI{95,6}{\celsius}, &
T_{f,\text{cil}} &= \SI{27,2}{\celsius},\\
T_{i,\text{agua}} &= \SI{23,3}{\celsius}, &
T_{f,\text{agua}} &= \SI{27,2}{\celsius},\\
c_\text{agua} &= \SI{4186}{\joule\per\kilogram\per\celsius}.
\end{align*}

% ------------------------------------------------------------
\subsection*{Fórmula}
El calor perdido por el cilindro es igual al calor ganado por el agua:
\[
m_\text{cil}\,c_\text{cil}\,(T_{i,\text{cil}} - T_{f,\text{cil}}) = 
m_\text{agua}\,c_\text{agua}\,(T_{f,\text{agua}} - T_{i,\text{agua}}).
\]
Despejando el calor específico del cilindro:
\[
c_\text{cil} =
\frac{m_\text{agua}\,c_\text{agua}\,(T_{f,\text{agua}} - T_{i,\text{agua}})}
{m_\text{cil}\,(T_{i,\text{cil}} - T_{f,\text{cil}})}.
\]

% ------------------------------------------------------------
\subsection*{Sustitución}
\[
c_\text{cil} =
\frac{(0,172)(4186)(27,2 - 23,3)}
{(0,318)(95,6 - 27,2)}.
\]

% ------------------------------------------------------------
\subsection*{Cálculo}
\[
\begin{aligned}
c_\text{cil} &=
\frac{(0,172)(4186)(3,9)}
{(0,318)(68,4)}\\[4pt]
&= \frac{(0,172)(16325,4)}{21,7512}\\[4pt]
&= \frac{2807,0}{21,7512}\\[4pt]
&= \SI{128,99}{\joule\per\kilogram\per\celsius}.
\end{aligned}
\]

% ------------------------------------------------------------
\subsection*{Interpretación}
El valor obtenido del calor específico es:
\[
\boxed{c_\text{cil} = \SI{129}{\joule\per\kilogram\per\celsius}}.
\]

Este resultado coincide estrechamente con el valor tabulado del \textbf{plomo} 
\[
(c_\text{Pb} \approx \SI{128}{\joule\per\kilogram\per\celsius}).
\]
Por tanto, la muestra desconocida corresponde probablemente a un cilindro de plomo.

\end{document}