% ==========================================================
% Tutoría de Física - Temperatura, equilibrio térmico y dilatación
% Formato con soluciones clickeables
% ==========================================================

\documentclass[11pt,letterpaper]{article}

% ---------------- Paquetes básicos ----------------
\usepackage[T1]{fontenc}
\usepackage[utf8]{inputenc}
\usepackage[spanish, es-nodecimaldot]{babel}

% Matemática y SI
\usepackage{amsmath, amssymb, bm}
\usepackage{siunitx}
\sisetup{
  locale = DE,                 % coma decimal
  output-decimal-marker = {,}, % coma decimal
  per-mode = symbol,
  exponent-product = \cdot,
  group-minimum-digits = 4
}

% Diseño de página
\usepackage[a4paper,margin=2.4cm]{geometry}
\setlength{\parindent}{0pt}
\setlength{\parskip}{6pt}

% Listas
\usepackage{enumitem}

% Cajas bonitas
\usepackage[most]{tcolorbox}
\tcbset{
  colback=white,
  colframe=black!70,
  boxrule=0.6pt,
  arc=2mm,
  left=6pt,right=6pt,top=6pt,bottom=6pt
}

% Encabezado y pie
\usepackage{fancyhdr}
\setlength{\headheight}{14.5pt}

% Botones / símbolos
\usepackage{pifont}

% Capas visibles/ocultas en PDF
\usepackage{ocgx2}

% Para símbolos unicode como ⇒
\usepackage{newunicodechar}
\newunicodechar{⇒}{\Rightarrow}

% Hipervínculos
\usepackage{hyperref}

% ---------------- Estilo de cajas ----------------
\newtcolorbox{ejercicio}[1][]{title=Ejercicio, #1}
\newtcolorbox{solucionbox}[1][]{title=Solución, colframe=green!60!black, #1}
\newtcolorbox{nota}[1][]{title=Nota, colframe=blue!60!black, #1}

% ---------------- Encabezado y pie ----------------
\pagestyle{fancy}
\fancyhf{}
\lhead{Tutoría Física — Temperatura y Dilatación térmica}
\rhead{Julián Arias Meza — 2025}
\cfoot{\thepage}

% ---------------- Entorno de solución clickeable ----------------
\newcounter{solnum}
\newcommand{\solbtn}{\fcolorbox{black}{gray!10}{\footnotesize Mostrar / ocultar solución}}
\newenvironment{solucionclick}{%
  \refstepcounter{solnum}%
  \par\noindent\textbf{Solución \#\thesolnum}\;
  \switchocg{sol:\thesolnum}{\solbtn}\\[2pt]%
  \begin{ocg}{Solución \#\thesolnum}{sol:\thesolnum}{0}%
  \begin{solucionbox}%
}{%
  \end{solucionbox}%
  \end{ocg}\par\medskip
}

% ---------------- Título reutilizable ----------------
\newcommand{\titulo}{
  \begin{center}
  {\Large \textbf{Tutoría de Física: Temperatura, Calor y Dilatación}}\\[4pt]
  {\large \textbf{Ley cero, escalas de temperatura, expansión térmica lineal y volumétrica}}\\[6pt]
  {\small Ejercicios tipo examen con soluciones ocultables}\\[6pt]
  \end{center}
  \vspace{-0.3em}\hrule\vspace{0.8em}
}

% ==========================================================
% ===================== DOCUMENTO ==========================
% ==========================================================
\begin{document}

\titulo

\begin{nota}
En este resumen trabajamos:
\begin{itemize}
  \item Qué es temperatura y qué significa equilibrio térmico.
  \item La ley cero de la termodinámica.
  \item Conversión entre escalas de temperatura: Celsius, Kelvin y Fahrenheit.
  \item Dilatación lineal de sólidos: \(\Delta L = \alpha L_0 \Delta T\).
  \item Dilatación volumétrica: \(\Delta V = \beta V_0 \Delta T\).
\end{itemize}

Cada ejercicio viene acompañado de una solución detallada paso a paso.  
Para mostrar / ocultar la solución: hacé clic en el recuadro gris.
\end{nota}

% ==========================================================
% 1. TEMPERATURA Y EQUILIBRIO TÉRMICO
% ==========================================================

\section*{1. Temperatura y equilibrio térmico}

\textbf{Idea clave:}  
La \textbf{temperatura} es una magnitud física que está relacionada con ``qué tan caliente'' o ``qué tan frío'' está un sistema \emph{y} determina la dirección del intercambio de energía térmica (calor).  
Cuando dos sistemas están en \textbf{equilibrio térmico}, ya no hay intercambio neto de energía por calor entre ellos.

\textbf{Ley cero de la termodinámica (forma práctica para laboratorio):}

Si el sistema A está en equilibrio térmico con el sistema C,  
y el sistema B está en equilibrio térmico con el mismo sistema C,  
entonces A y B están en equilibrio térmico entre sí.  

Conclusión útil: ¡Podemos usar el “sistema C’’ como \textbf{termómetro}! Si C no cambia al tocar A ni al tocar B, es que A y B tienen la misma temperatura.

% ---------- EJERCICIO 1 ----------
\begin{ejercicio}
Un termómetro se coloca primero en contacto con un bloque metálico caliente y se deja hasta que ambos alcanzan equilibrio térmico común a \(\SI{65,0}{^\circ C}\).  
Luego, el mismo termómetro se coloca en una taza de agua y, después de un rato, marca \(\SI{32,0}{^\circ C}\).

a) ¿Qué significa físicamente “alcanzar equilibrio térmico’’ entre el termómetro y el agua? \\
b) Según la ley cero de la termodinámica, ¿por qué confiamos en que la lectura del termómetro es la temperatura real del agua?
\end{ejercicio}

\begin{solucionclick}
\textbf{a) Equilibrio térmico}

Definición operativa: dos sistemas están en equilibrio térmico cuando \textbf{ya no hay intercambio neto de energía en forma de calor entre ellos}.  
Traducción al caso: el termómetro y el agua llegaron a la \textbf{misma} temperatura, \(\SI{32,0}{^\circ C}\).  
Antes de eso, hubo flujo de calor: el agua enfrió o calentó el termómetro hasta igualarlos.

\textbf{b) Ley cero}

La ley cero dice que si dos sistemas (termómetro y agua) alcanzan una misma temperatura común y dejan de intercambiar calor, entonces \textbf{esa} es la temperatura de ambos.  
Por eso podemos usar el termómetro como referencia externa: cuando deja de cambiar su lectura, esa lectura \(\SI{32,0}{^\circ C}\) es también la temperatura del agua.

\textbf{Interpretación:}  
La ley cero justifica la existencia de “temperatura’’ como magnitud física mensurable y permite construir escalas de temperatura con termómetros.
\end{solucionclick}

% ==========================================================
% 2. ESCALAS DE TEMPERATURA (°C, K, °F)
% ==========================================================

\section*{2. Escalas de temperatura}

Recordemos:

\[
T(\si{K}) = T(^{\circ}\text{C}) + 273{,}15
\]

\[
T(^{\circ}\text{F}) = \frac{9}{5} T(^{\circ}\text{C}) + 32
\qquad\Longleftrightarrow\qquad
T(^{\circ}\text{C}) = \frac{5}{9}\,\bigl[T(^{\circ}\text{F}) - 32\bigr]
\]

Importante: en Física usamos mucho la escala Kelvin porque es absoluta (0 K corresponde al “cero absoluto”: mínima energía térmica posible).

% ---------- EJERCICIO 2 ----------
\begin{ejercicio}
Convierta las siguientes temperaturas:

a) La temperatura más baja registrada en cierto laboratorio criogénico fue \(-120{,}0\,^{\circ}\text{C}\).  
Exprese ese valor en kelvin (K) y en grados Fahrenheit.

b) La fiebre de una paciente es \(\SI{39,5}{^\circ C}\).  
Exprese esa temperatura en kelvin y en grados Fahrenheit.

c) Una incubadora biomédica está ajustada a \(\SI{310}{K}\).  
¿Cuál es esa temperatura en \({^\circ}\text{C}\) y \({^\circ}\text{F}\)?
\end{ejercicio}

\begin{solucionclick}
\textbf{a) \(-120{,}0\,^{\circ}\text{C}\)}

Primero a kelvin:
\[
T(\si{K}) = T(^{\circ}\text{C}) + 273{,}15
= -120{,}0 + 273{,}15
= 153{,}15\ \si{K}.
\]

Ahora a Fahrenheit:
\[
T(^{\circ}\text{F})
= \frac{9}{5}(-120{,}0) + 32
= (-216{,}0) + 32
= -184{,}0\ ^{\circ}\text{F}.
\]

\textbf{b) \(\SI{39,5}{^\circ C}\)}

A kelvin:
\[
T(\si{K}) = 39{,}5 + 273{,}15
= 312{,}65\ \si{K}.
\]

A Fahrenheit:
\[
T(^{\circ}\text{F})
= \frac{9}{5}(39{,}5) + 32
= (71{,}1) + 32
= 103{,}1\ ^{\circ}\text{F}\ (\text{aprox.}).
\]

\textbf{c) \(\SI{310}{K}\)}

Primero a Celsius:
\[
T(^{\circ}\text{C})
= T(\si{K}) - 273{,}15
= 310 - 273{,}15
= 36{,}85\ ^{\circ}\text{C}
\approx 36{,}9\ ^{\circ}\text{C}.
\]

Ahora a Fahrenheit:
\[
T(^{\circ}\text{F})
= \frac{9}{5}(36{,}85) + 32
\approx 66{,}33 + 32
\approx 98{,}3\ ^{\circ}\text{F}.
\]

\textbf{Interpretación:}  
\(\sim 37{,}0\,^{\circ}\text{C}\) es temperatura corporal normal. Kelvin es muy usado en física; Fahrenheit es común en contexto biomédico en EEUU.
\end{solucionclick}

% ==========================================================
% 3. DILATACIÓN LINEAL
% ==========================================================

\section*{3. Dilatación lineal de sólidos}

Cuando aumenta la temperatura, la mayoría de los sólidos se expanden.  
Para cambios de temperatura moderados (digamos menos de \(\sim 100\,^{\circ}\text{C}\)) la \textbf{elongación} aproximada es:

\[
\Delta L = \alpha L_0 \,\Delta T
\]

donde:
\begin{itemize}
  \item \(L_0\): longitud inicial,
  \item \(\Delta L\): cambio de longitud,
  \item \(\alpha\): coeficiente de dilatación lineal \([\si{K^{-1}}]\),
  \item \(\Delta T = T_f - T_0\): cambio de temperatura.
\end{itemize}

También podemos escribir la longitud final:
\[
L_f = L_0 + \Delta L
= L_0 + \alpha L_0 \Delta T
= L_0 (1 + \alpha \Delta T).
\]

Valores típicos:  
aluminio \(\alpha \sim 2{,}4\times10^{-5}\ \si{K^{-1}}\),  
acero \(\alpha \sim 1{,}2\times10^{-5}\ \si{K^{-1}}\).

% ---------- EJERCICIO 3 ----------
\begin{ejercicio}
Una cinta métrica de acero mide exactamente \(\SI{50,000}{m}\) a \(\SI{20,0}{^\circ C}\).  
El coeficiente de dilatación lineal del acero es \(\alpha = 1{,}2\times10^{-5}\ \si{K^{-1}}\).

a) ¿Cuál será la longitud de la cinta a \(\SI{35,0}{^\circ C}\)?  
b) Explique por qué esto es importante en topografía / construcción.
\end{ejercicio}

\begin{solucionclick}
\textbf{Paso 1. Datos}

\[
L_0 = \SI{50,000}{m}, \quad
T_0 = \SI{20,0}{^\circ C}, \quad
T_f = \SI{35,0}{^\circ C},
\]
\[
\alpha = 1{,}2\times10^{-5}\ \si{K^{-1}}.
\]

Cambio de temperatura:
\[
\Delta T = T_f - T_0
= 35{,}0 - 20{,}0
= \SI{15,0}{K}.
\]

(Ojo: un cambio de \(\SI{15,0}{^\circ C}\) es también \(\SI{15,0}{K}\) en magnitud.)

\textbf{Paso 2. Longitud final usando } \(L_f = L_0(1+\alpha\Delta T)\)

\[
L_f
= (50{,}000\ \si{m})
\left[1 + (1{,}2\times10^{-5}\ \si{K^{-1}})(15{,}0\ \si{K})\right].
\]

Primero el producto:
\[
(1{,}2\times10^{-5})(15{,}0)
= 1{,}8\times10^{-4}.
\]

Entonces:
\[
L_f
= 50{,}000\ \si{m}
\left(1 + 1{,}8\times10^{-4}\right)
= 50{,}000\ \si{m}
\left(1{,}00018\right).
\]

\[
L_f \approx 50{,}009\ \si{m}.
\]

El aumento:
\[
\Delta L = L_f - L_0
\approx 50{,}009\ \si{m} - 50{,}000\ \si{m}
= 0{,}009\ \si{m}
= \SI{9,0}{mm}.
\]

\textbf{b) Interpretación física:}  
Si el topógrafo usa la cinta “a ojo” como si siguiera midiendo exactamente \(\SI{50,000}{m}\), en realidad está introduciendo un error de casi \(\SI{1}{cm}\).  
En ingeniería civil y puentes largos, ese error importa muchísimo.
\end{solucionclick}

% ==========================================================
% 4. APLICACIÓN: REMACHES EN AVIONES
% ==========================================================

\section*{4. Ajuste por contracción térmica (remaches)}

\begin{ejercicio}
En la fabricación de aviones se usan remaches de aluminio ligeramente más grandes que el orificio donde deben entrar, para que al calentarse en vuelo queden firmes.

Un remache debe ajustarse en un agujero de diámetro \(\SI{0,3500}{cm}\) en el fuselaje.  
A temperatura ambiente \(\SI{23}{^\circ C}\), el remache es \textbf{demasiado grande} para entrar.  
Se enfría el remache con “hielo seco’’ hasta \(\SI{-78}{^\circ C}\) para que se contraiga.

Suponga que el aluminio tiene \(\alpha = 2{,}4\times10^{-5}\ \si{K^{-1}}\).

Pregunta:  
¿Qué \textbf{diámetro máximo} puede tener el remache a \(\SI{23}{^\circ C}\) para que, al enfriarlo a \(\SI{-78}{^\circ C}\), su diámetro baje justo a \(\SI{0,3500}{cm}\) y pueda entrar?
\end{ejercicio}

\begin{solucionclick}
\textbf{Paso 1. Modelo físico}

El diámetro del remache se comporta igual que una longitud lineal \(L\).  
Usamos:
\[
L_f = L_0 (1 + \alpha \Delta T),
\]
donde:
\[
L_0 = \text{diámetro inicial a } T_0=23{^\circ}\text{C},
\quad
L_f = \text{diámetro final a } T_f=-78{^\circ}\text{C}.
\]

Queremos que \(L_f = \SI{0,3500}{cm}\).

\textbf{Paso 2. Calculamos \(\Delta T\)}

\[
\Delta T = T_f - T_0
= (-78) - 23
= -101\ ^{\circ}\text{C}
= -101\ \si{K}.
\]

\textbf{Paso 3. Planteo con números}

\[
L_f = L_0 (1 + \alpha \Delta T)
\quad\Longrightarrow\quad
L_0 = \frac{L_f}{1 + \alpha \Delta T}.
\]

Sustituimos:
\[
L_0
= \frac{0{,}3500\ \si{cm}}
       {1 + (2{,}4\times10^{-5}\ \si{K^{-1}})(-101\ \si{K}) }.
\]

Multiplicamos \(\alpha \Delta T\):
\[
(2{,}4\times10^{-5})(-101)
\approx -2{,}424\times10^{-3}
= -0{,}002424.
\]

Entonces:
\[
1 + \alpha\Delta T
\approx 1 - 0{,}002424
= 0{,}997576.
\]

\[
L_0
= \frac{0{,}3500\ \si{cm}}{0{,}997576}
\approx 0{,}3509\ \si{cm}.
\]

\textbf{Resultado:}  
El remache a \(\SI{23}{^\circ C}\) puede tener hasta \(\boxed{0{,}3509\ \text{ cm} \text{ de diámetro}}\).

\textbf{Interpretación física:}  
A temperatura ambiente es un poquito más ancho que el agujero (\(0{,}3509\ \text{ cm} > 0{,}3500\ \text{ cm}\)).  
Pero al enfriarlo \(\sim 100\,^{\circ}\text{C}\) se contrae lo suficiente para caber.  
Luego al calentarse otra vez en el avión vuelve a expandirse y queda ajustado.
\end{solucionclick}

% ==========================================================
% 5. DILATACIÓN VOLUMÉTRICA
% ==========================================================

\section*{5. Dilatación volumétrica (líquidos y sólidos)}

Para volúmenes (tanques, líquidos, bloques 3D) usamos:
\[
\Delta V = \beta V_0 \Delta T
\qquad\Longrightarrow\qquad
V_f = V_0 (1 + \beta \Delta T),
\]
donde \(\beta\) es el coeficiente de dilatación volumétrica \([\si{K^{-1}}]\).

Para muchos sólidos aproximadamente \(\beta \approx 3\alpha\), pero los líquidos suelen tener \(\beta\) mucho mayor.

% ---------- EJERCICIO 5 ----------
\begin{ejercicio}
Un recipiente de vidrio tiene volumen interno \(\,V_0 = \SI{1,000}{L}\) a \(\SI{20,0}{^\circ C}\).  
El coeficiente de dilatación volumétrica del vidrio es \(\beta_{\text{vidrio}} = 1{,}2\times10^{-5}\ \si{K^{-1}}\).  
Está lleno completamente de un aceite cuyo coeficiente volumétrico es \(\beta_{\text{aceite}} = 7{,}5\times10^{-4}\ \si{K^{-1}}\).

Si ambos se calientan uniformemente hasta \(\SI{50,0}{^\circ C}\):

a) ¿Cuál es el nuevo volumen del recipiente? \\
b) ¿Cuál sería el nuevo volumen del aceite si pudiera expandirse libremente? \\
c) ¿Se derrama aceite? Si sí, ¿cuánto?
\end{ejercicio}

\begin{solucionclick}
\textbf{Paso 1. Datos iniciales}

\[
V_0 = \SI{1,000}{L} = 1{,}000\ \text{L},
\quad
T_0 = \SI{20,0}{^\circ C},
\quad
T_f = \SI{50,0}{^\circ C}.
\]

\[
\Delta T = T_f - T_0 = 50,0 - 20,0 = \SI{30,0}{K}.
\]

Coeficientes:
\[
\beta_{\text{vidrio}} = 1{,}2\times10^{-5}\ \si{K^{-1}},
\quad
\beta_{\text{aceite}} = 7{,}5\times10^{-4}\ \si{K^{-1}}.
\]

\textbf{Paso 2. Volumen del recipiente (vidrio)}

\[
V_{\text{recip},f}
= V_0 \bigl(1 + \beta_{\text{vidrio}} \Delta T \bigr)
= 1{,}000\ \text{L}\,
\Bigl[1 + (1{,}2\times10^{-5})(30,0)\Bigr].
\]

Calculamos:
\[
(1{,}2\times10^{-5})(30,0)
= 3{,}6\times10^{-4}
= 0{,}00036.
\]

\[
V_{\text{recip},f}
= 1{,}000\ \text{L} \cdot (1{,}00036)
\approx 1{,}00036\ \text{L}.
\]

\textbf{Paso 3. Volumen del aceite libre}

\[
V_{\text{aceite},f}
= V_0 \bigl(1 + \beta_{\text{aceite}} \Delta T \bigr)
= 1{,}000\ \text{L}\,
\Bigl[1 + (7{,}5\times10^{-4})(30,0)\Bigr].
\]

\[
(7{,}5\times10^{-4})(30,0)
= 2{,}25\times10^{-2}
= 0{,}0225.
\]

\[
V_{\text{aceite},f}
= 1{,}000\ \text{L} \cdot (1{,}0225)
= 1{,}0225\ \text{L}.
\]

\textbf{Paso 4. ¿Se derrama?}

El recipiente “ahora’’ sólo tiene \(\approx 1{,}00036\ \text{L}\) de espacio,  
pero el aceite “quiere’’ ocupar \(\approx 1{,}0225\ \text{L}\).

Exceso:
\[
V_{\text{exceso}}
= 1{,}0225\ \text{L} - 1{,}00036\ \text{L}
\approx 0{,}02214\ \text{L}.
\]

Eso es \(\approx \SI{22,1}{mL}\).

\textbf{Interpretación física:}  
Los líquidos suelen expandirse MUCHO más que el sólido que los contiene.  
Por eso, al calentar un líquido en un recipiente lleno “hasta el tope”, se derrama.
\end{solucionclick}
% ==========================================================
% 6. CALOR Y CAMBIOS DE TEMPERATURA (LEY DE ENERGÍA TÉRMICA)
% ==========================================================

\section*{6. Calor y energía transferida}

Cuando un cuerpo recibe o cede calor sin cambio de fase, su temperatura varía según:

\[
Q = m\,c\,\Delta T
\]

donde:
\begin{itemize}
  \item \(Q\): cantidad de calor (energía térmica transferida) \([\si{J}]\),
  \item \(m\): masa del cuerpo \([\si{kg}]\),
  \item \(c\): calor específico del material \([\si{J/(kg\cdot K)}]\),
  \item \(\Delta T = T_f - T_0\): variación de temperatura.
\end{itemize}

Convenciones:
\[
Q>0 \text{ si el cuerpo gana calor,} \qquad Q<0 \text{ si lo pierde.}
\]

% ---------- EJERCICIO 6 ----------
\begin{ejercicio}
Se mezclan \(\SI{200}{g}\) de agua a \(\SI{80,0}{^\circ C}\) con \(\SI{300}{g}\) de agua a \(\SI{20,0}{^\circ C}\) en un recipiente aislado térmicamente.  
Suponiendo que no hay pérdidas de calor al ambiente, determine la \textbf{temperatura final de equilibrio} del sistema.  
Use \(c_{\text{agua}} = \SI{4186}{J/(kg\cdot K)}.\)
\end{ejercicio}

\begin{solucionclick}
\textbf{Paso 1. Datos y planteo físico}

\[
m_1 = 0{,}200\ \si{kg}, \quad T_1 = 80{,}0\,^{\circ}\text{C},
\]
\[
m_2 = 0{,}300\ \si{kg}, \quad T_2 = 20{,}0\,^{\circ}\text{C}.
\]

El recipiente es adiabático \(\Rightarrow\)el calor cedido por el agua caliente = calor absorbido por el agua fría:

\[
Q_{\text{perdido}} + Q_{\text{ganado}} = 0.
\]

\[
m_1 c (T_f - T_1) + m_2 c (T_f - T_2) = 0.
\]

\textbf{Paso 2. Simplificamos (el mismo $c$ se cancela):}

\[
m_1(T_f - T_1) + m_2(T_f - T_2) = 0.
\]

\[
m_1 T_f - m_1 T_1 + m_2 T_f - m_2 T_2 = 0.
\]

\[
T_f (m_1 + m_2) = m_1 T_1 + m_2 T_2.
\]

\textbf{Paso 3. Sustituimos valores numéricos:}

\[
T_f = \frac{m_1 T_1 + m_2 T_2}{m_1 + m_2}
= \frac{(0{,}200)(80{,}0) + (0{,}300)(20{,}0)}{0{,}200 + 0{,}300}.
\]

\[
T_f = \frac{16{,}0 + 6{,}0}{0{,}500}
= \frac{22{,}0}{0{,}500}
= 44{,}0\ ^{\circ}\text{C}.
\]

\textbf{Resultado:}  
\[
\boxed{T_f = 44{,}0\ ^{\circ}\text{C}}.
\]

\textbf{Interpretación física:}  
El equilibrio térmico se alcanza cuando el agua caliente cede tanto calor como el agua fría gana.  
La temperatura final cae entre ambas, más cercana a la de la masa mayor (\(\SI{300}{g}\) de agua fría \(\Rightarrow\) el resultado se inclina hacia los \(20^{\circ}\text{C}\)).
\end{solucionclick}

% =================== FIN DE LA PRÁCTICA ===================

\noindent\fbox{%
\parbox{\linewidth}{%
Esta práctica combina fundamentos de termodinámica temprana
(temperatura, ley cero) con modelos de ingeniería térmica
(dilatación lineal y volumétrica).  
Listo para PDF o para exportar a Word / plataforma virtual.

\medskip
\textbf{Contacto (WhatsApp):} \texttt{7076-9371}
}}
\end{document}
