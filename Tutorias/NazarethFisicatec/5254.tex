\documentclass[11pt,a4paper]{article}

%------------------------------------------------
% PAQUETES
%------------------------------------------------
\usepackage[spanish,es-nodecimaldot]{babel}
\usepackage[utf8]{inputenc}
\usepackage[T1]{fontenc}
\usepackage{amsmath,amssymb}
\usepackage{siunitx}
\usepackage[a4paper,margin=2.5cm]{geometry}

\sisetup{
  locale = DE,
  output-decimal-marker = {,},
  per-mode = symbol,
  exponent-product = \cdot,
  group-minimum-digits = 4
}

\begin{document}

%=================================================
\section*{Problema 8.52 – Centro de masa Sol–Júpiter}

\subsection*{Datos}

\begin{itemize}
  \item Masa del Sol:
  \[
    m_S = 1,99\times 10^{30}\ \text{kg}
  \]
  \item Masa de Júpiter:
  \[
    m_J = 1,90\times 10^{27}\ \text{kg}
  \]
  \item Distancia Sol–Júpiter:
  \[
    x_J = 7,78\times 10^{11}\ \text{m}
  \]
  \item Tomamos el origen en el centro del Sol:
  \[
    x_S = 0
  \]
\end{itemize}

\subsection*{Fórmula}

\[
x_{cm} = \frac{m_S x_S + m_J x_J}{m_S + m_J}
\]

\subsection*{Sustitución}

\[
x_{cm} = 
\frac{(1,90\times 10^{27})(7,78\times 10^{11})}
{1,99\times 10^{30} + 1,90\times 10^{27}}
\]

\subsection*{Cálculo}

\[
x_{cm} = \frac{1,478\times 10^{39}}
{1,9919\times 10^{30}}
= 7,42\times 10^{8}\ \text{m}
\]

\subsection*{Conclusión}

\[
\boxed{x_{cm} = 7,42\times 10^{8}\ \text{m}}
\]

El radio del Sol es:
\[
R_S = 6,96\times 10^{8}\ \text{m}
\]

Por lo tanto:
\[
x_{cm} > R_S
\]

El centro de masa del sistema Sol–Júpiter está \textbf{ligeramente fuera del Sol}.

%=================================================
\newpage

\section*{Problema 8.54 – Sistema de dos vehículos}

\subsection*{Datos}

\begin{itemize}
  \item Camioneta (A):
  \[
    m_A = 1200\ \text{kg},\quad v_A = 12,0\ \text{m/s},\quad x_A = 0
  \]
  \item Automóvil (B):
  \[
    m_B = 1800\ \text{kg},\quad v_B = 20,0\ \text{m/s},\quad x_B = 40,0\ \text{m}
  \]
  \item Masa total:
  \[
    M = m_A + m_B = 3000\ \text{kg}
  \]
\end{itemize}

\subsection*{(a) Posición del centro de masa}

\[
x_{cm} = \frac{m_A x_A + m_B x_B}{m_A + m_B}
\]

\[
x_{cm} = 
\frac{0 + (1800)(40)}
{1200 + 1800}
\]

\[
x_{cm} = \frac{72000}{3000} = 24,0\ \text{m}
\]

\[
\boxed{x_{cm} = 24,0\ \text{m}}
\]

El centro de masa está entre los vehículos.

\subsection*{(b) Cantidad de movimiento total}

\[
P_x = m_A v_A + m_B v_B
\]

\[
P_x = (1200)(12,0) + (1800)(20,0)
\]

\[
P_x = 14400 + 36000 = 50400\ \text{kg·m/s}
\]

\[
\boxed{P_x = 5,04\times 10^{4}\ \text{kg·m/s}}
\]

\subsection*{(c) Velocidad del centro de masa}

\[
v_{cm} = \frac{P_x}{M}
\]

\[
v_{cm} = \frac{50400}{3000}
\]

\[
\boxed{v_{cm} = 16,8\ \text{m/s}}
\]

\subsection*{(d) Verificación con \(P_x = M v_{cm}\)}

\[
P_x = (3000)(16,8)
\]

\[
P_x = 50400\ \text{kg·m/s}
\]

Coincide con el inciso (b).

\end{document}
