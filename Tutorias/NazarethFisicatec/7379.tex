\documentclass[11pt,a4paper]{article}

%------------------------------------------------
% PAQUETES
%------------------------------------------------
\usepackage[spanish,es-nodecimaldot]{babel}
\usepackage[utf8]{inputenc}
\usepackage[T1]{fontenc}
\usepackage{amsmath,amssymb}
\usepackage{siunitx}
\usepackage[a4paper,margin=2.5cm]{geometry}

\sisetup{
  locale = DE,
  output-decimal-marker = {,},
  per-mode = symbol,
  exponent-product = \cdot,
  group-minimum-digits = 4
}

\begin{document}

%=================================================
\section*{Problema 8.73 – Combinación de leyes de conservación}

\subsection*{Datos}

\begin{itemize}
  \item Masa del bloque en movimiento:
  \[
    m_1 = \SI{5,00}{kg}
  \]
  \item Velocidad inicial:
  \[
    v_1 = \SI{12,0}{m/s}
  \]
  \item Masa del bloque en reposo:
  \[
    m_2 = \SI{5,00}{kg}
  \]
  \item Velocidad inicial del segundo bloque:
  \[
    v_2 = 0
  \]
  \item Choque completamente inelástico: se adhieren.
\end{itemize}

\subsection*{Fórmula}

Conservación del momento lineal en el choque:
\[
m_1 v_1 = (m_1 + m_2) v_f.
\]

Después del choque, se conserva la energía mecánica porque no hay fricción:
\[
\frac{1}{2}(m_1+m_2) v_f^2 = (m_1+m_2) g h.
\]

\subsection*{Sustitución}

\[
(5,00)(12,0) = (10,0) v_f.
\]

\[
v_f = \frac{60}{10} = \SI{6,0}{m/s}.
\]

Ahora para la altura:
\[
\frac{1}{2}(10,0)(6,0)^2 = (10,0)(9,8) h.
\]

\subsection*{Cálculo}

\[
\frac{1}{2}(10)(36) = 180~\text{J},
\]

\[
180 = 98 h,
\]

\[
h = \frac{180}{98} = \SI{1,84}{m}.
\]

\subsection*{Conclusión}

Los bloques combinados alcanzan una altura:
\[
\boxed{h = \SI{1,8}{m}}
\]

%=================================================
\newpage

\section*{Problema 8.79 – Choque bala–bloque y resorte}

\subsection*{Datos}

\begin{itemize}
  \item Masa de la bala:
  \[
    m_b = \SI{8,00e-3}{kg}
  \]
  \item Masa del bloque:
  \[
    m_c = \SI{0,992}{kg}
  \]
  \item Compresión del resorte:
  \[
    x = \SI{15,0}{cm} = \SI{0,150}{m}
  \]
  \item Constante del resorte:
  \[
    F = \SI{0,750}{N}, \quad x_0 = \SI{0,250}{cm} = 2,50\times 10^{-3}\,\text{m}
  \]
\end{itemize}

Primero se calcula la constante elástica:
\[
k = \frac{F}{x_0}
    = \frac{0,750}{2,50\times10^{-3}}
    = \SI{300}{N/m}.
\]

\subsection*{(a) Velocidad del bloque después del impacto}

Conservación de energía para el resorte:
\[
\frac12 (m_b + m_c) V^2 = \frac12 k x^2.
\]

\subsection*{Sustitución}

\[
\frac12 (1,00) V^2 = \frac12 (300)(0,150)^2.
\]

\subsection*{Cálculo}

\[
V = 0,150 \sqrt{\frac{300}{1,00}} = \SI{2,60}{m/s}.
\]

\[
\boxed{V = \SI{2,60}{m/s}}
\]

\subsection*{(b) Velocidad inicial de la bala}

Conservación del momento lineal en el choque:
\[
m_b v_b = (m_b + m_c) V.
\]

\subsection*{Sustitución}

\[
(8,00\times10^{-3}) v_b = (1,00)(2,60).
\]

\[
v_b = \frac{2,60}{8,00\times 10^{-3}}
\]

\subsection*{Cálculo}

\[
v_b = \SI{325}{m/s}.
\]

\subsection*{Conclusión}

\[
\boxed{v_b = \SI{325}{m/s}}
\]

La bala inicialmente viajaba a aproximadamente \(\SI{325}{m/s}\).

\end{document}
