\documentclass[11pt,a4paper]{article}

%------------------------------------------------
% PAQUETES
%------------------------------------------------
\usepackage[spanish,es-nodecimaldot]{babel}
\usepackage[utf8]{inputenc}
\usepackage[T1]{fontenc}
\usepackage{amsmath,amssymb}
\usepackage{siunitx}
\usepackage[a4paper,margin=2.5cm]{geometry}

\sisetup{
  locale = DE,
  output-decimal-marker = {,},
  per-mode = symbol,
  exponent-product = \cdot,
  group-minimum-digits = 4
}

\begin{document}

\section*{Problema 8.17 – Cantidad de movimiento de los gases}

\subsection*{Datos}

\begin{itemize}
  \item Masa de la bala:
  \[
    m_b = \SI{0,00720}{kg}
  \]
  \item Rapidez de la bala respecto al cañón:
  \[
    v_{b/r} = \SI{601}{m/s}
  \]
  \item Masa del rifle:
  \[
    m_r = \SI{2,80}{kg}
  \]
  \item Velocidad del rifle respecto al suelo (retroceso):
  \[
    v_r = -\SI{1,85}{m/s}
  \]
  \item El sistema inicial (rifle + bala + gases) está en reposo:
  \[
    p_{\text{inicial}} = 0
  \]
\end{itemize}

Tomando el eje \(x\) en la dirección del movimiento de la bala,  
la rapidez de la bala respecto al suelo es
\[
v_b = v_{b/r} + v_r
    = \SI{601}{m/s} - \SI{1,85}{m/s}
    = \SI{599}{m/s}.
\]

\subsection*{Fórmula}

Se aplica conservación de la cantidad de movimiento en el eje \(x\):
\[
p_r + p_b + p_g = 0,
\]
donde \(p_r\) es el momento del rifle, \(p_b\) el de la bala y
\(p_g\) el de los gases.

Despejando la cantidad de movimiento de los gases:
\[
p_g = -\left(p_r + p_b\right).
\]

\subsection*{Sustitución}

Momento del rifle:
\[
p_r = m_r\,v_r = \left(\SI{2,80}{kg}\right)\left(-\SI{1,85}{m/s}\right).
\]

Momento de la bala:
\[
p_b = m_b\,v_b = \left(\SI{0,00720}{kg}\right)\left(\SI{599}{m/s}\right).
\]

Entonces:
\[
p_g = -\left[\,m_r\,v_r + m_b\,v_b\,\right].
\]

\subsection*{Cálculo}

Primero se calculan los momentos del rifle y de la bala:
\begin{align*}
p_r &= \left(\SI{2,80}{kg}\right)\left(-\SI{1,85}{m/s}\right)
     = -\SI{5,18}{kg\cdot m/s},\\[4pt]
p_b &= \left(\SI{0,00720}{kg}\right)\left(\SI{599}{m/s}\right)
     = \SI{4,31}{kg\cdot m/s}.
\end{align*}

Ahora la cantidad de movimiento de los gases:
\begin{align*}
p_g &= -\left(p_r + p_b\right) \\
    &= -\left(-\SI{5,18}{kg\cdot m/s} + \SI{4,31}{kg\cdot m/s}\right) \\
    &= -\left(-\SI{0,87}{kg\cdot m/s}\right) \\
    &= \SI{0,87}{kg\cdot m/s}.
\end{align*}

\subsection*{Interpretación}

La cantidad de movimiento de los gases al salir del cañón es
\[
\boxed{p_g = \SI{0,87}{kg\cdot m/s}},
\]
en la misma dirección en que se mueve la bala (sentido positivo del eje \(x\)).
Esto contribuye al retroceso del rifle para que se conserve
la cantidad de movimiento total del sistema.

\end{document}
