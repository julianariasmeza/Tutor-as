\documentclass[11pt,a4paper]{article}

%------------------------------------------------
% PAQUETES
%------------------------------------------------
\usepackage[spanish,es-nodecimaldot]{babel}
\usepackage[utf8]{inputenc}
\usepackage[T1]{fontenc}
\usepackage{amsmath,amssymb}
\usepackage{siunitx}
\usepackage[a4paper,margin=2.5cm]{geometry}

\sisetup{
  locale = DE,
  output-decimal-marker = {,},
  per-mode = symbol,
  exponent-product = \cdot,
  group-minimum-digits = 4
}

\begin{document}

%%%%%%%%%%%%%%%%%%%%%%%%%%%%%%%%%%%%%%%%%%%%%%%%%%%%%%%%%%%%
\section*{Problema 11.49 – Equilibrio de la viga y el letrero}
%%%%%%%%%%%%%%%%%%%%%%%%%%%%%%%%%%%%%%%%%%%%%%%%%%%%%%%%%%%%

\subsection*{Datos}

\begin{itemize}
    \item Masa de la viga: \(m_b = \SI{16,0}{kg}\)
    \item Longitud de la viga: \(L_b = \SI{1,50}{m}\)
    \item Masa del letrero: \(m_L = \SI{28,0}{kg}\)
    \item Longitud del letrero: \(\SI{1,20}{m}\)
    \item Distancia entre alambres: \(\SI{0,90}{m}\)
    \item Longitud de cada alambre: \(\SI{0,32}{m}\)
    \item Longitud del cable: \(L_c = \SI{2,00}{m}\)
    \item Ángulo del cable: \(\theta = 41,4^\circ\)
    \item Peso soportado por cada alambre:
    \[
    T_w = \frac{m_L g}{2} = \SI{137}{N}
    \]
\end{itemize}

\subsection*{Fórmulas}

\[
\sum \tau = 0,\qquad
\sum F_x = 0,\qquad
\sum F_y = 0
\]

\[
T_c \sin\theta,\qquad
T_c \cos\theta
\]

\subsection*{Sustitución}

\[
T_c (\sin\theta)(1,50)
- w_b(0,750)
- T_w(1,50)
- T_w(0,60)
= 0
\]

\subsection*{Cálculo}

\[
T_c =
\dfrac{(16,0)(9,80)(0,750) + 137(1,50+0,60)}
{1,50\sin41,4^\circ}
\]

\[
T_c = \SI{408,6}{N} \approx \SI{409}{N}
\]

\subsection*{Conclusión (a)}

\[
\boxed{T_c = \SI{409}{N}}
\]

%------------------------------------------
\subsection*{Fuerza vertical en la bisagra}

\[
\sum F_y = 0
\]

\[
F_v + T_c\sin\theta - w_b - 2T_w = 0
\]

\[
F_v = 2(137) + (16,0)(9,80) - (408,6)\sin41,4^\circ
\]

\[
F_v = \SI{161}{N}
\]

\subsection*{Conclusión (b)}

\[
\boxed{F_v = \SI{161}{N}}
\]

%%%%%%%%%%%%%%%%%%%%%%%%%%%%%%%%%%%%%%%%%%%%%%%%%%%%%%%%%%%%
\newpage
\section*{Problema 11.67 – Fuerzas al cargar una caja en escaleras}
%%%%%%%%%%%%%%%%%%%%%%%%%%%%%%%%%%%%%%%%%%%%%%%%%%%%%%%%%%%%

\subsection*{Datos}

\begin{itemize}
    \item Masa de la caja: \(m = \SI{200}{kg}\)
    \item Peso: \(w = mg = \SI{1960}{N}\)
    \item Longitud: \(\SI{1,25}{m}\)
    \item Altura: \(\SI{0,500}{m}\)
    \item Ángulo de inclinación: \(45^\circ\)
    \item Distancias:
    \[
    l_w = 0,375\cos45^\circ,
    \qquad
    l_2 = 1,25\cos45^\circ
    \]
\end{itemize}

\subsection*{Fórmulas}

\[
\sum F_y = 0 \quad\Rightarrow\quad F_1 + F_2 = w
\]

\[
\sum \tau_A = 0 \quad\Rightarrow\quad F_2 l_2 - w l_w = 0
\]

\subsection*{Sustitución}

\[
F_2 = w\left(\frac{l_w}{l_2}\right)
\]

\[
F_2 =
1960\left(
\frac{0,375\cos45^\circ}{1,25\cos45^\circ}
\right)
\]

\subsection*{Cálculo}

\[
F_2 = \SI{590}{N}
\]

\[
F_1 = w - F_2 = 1960 - 590 = \SI{1370}{N}
\]

\subsection*{Conclusión}

\[
\boxed{F_1 = \SI{1370}{N}}
\qquad
\boxed{F_2 = \SI{590}{N}}
\]

La persona inferior aplica más fuerza; es mejor ser la persona de arriba.

\end{document}
