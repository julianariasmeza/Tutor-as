% ==========================================================
% Práctica de Repaso — Trigonometría
% Basada en examen de selección única
% ==========================================================
\documentclass[11pt,letterpaper]{article}

% --- Paquetes y formato ---
\usepackage[T1]{fontenc}
\usepackage[utf8]{inputenc}
\usepackage[spanish, es-nodecimaldot]{babel}
\usepackage{amsmath,amssymb,siunitx}
\usepackage[a4paper,margin=2.4cm]{geometry}
\usepackage{setspace}
\usepackage{enumitem}
\usepackage{multicol}

\sisetup{
  locale=DE,
  output-decimal-marker={,},
  per-mode=symbol,
  exponent-product=\cdot,
  group-minimum-digits=4
}
\setlength{\parskip}{4pt}
\setlength{\parindent}{0pt}
\onehalfspacing

% --- Inicio ---
\begin{document}

\begin{center}
  {\Large \textbf{Práctica de Repaso — Trigonometría}}\\[2mm]
  \textit{Selección única • Nivel preuniversitario / universitario básico}\\[2mm]
  Tercer Cuatrimestre 2025
\end{center}

\hrule
\vspace{4mm}

\textbf{Instrucciones:} marque la alternativa correcta en cada ítem.  
Puede usar calculadora científica, pero justifique sus razonamientos en una hoja aparte.  
\bigskip

% ==========================================================
\section*{I. Ángulos y cuadrantes}
\begin{enumerate}[label=\textbf{\arabic*.}]

\item El punto de la circunferencia trigonométrica asociado al número real \(\tfrac{11\pi}{4}\) se localiza en el cuadrante:
\begin{multicols}{2}
A) I \\
B) II \\
C) III \\
D) IV
\end{multicols}

\item El ángulo \(-\tfrac{5\pi}{6}\) corresponde al cuadrante:
\begin{multicols}{2}
A) I \\
B) II \\
C) III \\
D) IV
\end{multicols}

\item Las coordenadas del punto de la circunferencia trigonométrica asociado al número real \(6\pi\) son:
\begin{multicols}{2}
A) (1,0) \\
B) (-1,0) \\
C) (0,1) \\
D) (0,-1)
\end{multicols}

\item Si \(x=\tfrac{17\pi}{3}\), ¿en qué cuadrante se localiza su punto asociado?
\begin{multicols}{2}
A) I \\
B) II \\
C) III \\
D) IV
\end{multicols}

\item Si \(\theta=-\tfrac{13\pi}{4}\), entonces su ángulo positivo coterminal es:
\begin{multicols}{2}
A) \(\tfrac{3\pi}{4}\) \\
B) \(\tfrac{7\pi}{4}\) \\
C) \(\tfrac{5\pi}{4}\) \\
D) \(\tfrac{9\pi}{4}\)
\end{multicols}

\end{enumerate}
\newpage
% ==========================================================
\section*{II. Razones trigonométricas}
\begin{enumerate}[label=\textbf{\arabic*.},resume]

\item Si \(\tan(\alpha)=-\tfrac{3}{4}\) y \(\sin(\alpha)>0\), entonces \(\cos(\alpha)=\)
\begin{multicols}{2}
A) \(\tfrac{4}{5}\) \\
B) \(-\tfrac{4}{5}\) \\
C) \(\tfrac{3}{5}\) \\
D) \(-\tfrac{3}{5}\)
\end{multicols}

\item Si \(\sec(\alpha)<0\) y \(\tan(\alpha)=\tfrac{5}{12}\), el valor de \(\sin(\alpha)\) es:
\begin{multicols}{2}
A) \(-\tfrac{5}{13}\) \\
B) \(\tfrac{5}{13}\) \\
C) \(-\tfrac{12}{13}\) \\
D) \(\tfrac{12}{13}\)
\end{multicols}

\item Si \(\pi<\alpha<\tfrac{3\pi}{2}\) y \(\cot(\alpha)=-1\), entonces \(\sin(2\alpha)=\)
\begin{multicols}{2}
A) \(-1\) \\
B) \(-\tfrac{1}{\sqrt{2}}\) \\
C) \(\tfrac{1}{\sqrt{2}}\) \\
D) \(1\)
\end{multicols}

\item Si \(\cos(\beta)=-\tfrac{1}{3}\), calcule \(\sec^2(\beta)-\tan^2(\beta)\).
\begin{multicols}{2}
A) 0 \\
B) 1 \\
C) 2 \\
D) 3
\end{multicols}

\end{enumerate}
\newpage
% ==========================================================
\section*{III. Identidades trigonométricas}
\begin{enumerate}[label=\textbf{\arabic*.},resume]

\item La expresión \(\sec(x)-\tan(x)\) es equivalente a:
\begin{multicols}{2}
A) \(\dfrac{1}{\sec(x)+\tan(x)}\) \\
B) \(\cos(x)\) \\
C) \(\sin(x)\) \\
D) \(\cos^2(x)\)
\end{multicols}

\item \(\dfrac{1+\cos(2x)}{\sin^2(2x)}\) es igual a:
\begin{multicols}{2}
A) \(\text{csc}^2(x)\) \\
B) \(\dfrac{\text{csc}^2(x)}{2}\) \\
C) \(\tan^2(x)\) \\
D) \(\sec^2(x)\)
\end{multicols}

\item \(\sin(\pi+u)-\sin(\pi-u)\) es equivalente a:
\begin{multicols}{2}
A) \(2\sin(u)\) \\
B) \(-2\sin(u)\) \\
C) \(-\sin^2(u)\) \\
D) \(0\)
\end{multicols}

\item \(\dfrac{1-\sin(x)}{1-\sin^2(x)}=\)
\begin{multicols}{2}
A) \(\dfrac{1}{1+\sin(x)}\) \\
B) \(\dfrac{1}{\cos(x)}\) \\
C) \(1-\sin(x)\) \\
D) \(\cos(x)\)
\end{multicols}

\item \(\dfrac{2\cos(\theta)}{\csc(\tfrac{\pi}{2}-\theta)}=\)
\begin{multicols}{2}
A) \(\sin(2\theta)\) \\
B) \(2\cos^2(\theta)\) \\
C) \(-2\cos^2(\theta)\) \\
D) \(2\sin(\theta)\)
\end{multicols}

\end{enumerate}

% ==========================================================
\newpage
\section*{IV. Funciones trigonométricas inversas}
\begin{enumerate}[label=\textbf{\arabic*.},resume]

\item El valor de \(\arccos\left(-\tfrac{\sqrt{2}}{2}\right)\) es:
\begin{multicols}{2}
A) \(\tfrac{\pi}{4}\) \\
B) \(\tfrac{3\pi}{4}\) \\
C) \(\tfrac{5\pi}{4}\) \\
D) \(-\tfrac{\pi}{4}\)
\end{multicols}

\item El valor de \(\arcsen\left[\cos\left(\tfrac{\pi}{3}\right)\right]\) es:
\begin{multicols}{2}
A) \(\tfrac{\pi}{6}\) \\
B) \(\tfrac{\pi}{3}\) \\
C) \(-\tfrac{\pi}{3}\) \\
D) \(-\tfrac{\pi}{6}\)
\end{multicols}

\item Si \(f(x)=\sec(x)\), una de sus asíntotas verticales está dada por:
\begin{multicols}{2}
A) \(x=\tfrac{\pi}{2}\) \\
B) \(x=\pi\) \\
C) \(x=\tfrac{3\pi}{2}\) \\
D) \(x=0\)
\end{multicols}

\item El rango de \(f(x)=\arcsen(x)\) es:
\begin{multicols}{2}
A) \([-\tfrac{\pi}{2},\tfrac{\pi}{2}]\) \\
B) \([0,\pi]\) \\
C) \((-\pi,\pi)\) \\
D) \([-\pi,\pi]\)
\end{multicols}

\end{enumerate}
\newpage
% ==========================================================
\section*{V. Ecuaciones trigonométricas}
\begin{enumerate}[label=\textbf{\arabic*.},resume]

\item En el intervalo \([0,\pi]\), ¿cuántas soluciones reales tiene \(\tan(x)\cdot\cos(x)+\cos(x)=0\)?
\begin{multicols}{2}
A) 0 \\
B) 1 \\
C) 2 \\
D) 3
\end{multicols}

\item En el intervalo \([0,2\pi]\), las soluciones de \(2\sin^2(x)=\sin(x)\) son:
\begin{multicols}{2}
A) \(0\) y \(\tfrac{\pi}{6}\) \\
B) \(0\) y \(\tfrac{5\pi}{6}\) \\
C) \(0\) y \(\tfrac{7\pi}{6}\) \\
D) \(0\) y \(\tfrac{11\pi}{6}\)
\end{multicols}

\item En \(\mathbb{R}\), el conjunto solución de \(\sec^2(x)-2\sec(x)+1=0\) es:
\begin{multicols}{2}
A) \(\{x\in\mathbb{R}\,/\,x=k\pi,\ k\in\mathbb{Z}\}\) \\
B) \(\{x\in\mathbb{R}\,/\,x=(2k+1)\pi,\ k\in\mathbb{Z}\}\) \\
C) \(\{x\in\mathbb{R}\,/\,x=2k\pi,\ k\in\mathbb{Z}\}\) \\
D) \(\emptyset\)
\end{multicols}

\item Si \(2\cos(x)+1=0\), una solución en el intervalo \([0,2\pi)\) es:
\begin{multicols}{2}
A) \(\tfrac{2\pi}{3}\) \\
B) \(\tfrac{4\pi}{3}\) \\
C) \(\tfrac{5\pi}{3}\) \\
D) \(\tfrac{\pi}{3}\)
\end{multicols}

\end{enumerate}
\newpage
% ==========================================================
\section*{VI. Razonamiento combinado}
\begin{enumerate}[label=\textbf{\arabic*.},resume]

\item Si \(\cos(x)=\tfrac{3}{5}\), halle \(\sin(2x)\).
\begin{multicols}{2}
A) \(\tfrac{12}{25}\) \\
B) \(\tfrac{24}{25}\) \\
C) \(\tfrac{7}{25}\) \\
D) \(\tfrac{9}{25}\)
\end{multicols}

\item Si \(\sin(x)=\tfrac{4}{5}\) y \(\cos(y)=\tfrac{5}{13}\), entonces \(\sin(x+y)=\)
\begin{multicols}{2}
A) \(\tfrac{56}{65}\) \\
B) \(\tfrac{33}{65}\) \\
C) \(\tfrac{60}{65}\) \\
D) \(\tfrac{12}{65}\)
\end{multicols}

\item Si \(\tan(x)=2\), calcule \(\tan(2x)\).
\begin{multicols}{2}
A) \(\tfrac{4}{3}\) \\
B) \(\tfrac{3}{4}\) \\
C) 2 \\
D) 4
\end{multicols}

\item Si \(\sin(x-y)=\tfrac{3}{5}\) y \(\cos(x+y)=\tfrac{4}{5}\), determine \(\sin(2x)\).
\begin{multicols}{2}
A) \(\tfrac{24}{25}\) \\
B) \(\tfrac{7}{25}\) \\
C) \(\tfrac{9}{25}\) \\
D) \(\tfrac{16}{25}\)
\end{multicols}

\end{enumerate}

% ==========================================================
\bigskip
\hrule
\vspace{3mm}
\textbf{Fin de la práctica.}\\
Verifique sus respuestas, revise los signos y los cuadrantes antes de entregar.
\newpage
% ==========================================================
% Sección de Respuestas — Práctica de Trigonometría
% (para anexar al final del documento)
% ==========================================================
\section*{Respuestas y justificaciones breves}

\begin{enumerate}[label=\textbf{\arabic*.}]
% ------------------ I. Ángulos y cuadrantes ------------------
\item \textbf{B}. \(\frac{11\pi}{4}=2\pi+\frac{3\pi}{4}\Rightarrow\) QII.
\item \textbf{C}. \(-\frac{5\pi}{6}=210^\circ\Rightarrow\) QIII.
\item \textbf{A}. \(6\pi=3\cdot 2\pi\Rightarrow (1,0)\).
\item \textbf{D}. \(\frac{17\pi}{3}-4\cdot\frac{3\pi}{3}=\frac{5\pi}{3}\Rightarrow\) QIV.
\item \textbf{A}. \(-\frac{13\pi}{4}+2\pi=-\frac{13\pi}{4}+\frac{8\pi}{4}=\frac{3\pi}{4}\).

% ------------------ II. Razones trigonométricas --------------
\setcounter{enumi}{5}
\item \textbf{B}. \(\tan\alpha=-\frac{3}{4}\), \(\sin\alpha>0\Rightarrow\) QII. \(|\cos|=\frac{4}{5}\Rightarrow \cos\alpha=-\frac{4}{5}\).
\item \textbf{A}. \(\tan\alpha=\frac{5}{12}>0\) y \(\sec\alpha<0\Rightarrow\) QIII \(\Rightarrow \sin\alpha<0\). \(|\sin|=\frac{5}{13}\Rightarrow \sin\alpha=-\frac{5}{13}\).
\item \textbf{A}. Si \(\cot\alpha=-1\Rightarrow\tan\alpha=-1\Rightarrow \sin(2\alpha)=\frac{2\tan\alpha}{1+\tan^2\alpha}=-1\).
\item \textbf{B}. Identidad pitagórica: \(\sec^2\beta-\tan^2\beta\equiv 1\).

% ------------------ III. Identidades -------------------------
\setcounter{enumi}{9}
\item \textbf{A}. \((\sec x-\tan x)(\sec x+\tan x)=1\Rightarrow \sec x-\tan x=\frac{1}{\sec x+\tan x}\).
\item \textbf{B}. \(\frac{1+\cos 2x}{\sin^2 2x}=\frac{2\cos^2 x}{4\sin^2 x\cos^2 x}=\frac{1}{2}\csc^2 x\).
\item \textbf{B}. \(\sin(\pi+u)=-\sin u\), \(\sin(\pi-u)=\sin u\Rightarrow -2\sin u\).
\item \textbf{A}. \(\frac{1-\sin x}{1-\sin^2 x}=\frac{1-\sin x}{(1-\sin x)(1+\sin x)}=\frac{1}{1+\sin x}\) (donde está definida).
\item \textbf{B}. \(\csc(\tfrac{\pi}{2}-\theta)=\frac{1}{\cos\theta}\Rightarrow \frac{2\cos\theta}{1/\cos\theta}=2\cos^2\theta\).

% ------------------ IV. Inversas -----------------------------
\setcounter{enumi}{14}
\item \textbf{B}. \(\arccos(-\tfrac{\sqrt{2}}{2})=\frac{3\pi}{4}\) (rango \([0,\pi]\)).
\item \textbf{A}. \(\cos(\frac{\pi}{3})=\frac{1}{2}\Rightarrow \arcsen(\frac{1}{2})=\frac{\pi}{6}\).
\item \textbf{A}. \(\sec x\) indefinida cuando \(\cos x=0\Rightarrow x=\frac{\pi}{2}+k\pi\).
\item \textbf{A}. Rango de \(\arcsen(x)\): \([-\frac{\pi}{2},\frac{\pi}{2}]\).

% ------------------ V. Ecuaciones ----------------------------
\setcounter{enumi}{18}
\item \textbf{C}. \(\cos x(\tan x+1)=0\). En \([0,\pi]\): \(x=\frac{\pi}{2}\) y \(x=\frac{3\pi}{4}\) \(\Rightarrow\) dos soluciones.
\item \textbf{(Ver nota)}. \(2\sin^2x=\sin x\Rightarrow \sin x=0\) o \(\sin x=\frac{1}{2}\).
En \([0,2\pi]\): \(x\in\{0,\pi,2\pi,\frac{\pi}{6},\frac{5\pi}{6}\}\).
\emph{Nota:} el ítem presenta más de dos soluciones; si se exige una alternativa única, \textbf{B} incluye dos soluciones válidas (\(0\) y \(\frac{5\pi}{6}\)).
\item \textbf{C}. \((\sec x-1)^2=0\Rightarrow \sec x=1\Rightarrow \cos x=1\Rightarrow x=2k\pi\).
\item \textbf{B}. \(2\cos x+1=0\Rightarrow \cos x=-\frac{1}{2}\Rightarrow x=\frac{2\pi}{3},\frac{4\pi}{3}\). Opción pedida: \(\frac{4\pi}{3}\).

% ------------------ VI. Razonamiento combinado ---------------
\setcounter{enumi}{22}
\item \textbf{B}. Si \(\cos x=\frac{3}{5}\Rightarrow |\sin x|=\frac{4}{5}\). Tomando \(\sin x>0\): \(\sin 2x=2\sin x\cos x=\frac{24}{25}\). 
\emph{Si} \(\sin x<0\), el signo cambiaría.
\item \textbf{A}. (Ángulos agudos) \(\cos y=\frac{5}{13}\Rightarrow \sin y=\frac{12}{13}\).
\(\sin(x+y)=\frac{4}{5}\cdot\frac{5}{13}+\frac{3}{5}\cdot\frac{12}{13}=\frac{56}{65}\).
\item \textbf{—}. \(\tan(2x)=\frac{2\tan x}{1-\tan^2 x}=\frac{4}{1-4}=-\frac{4}{3}\).
\emph{Nota:} la alternativa correcta \(-\frac{4}{3}\) no está listada; es un ajuste a realizar en el banco de ítems.
\item \textbf{A}. Dadas \(\sin(x-y)=\frac{3}{5}\Rightarrow \cos(x-y)=\frac{4}{5}\) (caso agudo), 
\(\cos(x+y)=\frac{4}{5}\Rightarrow \sin(x+y)=\frac{3}{5}\) (caso agudo).
Entonces
\[
\sin(2x)=\sin[(x+y)+(x-y)]=\sin(x+y)\cos(x-y)+\cos(x+y)\sin(x-y)=\frac{24}{25}.
\]
Si alguno fuese obtuso, podría cambiar el signo.
\end{enumerate}

\bigskip
\textbf{Resumen (clave por letra):}
\[
\begin{array}{r|cccccccccc}
\text{Ítem} & 1 & 2 & 3 & 4 & 5 & 6 & 7 & 8 & 9 & 10 \\
\hline
\text{Letra} & B & C & A & D & A & B & A & A & B & A
\end{array}
\]
\[
\begin{array}{r|cccccccccc}
\text{Ítem} & 11 & 12 & 13 & 14 & 15 & 16 & 17 & 18 & 19 & 20^\ast \\
\hline
\text{Letra} & B & B & A & B & B & A & A & A & C & \text{B (ver nota)}
\end{array}
\]
\[
\begin{array}{r|cccccc}
\text{Ítem} & 21 & 22 & 23 & 24 & 25^\dagger & 26^\ddagger \\
\hline
\text{Letra} & C & B & B & A & \text{—} & A
\end{array}
\]

\smallskip
\noindent
\(^\ast\)\,Ítem 20: el enunciado tiene varias soluciones; se marcó \textbf{B} como opción válida representativa.\\
\(^\dagger\)\,Ítem 25: la respuesta correcta es \(-\frac{4}{3}\) (no incluida en las opciones dadas).\\
\(^\ddagger\)\,Ítem 26: la respuesta \textbf{A} asume ángulos agudos; con otros cuadrantes puede cambiar el signo.
\end{document}