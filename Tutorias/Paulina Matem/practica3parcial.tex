\documentclass[11pt,a4paper]{article}

% ------------------------------------------------
% PAQUETES
% ------------------------------------------------
\usepackage[spanish,es-nodecimaldot]{babel}
\usepackage[utf8]{inputenc}
\usepackage[T1]{fontenc}
\usepackage{amsmath,amssymb}
\usepackage{siunitx}
\usepackage[a4paper,margin=2.3cm]{geometry}

\sisetup{
  locale = DE,
  output-decimal-marker = {,},
  per-mode = symbol,
  exponent-product = \cdot,
  group-minimum-digits=4
}

\begin{document}

\section*{Práctica de Repaso -- Exponentes, Logaritmos y Geometría}

% ============================================================
% EJERCICIOS
% ============================================================

\begin{enumerate}

% =======================
\item Considere la función \(f:\mathbb R \to \mathbb R\) dada por
\[
f(x)=3^{1-x}+4.
\]
El ámbito de la función corresponde a:

\begin{enumerate}
\item \(]4,+\infty[\)
\item \(]-\infty,4[\)
\item \(]3,+\infty[\)
\item \(]-\infty,3[\)
\end{enumerate}

% =======================
\item La gráfica de la función
\[
g(x)=-e^{2-x}-3
\]
es asintótica a la recta:

\begin{enumerate}
\item \(y=-3\)
\item \(y=-1\)
\item \(y=0\)
\item \(y=3\)
\end{enumerate}

% =======================
\item El corte con el eje \(X\) de la función
\[
h(x)=\left(\frac{1}{2}\right)^{x+3}-5
\]
corresponde al punto:

\begin{enumerate}
\item \((-3,0)\)
\item \((-8,0)\)
\item \((3,0)\)
\item \((8,0)\)
\end{enumerate}

% =======================
\item Si una función exponencial cumple \(f(x)=b^{-x}\) y se sabe que
\(f(-3)>f(2)\), entonces el valor de \(b\) satisface:

\begin{enumerate}
\item \(b>\sqrt{2}\)
\item \(0<b<1\)
\item \(b>1\)
\item \(b=\dfrac{3}{2}\)
\end{enumerate}

% =======================
\item Considere la función
\[
m(x)=\log_{7}(x-5)+2.
\]
El dominio máximo de \(m\) corresponde a:

\begin{enumerate}
\item \(]5,+\infty[\)
\item \(]-5,+\infty[\)
\item \(]2,+\infty[\)
\item \(]-\infty,5[\)
\end{enumerate}

% =======================
\item Sea \(p(x)=\log_{5}(3x)\). Analice las proposiciones:

\begin{itemize}
\item[I.] \(p(x)>1\) para todo \(x> \dfrac{5}{3}\).
\item[II.] \(p\) es creciente en su dominio.
\end{itemize}

Determine cuáles son verdaderas:

\begin{enumerate}
\item Solo I
\item Solo II
\item Ambas
\item Ninguna
\end{enumerate}

% =======================
\item Considere la función
\[
h(x)=\ln (x-4)-7.
\]
La ecuación de la asíntota vertical corresponde a:

\begin{enumerate}
\item \(x=-4\)
\item \(x=4\)
\item \(x=7\)
\item \(x=0\)
\end{enumerate}

% =======================
\item La expresión
\[
\log_{a} \sqrt[n]{a^{3m}}
\]
es equivalente a:

\begin{enumerate}
\item \(\dfrac{3m}{n}\)
\item \(\dfrac{n}{3m}\)
\item \(3mn\)
\item \(\dfrac{m}{3n}\)
\end{enumerate}

% =======================
\item La expresión
\[
\frac{1}{2}\left[ \ln\left(\frac{p^3}{q}\right)+\ln(q^2)-\ln\left(\frac{p}{q^5}\right)\right]
\]
es igual a:

\begin{enumerate}
\item \(1\)
\item \(2\)
\item \(-1\)
\item \(\ln(p^4)\)
\end{enumerate}
\newpage
% =======================
\item El valor de
\[
\log_{3}\sqrt{27}+\sqrt{\log_{3}(81)}
\]
es igual a:

\begin{enumerate}
\item \(\dfrac{5}{2}\)
\item \(3\)
\item \(4\)
\item \(2\)
\end{enumerate}

% =======================
\item El valor aproximado de
\[
3\log 4+\dfrac{1}{2}\log 50
\]
(Use \(\log 2\approx 0{,}3010\), \(\log 5\approx 0{,}6989\)) es:

\begin{enumerate}
\item \(1{,}4081\)
\item \(2{,}3010\)
\item \(2{,}8895\)
\item \(3{,}0040\)
\end{enumerate}

% =======================
\item ¿Cuántas soluciones enteras tiene la ecuación
\[
(4^x-16)(2^x+1)=0 ?
\]

\begin{enumerate}
\item 0
\item 1
\item 2
\item 3
\end{enumerate}

% =======================
\item Si \(8^{x}+8^{x+1}=96\), entonces el valor de \(x\) pertenece al intervalo:

\begin{enumerate}
\item \(]0,1[\)
\item \(]1,2[\)
\item \(]2,3[\)
\item \(]3,4[\)
\end{enumerate}

% =======================
\item Si \((3/5)^{x-2}(25/9)^{x}=1\), el conjunto solución es:

\begin{enumerate}
\item \(\{0\}\)
\item \(\{2\}\)
\item \(\{5\}\)
\item \(\{4\}\)
\end{enumerate}
\newpage
% =======================
\item Un cubo de arista \(a\) y un cono de radio \(r\) tienen igual volumen. Si la altura del cono es igual a \(3r\), entonces \(r\) es igual a:

\begin{enumerate}
\item \(a\sqrt[3]{\dfrac{1}{\pi}}\)
\item \(a\sqrt[3]{\dfrac{3}{\pi}}\)
\item \(a\sqrt[3]{\dfrac{\pi}{3}}\)
\item \(\dfrac{a}{\pi}\)
\end{enumerate}

% =======================
\item En un cono circular recto de 18 cm de altura y área de base \(36\pi\ \text{cm}^2\), el área lateral del cono es:

\begin{enumerate}
\item \(72\pi\ \text{cm}^2\)
\item \(90\pi\ \text{cm}^2\)
\item \(108\pi\ \text{cm}^2\)
\item \(144\pi\ \text{cm}^2\)
\end{enumerate}

% =======================
\item Si en un polígono regular la suma de los ángulos internos es \(2700^\circ\), entonces el número de diagonales es:

\begin{enumerate}
\item 35
\item 54
\item 72
\item 90
\end{enumerate}

% =======================
\item En un cuadrado de área \(144\ \text{cm}^2\), el radio de la circunferencia circunscrita mide:

\begin{enumerate}
\item 6 cm
\item \(3\sqrt{2}\) cm
\item \(6\sqrt{2}\) cm
\item 12 cm
\end{enumerate}

% =======================
\item ¿Cuál es el área de un círculo inscrito en un cuadrado de lado 12 cm?

\begin{enumerate}
\item \(36\pi\)
\item \(72\pi\)
\item \(144\pi\)
\item \(18\pi\)
\end{enumerate}

\end{enumerate}
\newpage
\section*{Respuestas}

\begin{enumerate}
\item A
\item A
\item B
\item B
\item A
\item C
\item B
\item A
\item D
\item B
\item C
\item B
\item B
\item A
\item B
\item C
\item B
\item A
\item A
\end{enumerate}
\end{document}