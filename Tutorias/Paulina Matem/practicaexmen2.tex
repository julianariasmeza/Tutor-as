% ==========================================================
% Práctica de Repaso — Ángulos y Cuadrantes
% Selección única • Nivel preuniversitario / universitario básico
% ==========================================================
\documentclass[11pt,letterpaper]{article}

% --- Paquetes y formato ---
\usepackage[T1]{fontenc}
\usepackage[utf8]{inputenc}
\usepackage[spanish, es-nodecimaldot]{babel}
\usepackage{amsmath,amssymb,siunitx}
\usepackage[a4paper,margin=2.4cm]{geometry}
\usepackage{setspace}
\usepackage{enumitem}
\usepackage{multicol}

\sisetup{
  locale=DE,
  output-decimal-marker={,},
  per-mode=symbol,
  exponent-product=\cdot,
  group-minimum-digits=4
}

% --- Espaciado general ---
\setlength{\parskip}{4pt}
\setlength{\parindent}{0pt}
\setlength{\columnsep}{1cm}
\setlength{\multicolsep}{8pt}
\renewcommand{\baselinestretch}{1.4}

% --- Inicio ---
\begin{document}

\begin{center}
  {\Large \textbf{Práctica — Ángulos y Cuadrantes}}\\[2mm]
  \textit{Selección única • Nivel preuniversitario}\\[2mm]
  Tercer Cuatrimestre 2025
\end{center}

\hrule
\vspace{4mm}

\textbf{Instrucciones:} seleccione la alternativa correcta.  
Salvo indicación, los ángulos están expresados en radianes.

% ==========================================================
\section*{I. Ángulos y Cuadrantes}
\begin{enumerate}[label=\textbf{\arabic*.},resume]


\item El ángulo \(\dfrac{11\pi}{4}\) está en el cuadrante:
\begin{multicols}{2}
A) I \\[5mm]
B) II \\[5mm]
C) III \\[5mm]
D) IV
\end{multicols}

\item El ángulo \(-\dfrac{5\pi}{6}\) corresponde al cuadrante:
\begin{multicols}{2}
A) I \\[5mm]
B) II \\[5mm]
C) III \\[5mm]
D) IV
\end{multicols}

\item Las coordenadas del punto de la circunferencia trigonométrica asociado a \(6\pi\) son:
\begin{multicols}{2}
A) (1, 0) \\[5mm]
B) (-1, 0) \\[5mm]
C) (0, 1) \\[5mm]
D) (0, -1)
\end{multicols}

\item Si \(x = \dfrac{17\pi}{3}\), el punto correspondiente se encuentra en el cuadrante:
\begin{multicols}{2}
A) I \\[5mm]
B) II \\[5mm]
C) III \\[5mm]
D) IV
\end{multicols}

\item Si \(\theta = -\dfrac{13\pi}{4}\), su ángulo positivo coterminal es:
\begin{multicols}{2}
A) \(\dfrac{3\pi}{4}\) \\[5mm]
B) \(\dfrac{5\pi}{4}\) \\[5mm]
C) \(\dfrac{7\pi}{4}\) \\[5mm]
D) \(\dfrac{9\pi}{4}\)
\end{multicols}

\end{enumerate}
\newpage
% ==========================================================
% Práctica — Razones Trigonométricas
% ==========================================================
\section*{II. Razones Trigonométricas}
\begin{enumerate}[label=\textbf{\arabic*.},resume]


\item Si \(\tan(\alpha)=-\dfrac{3}{4}\) y \(\sin(\alpha)>0\), entonces \(\cos(\alpha)=\)
\begin{multicols}{2}
A) \(\dfrac{4}{5}\) \\[5mm]
B) \(-\dfrac{4}{5}\) \\[5mm]
C) \(\dfrac{3}{5}\) \\[5mm]
D) \(-\dfrac{3}{5}\)
\end{multicols}

\item Si \(\sec(\alpha)<0\) y \(\tan(\alpha)=\dfrac{5}{12}\), el valor de \(\sin(\alpha)\) es:
\begin{multicols}{2}
A) \(-\dfrac{5}{13}\) \\[5mm]
B) \(\dfrac{5}{13}\) \\[5mm]
C) \(-\dfrac{12}{13}\) \\[5mm]
D) \(\dfrac{12}{13}\)
\end{multicols}

\item Si \(\pi<\alpha<\tfrac{3\pi}{2}\) y \(\cot(\alpha)=-1\), entonces \(\sin(2\alpha)=\)
\begin{multicols}{2}
A) \(-1\) \\[5mm]
B) \(-\dfrac{1}{\sqrt{2}}\) \\[5mm]
C) \(\dfrac{1}{\sqrt{2}}\) \\[5mm]
D) \(1\)
\end{multicols}

\item Si \(\cos(\beta)=-\dfrac{1}{3}\), calcule \(\sec^2(\beta)-\tan^2(\beta)\).
\begin{multicols}{2}
A) 0 \\[5mm]
B) 1 \\[5mm]
C) 2 \\[5mm]
D) 3
\end{multicols}

\item Si \(\sin(\theta)=\dfrac{3}{5}\), determine \(\tan(\theta)\).
\begin{multicols}{2}
A) \(\dfrac{3}{4}\) \\[5mm]
B) \(-\dfrac{3}{4}\) \\[5mm]
C) \(\dfrac{4}{3}\) \\[5mm]
D) \(-\dfrac{4}{3}\)
\end{multicols}

\end{enumerate}
\newpage
% ==========================================================
% Práctica — Propiedades de las Funciones Trigonométricas
% ==========================================================
\section*{III. Propiedades de las Funciones Trigonométricas}
\begin{enumerate}[label=\textbf{\arabic*.},resume]

\item La función \(f(x)=\sin(x)\) es:
\begin{multicols}{2}
A) Par y periódica de período \(\pi\) \\[5mm]
B) Impar y periódica de período \(2\pi\) \\[5mm]
C) Par y no periódica \\[5mm]
D) Impar y no periódica
\end{multicols}

\item La función \(f(x)=\cos(x)\) cumple que:
\begin{multicols}{2}
A) Es par y su período es \(2\pi\) \\[5mm]
B) Es impar y su período es \(\pi\) \\[5mm]
C) Es par y su período es \(\pi\) \\[5mm]
D) Es impar y no periódica
\end{multicols}

\item El dominio de la función \(f(x)=\tan(x)\) es:
\begin{multicols}{2}
A) \(\mathbb{R}\) \\[5mm]
B) \(\mathbb{R}-\{\tfrac{\pi}{2}+k\pi,\ k\in\mathbb{Z}\}\) \\[5mm]
C) \(\mathbb{R}-\{k\pi,\ k\in\mathbb{Z}\}\) \\[5mm]
D) \((0,\pi)\)
\end{multicols}

\item El rango de la función \(f(x)=\sec(x)\) es:
\begin{multicols}{2}
A) \((-1,1)\) \\[5mm]
B) \([0,1)\cup(1,\infty)\) \\[5mm]
C) \((-\infty,-1]\cup[1,\infty)\) \\[5mm]
D) \((0,\infty)\)
\end{multicols}

\item La función \(f(x)=\csc(x)\) presenta asíntotas verticales en:
\begin{multicols}{2}
A) \(x=0,\pi,2\pi,\ldots\) \\[5mm]
B) \(x=\tfrac{\pi}{2},\tfrac{3\pi}{2},\ldots\) \\[5mm]
C) \(x=\tfrac{\pi}{4},\tfrac{3\pi}{4},\ldots\) \\[5mm]
D) \(x=\tfrac{\pi}{3},\tfrac{4\pi}{3},\ldots\)
\end{multicols}

\end{enumerate}
\newpage
% ==========================================================
% Práctica — Funciones Trigonométricas Inversas
% ==========================================================
\section*{IV. Funciones Trigonométricas Inversas}
\begin{enumerate}[label=\textbf{\arabic*.},resume]

\item El valor de \(\arccos\left(-\dfrac{\sqrt{2}}{2}\right)\) es:
\begin{multicols}{2}
A) \(\dfrac{\pi}{4}\) \\[5mm]
B) \(\dfrac{3\pi}{4}\) \\[5mm]
C) \(\dfrac{5\pi}{4}\) \\[5mm]
D) \(-\dfrac{\pi}{4}\)
\end{multicols}

\item El valor de \(\arcsen\left[\cos\left(\dfrac{\pi}{3}\right)\right]\) es:
\begin{multicols}{2}
A) \(\dfrac{\pi}{6}\) \\[5mm]
B) \(\dfrac{\pi}{3}\) \\[5mm]
C) \(-\dfrac{\pi}{3}\) \\[5mm]
D) \(-\dfrac{\pi}{6}\)
\end{multicols}

\item Si \(f(x)=\sec(x)\), una de sus asíntotas verticales está dada por:
\begin{multicols}{2}
A) \(x=\dfrac{\pi}{2}\) \\[5mm]
B) \(x=\pi\) \\[5mm]
C) \(x=\dfrac{3\pi}{2}\) \\[5mm]
D) \(x=0\)
\end{multicols}

\item El rango de \(f(x)=\arcsen(x)\) es:
\begin{multicols}{2}
A) \([-\dfrac{\pi}{2},\dfrac{\pi}{2}]\) \\[5mm]
B) \([0,\pi]\) \\[5mm]
C) \((-\pi,\pi)\) \\[5mm]
D) \([-\pi,\pi]\)
\end{multicols}

\item Si \(y=\arctan(1)\), entonces \(y=\)
\begin{multicols}{2}
A) \(\dfrac{\pi}{2}\) \\[5mm]
B) \(\dfrac{\pi}{4}\) \\[5mm]
C) \(\dfrac{3\pi}{4}\) \\[5mm]
D) \(1\)
\end{multicols}

\end{enumerate}
\newpage
% ==========================================================
% Práctica — Identidades Trigonométricas
% ==========================================================
\section*{V. Identidades Trigonométricas}
\begin{enumerate}[label=\textbf{\arabic*.},resume]


\item La expresión \(\sec(x)-\tan(x)\) es equivalente a:
\begin{multicols}{2}
A) \(\dfrac{1}{\sec(x)+\tan(x)}\) \\[5mm]
B) \(\cos(x)\) \\[5mm]
C) \(\sin(x)\) \\[5mm]
D) \(\cos^2(x)\)
\end{multicols}

\item Simplifique \(\dfrac{1-\cos(2x)}{2}\).
\begin{multicols}{2}
A) \(\sin^2(x)\) \\[5mm]
B) \(\cos^2(x)\) \\[5mm]
C) \(\dfrac{1}{2}\sec^2(x)\) \\[5mm]
D) \(\tan^2(x)\)
\end{multicols}

\item \(\sin(\pi+u)-\sin(\pi-u)\) es equivalente a:
\begin{multicols}{2}
A) \(2\sin(u)\) \\[5mm]
B) \(-2\sin(u)\) \\[5mm]
C) \(-\sin^2(u)\) \\[5mm]
D) \(0\)
\end{multicols}

\item \(\dfrac{1-\sin(x)}{1-\sin^2(x)}=\)
\begin{multicols}{2}
A) \(\dfrac{1}{1+\sin(x)}\) \\[5mm]
B) \(\dfrac{1}{\cos(x)}\) \\[5mm]
C) \(1-\sin(x)\) \\[5mm]
D) \(\cos(x)\)
\end{multicols}

\item \(\dfrac{2\cos(\theta)}{\csc(\tfrac{\pi}{2}-\theta)}=\)
\begin{multicols}{2}
A) \(\sin(2\theta)\) \\[5mm]
B) \(2\cos^2(\theta)\) \\[5mm]
C) \(-2\cos^2(\theta)\) \\[5mm]
D) \(2\sin(\theta)\)
\end{multicols}

\end{enumerate}
\newpage
% ==========================================================
% Práctica — Ecuaciones Trigonométricas
% ==========================================================
\section*{VI. Ecuaciones Trigonométricas}
\begin{enumerate}[label=\textbf{\arabic*.},resume]

\item En el intervalo \([0,2\pi)\), las soluciones de \(2\sin^2(x)=\sin(x)\) son:
\begin{multicols}{2}
A) \(0,\ \tfrac{\pi}{6}\) \\[5mm]
B) \(0,\ \tfrac{5\pi}{6}\) \\[5mm]
C) \(0,\ \pi,\ \tfrac{\pi}{6},\ \tfrac{5\pi}{6}\) \\[5mm]
D) \(\tfrac{\pi}{6},\ \tfrac{11\pi}{6}\)
\end{multicols}

\item En el intervalo \([0,2\pi)\), las soluciones de \(2\cos(x)+1=0\) son:
\begin{multicols}{2}
A) \(\tfrac{2\pi}{3},\ \tfrac{4\pi}{3}\) \\[5mm]
B) \(\tfrac{\pi}{3},\ \tfrac{5\pi}{3}\) \\[5mm]
C) \(\tfrac{\pi}{2},\ \tfrac{3\pi}{2}\) \\[5mm]
D) \(\pi,\ 2\pi\)
\end{multicols}

\item En el intervalo \([0,2\pi)\), resuelva \(\tan(x)\cdot\cos(x)+\cos(x)=0\).
\begin{multicols}{2}
A) \(0\) \\[5mm]
B) \(\tfrac{\pi}{2}\) \\[5mm]
C) \(\tfrac{\pi}{2},\ \tfrac{3\pi}{4}\) \\[5mm]
D) \(\tfrac{\pi}{2},\ \tfrac{3\pi}{4},\ \tfrac{7\pi}{4}\)
\end{multicols}

\item En el intervalo \([0,2\pi)\), las soluciones de \(\sin(2x)=\cos(x)\) son:
\begin{multicols}{2}
A) \(\tfrac{\pi}{6},\ \tfrac{5\pi}{6}\) \\[5mm]
B) \(\tfrac{\pi}{2},\ \tfrac{3\pi}{2}\) \\[5mm]
C) \(\tfrac{\pi}{6},\ \tfrac{5\pi}{6},\ \tfrac{\pi}{2},\ \tfrac{3\pi}{2}\) \\[5mm]
D) \(\tfrac{\pi}{3},\ \tfrac{4\pi}{3}\)
\end{multicols}

\item En el intervalo \([0,2\pi)\), resuelva \(\sec^2(x)-2\sec(x)+1=0\).
\begin{multicols}{2}
A) \(\emptyset\) \\[5mm]
B) \(x=k\pi,\ k\in\mathbb{Z}\) \\[5mm]
C) \(x=2k\pi,\ k\in\mathbb{Z}\) \\[5mm]
D) \(x=(2k+1)\pi,\ k\in\mathbb{Z}\)
\end{multicols}

\end{enumerate}
\newpage
% ==========================================================
% Práctica — Razonamiento Combinado
% ==========================================================
\section*{VII. Razonamiento Combinado}
\begin{enumerate}[label=\textbf{\arabic*.},resume]

% \setcounter{enumi}{22} % ← descomentá para empezar en 23

\item Si \(\cos(x)=\dfrac{3}{5}\), halle \(\sin(2x)\).
\begin{multicols}{2}
A) \(\dfrac{12}{25}\) \\[5mm]
B) \(\dfrac{24}{25}\) \\[5mm]
C) \(\dfrac{7}{25}\) \\[5mm]
D) \(\dfrac{9}{25}\)
\end{multicols}

\item Si \(\sin(x)=\dfrac{4}{5}\) y \(\cos(y)=\dfrac{5}{13}\), entonces \(\sin(x+y)=\)
\begin{multicols}{2}
A) \(\dfrac{56}{65}\) \\[5mm]
B) \(\dfrac{33}{65}\) \\[5mm]
C) \(\dfrac{60}{65}\) \\[5mm]
D) \(\dfrac{12}{65}\)
\end{multicols}

\item Si \(\tan(x)=2\), calcule \(\tan(2x)\).
\begin{multicols}{2}
A) \(\dfrac{4}{3}\) \\[5mm]
B) \(-\dfrac{4}{3}\) \\[5mm]
C) \(2\) \\[5mm]
D) \(4\)
\end{multicols}

\item Si \(\sin(x-y)=\dfrac{3}{5}\) y \(\cos(x+y)=\dfrac{4}{5}\), determine \(\sin(2x)\).
\begin{multicols}{2}
A) \(\dfrac{24}{25}\) \\[5mm]
B) \(\dfrac{7}{25}\) \\[5mm]
C) \(\dfrac{9}{25}\) \\[5mm]
D) \(\dfrac{16}{25}\)
\end{multicols}

\item Sean \(x\) en QII y \(y\) en QI, con \(\cos x=-\dfrac{7}{25}\) y \(\sin y=\dfrac{3}{5}\). Calcule \(\cos(x+y)\).
\begin{multicols}{2}
A) \(-\dfrac{4}{5}\) \\[5mm]
B) \(-\dfrac{24}{25}\) \\[5mm]
C) \(\dfrac{4}{5}\) \\[5mm]
D) \(\dfrac{24}{25}\)
\end{multicols}

\end{enumerate}

\bigskip
\hrule
\vspace{3mm}
\textbf{Fin de la práctica.}\\
Verifique los signos y los cuadrantes antes de entregar.
% ==========================================================
% Sección de Respuestas — Práctica de Trigonometría
% ==========================================================
\newpage
\section*{Respuestas y justificaciones breves}

\begin{enumerate}[label=\textbf{\arabic*.}]
% ------------------ I. Ángulos y Cuadrantes (1–5) -----------
\item \textbf{B}. \(\tfrac{11\pi}{4}=2\pi+\tfrac{3\pi}{4}\Rightarrow\) QII.
\item \textbf{C}. \(-\tfrac{5\pi}{6}\equiv \tfrac{7\pi}{6}\Rightarrow\) QIII.
\item \textbf{A}. \(6\pi=3\cdot 2\pi\Rightarrow (1,0)\).
\item \textbf{D}. \(\tfrac{17\pi}{3}-4\cdot\tfrac{3\pi}{3}=\tfrac{5\pi}{3}\Rightarrow\) QIV.
\item \textbf{A}. \(-\tfrac{13\pi}{4}+2\cdot 2\pi=\tfrac{3\pi}{4}\).

% ------------------ II. Razones Trigonométricas (6–10) ------
\setcounter{enumi}{5}
\item \textbf{B}. \(\tan\alpha=-\tfrac{3}{4}\) y \(\sin\alpha>0\Rightarrow\) QII \(\Rightarrow \cos\alpha<0\). \(|\cos|=\tfrac{4}{5}\Rightarrow \cos\alpha=-\tfrac{4}{5}\).
\item \textbf{A}. \(\sec\alpha<0\Rightarrow \cos\alpha<0\). \(\tan\alpha>0\Rightarrow \sin\alpha\) y \(\cos\alpha\) mismo signo \(\Rightarrow \sin\alpha<0\). \(|\sin|=\tfrac{5}{13}\Rightarrow -\tfrac{5}{13}\).
\item \textbf{A}. \(\cot\alpha=-1\Rightarrow \tan\alpha=-1\Rightarrow \sin(2\alpha)=\dfrac{2\tan\alpha}{1+\tan^2\alpha}=-1\).
\item \textbf{B}. Identidad pitagórica: \(\sec^2\beta-\tan^2\beta\equiv 1\).
\item \textbf{A}. \(\sin\theta=\tfrac{3}{5}\Rightarrow |\cos\theta|=\tfrac{4}{5}\Rightarrow |\tan\theta|=\tfrac{3}{4}\) (signo según cuadrante).

% ------------------ III. Propiedades (11–15) ----------------
\setcounter{enumi}{10}
\item \textbf{B}. \(\sin x\) es \textit{impar} y de período \(2\pi\).
\item \textbf{A}. \(\cos x\) es \textit{par} y de período \(2\pi\).
\item \textbf{B}. \(\tan x\) indefinida en \(x=\tfrac{\pi}{2}+k\pi\).
\item \textbf{C}. \(\sec x=\dfrac{1}{\cos x}\Rightarrow |\sec x|\ge 1\).
\item \textbf{A}. \(\csc x=\dfrac{1}{\sin x}\Rightarrow\) asíntotas donde \(\sin x=0\): \(x=k\pi\).

% ------------------ IV. Inversas (16–20) --------------------
\setcounter{enumi}{15}
\item \textbf{B}. \(\arccos\) toma valores en \([0,\pi]\): \(-\tfrac{\sqrt{2}}{2}\mapsto \tfrac{3\pi}{4}\).
\item \textbf{A}. \(\cos(\tfrac{\pi}{3})=\tfrac{1}{2}\Rightarrow \arcsen(\tfrac{1}{2})=\tfrac{\pi}{6}\).
\item \textbf{A}. \(\sec x\) indefinida si \(\cos x=0\Rightarrow x=\tfrac{\pi}{2}+k\pi\).
\item \textbf{A}. Rango: \([-\tfrac{\pi}{2},\tfrac{\pi}{2}]\).
\item \textbf{B}. \(\arctan(1)=\tfrac{\pi}{4}\).

% ------------------ V. Identidades (21–25) -------------------
\setcounter{enumi}{20}
\item \textbf{A}. \((\sec x-\tan x)(\sec x+\tan x)=1\).
\item \textbf{A}. \(\dfrac{1-\cos 2x}{2}=\sin^2 x\).
\item \textbf{B}. \(\sin(\pi+u)=-\sin u,\ \sin(\pi-u)=\sin u\Rightarrow -2\sin u\).
\item \textbf{A}. \(\dfrac{1-\sin x}{1-\sin^2 x}=\dfrac{1}{1+\sin x}\) (donde está definida).
\item \textbf{B}. \(\csc(\tfrac{\pi}{2}-\theta)=\dfrac{1}{\cos\theta}\Rightarrow \dfrac{2\cos\theta}{1/\cos\theta}=2\cos^2\theta\).

% ------------------ VI. Ecuaciones (26–30) -------------------
\setcounter{enumi}{25}
\item \textbf{C}. \(2\sin^2x=\sin x\Rightarrow \sin x=0\) o \(\sin x=\tfrac{1}{2}\Rightarrow x=0,\pi,\tfrac{\pi}{6},\tfrac{5\pi}{6}\).
\item \textbf{A}. \(2\cos x+1=0\Rightarrow \cos x=-\tfrac{1}{2}\Rightarrow x=\tfrac{2\pi}{3},\tfrac{4\pi}{3}\).
\item \textbf{D}. \(\cos x(\tan x+1)=0\Rightarrow x=\tfrac{\pi}{2},\tfrac{3\pi}{2}\) o \(x=\tfrac{3\pi}{4},\tfrac{7\pi}{4}\). \emph{Nota:} la opción D omite \(x=\tfrac{3\pi}{2}\); el conjunto completo incluye cuatro soluciones.
\item \textbf{C}. \(\sin 2x=\cos x\Rightarrow \cos x(2\sin x-1)=0\Rightarrow x=\tfrac{\pi}{2},\tfrac{3\pi}{2},\tfrac{\pi}{6},\tfrac{5\pi}{6}\).
\item \textbf{C}. \((\sec x-1)^2=0\Rightarrow \sec x=1\Rightarrow \cos x=1\Rightarrow x=2k\pi\).

% ------------------ VII. Razonamiento Combinado (31–35) -----
\setcounter{enumi}{30}
\item \textbf{B}. \(\sin 2x=2\sin x\cos x=2\cdot \tfrac{4}{5}\cdot \tfrac{3}{5}=\tfrac{24}{25}\).
\item \textbf{A}. \(\sin(x+y)=\sin x\cos y+\cos x\sin y=\tfrac{4}{5}\cdot\tfrac{5}{13}+\tfrac{3}{5}\cdot\tfrac{12}{13}=\tfrac{56}{65}\).
\item \textbf{B}. \(\tan 2x=\dfrac{2\tan x}{1-\tan^2 x}=\dfrac{4}{1-4}=-\tfrac{4}{3}\).
\item \textbf{A}. \(\sin(2x)=\sin[(x+y)+(x-y)]=\sin(x+y)\cos(x-y)+\cos(x+y)\sin(x-y)=\tfrac{24}{25}\).
\item \textbf{A}. \(\cos(x+y)=\cos x\cos y-\sin x\sin y=(-\tfrac{7}{25})(\tfrac{4}{5})-(\tfrac{24}{25})(\tfrac{3}{5})=-\tfrac{4}{5}\).
\end{enumerate}

\bigskip
\textbf{Resumen (clave por letra):}

\[
\begin{array}{r|ccccc}
\text{Ítem} & 1 & 2 & 3 & 4 & 5 \\
\hline
\text{Letra} & B & C & A & D & A
\end{array}
\qquad
\begin{array}{r|ccccc}
\text{Ítem} & 6 & 7 & 8 & 9 & 10 \\
\hline
\text{Letra} & B & A & A & B & A
\end{array}
\]

\[
\begin{array}{r|ccccc}
\text{Ítem} & 11 & 12 & 13 & 14 & 15 \\
\hline
\text{Letra} & B & A & B & C & A
\end{array}
\qquad
\begin{array}{r|ccccc}
\text{Ítem} & 16 & 17 & 18 & 19 & 20 \\
\hline
\text{Letra} & B & A & A & A & B
\end{array}
\]

\[
\begin{array}{r|ccccc}
\text{Ítem} & 21 & 22 & 23 & 24 & 25 \\
\hline
\text{Letra} & A & A & B & A & B
\end{array}
\qquad
\begin{array}{r|ccccc}
\text{Ítem} & 26 & 27 & 28 & 29 & 30 \\
\hline
\text{Letra} & C & A & D^\ast & C & C
\end{array}
\]

\[
\begin{array}{r|ccccc}
\text{Ítem} & 31 & 32 & 33 & 34 & 35 \\
\hline
\text{Letra} & B & A & B & A & A
\end{array}
\]

\smallskip
\noindent
\(^\ast\)\,Ítem 28: la opción D **no incluye** \(x=\tfrac{3\pi}{2}\); el conjunto correcto es \(\{\tfrac{\pi}{2},\tfrac{3\pi}{2},\tfrac{3\pi}{4},\tfrac{7\pi}{4}\}\).

\end{document}
