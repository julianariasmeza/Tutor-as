% ==========================================================
% PRÁCTICA — FUNCIONES TRIGONOMÉTRICAS BÁSICAS
% (características, valores en puntos clave, crecimiento/decrecimiento)
% ==========================================================
\documentclass[11pt,letterpaper]{article}

% --- Codificación e idioma ---
\usepackage[T1]{fontenc}
\usepackage[utf8]{inputenc}
\usepackage[spanish, es-nodecimaldot]{babel}

% --- Matemática y unidades ---
\usepackage{amsmath, amssymb}
\usepackage{siunitx}
\sisetup{
  locale = DE,
  output-decimal-marker = {,}
}

% --- Diseño de página y tablas ---
\usepackage{geometry}
\geometry{margin=2.2cm}
\usepackage{booktabs, array}
\renewcommand{\arraystretch}{1.2}

% --- Estética opcional ---
\usepackage{microtype}

\begin{document}

\begin{center}
  {\Large \textbf{Práctica — Funciones trigonométricas básicas}}\\[4pt]
  \small (Características, valores de la imagen y crecimiento/decrecimiento)
\end{center}

\hrule
\vspace{0.6em}

\section*{Parte A — Características principales}
\noindent
Complete la tabla con \textbf{dominio}, \textbf{recorrido}, \textbf{amplitud}, \textbf{período} y si la función es \textbf{par}, \textbf{impar} o \textbf{ninguna}.
\[
\text{Considere solo: } \sin x,\; \cos x,\; 2\sin x,\; -3\cos x \quad (\text{sin desplazamientos de fase})
\]

\begin{center}
\small
\begin{tabular}{@{}>{\centering}m{3cm} >{\centering}m{3.2cm} >{\centering}m{3.2cm} >{\centering}m{2.1cm} >{\centering}m{2.1cm} >{\centering\arraybackslash}m{2.2cm}@{}}
\toprule
\textbf{Función} & \textbf{Dominio} & \textbf{Recorrido} & \textbf{Amplitud} & \textbf{Período} & \textbf{Par/Impar} \\
\midrule
$y=\sin x$    & \rule{3.0cm}{0.4pt} & \rule{3.0cm}{0.4pt} & \rule{1.8cm}{0.4pt} & \rule{1.8cm}{0.4pt} & \rule{2.0cm}{0.4pt} \\
$y=\cos x$    & \rule{3.0cm}{0.4pt} & \rule{3.0cm}{0.4pt} & \rule{1.8cm}{0.4pt} & \rule{1.8cm}{0.4pt} & \rule{2.0cm}{0.4pt} \\
$y=2\sin x$   & \rule{3.0cm}{0.4pt} & \rule{3.0cm}{0.4pt} & \rule{1.8cm}{0.4pt} & \rule{1.8cm}{0.4pt} & \rule{2.0cm}{0.4pt} \\
$y=-3\cos x$  & \rule{3.0cm}{0.4pt} & \rule{3.0cm}{0.4pt} & \rule{1.8cm}{0.4pt} & \rule{1.8cm}{0.4pt} & \rule{2.0cm}{0.4pt} \\
\bottomrule
\end{tabular}
\end{center}

\vspace{0.8em}
\noindent\textbf{Indicaciones rápidas (para recordar):}
\begin{itemize}
  \item Dominio típico de $\sin x$ y $\cos x$: todos los reales.
  \item Período básico de $\sin x$ y $\cos x$: $2\pi$.
  \item Amplitud: valor absoluto del coeficiente que multiplica a $\sin$ o $\cos$.
\end{itemize}

\bigskip

\section*{Parte B — Valores de la imagen}
\noindent
Calcule los valores numéricos con \textbf{tres cifras decimales}. Use $\pi$ en radianes.

\subsection*{2(a) \; $y=\sin x$}
\[
x \in \left\{\, 0,\; \tfrac{\pi}{6},\; \tfrac{\pi}{4},\; \tfrac{\pi}{3},\; \tfrac{\pi}{2} \,\right\}
\]
\begin{center}
\small
\begin{tabular}{@{}cccccc@{}}
\toprule
$x$ & $0$ & $\tfrac{\pi}{6}$ & $\tfrac{\pi}{4}$ & $\tfrac{\pi}{3}$ & $\tfrac{\pi}{2}$ \\
\midrule
$\sin x$ & \rule{1.6cm}{0.4pt} & \rule{1.6cm}{0.4pt} & \rule{1.6cm}{0.4pt} & \rule{1.6cm}{0.4pt} & \rule{1.6cm}{0.4pt} \\
\bottomrule
\end{tabular}
\end{center}

\subsection*{2(b) \; $y=\cos x$}
\[
x \in \left\{\, 0,\; \tfrac{\pi}{6},\; \tfrac{\pi}{4},\; \tfrac{\pi}{3},\; \tfrac{\pi}{2} \,\right\}
\]
\begin{center}
\small
\begin{tabular}{@{}cccccc@{}}
\toprule
$x$ & $0$ & $\tfrac{\pi}{6}$ & $\tfrac{\pi}{4}$ & $\tfrac{\pi}{3}$ & $\tfrac{\pi}{2}$ \\
\midrule
$\cos x$ & \rule{1.6cm}{0.4pt} & \rule{1.6cm}{0.4pt} & \rule{1.6cm}{0.4pt} & \rule{1.6cm}{0.4pt} & \rule{1.6cm}{0.4pt} \\
\bottomrule
\end{tabular}
\end{center}

\subsection*{2(c) \; $y=2\sin x$}
\[
x \in \left\{\, 0,\; \tfrac{\pi}{2},\; \pi,\; \tfrac{3\pi}{2},\; 2\pi \,\right\}
\]
\begin{center}
\small
\begin{tabular}{@{}cccccc@{}}
\toprule
$x$ & $0$ & $\tfrac{\pi}{2}$ & $\pi$ & $\tfrac{3\pi}{2}$ & $2\pi$ \\
\midrule
$2\sin x$ & \rule{1.6cm}{0.4pt} & \rule{1.6cm}{0.4pt} & \rule{1.6cm}{0.4pt} & \rule{1.6cm}{0.4pt} & \rule{1.6cm}{0.4pt} \\
\bottomrule
\end{tabular}
\end{center}

\bigskip

\section*{Parte C — Crecimiento y decrecimiento}
\noindent
Indique los intervalos de \textbf{crecimiento} y \textbf{decrecimiento} en $[0,\,2\pi]$.

\subsection*{3(a) \; $y=\sin x$}
\noindent
Creciente en: \rule{8cm}{0.4pt} \\
Decreciente en: \rule{8cm}{0.4pt}

\subsection*{3(b) \; $y=\cos x$}
\noindent
Creciente en: \rule{8cm}{0.4pt} \\
Decreciente en: \rule{8cm}{0.4pt}

\subsection*{3(c) \; $y=-\sin x$}
\noindent
Creciente en: \rule{8cm}{0.4pt} \\
Decreciente en: \rule{8cm}{0.4pt}

\bigskip

\section*{Parte D — Representación gráfica}
\noindent
Trace a mano (o en GeoGebra) y marque:
\[
\text{(i) } y=\sin x,\qquad
\text{(ii) } y=\cos x,\qquad
\text{(iii) } y=2\sin x.
\]
\begin{itemize}
  \item Señale los \textbf{cortes con el eje $x$} en un período.
  \item Marque \textbf{máximos} y \textbf{mínimos} en un período.
  \item Indique con flechas los \textbf{tramos crecientes} y \textbf{decrecientes}.
\end{itemize}

\vfill
\hrule
\small
\noindent\textbf{Sugerencias para el docente (opcional):} Puede pedir aproximaciones con tres cifras decimales: 
$\sin 30^\circ=0{,}500$, $\sin 45^\circ\approx 0{,}707$, $\sin 60^\circ\approx 0{,}866$, 
$\cos 30^\circ\approx 0{,}866$, $\cos 45^\circ\approx 0{,}707$, $\cos 60^\circ=0{,}500$.

\end{document}
