% ==========================================================
% PRÁCTICA — FUNCIONES TRIGONOMÉTRICAS (Estilo MATEM-UCR)
% ==========================================================
\documentclass[11pt,letterpaper]{article}

% --- Codificación e idioma ---
\usepackage[T1]{fontenc}
\usepackage[utf8]{inputenc}
\usepackage[spanish, es-nodecimaldot]{babel}

% --- Matemática y unidades ---
\usepackage{amsmath, amssymb}
\usepackage{siunitx}
\sisetup{
  locale = DE,
  output-decimal-marker = {,}
}

% --- Diseño de página y tablas ---
\usepackage{geometry, booktabs, array, microtype}
\geometry{margin=2.2cm}
\renewcommand{\arraystretch}{1.2}

% --- Encabezado opcional ---
\usepackage{fancyhdr}
\pagestyle{fancy}
\fancyhead[L]{Práctica — Funciones Trigonométricas}
\fancyhead[R]{MATEM-UCR}
\fancyfoot[C]{\thepage}

\begin{document}

\begin{center}
  {\Large \textbf{Práctica — Funciones trigonométricas (nivel admisión UCR)}}\\[4pt]
  \small (Circunferencia, razones trigonométricas, arcos y propiedades)
\end{center}

\hrule
\vspace{0.8em}

\section*{Parte A — Puntos en la circunferencia trigonométrica}

\begin{enumerate}
  \item Al número real \(\dfrac{25\pi}{3}\) le corresponde el punto de la circunferencia trigonométrica cuyas coordenadas son:
  \[
  (\,\rule{2cm}{0.4pt},\; \rule{2cm}{0.4pt}\,)
  \]
  Indique el cuadrante donde se encuentra y determine el ángulo de referencia.

  \item Si el punto asociado al número real \(\alpha\) pertenece a la recta que pasa por el origen y el punto \((-3,6)\), determine \(\sin(\alpha)\).

  \item El punto \((\cos 3,\; \sin 3)\) se ubica en el cuadrante: \rule{3cm}{0.4pt}.
\end{enumerate}

\bigskip

\section*{Parte B — Razones y valores trigonométricos}

\begin{enumerate}
  \setcounter{enumi}{3}
  \item Calcule el valor de:
  \[
  \tan\!\left(\dfrac{17\pi}{6}\right) = \rule{3cm}{0.4pt}
  \]

  \item Determine el \textbf{rango} de la función \( f(x) = -4\sin(3x - \pi) + 2 \).

  \item Evalúe: \(\arcsen\!\left(-\dfrac{\sqrt{3}}{2}\right)\).

  \item Calcule el valor de: \(\sen[4\arctan(1)]\).
\end{enumerate}

\bigskip

\section*{Parte C — Identidades y equivalencias}

\begin{enumerate}
  \setcounter{enumi}{7}
  \item Simplifique la expresión:
  \[
  \dfrac{1 + \cos^2(x) - \sen^2(x)}{\sen(2x)\cdot\cos(x)} = \rule{5cm}{0.4pt}
  \]

  \item Simplifique:
  \[
  \dfrac{1 - \cos^2(x)}{\tan(x)} = \rule{4cm}{0.4pt}
  \]

  \item Verifique cuáles de las siguientes igualdades son verdaderas:
  \begin{enumerate}
    \item \(\sen(4\theta) = 2\sen(2\theta)\)
    \item \(\cos^2(10) - \sen^2(10) = \cos(20)\)
    \item \(\sen(\alpha + 2) = \sen(\alpha)\cos(2) + \sen(2)\cos(\alpha)\)
  \end{enumerate}
  Marque las correctas: \rule{5cm}{0.4pt}.
\end{enumerate}

\bigskip

\section*{Parte D — Propiedades y análisis de funciones}

\begin{enumerate}
  \setcounter{enumi}{10}
  \item Si \(f(x)=\sen(x)+\cos(x)\), determine un punto donde la gráfica corta al eje \(x\).

  \item El período de \(f(x)=2\sen(x-\tfrac{\pi}{2})\) es igual a: \rule{3cm}{0.4pt}.

  \item Si \(f(x)=-3\sen(x - \tfrac{\pi}{6})\), indique la \textbf{amplitud}.

  \item Si \(\tan(\alpha)=-\dfrac{5}{3}\) y \(\cos(\alpha)<0\), determine \(\sen(\alpha)\).
\end{enumerate}

\bigskip

\section*{Parte E — Aplicaciones y ecuaciones trigonométricas}

\begin{enumerate}
  \setcounter{enumi}{14}
  \item Sea \(f(x)=\sen(x)\) en el intervalo \([\pi, \tfrac{3\pi}{2}]\). Indique si la función es creciente o decreciente.

  \item Determine un intervalo donde \(f(x)=\cos(x)\) sea estrictamente creciente.

  \item Resuelva en \([0,2\pi]\): \(-2\sen(t)+\sqrt{3}=0\).

  \item En el intervalo \(\left[-\dfrac{3\pi}{2}, 0\right]\), determine cuántas soluciones tiene la ecuación:
  \[
  2\cos^2(x) - \cos(x) = 0
  \]
\end{enumerate}

\bigskip

\section*{Parte F — Expresiones equivalentes}

\begin{enumerate}
  \setcounter{enumi}{18}
  \item Determine el número que representa el mismo punto en la circunferencia que \(\dfrac{5\pi}{12}\).

  \item Calcule:
  \[
  \sen^2\!\left(\dfrac{5\pi}{4}\right) - \cos\!\left(\dfrac{7\pi}{3}\right) + \tan\!\left(\dfrac{3\pi}{4}\right)
  \]

  \item Simplifique:
  \[
  \tan(\alpha) - \cot(\alpha)
  \]

  \item Simplifique:
  \[
  \dfrac{\sen(y) + \tan(y)}{1 + \sec(y)} = \rule{4cm}{0.4pt}
  \]
\end{enumerate}
\newpage
\section*{Parte G — Ejercicios de selección múltiple}

Seleccione la opción correcta en cada caso.  
Recuerde expresar los resultados en radianes y usar tres cifras decimales si es necesario.

\begin{enumerate}
  \setcounter{enumi}{22}

  \item El valor de \(\sen\!\left(\dfrac{11\pi}{6}\right)\) es:
  \begin{enumerate}
    \item \(\dfrac{1}{2}\)
    \item \(-\dfrac{1}{2}\)
    \item \(\dfrac{\sqrt{3}}{2}\)
    \item \(-\dfrac{\sqrt{3}}{2}\)
  \end{enumerate}

  \item La razón \(\tan(300^\circ)\) es igual a:
  \begin{enumerate}
    \item \(\sqrt{3}\)
    \item \(-\sqrt{3}\)
    \item \(\dfrac{1}{\sqrt{3}}\)
    \item \(-\dfrac{1}{\sqrt{3}}\)
  \end{enumerate}

  \item El ángulo en radianes que corresponde a \(240^\circ\) es:
  \begin{enumerate}
    \item \(\dfrac{2\pi}{3}\)
    \item \(\dfrac{3\pi}{2}\)
    \item \(\dfrac{4\pi}{3}\)
    \item \(\dfrac{5\pi}{3}\)
  \end{enumerate}

  \item Si \(\sen(\alpha)=\dfrac{3}{5}\) y \(\alpha\) está en el segundo cuadrante, entonces \(\cos(\alpha)\) es:
  \begin{enumerate}
    \item \(\dfrac{4}{5}\)
    \item \(-\dfrac{4}{5}\)
    \item \(\dfrac{3}{5}\)
    \item \(-\dfrac{3}{5}\)
  \end{enumerate}
 \newpage
  \item El período de la función \(f(x)=2\cos(4x)\) es:
  \begin{enumerate}
    \item \(\pi\)
    \item \(\dfrac{\pi}{2}\)
    \item \(2\pi\)
    \item \(\dfrac{\pi}{4}\)
  \end{enumerate}

  \item Si \(f(x)=3\sen(x)+1\), el \textbf{rango} de la función es:
  \begin{enumerate}
    \item \([-3,3]\)
    \item \([1,4]\)
    \item \([-2,4]\)
    \item \([0,3]\)
  \end{enumerate}

  \item La expresión \(\dfrac{1-\cos(2x)}{2}\) es equivalente a:
  \begin{enumerate}
    \item \(\sen^2(x)\)
    \item \(\cos^2(x)\)
    \item \(\tan^2(x)\)
    \item \(\cot^2(x)\)
  \end{enumerate}

  \item La función \(f(x)=\sen(x)\) es \textbf{creciente} en el intervalo:
  \begin{enumerate}
    \item \((-\pi,0)\)
    \item \((0,\pi)\)
    \item \((\pi,2\pi)\)
    \item \((-\tfrac{\pi}{2},\tfrac{\pi}{2})\)
  \end{enumerate}

  \item Si \(\tan(\alpha)=2\), entonces \(\cot(\alpha)\) es:
  \begin{enumerate}
    \item \(\dfrac{1}{2}\)
    \item \(-2\)
    \item \(2\)
    \item \(-\dfrac{1}{2}\)
  \end{enumerate}

  \item Si \(f(x)=-\sen(x)\), los máximos de la función ocurren en:
  \begin{enumerate}
    \item \(x=\dfrac{\pi}{2}+2k\pi\)
    \item \(x=\dfrac{3\pi}{2}+2k\pi\)
    \item \(x=2k\pi\)
    \item \(x=\pi+2k\pi\)
  \end{enumerate}
\end{enumerate}

\vspace{1cm}
\hrule
\small
\noindent\textbf{Consejo:}  
Verifique siempre el cuadrante y el signo de la razón trigonométrica antes de seleccionar la respuesta.
\vspace{1cm}
\hrule
\small
\noindent\textbf{Recomendación:}  
Exprese todos los resultados en radianes y use aproximaciones con tres cifras decimales si es necesario.  
Recuerde que las funciones seno y coseno tienen período \(2\pi\) y amplitud 1.

\end{document}