\documentclass[11pt,a4paper]{article}
\usepackage{hyperref}
%------------------------------------------------
% PAQUETES
%------------------------------------------------
\usepackage[spanish,es-nodecimaldot]{babel}
\usepackage[utf8]{inputenc}
\usepackage[T1]{fontenc}
\usepackage{amsmath,amssymb}
\usepackage{siunitx}
\usepackage[a4paper,margin=2.5cm]{geometry}

% Configuración SI y siunitx
\sisetup{
  locale = DE,                      % coma decimal
  output-decimal-marker = {,},      % separador decimal coma
  per-mode=symbol,
  exponent-product=\cdot,
  group-minimum-digits=4
}

\begin{document}

\section*{Cálculo del pH de una disolución amortiguadora HCOOH/HCOOK}

\subsection*{Datos}

\begin{itemize}
  \item Concentración del ácido fórmico: 
  \[
    [\mathrm{HCOOH}] = \SI{0,30}{mol/L}
  \]
  \item Concentración del formiato de potasio:
  \[
    [\mathrm{HCOO^-}] = [\mathrm{HCOOK}] = \SI{0,52}{mol/L}
  \]
  \item Constante de acidez del ácido fórmico (dato de tablas):
  \[
    K_a = 1{,}8 \cdot 10^{-4}
  \]
\end{itemize}

\subsection*{Ecuación del equilibrio ácido--base}

El equilibrio del par ácido--base es:
\[
\mathrm{HCOOH_{(ac)} \rightleftharpoons H^+_{(ac)} + HCOO^-_{(ac)}} 
\]

\subsection*{Cálculo de \texorpdfstring{$\mathrm{p}K_a$}{pKa}}

\[
  \mathrm{p}K_a = -\log_{10}(K_a)
\]

\[
  \mathrm{p}K_a = -\log_{10}\bigl(1{,}8 \cdot 10^{-4}\bigr)
\]

\[
  \mathrm{p}K_a \approx 3{,}74
\]

\subsection*{Fórmula para el pH de un amortiguador}

Para una disolución amortiguadora ácido débil / base conjugada se utiliza
la ecuación de Henderson--Hasselbalch:
\[
  \mathrm{pH} = \mathrm{p}K_a + \log_{10}\!\left(
    \frac{[\text{base conjugada}]}{[\text{ácido débil}]}
  \right)
\]

En este caso:
\[
  \text{base conjugada} \equiv \mathrm{HCOO^-},
  \qquad
  \text{ácido débil} \equiv \mathrm{HCOOH}
\]

\subsection*{Sustitución de valores}

\[
  \mathrm{pH} = 3{,}74 + \log_{10}\!\left(
    \frac{[\mathrm{HCOO^-}]}{[\mathrm{HCOOH}]}
  \right)
\]

\[
  \mathrm{pH} = 3{,}74 + \log_{10}\!\left(
    \frac{0{,}52}{0{,}30}
  \right)
\]

\subsection*{Cálculo numérico}

Primero se calcula el cociente de concentraciones:
\[
  \frac{0{,}52}{0{,}30} \approx 1{,}73
\]

Luego se evalúa el logaritmo decimal:
\[
  \log_{10}(1{,}73) \approx 0{,}24
\]

Sustituyendo en la expresión del pH:
\[
  \mathrm{pH} = 3{,}74 + 0{,}24
\]

\[
  \boxed{\mathrm{pH} \approx 3{,}98}
\]

\subsection*{Interpretación}

La disolución formada por ácido fórmico (\(\mathrm{HCOOH}\)) y formiato
de potasio (\(\mathrm{HCOO^-}\)) actúa como un \emph{amortiguador ácido}.
El valor de \(\mathrm{pH} \approx 4\) indica que la disolución es
ligeramente ácida, como se espera para un sistema ácido débil / base
conjugada donde la concentración de la base es algo mayor que la del ácido.
\vspace{0.5cm}

\noindent
Por tanto, el pH de la disolución es aproximadamente:
\[
  \boxed{\mathrm{pH} \approx 4{,}0}
\]

\end{document}
