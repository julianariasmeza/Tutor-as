\documentclass[11pt,a4paper]{article}

%------------------------------------------------
% PAQUETES
%------------------------------------------------
\usepackage[spanish,es-nodecimaldot]{babel}
\usepackage[utf8]{inputenc}
\usepackage[T1]{fontenc}
\usepackage{amsmath,amssymb}
\usepackage{siunitx}
\usepackage[a4paper,margin=2.5cm]{geometry}

\sisetup{
  locale = DE,
  output-decimal-marker = {,},
  per-mode = symbol,
  exponent-product = \cdot,
  group-minimum-digits = 4
}

\begin{document}

\section*{pH del buffer NH\(_3\)/NH\(_4^+\) después de agregar NaOH}

\subsection*{Datos iniciales del buffer}

\begin{itemize}
  \item Concentración de la base débil:
  \[
    [\mathrm{NH_3}] = \SI{0,30}{mol/L}
  \]
  \item Concentración del ácido conjugado:
  \[
    [\mathrm{NH_4^+}] = \SI{0,36}{mol/L}
  \]
  \item Volumen de disolución buffer:
  \[
    V_{\text{buffer}} = \SI{80,0}{mL} = \SI{0,0800}{L}
  \]
  \item Volumen y concentración de la base fuerte añadida:
  \[
    V_{\text{NaOH}} = \SI{20,0}{mL} = \SI{0,0200}{L},
    \quad
    C_{\text{NaOH}} = \SI{0,050}{mol/L}
  \]
  \item Constante de acidez del par NH\(_4^+\)/NH\(_3\):
  \[
    pK_a = 9{,}25
  \]
\end{itemize}

\subsection*{Reacción entre el buffer y la base fuerte}

La base fuerte reacciona con el ácido del buffer:
\[
  \mathrm{{NH_4^+}_{(ac)} + OH^-_{(ac)} \rightarrow H_2O_{(l)} + NH_3{}_{(ac)}}
\]

\subsection*{Cálculo de moles iniciales}

\paragraph{Moles en el buffer}

\[
  n(\mathrm{NH_3})_{\text{inicial}} =
  [\mathrm{NH_3}] \cdot V_{\text{buffer}}
  = 0{,}30 \cdot 0{,}0800
  = \SI{0,0240}{mol}
\]

\[
  n(\mathrm{NH_4^+})_{\text{inicial}} =
  [\mathrm{NH_4^+}] \cdot V_{\text{buffer}}
  = 0{,}36 \cdot 0{,}0800
  = \SI{0,0288}{mol}
\]

\paragraph{Moles de NaOH añadidos}

\[
  n(\mathrm{OH^-}) =
  C_{\text{NaOH}} \cdot V_{\text{NaOH}}
  = 0{,}050 \cdot 0{,}0200
  = \SI{0,00100}{mol}
\]

\subsection*{Cambio de moles por la reacción}

El ion \(\mathrm{OH^-}\) es el reactivo limitante, por lo que:
\[
  \Delta n = \SI{0,00100}{mol}
\]

\[
  n(\mathrm{NH_4^+})_{\text{final}} =
  0{,}0288 - 0{,}00100
  = \SI{0,0278}{mol}
\]

\[
  n(\mathrm{NH_3})_{\text{final}} =
  0{,}0240 + 0{,}00100
  = \SI{0,0250}{mol}
\]

\subsection*{Concentraciones después de la mezcla}

El volumen total final es:
\[
  V_{\text{total}} =
  V_{\text{buffer}} + V_{\text{NaOH}}
  = 0{,}0800 + 0{,}0200
  = \SI{0,1000}{L}
\]

\[
  [\mathrm{NH_3}]_{\text{final}} =
  \frac{n(\mathrm{NH_3})_{\text{final}}}{V_{\text{total}}}
  = \frac{0{,}0250}{0{,}1000}
  = \SI{0,250}{mol/L}
\]

\[
  [\mathrm{NH_4^+}]_{\text{final}} =
  \frac{n(\mathrm{NH_4^+})_{\text{final}}}{V_{\text{total}}}
  = \frac{0{,}0278}{0{,}1000}
  = \SI{0,278}{mol/L}
\]

\subsection*{Cálculo del pH con Henderson--Hasselbalch}

Se aplica la ecuación de Henderson--Hasselbalch para el par
\(\mathrm{NH_4^+/NH_3}\):

\[
  \mathrm{pH} = pK_a +
  \log_{10}\!\left(
    \frac{[\text{base}]}{[\text{ácido}]}
  \right)
\]

\[
  \mathrm{pH} =
  9{,}25 +
  \log_{10}\!\left(
    \frac{[\mathrm{NH_3}]_{\text{final}}}{[\mathrm{NH_4^+}]_{\text{final}}}
  \right)
  =
  9{,}25 +
  \log_{10}\!\left(
    \frac{0{,}250}{0{,}278}
  \right)
\]

Primero el cociente:
\[
  \frac{0{,}250}{0{,}278} \approx 0{,}90
\]

Luego el logaritmo:
\[
  \log_{10}(0{,}90) \approx -0{,}05
\]

Por tanto:
\[
  \mathrm{pH} \approx 9{,}25 - 0{,}05 = 9{,}20
\]

\[
  \boxed{\mathrm{pH}_{\text{después de agregar NaOH}} \approx 9{,}20}
\]

\end{document}
