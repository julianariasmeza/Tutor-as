\documentclass[11pt,a4paper]{article}

%------------------------------------------------
% PAQUETES
%------------------------------------------------
\usepackage[spanish,es-nodecimaldot]{babel}
\usepackage[utf8]{inputenc}
\usepackage[T1]{fontenc}
\usepackage{amsmath,amssymb}
\usepackage{siunitx}
\usepackage[a4paper,margin=2.5cm]{geometry}

\sisetup{
  locale = DE,
  output-decimal-marker = {,},
  per-mode = symbol,
  exponent-product = \cdot,
  group-minimum-digits = 4
}

\begin{document}

\section*{Ejercicio 1: Concentración de HCl mediante titulación}

Calcule la concentración en mol/L de una disolución de HCl si, durante la titulación de 5 mL de este ácido, el punto final se alcanzó cuando se consumieron 10 mL de NaOH 0,20 mol/L.

\subsection*{Datos}

\[
V_{\text{HCl}} = \SI{5,0}{mL} = \SI{0,0050}{L}
\]

\[
V_{\text{NaOH}} = \SI{10,0}{mL} = \SI{0,0100}{L}
\]

\[
C_{\text{NaOH}} = \SI{0,20}{mol/L}
\]

Reacción química:

\[
\mathrm{HCl + NaOH \rightarrow NaCl + H_2O}
\]

\subsection*{Cálculo de moles de NaOH}

\[
n_{\text{NaOH}} = C_{\text{NaOH}} \cdot V_{\text{NaOH}}
\]

\[
n_{\text{NaOH}} = 0,20 \cdot 0,0100 = \SI{0,0020}{mol}
\]

\subsection*{Moles de HCl}

\[
n_{\text{HCl}} = \SI{0,0020}{mol}
\]

\subsection*{Cálculo de la concentración del HCl}

\[
C_{\text{HCl}} = \frac{n_{\text{HCl}}}{V_{\text{HCl}}}
\]

\[
C_{\text{HCl}} = \frac{0,0020}{0,0050} = \SI{0,40}{mol/L}
\]

\subsection*{Conclusión}

\[
C_{\text{HCl}} = \SI{0,40}{mol/L}
\]
\newpage
%=============================================================

\section*{Ejercicio 2: Volumen de Ba(OH)\(_2\) necesario para neutralizar HCl}

Calcule los mililitros que se consumirían de Ba(OH)\(_2\) 0,20 mol/L al titular 50 mL de una muestra de HCl 0,15 mol/L.

\subsection*{Datos}

\[
V_{\text{HCl}} = \SI{50}{mL} = \SI{0,0500}{L}
\]

\[
C_{\text{HCl}} = \SI{0,15}{mol/L}
\]

\[
C_{\text{Ba(OH)\_2}} = \SI{0,20}{mol/L}
\]

Reacción química:

\[
\mathrm{Ba(OH)_2 + 2HCl \rightarrow BaCl_2 + 2H_2O}
\]

\subsection*{Moles de HCl}

\[
n_{\text{HCl}} = C_{\text{HCl}} \cdot V_{\text{HCl}}
\]

\[
n_{\text{HCl}} = 0,15 \cdot 0,0500 = \SI{0,00750}{mol}
\]

\subsection*{Relación molar con Ba(OH)\(_2\)}

\[
n_{\text{Ba(OH)\_2}} = \frac{n_{\text{HCl}}}{2}
\]

\[
n_{\text{Ba(OH)\_2}} = \frac{0,00750}{2} = \SI{0,00375}{mol}
\]

\subsection*{Cálculo del volumen requerido}

\[
V = \frac{n_{\text{Ba(OH)\_2}}}{C_{\text{Ba(OH)\_2}}}
\]

\[
V = \frac{0,00375}{0,20} = \SI{0,01875}{L}
\]

Conversión a mililitros:

\[
V = 0,01875 \cdot 1000 = \SI{18,75}{mL}
\]

\subsection*{Conclusión}

\[
V_{\text{Ba(OH)\_2}} = \SI{18,75}{mL}
\]

\end{document}
