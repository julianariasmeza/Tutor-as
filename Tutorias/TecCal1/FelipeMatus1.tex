% ==========================================================
% Tutoría de Cálculo I - TEC
% Tema: Máximos, mínimos, concavidad y optimización
% Formato con soluciones clickeables
% ==========================================================

\documentclass[12pt]{article}

% ---------------- Paquetes básicos ----------------
\usepackage[spanish]{babel}
\usepackage[utf8]{inputenc}
\usepackage[T1]{fontenc}

% Matemática y símbolos
\usepackage{amsmath, amssymb}
\usepackage{siunitx}

% Diseño de página
\usepackage{geometry}
\geometry{letterpaper, margin=2.5cm}

% Listas
\usepackage{enumitem}

% Cajas bonitas
\usepackage[most]{tcolorbox}

% Encabezado y pie
\usepackage{fancyhdr}
\setlength{\headheight}{14.5pt}

% Botones / símbolos
\usepackage{pifont}

% Capas visibles/ocultas en PDF
\usepackage{ocgx2}

% Para permitir unicode tipo “⇒” o “✗”
\usepackage{newunicodechar}
\newunicodechar{✗}{\ding{55}}
\newunicodechar{⇒}{\Rightarrow}

% Hipervínculos (lo cargamos al final del preámbulo normalmente,
% pero lo dejamos aquí para que el archivo compile de una vez)
\usepackage{hyperref}

% ---------------- Configuración siunitx ----------------
\sisetup{
  locale = DE,                 % coma decimal
  output-decimal-marker = {,}, % forzar coma
  per-mode = symbol,
  exponent-product = \cdot,
  group-minimum-digits = 4
}

% ---------------- Estilo de cajas ----------------
\tcbset{
  colback=white,
  colframe=black!70,
  boxrule=0.6pt,
  arc=2mm,
  left=6pt,right=6pt,top=6pt,bottom=6pt
}

% Caja para enunciados / ejemplos
\newtcolorbox{ejercicio}[1][]{title=Ejercicio, #1}

% Caja para soluciones desarrolladas
\newtcolorbox{solucionbox}[1][]{title=Solución, colframe=green!60!black, #1}

% Caja tipo nota / advertencia / recordatorio teórico
\newtcolorbox{nota}[1][]{title=Nota, colframe=blue!60!black, #1}

% ---------------- Encabezado y pie ----------------
\pagestyle{fancy}
\fancyhf{}
\lhead{Tutoría Cálculo I — Máximos, mínimos y optimización}
\rhead{Julián Arias Meza — 2025}
\cfoot{\thepage}

% ---------------- Entorno de solución clickeable ----------------
% Uso:
% \begin{solucionclick}
%   ... desarrollo paso a paso ...
% \end{solucionclick}
%
% Esto crea un botón "Mostrar / ocultar solución" que funciona en visores PDF compatibles.
%
\newcounter{solnum}
\newcommand{\solbtn}{\fcolorbox{black}{gray!10}{\footnotesize Mostrar / ocultar solución}}
\newenvironment{solucionclick}{%
  \refstepcounter{solnum}%
  \par\noindent\textbf{Solución \#\thesolnum}\;
  \switchocg{sol:\thesolnum}{\solbtn}\\[2pt]%
  \begin{ocg}{Solución \#\thesolnum}{sol:\thesolnum}{0}%
  \begin{solucionbox}%
}{%
  \end{solucionbox}%
  \end{ocg}\par\medskip
}

% ---------------- Título reutilizable ----------------
\newcommand{\titulo}{
  \begin{center}
  {\Large \textbf{Tutoría de Cálculo I (TEC)}}\\[4pt]
  {\large \textbf{Máximos, mínimos, concavidad y optimización}}\\[6pt]
  {\small Análisis con derivadas y aplicaciones prácticas}\\[6pt]
  \end{center}
  \vspace{-0.3em}\hrule\vspace{0.8em}
}

% ==========================================================
% ===================== DOCUMENTO ==========================
% ==========================================================
\begin{document}

\titulo

\begin{nota}
Este material resume los puntos clave de Cálculo Diferencial en Cálculo I (TEC): \textbf{máximos y mínimos}, \textbf{concavidad} y \textbf{optimización}.  
Luego se incluyen ejercicios con soluciones que se pueden \textbf{mostrar / ocultar} haciendo clic.  
Para que el botón funcione, se debe abrir el PDF en un visor compatible con \textit{capas OCG} (por ejemplo \textbf{Adobe Acrobat Reader} o \textbf{PDF-XChange}). En visores del navegador o Vista Previa (macOS) puede no funcionar.
\end{nota}

% ----------------------------------------------------------
\section*{1. Máximos y mínimos}

El cálculo diferencial permite localizar dónde una función alcanza \textbf{puntos altos} (máximos) o \textbf{puntos bajos} (mínimos).  
Estos valores son importantes tanto para graficar como para aplicaciones físicas, geométricas y económicas.

\subsection*{1.1 Definiciones básicas}

Sea \( f(x) \) una función.

\begin{itemize}
  \item \( f(x_0) \) es un \textbf{máximo local} si \( f(x_0) \geq f(x) \) para todo \( x \) suficientemente cercano a \( x_0 \).
  \item \( f(x_0) \) es un \textbf{mínimo local} si \( f(x_0) \leq f(x) \) para todo \( x \) suficientemente cercano a \( x_0 \).
\end{itemize}

Un \textbf{máximo absoluto (global)} es el valor más grande que toma la función en todo el intervalo que estamos estudiando.  
Un \textbf{mínimo absoluto (global)} es el valor más pequeño que toma la función en ese intervalo.

En un intervalo cerrado \([a,b]\) y con \(f\) continua, los máximos/mínimos absolutos pueden ocurrir:
\begin{itemize}
  \item en puntos críticos interiores,
  \item o en los extremos \(x=a\) y \(x=b\).
\end{itemize}

\subsection*{1.2 Puntos críticos}

Un \textbf{punto crítico} de \( f \) es un valor \( x = c \) donde:
\[
f'(c) = 0
\quad \text{o} \quad
f'(c) \text{ no existe.}
\]

En esos puntos \emph{puede} haber máximo o mínimo local (pero no siempre; hay que clasificar).

\subsection*{1.3 Prueba de la primera derivada}

Para decidir si hay máximo o mínimo en un punto crítico \(x_0\), observamos el signo de \(f'(x)\) alrededor de \(x_0\):

\begin{itemize}
  \item Si \( f'(x) \) pasa de \(+\) a \(-\) (crecía y luego decrece), en \(x_0\) hay \textbf{máximo local}.
  \item Si \( f'(x) \) pasa de \(-\) a \(+\) (decrecía y luego crece), en \(x_0\) hay \textbf{mínimo local}.
  \item Si \( f'(x) \) no cambia de signo, no es extremo local (puede ser una meseta).
\end{itemize}

\subsection*{1.4 Prueba de la segunda derivada}

Otra forma muy usada en exámenes del TEC:

Calculamos la segunda derivada \( f''(x) \).  
Sea \(x_0\) un punto crítico con \( f'(x_0)=0 \).

\[
\begin{array}{ll}
f''(x_0) > 0 & \Rightarrow \text{cóncava hacia arriba en } x_0 \Rightarrow \text{mínimo local} \\
f''(x_0) < 0 & \Rightarrow \text{cóncava hacia abajo en } x_0 \Rightarrow \text{máximo local}
\end{array}
\]

Si \( f''(x_0) = 0 \), esta prueba no decide y se debe usar otra estrategia (por ejemplo, la prueba de la primera derivada).

% ----------------------------------------------------------
\section*{2. Concavidad y puntos de inflexión}

\subsection*{2.1 Concavidad}

La \textbf{concavidad} describe cómo se curva la gráfica.

\[
f''(x) > 0 \quad \Rightarrow \quad \text{La función es cóncava hacia arriba.}
\]

\[
f''(x) < 0 \quad \Rightarrow \quad \text{La función es cóncava hacia abajo.}
\]

Interpretación geométrica:
\begin{itemize}
  \item Cóncava hacia arriba: la gráfica “abre hacia arriba”, como una sonrisa.
  \item Cóncava hacia abajo: la gráfica “abre hacia abajo”, como una carpa invertida.
\end{itemize}

\subsection*{2.2 Punto de inflexión}

Un \textbf{punto de inflexión} es un valor \(x = x_0\) donde la concavidad cambia:
pasa de cóncava hacia arriba a cóncava hacia abajo, o viceversa.

Típicamente, en un punto de inflexión:
\[
f''(x_0) = 0 \quad \text{o } f''(x_0) \text{ no existe,}
\]
\textbf{y además} el signo de \(f''(x)\) realmente cambia al cruzar \(x_0\).

% ----------------------------------------------------------
\section*{3. Problemas de optimización}

Los problemas de optimización consisten en \textbf{maximizar} o \textbf{minimizar} alguna cantidad física, geométrica o económica (área, volumen, costo, distancia, ganancia, etc.).

Ejemplos típicos:
\begin{itemize}
  \item Máxima área posible con cierta cantidad fija de cerca.
  \item Mínimo costo de un envase con volumen fijo.
  \item Distancia mínima entre un punto y una recta / curva.
  \item Dimensiones que minimizan material.
\end{itemize}

\subsection*{3.1 Procedimiento general de optimización}

Este es el flujo estándar que vamos a usar en los ejercicios:

\begin{enumerate}[label=\arabic*.]
  \item \textbf{Comprender el problema.}  
    Lea con cuidado. ¿Qué se está pidiendo maximizar o minimizar?  
    ¿Área? ¿Volumen? ¿Costo? ¿Distancia?
  \item \textbf{Hacer un diagrama.}  
    Dibujar ayuda muchísimo para identificar relaciones geométricas entre las variables.
  \item \textbf{Definir las variables.}  
    Ponga nombre a las longitudes, radios, alturas, etc.  
    Señale claramente cuál es la \textbf{magnitud objetivo} (la que vamos a optimizar).
  \item \textbf{Plantear la función objetivo.}  
    Escriba la expresión matemática de esa magnitud objetivo.  
    Ejemplo: área \(A(x)\), costo \(C(x)\), volumen \(V(x)\), distancia \(D(x)\).
  \item \textbf{Usar las restricciones.}  
    Normalmente hay una restricción física: “solo hay \SI{800}{m} de cerca”,  
    “el volumen debe ser \SI{100}{cm^3}”, etc.  
    Con esa restricción se debe \textbf{reducir la función a una sola variable}.
  \item \textbf{Derivar e igualar a cero.}  
    Calcular \(f'(x)=0\) para encontrar puntos críticos.
  \item \textbf{Clasificar el extremo.}  
    Verificar que ese punto crítico da máximo o mínimo (por ejemplo usando \(f''(x)\)).
  \item \textbf{Interpretar la respuesta.}  
    Responder con palabras y con unidades claras.  
    Ejemplo: “El rectángulo de área máxima mide \(\SI{200}{m} \times \SI{400}{m}\)”.
\end{enumerate}

\begin{nota}
En todos los problemas de optimización vamos a presentar primero el enunciado (caja tipo Ejercicio)  
y luego la solución completa paso a paso dentro de un bloque ocultable con clic.
\end{nota}

% ----------------------------------------------------------
\section*{4. Resumen rápido para el estudiante}

\begin{tcolorbox}[
  colback=green!5!white,
  colframe=green!60!black,
  title={Resumen de criterios},
  boxrule=0.7pt,
  arc=2mm
]
\textbf{Crecimiento y decrecimiento:}\\
Si \( f'(x) > 0 \) en un intervalo, \(f\) es creciente allí.  
Si \( f'(x) < 0 \) en un intervalo, \(f\) es decreciente allí.\\[6pt]

\textbf{Máximo / mínimo local:}\\
Punto crítico \(x_0\) con \(f'(x_0)=0\).  
Si \( f''(x_0) < 0 \Rightarrow\) máximo local.  
Si \( f''(x_0) > 0 \Rightarrow\) mínimo local.\\[6pt]

\textbf{Concavidad:}\\
\( f''(x) > 0 \Rightarrow\) cóncava hacia arriba.  
\( f''(x) < 0 \Rightarrow\) cóncava hacia abajo.\\[6pt]

\textbf{Punto de inflexión:}\\
Lugar donde cambia la concavidad (el signo de \(f''(x)\) cambia).\\[6pt]

\textbf{Optimización aplicada:}\\
1) Plantee la función objetivo.  
2) Reduzca a una sola variable.  
3) Derive y halle críticos.  
4) Concluya en el contexto físico/económico con unidades.
\end{tcolorbox}

\vspace{1cm}

\begin{center}
\textit{A continuación vienen los ejercicios tipo examen con soluciones clickeables.}
\end{center}
% ==========================================================
% 5. MÁXIMOS Y MÍNIMOS
% ==========================================================

\section*{5. Máximos y mínimos (prueba de derivadas)}

En los siguientes ejercicios se busca:
\begin{itemize}
  \item Encontrar puntos críticos (\(f'(x)=0\)).
  \item Clasificar cada punto crítico como máximo local o mínimo local.
  \item (Cuando aplique) dar el valor máximo o mínimo.
\end{itemize}

% ---------- EJERCICIO 5.1 ----------
\begin{ejercicio}
Considere la función
\[
f(x)=x^{3}-3x^{2}+2.
\]
a) Encuentre los puntos críticos. \\
b) Determine cuáles son máximos locales y cuáles son mínimos locales. \\
c) Indique el valor de la función en esos puntos.
\end{ejercicio}

\begin{solucionclick}
\textbf{Paso 1. Derivada primera}

\[
f(x)=x^{3}-3x^{2}+2
\quad\Longrightarrow\quad
f'(x)=3x^{2}-6x=3x(x-2).
\]

\textbf{Paso 2. Puntos críticos}

Resolvemos \(f'(x)=0\):
\[
3x(x-2)=0
\quad\Longrightarrow\quad
x=0 \quad\text{o}\quad x=2.
\]

Entonces los candidatos a extremos son \(x=0\) y \(x=2\).

\textbf{Paso 3. Segunda derivada}

\[
f''(x)=\frac{d}{dx}\bigl(3x^{2}-6x\bigr)=6x-6.
\]

\textbf{Paso 4. Clasificación con \(f''(x)\)}

Para \(x=0\):
\[
f''(0)=6\cdot 0-6=-6<0
\quad\Rightarrow\quad
\text{cóncava hacia abajo en }x=0,
\text{ entonces }x=0\text{ es máximo local.}
\]

Para \(x=2\):
\[
f''(2)=6\cdot 2-6=12-6=6>0
\quad\Rightarrow\quad
\text{cóncava hacia arriba en }x=2,
\text{ entonces }x=2\text{ es mínimo local.}
\]

\textbf{Paso 5. Valores de la función}

\[
f(0)=0^{3}-3\cdot 0^{2}+2=2.
\]

\[
f(2)=2^{3}-3\cdot 2^{2}+2
=8-12+2=-2.
\]

\textbf{Conclusión:}

Máximo local en \((0,2)\). \\
Mínimo local en \((2,-2)\).
\end{solucionclick}


% ---------- EJERCICIO 5.2 ----------
\begin{ejercicio}
Considere la función
\[
g(x)=x^{4}-4x^{2}+1.
\]
a) Encuentre todos los puntos críticos. \\
b) Clasifique cada punto crítico usando la segunda derivada. \\
c) Indique cuáles son máximos/mínimos locales y sus valores.
\end{ejercicio}

\begin{solucionclick}
\textbf{Paso 1. Derivada primera}

\[
g'(x)=4x^{3}-8x=4x(x^{2}-2).
\]

\textbf{Paso 2. Puntos críticos}

Resolvemos \(g'(x)=0\):
\[
4x(x^{2}-2)=0
\quad\Longrightarrow\quad
x=0
\quad\text{o}\quad
x^{2}-2=0\ \Longrightarrow\ x=\pm\sqrt{2}.
\]

Entonces candidatos: \(x=-\sqrt{2},\ 0,\ \sqrt{2}\).

\textbf{Paso 3. Segunda derivada}

\[
g''(x)=\frac{d}{dx}(4x^{3}-8x)=12x^{2}-8.
\]

Evaluamos:

Para \(x=0\):
\[
g''(0)=12\cdot 0^{2}-8=-8<0
\quad\Rightarrow\quad
\text{máximo local en }x=0.
\]

Para \(x=\sqrt{2}\):
\[
g''(\sqrt{2})=12(\sqrt{2})^{2}-8=12\cdot 2-8=24-8=16>0
\quad\Rightarrow\quad
\text{mínimo local en }x=\sqrt{2}.
\]

Para \(x=-\sqrt{2}\):
\[
g''(-\sqrt{2})=12(-\sqrt{2})^{2}-8=12\cdot 2-8=24-8=16>0
\quad\Rightarrow\quad
\text{mínimo local en }x=-\sqrt{2}.
\]

\textbf{Paso 4. Valores de la función}

\[
g(0)=0^{4}-4\cdot 0^{2}+1=1.
\]

\[
g(\sqrt{2})=(\sqrt{2})^{4}-4(\sqrt{2})^{2}+1
= (2^{2})-4\cdot 2+1
=4-8+1=-3.
\]

\[
g(-\sqrt{2})=\text{igual que }g(\sqrt{2})=-3.
\]

\textbf{Conclusión:}

Máximo local en \((0,1)\). \\
Mínimos locales en \(\bigl(-\sqrt{2},-3\bigr)\) y \(\bigl(\sqrt{2},-3\bigr)\).
\end{solucionclick}


% ---------- EJERCICIO 5.3 ----------
\begin{ejercicio}
Sea
\[
h(x)=x^{3}-9x.
\]
a) Determine los intervalos donde \(h(x)\) es creciente y decreciente. \\
b) Determine los máximos y mínimos locales.
\end{ejercicio}

\begin{solucionclick}
\textbf{Paso 1. Derivada primera}

\[
h'(x)=3x^{2}-9=3(x^{2}-3).
\]

\textbf{Paso 2. Zonas de crecimiento / decrecimiento}

Analizamos el signo de \(h'(x)\).

Resolvemos \(h'(x)=0\):
\[
3(x^{2}-3)=0
\quad\Longrightarrow\quad
x^{2}-3=0
\quad\Longrightarrow\quad
x=\pm\sqrt{3}.
\]

Tomamos intervalos:
\[
(-\infty,-\sqrt{3}),\quad (-\sqrt{3},\sqrt{3}),\quad (\sqrt{3},\infty).
\]

Escogemos un punto de prueba en cada intervalo:

1. Si \(x<- \sqrt{3}\), por ejemplo \(x=-10\):
\[
h'(-10)=3(100-3)=3\cdot 97>0
\quad\Rightarrow\quad
h \text{ crece en }(-\infty,-\sqrt{3}).
\]

2. Si \(-\sqrt{3}<x<\sqrt{3}\), por ejemplo \(x=0\):
\[
h'(0)=3(0-3)=-9<0
\quad\Rightarrow\quad
h \text{ decrece en }(-\sqrt{3},\sqrt{3}).
\]

3. Si \(x>\sqrt{3}\), por ejemplo \(x=10\):
\[
h'(10)=3(100-3)=291>0
\quad\Rightarrow\quad
h \text{ crece en }(\sqrt{3},\infty).
\]

\textbf{Paso 3. Clasificación de los extremos}

- En \(x=-\sqrt{3}\), la función pasa de creciente (\(+\)) a decreciente (\(-\)).  
  Entonces \(x=-\sqrt{3}\) es \textbf{máximo local}.

- En \(x=\sqrt{3}\), la función pasa de decreciente (\(-\)) a creciente (\(+\)).  
  Entonces \(x=\sqrt{3}\) es \textbf{mínimo local}.

\textbf{Conclusión:}

\[
\text{Creciente en }(-\infty,-\sqrt{3})\cup(\sqrt{3},\infty).
\]

\[
\text{Decreciente en }(-\sqrt{3},\sqrt{3}).
\]

Máximo local en \(x=-\sqrt{3}\).  
Mínimo local en \(x=\sqrt{3}\).
\end{solucionclick}


% ==========================================================
% 6. CONCAVIDAD Y PUNTOS DE INFLEXIÓN
% ==========================================================

\section*{6. Concavidad y puntos de inflexión}

En estos ejercicios se busca:
\begin{itemize}
  \item Calcular \(f''(x)\).
  \item Determinar intervalos de concavidad hacia arriba / hacia abajo.
  \item Localizar puntos de inflexión.
\end{itemize}

% ---------- EJERCICIO 6.1 ----------
\begin{ejercicio}
Considere
\[
f(x)=x^{3}-6x^{2}+9x.
\]
a) Calcule \(f'(x)\) y \(f''(x)\). \\
b) Determine en qué intervalos la función es cóncava hacia arriba y hacia abajo. \\
c) Encuentre los puntos de inflexión.
\end{ejercicio}

\begin{solucionclick}
\textbf{Paso 1. Derivadas}

\[
f'(x)=3x^{2}-12x+9.
\]

\[
f''(x)=6x-12.
\]

\textbf{Paso 2. Concavidad}

Para saber dónde es cóncava hacia arriba o hacia abajo, analizamos el signo de \(f''(x)=6x-12\).

Resolvemos \(f''(x)=0\):
\[
6x-12=0
\quad\Longrightarrow\quad
6x=12
\quad\Longrightarrow\quad
x=2.
\]

Tomamos intervalos:
\[
(-\infty,2)\quad\text{y}\quad (2,\infty).
\]

Escogemos un punto de prueba en cada intervalo:

1. Si \(x<2\), por ejemplo \(x=0\):
\[
f''(0)=6\cdot 0-12=-12<0
\quad\Rightarrow\quad
\text{cóncava hacia abajo en }(-\infty,2).
\]

2. Si \(x>2\), por ejemplo \(x=5\):
\[
f''(5)=6\cdot 5-12=30-12=18>0
\quad\Rightarrow\quad
\text{cóncava hacia arriba en }(2,\infty).
\]

\textbf{Paso 3. Punto de inflexión}

Como la concavidad cambia en \(x=2\), allí hay un punto de inflexión.

Coordenada \(y\):
\[
f(2)=2^{3}-6\cdot 2^{2}+9\cdot 2
=8-24+18=2.
\]

\textbf{Conclusión:}

Cóncava hacia abajo en \((-\infty,2)\). \\
Cóncava hacia arriba en \((2,\infty)\). \\
Punto de inflexión en \((2,2)\).
\end{solucionclick}


% ---------- EJERCICIO 6.2 ----------
\begin{ejercicio}
Sea
\[
g(x)=x^{4}-4x^{3}.
\]
a) Determine \(g''(x)\). \\
b) Determine los intervalos de concavidad. \\
c) Determine los puntos de inflexión (si existen).
\end{ejercicio}

\begin{solucionclick}
\textbf{Paso 1. Derivadas}

\[
g'(x)=4x^{3}-12x^{2}=4x^{2}(x-3).
\]

\[
g''(x)=12x^{2}-24x=12x(x-2).
\]

\textbf{Paso 2. Zonas según el signo de \(g''(x)\)}

Resolvemos \(g''(x)=0\):
\[
12x(x-2)=0
\quad\Longrightarrow\quad
x=0
\quad\text{o}\quad
x=2.
\]

Analizamos intervalos:
\[
(-\infty,0),\quad (0,2),\quad (2,\infty).
\]

Tomemos puntos de prueba:

1. Para \(x<0\), por ejemplo \(x=-1\):
\[
g''(-1)=12(-1)(-1-2)=12(-1)(-3)=36>0
\quad\Rightarrow\quad
\text{cóncava hacia arriba en }(-\infty,0).
\]

2. Para \(0<x<2\), por ejemplo \(x=1\):
\[
g''(1)=12(1)(1-2)=12(-1)=-12<0
\quad\Rightarrow\quad
\text{cóncava hacia abajo en }(0,2).
\]

3. Para \(x>2\), por ejemplo \(x=3\):
\[
g''(3)=12(3)(3-2)=12\cdot 3\cdot 1=36>0
\quad\Rightarrow\quad
\text{cóncava hacia arriba en }(2,\infty).
\]

\textbf{Paso 3. Puntos de inflexión}

Hay cambio de concavidad en \(x=0\) y en \(x=2\), así que ambos son puntos de inflexión.

Coordenadas:

\[
g(0)=0^{4}-4\cdot 0^{3}=0.
\]

\[
g(2)=2^{4}-4\cdot 2^{3}
=16-32=-16.
\]

\textbf{Conclusión:}

Cóncava hacia arriba en \((-\infty,0)\cup(2,\infty)\). \\
Cóncava hacia abajo en \((0,2)\). \\
Puntos de inflexión en \((0,0)\) y \((2,-16)\).
\end{solucionclick}


% ---------- EJERCICIO 6.3 ----------
\begin{ejercicio}
Para la función
\[
h(x)=x^{3}-3x,
\]
a) halle los puntos de inflexión; \\
b) describa brevemente la forma cualitativa de la gráfica (sin dibujar con precisión).
\end{ejercicio}

\begin{solucionclick}
\textbf{Paso 1. Derivadas}

\[
h'(x)=3x^{2}-3,
\qquad
h''(x)=6x.
\]

\textbf{Paso 2. Inflección}

Resolvemos \(h''(x)=0\):
\[
6x=0
\quad\Longrightarrow\quad
x=0.
\]

Analizamos el signo de \(h''(x)=6x\):

- Si \(x<0\), por ejemplo \(x=-1\):
\[
h''(-1)=6(-1)=-6<0
\quad\Rightarrow\quad
\text{cóncava hacia abajo a la izquierda de }0.
\]

- Si \(x>0\), por ejemplo \(x=1\):
\[
h''(1)=6(1)=6>0
\quad\Rightarrow\quad
\text{cóncava hacia arriba a la derecha de }0.
\]

Entonces en \(x=0\) cambia la concavidad \(\Rightarrow\) punto de inflexión.

Coordenada \(y\):
\[
h(0)=0^{3}-3\cdot 0=0.
\]

\textbf{Paso 3. Forma cualitativa}

- Para \(x \ll 0\): la función viene muy abajo y sube (\(h'(x)>0\) cuando \(|x|\) grande).  
- Cerca de \(x=-\sqrt{1}\approx -1\) y \(x=\sqrt{1}\approx 1\), hay cambios de crecimiento/decrecimiento (de hecho máximos y mínimos locales).  
- Cambia de concavidad en \((0,0)\).

\textbf{Conclusión:}

Punto de inflexión en \((0,0)\).  
Gráfica en forma de “S”: baja, luego sube, con cambio de curvatura en el origen.
\end{solucionclick}


% ==========================================================
% 7. OPTIMIZACIÓN
% ==========================================================

\section*{7. Problemas de optimización}

En los siguientes ejercicios se debe:
\begin{itemize}
  \item Plantear la función que se quiere optimizar (área, costo, etc.).
  \item Usar las restricciones físicas para escribir esa función con una sola variable.
  \item Derivar, igualar a cero, y verificar máximo / mínimo.
  \item Responder con interpretación física (unidades).
\end{itemize}

% ---------- EJERCICIO 7.1 (CERCA Y RÍO) ----------
\begin{ejercicio}
Un agricultor tiene \SI{800}{m} de cerca para cerrar un lote rectangular junto a un río recto.
No necesita cercar el lado que colinda con el río.
¿Cuáles deben ser las dimensiones del rectángulo para que el área sea máxima?
\end{ejercicio}

\begin{solucionclick}
\textbf{Paso 1. Definición de variables}

Sea \(x\) el ancho perpendicular al río, y \(y\) el lado paralelo al río.

Como el lado que da al río no se cerca, la cerca se usa en:
\[
\text{perímetro cercado } = 2x + y = 800.
\quad (1)
\]

Área del lote:
\[
A = x \cdot y.
\quad (2)
\]

\textbf{Paso 2. Resolver la restricción para una sola variable}

De (1):
\[
y = 800 - 2x.
\]

Sustituimos en (2):
\[
A(x)= x(800-2x)=800x-2x^{2}.
\]

Dominio físico: \(x>0\), \(y>0 \Rightarrow 800-2x>0 \Rightarrow x<400\).  
Entonces \(0<x<400\).

\textbf{Paso 3. Derivar e igualar a cero}

\[
A'(x)=800-4x.
\]

Punto crítico:
\[
800-4x=0
\quad\Longrightarrow\quad
4x=800
\quad\Longrightarrow\quad
x=200.
\]

\textbf{Paso 4. Verificar máximo}

\[
A''(x)= -4 <0,
\]
constante negativa. Entonces es un \textbf{máximo}.

\textbf{Paso 5. Hallar \(y\)}

\[
y=800-2x=800-2\cdot 200=800-400=400.
\]

\textbf{Conclusión (con unidades):}

El área máxima se logra con
\[
x=\SI{200}{m},\qquad y=\SI{400}{m}.
\]

Dimensiones óptimas: \(\SI{200}{m} \times \SI{400}{m}\).
\end{solucionclick}


% ---------- EJERCICIO 7.2 (LATA CILÍNDRICA) ----------
\begin{ejercicio}
Se desea fabricar una lata cilíndrica con volumen fijo de \(\SI{100}{cm^{3}}\).
Encuentre el radio \(r\) y la altura \(h\) que minimizan el área de material usado (costo mínimo).
Suponga que la lata tiene tapa y base circulares.
\end{ejercicio}

\begin{solucionclick}
\textbf{Paso 1. Datos geométricos}

Para un cilindro:
\[
V = \pi r^{2} h,
\qquad
S = 2\pi r^{2} + 2\pi r h,
\]
donde \(S\) es el área superficial total (costo de material).

Tenemos el volumen fijo:
\[
V = \SI{100}{cm^{3}}.
\]

\textbf{Paso 2. Despejar \(h\) en función de \(r\)}

\[
V = \pi r^{2} h = 100
\quad\Longrightarrow\quad
h = \frac{100}{\pi r^{2}}.
\]

\textbf{Paso 3. Sustituir en el área}

\[
S(r) = 2\pi r^{2} + 2\pi r \left( \frac{100}{\pi r^{2}} \right)
= 2\pi r^{2} + \frac{200}{r}.
\]

Dominio físico: \(r>0\).

\textbf{Paso 4. Derivar e igualar a cero}

\[
S'(r)= 4\pi r - \frac{200}{r^{2}}.
\]

Buscamos \(S'(r)=0\):
\[
4\pi r - \frac{200}{r^{2}}=0
\quad\Longrightarrow\quad
4\pi r = \frac{200}{r^{2}}
\quad\Longrightarrow\quad
4\pi r^{3}=200
\quad\Longrightarrow\quad
r^{3}=\frac{200}{4\pi}=\frac{50}{\pi}.
\]

Entonces:
\[
r = \left(\frac{50}{\pi}\right)^{1/3}\ \text{cm}.
\]

\textbf{Paso 5. Altura correspondiente}

\[
h = \frac{100}{\pi r^{2}}
= \frac{100}{\pi \left( \frac{50}{\pi} \right)^{2/3}}
= \frac{100}{ \pi^{1/3} \, 50^{2/3} }.
\]

(Se puede dejar así; lo importante en la tutoría es el método.)

\textbf{Paso 6. Clasificar (mínimo)}

\[
S''(r)=4\pi + \frac{400}{r^{3}} >0 \quad\text{para } r>0.
\]

Como \(S''(r)>0\), el punto crítico produce un \textbf{mínimo}.

\textbf{Conclusión (con unidades):}

El costo de material es mínimo cuando
\[
r = \left(\frac{50}{\pi}\right)^{1/3}\ \text{cm},
\qquad
h = \frac{100}{\pi r^{2}}\ \text{cm}.
\]

Ese par \((r,h)\) minimiza el área superficial (por tanto, el material usado).
\end{solucionclick}


% ---------- EJERCICIO 7.3 (DOS NÚMEROS) ----------
\begin{ejercicio}
Halle dos números reales \(x\) y \(y\) cuya diferencia sea \(100\) y cuyo producto sea mínimo.
Es decir, encuentre \(x\) y \(y\) que minimicen \(P = x\cdot y\) bajo la condición \(x-y=100\).
\end{ejercicio}

\begin{solucionclick}
\textbf{Paso 1. Restricción}

Tenemos:
\[
x-y=100
\quad\Longrightarrow\quad
x = y+100.
\]

\textbf{Paso 2. Función a minimizar}

Producto:
\[
P = x\cdot y = (y+100)\,y = y^{2}+100y.
\]

Entonces
\[
P(y)=y^{2}+100y.
\]

\textbf{Paso 3. Derivar e igualar a cero}

\[
P'(y)=2y+100.
\]

Punto crítico:
\[
2y+100=0
\quad\Longrightarrow\quad
2y=-100
\quad\Longrightarrow\quad
y=-50.
\]

Luego:
\[
x = y+100 = -50+100 = 50.
\]

\textbf{Paso 4. Clasificar el extremo}

\[
P''(y)=2>0,
\]
constante positiva, entonces el punto crítico da \textbf{mínimo}.

\textbf{Paso 5. Conclusión}

Los dos números son
\[
x=50,\qquad y=-50.
\]

El producto mínimo es
\[
P_{\min}=50\cdot(-50)=-2500.
\]

Interpretación:
entre todos los pares \((x,y)\) con diferencia \(100\), el par simétrico respecto a \(0\), es decir
\((50,-50)\), produce el producto más pequeño.
\end{solucionclick}

% =================== FIN DE EJERCICIOS ===================
\noindent\fbox{%
\parbox{\linewidth}{%
%\textbf{Julián Arias Meza} es profesor de Matemática graduado de la Universidad de Costa Rica. 
%Cursó la carrera de Física en la misma universidad y actualmente cursa la licenciatura en 
%Ingeniería Física en el Tecnológico de Costa Rica (TEC). 
Integración  del uso de herramientas digitales y de inteligencia artificial (IA), 
enfocadas en la mejora de las capacidades pedagógicas: 
diseño de materiales adaptativos, retroalimentación automática y 
optimización de la evaluación formativa.

\medskip
\textbf{Áreas}
\begin{itemize}
  \item \textbf{Matemática:} aritmética, álgebra, trigonometría, geometría analítica, cálculo (diferencial e integral), probabilidad y estadística.
  \item \textbf{Física:} mecánica, ondas, electricidad y magnetismo, termodinámica y óptica; con resolución de problemas, interpretación física y uso riguroso del SI.
  \item \textbf{Química:} nomenclatura con sistema Stock y sistema estequiométrico, óxidos, hidruros, hidróxidos, hidrácidos, sales binarias, tipos de reacciones, balanceo y estequiometría.
  \item \textbf{Admisión (universidades y colegios):} preparación integral para exámenes de ingreso (UCR, TEC, UNA, Colegios Científicos, COVAO, entre otros): diagnóstico inicial, plan de estudio, técnicas de resolución, simulacros cronometrados y análisis de errores.
  \item \textbf{Formatos:} guías teóricas, bancos de ejercicios, prácticas con soluciones paso a paso, simulacros, rúbricas, presentaciones y resúmenes ejecutivos.
\end{itemize}

\text{Se entrega en:} formato PDF listo para imprimir, o \texttt{.docx} (Word).

\medskip
\textbf{Contacto (WhatsApp):} \texttt{7076-9371}

}}
\end{document}
