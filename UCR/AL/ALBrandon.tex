% ==========================================================
% Subespacio vectorial en M(2x2,R) — Desarrollo completo
% ==========================================================
\documentclass[11pt,letterpaper]{article}

% --- Idioma y formato numérico ---
\usepackage[T1]{fontenc}
\usepackage[utf8]{inputenc}
\usepackage[spanish, es-nodecimaldot]{babel}
\usepackage{amsmath, amssymb, siunitx}

\sisetup{
  locale=DE,
  output-decimal-marker={,},
  per-mode=symbol,
  exponent-product=\cdot,
  group-minimum-digits=4
}

\usepackage[a4paper,margin=2.4cm]{geometry}

% ----------------------------------------------------------
\begin{document}

\begin{center}
  {\Large \textbf{Ejercicio — Subespacio vectorial en $M(2\times2,\mathbb{R})$}}\\[2mm]
\end{center}

Se define el subespacio:
\[
V = \left\{
\begin{pmatrix}
a & b \\[2pt]
c & d
\end{pmatrix}
\in M(2\times2,\mathbb{R})
\;:\;
d = 2a - c,\; c = a + 2b
\right\}.
\]

---

\section*{(a) Escribir \(V\) como un conjunto generado}

\textbf{Condiciones dadas:}
\[
c = a + 2b, \qquad d = 2a - c.
\]

\textbf{Sustitución de \(c\) en \(d\):}
\[
d = 2a - (a + 2b) = a - 2b.
\]

\textbf{Por tanto:}
\[
\begin{pmatrix}
a & b \\[2pt]
c & d
\end{pmatrix}
=
\begin{pmatrix}
a & b \\[2pt]
a + 2b & a - 2b
\end{pmatrix}.
\]

\textbf{Separando términos:}
\[
\begin{pmatrix}
a & b \\[2pt]
a + 2b & a - 2b
\end{pmatrix}
=
a
\begin{pmatrix}
1 & 0 \\[2pt]
1 & 1
\end{pmatrix}
+
b
\begin{pmatrix}
0 & 1 \\[2pt]
2 & -2
\end{pmatrix}.
\]

\textbf{Entonces:}
\[
V = \operatorname{gen}\left\{
\begin{pmatrix}
1 & 0 \\[2pt]
1 & 1
\end{pmatrix},
\begin{pmatrix}
0 & 1 \\[2pt]
2 & -2
\end{pmatrix}
\right\}.
\]

---

\section*{(b) Base de \(V\) y su dimensión}

\textbf{Propuesta de base:}
\[
\mathcal{B} =
\left\{
A_1 =
\begin{pmatrix}
1 & 0 \\[2pt]
1 & 1
\end{pmatrix},
\;
A_2 =
\begin{pmatrix}
0 & 1 \\[2pt]
2 & -2
\end{pmatrix}
\right\}.
\]

\textbf{Verificación de independencia lineal:}
\[
\alpha A_1 + \beta A_2 =
\begin{pmatrix}
0 & 0 \\[2pt]
0 & 0
\end{pmatrix}.
\]

\textbf{Sustituyendo:}
\[
\alpha
\begin{pmatrix}
1 & 0 \\[2pt]
1 & 1
\end{pmatrix}
+
\beta
\begin{pmatrix}
0 & 1 \\[2pt]
2 & -2
\end{pmatrix}
=
\begin{pmatrix}
0 & 0 \\[2pt]
0 & 0
\end{pmatrix}.
\]

\textbf{Igualando componentes:}
\[
\begin{cases}
\alpha = 0,\\[2pt]
\beta = 0.
\end{cases}
\]

\textbf{Conclusión:} \(\alpha = \beta = 0 \Rightarrow A_1, A_2\) son linealmente independientes.

\[
\boxed{
\mathcal{B} =
\left\{
\begin{pmatrix}
1 & 0 \\[2pt]
1 & 1
\end{pmatrix},
\begin{pmatrix}
0 & 1 \\[2pt]
2 & -2
\end{pmatrix}
\right\},
\qquad
\dim(V) = 2.
}
\]

---

\section*{(c) Dos matrices linealmente independientes que no pertenezcan a \(V\)}

\textbf{Sea:}
\[
M_1 =
\begin{pmatrix}
0 & 0 \\[2pt]
1 & 0
\end{pmatrix},
\qquad
M_2 =
\begin{pmatrix}
0 & 0 \\[2pt]
0 & 1
\end{pmatrix}.
\]

\textbf{Verificación de pertenencia:}

Para \(M_1\):
\[
a=0,\; b=0,\; c=1,\; d=0.
\]
Se verifica \(c=a+2b \Rightarrow 1 \neq 0\), por lo tanto \(M_1 \notin V.\)

Para \(M_2\):
\[
a=0,\; b=0,\; c=0,\; d=1.
\]
Se verifica \(d=2a-c \Rightarrow 1 \neq 0\), por lo tanto \(M_2 \notin V.\)

\textbf{Verificación de independencia lineal:}
\[
\lambda M_1 + \mu M_2 = 0
\quad\Rightarrow\quad
\begin{cases}
\lambda = 0,\\[2pt]
\mu = 0.
\end{cases}
\]

\textbf{Conclusión:} son linealmente independientes y no pertenecen a \(V\).

\[
\boxed{
M_1 =
\begin{pmatrix}
0 & 0 \\[2pt]
1 & 0
\end{pmatrix},
\quad
M_2 =
\begin{pmatrix}
0 & 0 \\[2pt]
0 & 1
\end{pmatrix}.
}
\]

---

\textbf{Resumen final:}
\[
\begin{aligned}
V &= \operatorname{gen}\left\{
\begin{pmatrix}
1 & 0 \\[2pt]
1 & 1
\end{pmatrix},
\begin{pmatrix}
0 & 1 \\[2pt]
2 & -2
\end{pmatrix}
\right\}, \\[4pt]
\mathcal{B} &= \left\{
\begin{pmatrix}
1 & 0 \\[2pt]
1 & 1
\end{pmatrix},
\begin{pmatrix}
0 & 1 \\[2pt]
2 & -2
\end{pmatrix}
\right\},
\qquad
\dim(V) = 2,\\[4pt]
M_1 &=
\begin{pmatrix}
0 & 0 \\[2pt]
1 & 0
\end{pmatrix},
\quad
M_2 =
\begin{pmatrix}
0 & 0 \\[2pt]
0 & 1
\end{pmatrix}.
\end{aligned}
\]

\end{document}