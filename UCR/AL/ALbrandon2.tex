% ==========================================================
% Ejercicio — Base en P2 y coordenadas de un polinomio
% Desarrollo paso a paso (forma larga)
% ==========================================================
\documentclass[11pt,letterpaper]{article}

\usepackage[T1]{fontenc}
\usepackage[utf8]{inputenc}
\usepackage[spanish, es-nodecimaldot]{babel}
\usepackage{amsmath, amssymb, siunitx}
\usepackage[a4paper,margin=2.4cm]{geometry}

\sisetup{
  locale=DE,
  output-decimal-marker={,},
  per-mode=symbol,
  exponent-product=\cdot,
  group-minimum-digits=4
}

\begin{document}

\begin{center}
  {\Large \textbf{Ejercicio — Coordenadas de un polinomio en una base de $\mathcal{P}_2$}}\\[2mm]
\end{center}

Se define
\[
\mathcal{B} = \{\,1 + x^2,\; 1 + 2x + x^2,\; -1 + x\,\} \subset \mathcal{P}_2.
\]

% ==========================================================
\section*{(a) Verificar que $\mathcal{B}$ es base de $\mathcal{P}_2$}

\subsection*{Paso 1. Expresar cada polinomio en la base canónica $\{1,x,x^2\}$}
\[
p_1(x)=1+x^2 \quad\Rightarrow\quad (1,0,1),
\]
\[
p_2(x)=1+2x+x^2 \quad\Rightarrow\quad (1,2,1),
\]
\[
p_3(x)=-1+x \quad\Rightarrow\quad (-1,1,0).
\]

\subsection*{Paso 2. Construir la matriz con esos vectores como columnas}
\[
M=
\begin{pmatrix}
1 & 1 & -1\\
0 & 2 & 1\\
1 & 1 & 0
\end{pmatrix}.
\]

\subsection*{Paso 3. Calcular el determinante de $M$}
Usamos expansión por la primera fila:
\[
\det(M)
=
1
\begin{vmatrix}
2 & 1\\
1 & 0
\end{vmatrix}
-
1
\begin{vmatrix}
0 & 1\\
1 & 0
\end{vmatrix}
+
(-1)
\begin{vmatrix}
0 & 2\\
1 & 1
\end{vmatrix}.
\]

Ahora cada menor $2\times 2$:
\[
\begin{vmatrix}
2 & 1\\
1 & 0
\end{vmatrix}
=2\cdot0-1\cdot1=-1,
\qquad
\begin{vmatrix}
0 & 1\\
1 & 0
\end{vmatrix}
=0\cdot0-1\cdot1=-1,
\qquad
\begin{vmatrix}
0 & 2\\
1 & 1
\end{vmatrix}
=0\cdot1-2\cdot1=-2.
\]

Sustituyendo:
\[
\det(M)
=
1(-1)
-
1(-1)
+
(-1)(-2)
=
(-1) - (-1) + 2
=
-1+1+2
=
2.
\]

\subsection*{Conclusión}
Como $\det(M)=2\neq0$, las tres funciones polinomiales son linealmente independientes.

Además, $\mathcal{P}_2$ tiene dimensión $3$, y nuestro conjunto tiene $3$ elementos independientes.

\[
\boxed{
\mathcal{B} \text{ es base de } \mathcal{P}_2.
}
\]

% ==========================================================
\section*{(b) Vector de coordenadas de $p(x)=-3x^2+7x+6$ en la base $\mathcal{B}$}

Buscamos escalares $a,b,c\in\mathbb{R}$ tales que
\[
p(x)
=
a(1+x^2)
+
b(1+2x+x^2)
+
c(-1+x),
\]
donde
\[
p(x)=-3x^2+7x+6.
\]

\subsection*{Paso 1. Desarrollar el lado derecho}
Primero expandimos:
\[
a(1+x^2) = a + a x^2,
\]
\[
b(1+2x+x^2) = b + 2b x + b x^2,
\]
\[
c(-1+x) = -c + c x.
\]

Sumamos término a término:
\[
a(1+x^2)+b(1+2x+x^2)+c(-1+x)
=
(a+b-c)
+
(2b+c)x
+
(a+b)x^2.
\]

\subsection*{Paso 2. Igualar coeficientes con $p(x)=6+7x-3x^2$}
Comparando coeficientes de $1$, $x$, $x^2$:
\[
\begin{cases}
a + b - c = 6,\\[4pt]
2b + c = 7,\\[4pt]
a + b = -3.
\end{cases}
\tag{$\ast$}
\]

Este es un sistema lineal en $a,b,c$.

\subsection*{Paso 3. Resolver el sistema con matriz aumentada (Gauss)}
Escribimos la matriz aumentada asociada a $(\ast)$ en el orden de ecuaciones dado:
\[
\left[
\begin{array}{ccc|c}
1 & 1 & -1 & 6\\
0 & 2 & 1 & 7\\
1 & 1 & 0 & -3
\end{array}
\right].
\]

Llamemos a las filas $F_1,F_2,F_3$.

\subsubsection*{Paso 3.1. Hacer ceros debajo del pivote de la primera columna}
Restamos $F_1$ a $F_3$:
\[
F_3 \leftarrow F_3 - F_1
=
\left[
\begin{array}{ccc|c}
1 & 1 & -1 & 6\\
0 & 2 & 1 & 7\\
0 & 0 & 1 & -9
\end{array}
\right].
\]

Justificación:
\[
F_3 - F_1 =
[1-1,\ 1-1,\ 0-(-1),\ -3-6]
=
[0,\ 0,\ 1,\ -9].
\]

\subsubsection*{Paso 3.2. Usar $F_3$ para eliminar la tercera columna en $F_1$ y $F_2$}
Primero en $F_1$:
\[
F_1 \leftarrow F_1 + F_3
=
[1,\ 1,\ -1,\ 6] + [0,\ 0,\ 1,\ -9]
=
[1,\ 1,\ 0,\ -3].
\]

Luego en $F_2$:
\[
F_2 \leftarrow F_2 - F_3
=
[0,\ 2,\ 1,\ 7] - [0,\ 0,\ 1,\ -9]
=
[0,\ 2,\ 0,\ 16].
\]

La matriz ahora es:
\[
\left[
\begin{array}{ccc|c}
1 & 1 & 0 & -3\\
0 & 2 & 0 & 16\\
0 & 0 & 1 & -9
\end{array}
\right].
\]

\subsubsection*{Paso 3.3. Resolver por sustitución hacia atrás}
De la tercera fila:
\[
c = -9.
\]

De la segunda fila:
\[
2b = 16
\quad\Rightarrow\quad
b = 8.
\]

De la primera fila:
\[
a + b = -3
\quad\Rightarrow\quad
a + 8 = -3
\quad\Rightarrow\quad
a = -11.
\]

\subsection*{Paso 4. Coordenadas finales}
Hemos encontrado:
\[
a=-11,\qquad b=8,\qquad c=-9.
\]

Por definición,
\[
[p(x)]_{\mathcal{B}}
=
\begin{pmatrix}
a\\ b\\ c
\end{pmatrix}
=
\begin{pmatrix}
-11\\
8\\
-9
\end{pmatrix}.
\]

\[
\boxed{
[p(x)]_{\mathcal{B}}
=
\begin{pmatrix}
-11\\[2pt]
8\\[2pt]
-9
\end{pmatrix}
}
\]

% ==========================================================
\section*{(c) Polinomio con vector de coordenadas $(-1,2,-3)^T$ en la base $\mathcal{B}$}

Ahora se nos da el vector de coordenadas directamente:
\[
[p(x)]_{\mathcal{B}}=
\begin{pmatrix}
-1\\
2\\
-3
\end{pmatrix}.
\]

Esto significa:
\[
p(x)
=
(-1)\,(1+x^2)
+
2\,(1+2x+x^2)
+
(-3)\,(-1+x).
\]

\subsection*{Paso 1. Desarrollar cada término}
\[
(-1)(1+x^2) = -1 - x^2,
\]
\[
2(1+2x+x^2) = 2 + 4x + 2x^2,
\]
\[
(-3)(-1+x) = 3 - 3x.
\]

\subsection*{Paso 2. Sumar todos los términos}
Sumamos constantes, términos en $x$ y términos en $x^2$:
\[
p(x)
=
(-1 - x^2)
+
(2 + 4x + 2x^2)
+
(3 - 3x).
\]

Agrupando:
\[
p(x)
=
\underbrace{(-1+2+3)}_{\text{término constante}}
+
\underbrace{(4x-3x)}_{\text{término en }x}
+
\underbrace{(-x^2+2x^2)}_{\text{término en }x^2}.
\]

Calculando:
\[
-1 + 2 + 3 = 4,\qquad
4x - 3x = x,\qquad
-x^2 + 2x^2 = x^2.
\]

Por tanto:
\[
p(x) = 4 + x + x^2.
\]

\[
\boxed{
p(x)=x^2 + x + 4
}
\]

% ==========================================================
\section*{Resumen final}

\[
\begin{aligned}
\mathcal{B} &= \{\,1+x^2,\;1+2x+x^2,\;-1+x\,\}
\text{ es base de } \mathcal{P}_2.\\[6pt]
[p(x)]_{\mathcal{B}} &= (-11,\,8,\,-9)^T
\quad\text{para }p(x)=-3x^2+7x+6.\\[6pt]
p_{\text{con coords }(-1,2,-3)^T}(x) &= x^2 + x + 4.
\end{aligned}
\]

\end{document}