% ==========================================================
% Ejercicio — Subespacio en R3 y coordenadas en una base
% ==========================================================
\documentclass[11pt,letterpaper]{article}

\usepackage[T1]{fontenc}
\usepackage[utf8]{inputenc}
\usepackage[spanish, es-nodecimaldot]{babel}
\usepackage{amsmath, amssymb, siunitx}
\usepackage[a4paper,margin=2.4cm]{geometry}

\sisetup{
  locale=DE,
  output-decimal-marker={,},
  per-mode=symbol,
  exponent-product=\cdot,
  group-minimum-digits=4
}

\begin{document}

\begin{center}
  {\Large \textbf{Ejercicio — Subespacio vectorial y coordenadas en una base}}\\[2mm]
\end{center}

\[
S = \{(x,y,z)\in\mathbb{R}^3 : x+y+z=0\}.
\]

% ----------------------------------------------------------
\section*{(a) Verificar que $S$ es subespacio de $\mathbb{R}^3$}

\textbf{1. Vector cero:}
\[
(0,0,0)\in S \text{ porque } 0+0+0=0.
\]

\textbf{2. Cierre bajo suma:}
\[
(x_1+y_1+z_1=0,\; x_2+y_2+z_2=0)
\Rightarrow
(x_1+x_2)+(y_1+y_2)+(z_1+z_2)=0.
\]

\textbf{3. Cierre bajo producto escalar:}
\[
x+y+z=0 \Rightarrow kx+ky+kz=k(x+y+z)=0.
\]

\[
\boxed{S \text{ es subespacio de } \mathbb{R}^3.}
\]

% ----------------------------------------------------------
\section*{(b) Verificar que $\mathcal{B}=\{(-1,1,0),(-1,0,1)\}$ es base de $S$}

\textbf{Pertenencia:}
\[
(-1)+1+0=0,\quad (-1)+0+1=0 \Rightarrow \text{ambos pertenecen a }S.
\]

\textbf{Independencia lineal:}
\[
a(-1,1,0)+b(-1,0,1)=(0,0,0).
\]
\[
(-a-b,\; a,\; b)=(0,0,0)
\Rightarrow
\begin{cases}
-a-b=0,\\
a=0,\\
b=0.
\end{cases}
\Rightarrow a=b=0.
\]

\textbf{Conclusión:} Son linealmente independientes y $S$ tiene dimensión $2$,
por tanto
\[
\boxed{\mathcal{B}\text{ es base de }S.}
\]

% ----------------------------------------------------------
\section*{(c) Coordenadas de $(0,1,-1)$ en la base $\mathcal{B}$}

Buscamos $a,b$ tales que:
\[
(0,1,-1)=a(-1,1,0)+b(-1,0,1).
\]

Desarrollamos:
\[
(-a-b,\; a,\; b)=(0,1,-1).
\]
\[
\begin{cases}
-a-b=0,\\
a=1,\\
b=-1.
\end{cases}
\]

\[
\boxed{
[(0,1,-1)]_{\mathcal{B}}=
\begin{pmatrix}
1\\
-1
\end{pmatrix}
}
\]

% ----------------------------------------------------------
\section*{Resumen final}
\[
\begin{aligned}
S &= \{(x,y,z)\in\mathbb{R}^3:x+y+z=0\}\text{ subespacio de }\mathbb{R}^3,\\[4pt]
\mathcal{B} &= \{(-1,1,0),(-1,0,1)\}\text{ base de }S,\\[4pt]
[(0,1,-1)]_{\mathcal{B}} &= (1,-1)^T.
\end{aligned}
\]

\end{document}