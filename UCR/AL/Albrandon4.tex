% ==========================================================
% Ejercicio — Intersección de dos planos y punto común con una recta
% ==========================================================
\documentclass[11pt,letterpaper]{article}

\usepackage[T1]{fontenc}
\usepackage[utf8]{inputenc}
\usepackage[spanish, es-nodecimaldot]{babel}
\usepackage{amsmath, amssymb, siunitx}
\usepackage[a4paper,margin=2.4cm]{geometry}

\sisetup{
  locale=DE,
  output-decimal-marker={,},
  per-mode=symbol,
  exponent-product=\cdot,
  group-minimum-digits=4
}

\begin{document}

\begin{center}
  {\Large \textbf{Ejercicio — Intersección de planos y punto de intersección con una recta}}\\[2mm]
\end{center}

Sean los planos:
\[
\pi: 3x + y - 8z = -14, \qquad
\rho: 6x + 2y - z = 17.
\]

% ----------------------------------------------------------
\section*{(a) Ecuaciones paramétricas de la recta $\ell = \pi \cap \rho$}

\[
\begin{cases}
3x + y - 8z = -14,\\[3pt]
6x + 2y - z = 17.
\end{cases}
\]

De la primera ecuación:
\[
y = -3x + 8z - 14.
\]

Sustituyendo en la segunda:
\[
6x + 2(-3x + 8z - 14) - z = 17
\Rightarrow 15z = 45
\Rightarrow z = 3.
\]

Sustituyendo $z=3$ en la primera:
\[
3x + y - 24 = -14
\Rightarrow 3x + y = 10
\Rightarrow y = 10 - 3x.
\]

Sea $x=t$, entonces:
\[
\boxed{
\begin{cases}
x = t,\\
y = 10 - 3t,\\
z = 3.
\end{cases}
}
\]

Forma vectorial:
\[
\ell:\quad (x,y,z) = (0,10,3) + t(1,-3,0).
\]

% ----------------------------------------------------------
\section*{(b) Punto de intersección entre $\ell$ y la recta dada}

Recta dada en forma simétrica:
\[
\frac{1-x}{2} = \frac{y-7}{6} = \frac{z+3}{3} = s.
\]

Ecuaciones paramétricas:
\[
\begin{cases}
x = 1 - 2s,\\
y = 7 + 6s,\\
z = -3 + 3s.
\end{cases}
\]

Igualando con $\ell$:
\[
\begin{cases}
t = 1 - 2s,\\
10 - 3t = 7 + 6s,\\
3 = -3 + 3s.
\end{cases}
\]

De la tercera ecuación:
\[
s = 2.
\]
Sustituyendo en la primera:
\[
t = 1 - 2(2) = -3.
\]
Verificamos en la segunda:
\[
10 - 3(-3) = 19,\quad 7 + 6(2) = 19.
\]

Coinciden, por tanto las rectas se intersectan en:
\[
\boxed{P(-3,\,19,\,3)}.
\]

% ----------------------------------------------------------
\section*{Resumen final}
\[
\begin{aligned}
\ell &:
\begin{cases}
x = t,\\
y = 10 - 3t,\\
z = 3
\end{cases},
\quad
r:
\begin{cases}
x = 1 - 2s,\\
y = 7 + 6s,\\
z = -3 + 3s
\end{cases},
\\[6pt]
\ell \cap r &= \{(-3,\,19,\,3)\}.
\end{aligned}
\]

\end{document}