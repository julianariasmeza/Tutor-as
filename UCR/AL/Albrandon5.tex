% ==========================================================
% Ejercicio — Intersección, ángulo, distancia y plano de dos rectas
% ==========================================================
\documentclass[11pt,letterpaper]{article}

\usepackage[T1]{fontenc}
\usepackage[utf8]{inputenc}
\usepackage[spanish, es-nodecimaldot]{babel}
\usepackage{amsmath, amssymb, siunitx}
\usepackage[a4paper,margin=2.4cm]{geometry}

\sisetup{
  locale=DE,
  output-decimal-marker={,},
  per-mode=symbol,
  exponent-product=\cdot,
  group-minimum-digits=4
}

\begin{document}

\begin{center}
  {\Large \textbf{Ejercicio — Intersección y plano determinado por dos rectas en $\mathbb{R}^3$}}\\[2mm]
\end{center}

Sean:
\[
\ell_1:\ \frac{x-1}{2}=\frac{y}{-1}=\frac{z+1}{3},
\qquad
\ell_2:\ \frac{x}{3}=\frac{y-2}{-3}=z-1.
\]

% ----------------------------------------------------------
\section*{(a) Punto de intersección}

Forma paramétrica:
\[
\ell_1:
\begin{cases}
x=1+2t,\\
y=-t,\\
z=-1+3t,
\end{cases}
\qquad
\ell_2:
\begin{cases}
x=3s,\\
y=2-3s,\\
z=1+s.
\end{cases}
\]

Igualando:
\[
\begin{cases}
1+2t=3s,\\
-t=2-3s,\\
-1+3t=1+s.
\end{cases}
\]
De la primera $s=\tfrac{1+2t}{3}$, sustituyendo:
\[
-t=2-3\!\left(\tfrac{1+2t}{3}\right)\Rightarrow t=1,\quad s=1.
\]
Punto común:
\[
P(3,-1,2).
\]
\[
\boxed{P(3,-1,2)}
\]

% ----------------------------------------------------------
\section*{(b) Ángulo entre las rectas}

\[
\vec{v}_1=(2,-1,3),\qquad \vec{v}_2=(3,-3,1).
\]
\[
\vec{v}_1\cdot\vec{v}_2=12,\qquad
|\vec{v}_1|=\sqrt{14},\quad |\vec{v}_2|=\sqrt{19}.
\]
\[
\boxed{
\theta=\arccos\!\left(\frac{12}{\sqrt{266}}\right).
}
\]

% ----------------------------------------------------------
\section*{(c) Distancia entre las rectas}

Como se intersecan en $P(3,-1,2)$:
\[
\boxed{d(\ell_1,\ell_2)=0.}
\]

% ----------------------------------------------------------
\section*{(d) Plano que contiene a $\ell_1$ y $\ell_2$}

Vectores directores:
\[
\vec{v}_1=(2,-1,3),\qquad \vec{v}_2=(3,-3,1).
\]

Vector normal:
\[
\vec{n}=\vec{v}_1\times\vec{v}_2=
\begin{vmatrix}
\mathbf{i} & \mathbf{j} & \mathbf{k}\\
2 & -1 & 3\\
3 & -3 & 1
\end{vmatrix}
=(8,7,-3).
\]

Ecuación del plano que pasa por $P(3,-1,2)$:
\[
8(x-3)+7(y+1)-3(z-2)=0
\Rightarrow 8x+7y-3z=11.
\]

\[
\boxed{8x+7y-3z=11.}
\]

% ----------------------------------------------------------
\section*{Resumen final}
\[
\begin{aligned}
P_{\text{intersección}} &= (3,-1,2),\\[3pt]
\theta &= \arccos\!\left(\frac{12}{\sqrt{266}}\right),\\[3pt]
d(\ell_1,\ell_2) &= 0,\\[3pt]
\pi &: 8x+7y-3z=11.
\end{aligned}
\]

\end{document}