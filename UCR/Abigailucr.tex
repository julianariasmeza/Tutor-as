% ==========================================================
% Secciones cónicas y coordenadas polares
% ==========================================================
\documentclass[11pt,letterpaper]{article}

% --- Idioma y codificación ---
\usepackage[T1]{fontenc}
\usepackage[utf8]{inputenc}
\usepackage[spanish, es-nodecimaldot]{babel}

% --- Matemática, unidades y símbolos ---
\usepackage{amsmath, amssymb, bm}
\usepackage{siunitx}
\sisetup{
  locale = DE,                 % coma como separador decimal
  output-decimal-marker = {,}, % coma decimal
  per-mode = symbol,
  exponent-product = \cdot,
  group-minimum-digits = 4
}

% --- Diseño de página ---
\usepackage[a4paper,margin=2.4cm]{geometry}
\usepackage{graphicx}
\usepackage{booktabs}
\usepackage{xcolor}

\renewcommand{\figurename}{Figura}
\renewcommand{\tablename}{Cuadro}

\begin{document}

\begin{center}
  {\Large\bfseries Ejercicios de Secciones Cónicas y Coordenadas Polares}\\[2mm]
\end{center}

% ==========================================================
\section*{1) Ecuación cónica \(x^{2}+12x+9y^{2}=0\)}

\textbf{Datos.}
\[
x^{2}+12x+9y^{2}=0
\]

\textbf{Fórmula.}
\[
x^{2}+12x = (x+6)^{2}-36
\]

\textbf{Sustitución.}
\[
(x+6)^{2}-36+9y^{2}=0 \quad \Rightarrow \quad (x+6)^{2}+9y^{2}=36
\]

\textbf{Cálculo (forma canónica).}
\[
\frac{(x+6)^{2}}{36}+\frac{y^{2}}{4}=1
\]
\noindent
Elipse con centro \((-6,0)\), semiejes:
\[
a=6,\quad b=2
\]
\[
c=\sqrt{a^{2}-b^{2}}=\sqrt{36-4}=4\sqrt{2}
\]
\[
F_{1,2}=(-6\mp4\sqrt{2},\,0)
\]

\textbf{Vértices y covértices.}
\[
V_{1}=(-12,0),\quad V_{2}=(0,0),\quad C_{1}=(-6,-2),\quad C_{2}=(-6,2)
\]

\textbf{Intersecciones con los ejes.}
\begin{itemize}
  \item Eje \(x\): \(y=0 \Rightarrow x(x+12)=0 \Rightarrow x=0,-12\)
  \item Eje \(y\): \(x=0 \Rightarrow y=0\)
\end{itemize}

\textbf{Conclusión.}
La elipse está centrada en \((-6,0)\), se extiende horizontalmente de \((-12,0)\) a \((0,0)\) y tiene altura total de \(4\) unidades (\(y=\pm2\)).

\vspace{5mm}
\hrule
\vspace{5mm}

% ==========================================================
\section*{2) Ecuaciones en coordenadas polares}

\[
r=\cos(2\theta), \qquad r=\sen^{2}\theta
\]

\subsection*{a) Asociación con las figuras}
\begin{itemize}
  \item \(\boxed{r=\cos(2\theta)} \rightarrow\) Figura (b): rosa de 4 pétalos, simétrica respecto a los ejes.
  \item \(\boxed{r=\sen^{2}\theta} \rightarrow\) Figura (a): curva tipo “gota” centrada sobre el eje \(y\), \(r\ge0\).
\end{itemize}

\subsection*{b) Expresión en coordenadas cartesianas}

\textbf{Para \(r=\cos(2\theta)\):}
\[
\cos(2\theta)=\frac{x^{2}-y^{2}}{x^{2}+y^{2}},\qquad r=\sqrt{x^{2}+y^{2}}
\]
\[
\sqrt{x^{2}+y^{2}}=\frac{x^{2}-y^{2}}{x^{2}+y^{2}}
\quad\Rightarrow\quad
(x^{2}+y^{2})^{3/2}=x^{2}-y^{2}
\]
\[
\boxed{(x^{2}+y^{2})^{3}=(x^{2}-y^{2})^{2}}
\]

\textbf{Para \(r=\sen^{2}\theta\):}
\[
\sen\theta=\frac{y}{r}\Rightarrow\sen^{2}\theta=\frac{y^{2}}{r^{2}}
\]
\[
r=\frac{y^{2}}{r^{2}}
\Rightarrow r^{3}=y^{2}
\Rightarrow (x^{2}+y^{2})^{3/2}=y^{2}
\]
\[
\boxed{(x^{2}+y^{2})^{3}=y^{4}}
\]

\subsection*{c) Completar los dibujos}
\begin{itemize}
  \item Para \(r=\cos(2\theta)\): agregar el pétalo superior, completando la rosa de 4 pétalos.
  \item Para \(r=\sen^{2}\theta\): lazo cerrado en la parte superior, simétrico respecto al eje \(y\), con extremos \((0,0)\) y \((0,1)\).
\end{itemize}

\vspace{5mm}
\hrule
\vspace{5mm}

\textbf{Conclusión general.}
Las ecuaciones polares permiten describir curvas cerradas o simétricas de forma compacta; su conversión a coordenadas cartesianas revela su estructura algebraica y facilita el análisis geométrico.

\end{document}
