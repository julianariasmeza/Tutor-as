\documentclass[12pt]{article}

%––––––––––––– PAQUETES –––––––––––––%
\usepackage[spanish]{babel}
\usepackage[utf8]{inputenc}
\usepackage{amsmath, amssymb, amsfonts}
\usepackage{lipsum}
\usepackage{geometry}
\usepackage{setspace}
\usepackage{hyperref}
\usepackage{bm}

\geometry{letterpaper, margin=2.5cm}
\setstretch{1.25}

%––––––––––––– DOCUMENTO –––––––––––––%
\begin{document}

\begin{center}
{\Large \textbf{Series de Potencias: Radio e Intervalo de Convergencia}}\\[4pt]
{\large Desarrollo de los incisos más complejos}
\end{center}

\bigskip

\section*{1. Inciso (c)}
\[
\sum_{n=1}^{\infty} 2^{n}n^{2}x^{n}
\]

\subsection*{Datos}
\[
a_n = 2^{n} n^{2} x^{n}
\]

\subsection*{Criterio de la razón}
\[
L=\lim_{n\to\infty}\left|\frac{a_{n+1}}{a_n}\right|
\]

\subsection*{Sustitución}
\[
\frac{a_{n+1}}{a_n}
=2|x|\frac{(n+1)^{2}}{n^{2}}
\]

\[
L=2|x|\lim_{n\to\infty}\left(1+\frac1n\right)^{2}
=2|x|
\]

\subsection*{Condición}
\[
2|x|<1
\quad\Longrightarrow\quad
|x|<\frac12
\]

\[
R=\frac12
\]

\subsection*{Extremos}
\begin{itemize}
\item \(x=\frac12\): la serie se convierte en \(n^{2}\), diverge.
\item \(x=-\frac12\): la serie es \(n^{2}(-1)^{n}\), diverge.
\end{itemize}

\subsection*{Conclusión}
\[
\boxed{R=\frac12,\qquad (-\tfrac12,\tfrac12)}
\]


%=========================================================
\section*{2. Inciso (d)}
\[
\sum_{n=1}^{\infty}\frac{(-1)^n 4^n x^n}{\sqrt{n}}
\]

\subsection*{Datos}
\[
a_n = \frac{(-1)^{n}4^{n}x^{n}}{\sqrt{n}}
\]

\subsection*{Criterio de la razón}
\[
L=\lim_{n\to\infty}\left|\frac{a_{n+1}}{a_n}\right|
\]

\subsection*{Sustitución}
\[
\frac{a_{n+1}}{a_n}
=4|x|\sqrt{\frac{n}{n+1}}
\]

\[
L=4|x|
\]

\subsection*{Condición}
\[
4|x|<1
\quad\Longrightarrow\quad
|x|<\frac14
\]

\[
R=\frac14
\]

\subsection*{Extremos}
\begin{itemize}
\item \(x=\frac14\): \(\displaystyle a_n=\frac{(-1)^n}{\sqrt{n}}\), serie alternante → converge.
\item \(x=-\frac14\): \(\displaystyle a_n=\frac1{\sqrt{n}}\), diverge.
\end{itemize}

\subsection*{Conclusión}
\[
\boxed{R=\frac14,\qquad (-\tfrac14,\tfrac14]}
\]


%=========================================================
\section*{3. Inciso (f)}
\[
\sum_{n=1}^{\infty}\frac{(-1)^n(x-1)^n}{(2n-1)2^{n}}
\]

\subsection*{Datos}
\[
a_n=\frac{(-1)^n(x-1)^n}{(2n-1)2^{n}}
\]

\subsection*{Criterio de la razón}
\[
L=\lim_{n\to\infty}\left|\frac{a_{n+1}}{a_n}\right|
\]

\subsection*{Sustitución}
\[
\left|\frac{a_{n+1}}{a_n}\right|
=\frac{|x-1|}{2}\cdot \frac{2n-1}{2n+1}
\]

\[
L=\frac{|x-1|}{2}
\]

\subsection*{Condición}
\[
\frac{|x-1|}{2}<1
\quad\Longrightarrow\quad
|x-1|<2
\]

\[
R=2
\]

\subsection*{Extremos}
\begin{itemize}
\item \(x=-1\): \(\displaystyle a_n=\frac1{2n-1}\), diverge.
\item \(x=3\): \(\displaystyle a_n=\frac{(-1)^n}{2n-1}\), serie alternante → converge.
\end{itemize}

\subsection*{Conclusión}
\[
\boxed{R=2,\qquad (-1,3]}
\]


%=========================================================
\section*{4. Inciso (g)}
\[
\sum_{n=2}^{\infty}\frac{(x+2)^{n}}{2^{n}\ln(n)}
\]

\subsection*{Datos}
\[
a_n = \frac{(x+2)^{n}}{2^{n}\ln(n)}
\]

\subsection*{Criterio de la razón}
\[
L=\lim_{n\to\infty}\left|\frac{a_{n+1}}{a_n}\right|
\]

\subsection*{Sustitución}
\[
\left|\frac{a_{n+1}}{a_n}\right|
=\frac{|x+2|}{2}\cdot\frac{\ln(n)}{\ln(n+1)}
\]

\[
L=\frac{|x+2|}{2}
\]

\subsection*{Condición}
\[
\frac{|x+2|}{2}<1
\quad\Longrightarrow\quad
|x+2|<2
\]

\[
R=2
\]

\subsection*{Extremos}
\begin{itemize}
\item \(x=-4\): \(\displaystyle a_n=\frac{(-1)^n}{\ln(n)}\), serie alternante → converge.
\item \(x=0\): \(\displaystyle a_n=\frac1{\ln(n)}\), diverge.
\end{itemize}

\subsection*{Conclusión}
\[
\boxed{R=2,\qquad [-4,0)}
\]
\newpage
\section*{Problema 2: Representación como series de potencias}

Ahora se busca representar algunas funciones como series de potencias (alrededor de \(x=0\)) e indicar el radio de convergencia.

%---------------------------
\subsection*{Inciso 2(c)}
\[
f(x)=\frac{x^{2}}{x^{4}+16}
\]

\subsubsection*{Reescritura algebraica}
\[
x^{4}+16 = 16\left(1+\frac{x^{4}}{16}\right)
\]

\[
f(x)
=\frac{x^{2}}{16}\cdot\frac{1}{1+\frac{x^{4}}{16}}
=\frac{x^{2}}{16}\cdot\frac{1}{1- \left(-\frac{x^{4}}{16}\right)}
\]

\subsubsection*{Uso de la serie geométrica}
Para \(|r|<1\),
\[
\frac{1}{1-r}=\sum_{n=0}^{\infty}r^{n}.
\]

Aquí \(r=-\dfrac{x^{4}}{16}\), por lo que
\[
\frac{1}{1-\left(-\frac{x^{4}}{16}\right)}
=\sum_{n=0}^{\infty}\left(-\frac{x^{4}}{16}\right)^{n}.
\]

Entonces
\[
f(x)=\frac{x^{2}}{16}\sum_{n=0}^{\infty}\left(-\frac{x^{4}}{16}\right)^{n}
=\sum_{n=0}^{\infty}(-1)^{n}\frac{x^{4n+2}}{16^{n+1}}.
\]

\subsubsection*{Radio de convergencia}
La condición es
\[
\left|-\frac{x^{4}}{16}\right|<1
\quad\Longrightarrow\quad
|x|^{4}<16
\quad\Longrightarrow\quad
|x|<2.
\]

\[
\boxed{f(x)=\displaystyle\sum_{n=0}^{\infty}(-1)^{n}\frac{x^{4n+2}}{16^{n+1}},\qquad R=2}
\]

%---------------------------
\subsection*{Inciso 2(d)}
\[
f(x)=\frac{x}{2x^{2}+1}
\]

\subsubsection*{Reescritura algebraica}
\[
2x^{2}+1=1+2x^{2}=1-\left(-2x^{2}\right)
\]
y
\[
f(x)=x\cdot\frac{1}{1+2x^{2}}
=x\cdot\frac{1}{1-\left(-2x^{2}\right)}.
\]

\subsubsection*{Uso de la serie geométrica}
Con \(r=-2x^{2}\),
\[
\frac{1}{1-\left(-2x^{2}\right)}
=\sum_{n=0}^{\infty}(-2x^{2})^{n}.
\]

Entonces
\[
f(x)=x\sum_{n=0}^{\infty}(-2x^{2})^{n}
=\sum_{n=0}^{\infty}(-1)^{n}2^{n}x^{2n+1}.
\]

\subsubsection*{Radio de convergencia}
La condición es
\[
|-2x^{2}|<1
\quad\Longrightarrow\quad
2|x|^{2}<1
\quad\Longrightarrow\quad
|x|<\frac{1}{\sqrt{2}}.
\]

\[
\boxed{f(x)=\displaystyle\sum_{n=0}^{\infty}(-1)^{n}2^{n}x^{2n+1},\qquad R=\frac{1}{\sqrt{2}}}
\]

%---------------------------
\subsection*{Inciso 2(e)}
\[
f(x)=\frac{x-1}{x+2}
\]

\subsubsection*{Reescritura algebraica}
\[
\frac{x-1}{x+2}
=\frac{(x+2)-3}{x+2}
=1-\frac{3}{x+2}.
\]

Además
\[
\frac{1}{x+2}
=\frac{1}{2}\cdot\frac{1}{1+\frac{x}{2}}
=\frac{1}{2}\cdot\frac{1}{1-\left(-\frac{x}{2}\right)}.
\]

\subsubsection*{Uso de la serie geométrica}
Con \(r=-\dfrac{x}{2}\),
\[
\frac{1}{1-\left(-\frac{x}{2}\right)}
=\sum_{n=0}^{\infty}\left(-\frac{x}{2}\right)^{n}.
\]

Entonces
\[
\frac{1}{x+2}
=\frac12\sum_{n=0}^{\infty}\left(-\frac{x}{2}\right)^{n}
=\sum_{n=0}^{\infty}\frac{(-1)^{n}}{2^{n+1}}x^{n}.
\]

Por tanto
\[
f(x)=1-3\sum_{n=0}^{\infty}\frac{(-1)^{n}}{2^{n+1}}x^{n}
=-\frac12-3\sum_{n=1}^{\infty}\frac{(-1)^{n}}{2^{n+1}}x^{n}.
\]

\subsubsection*{Radio de convergencia}
La condición es
\[
\left|\frac{x}{2}\right|<1
\quad\Longrightarrow\quad
|x|<2.
\]

\[
\boxed{f(x)=-\frac12-3\displaystyle\sum_{n=1}^{\infty}\frac{(-1)^{n}}{2^{n+1}}x^{n},\qquad R=2}
\]

%---------------------------
\subsection*{Inciso 2(f)}
\[
f(x)=\frac{2x-4}{x^{2}-4x+3}
\]

\subsubsection*{Fracciones parciales}
Se factoriza el denominador:
\[
x^{2}-4x+3=(x-1)(x-3).
\]

Buscamos \(A,B\) tales que
\[
\frac{2x-4}{(x-1)(x-3)}=\frac{A}{x-1}+\frac{B}{x-3}.
\]

Entonces
\[
2x-4=A(x-3)+B(x-1).
\]

Tomando \(x=1\):
\[
2-4=-2= A(1-3)=-2A \Rightarrow A=1.
\]

Tomando \(x=3\):
\[
6-4=2=B(3-1)=2B \Rightarrow B=1.
\]

Luego
\[
f(x)=\frac{1}{x-1}+\frac{1}{x-3}.
\]

\subsubsection*{Escritura tipo geométrica}
Primero
\[
\frac{1}{x-1}=-\frac{1}{1-x}.
\]

Usando
\(\displaystyle \frac{1}{1-x}=\sum_{n=0}^{\infty}x^{n}\) para \(|x|<1\),
\[
\frac{1}{x-1}=-\sum_{n=0}^{\infty}x^{n}.
\]

Para el otro término:
\[
\frac{1}{x-3}=-\frac{1}{3-x}
=-\frac{1}{3}\cdot\frac{1}{1-\frac{x}{3}}
=-\frac{1}{3}\sum_{n=0}^{\infty}\left(\frac{x}{3}\right)^{n}
=-\sum_{n=0}^{\infty}\frac{x^{n}}{3^{n+1}}.
\]

\subsubsection*{Serie resultante}
Sumando ambas series:
\[
f(x)=-\sum_{n=0}^{\infty}x^{n}
-\sum_{n=0}^{\infty}\frac{x^{n}}{3^{n+1}}
=\sum_{n=0}^{\infty}\left(-1-\frac{1}{3^{n+1}}\right)x^{n}.
\]

\subsubsection*{Radio de convergencia}
Las dos series geométricas usadas requieren:
\[
|x|<1 \quad\text{y}\quad \left|\frac{x}{3}\right|<1.
\]
La condición más restrictiva es \(|x|<1\), así que
\[
R=1.
\]

\[
\boxed{f(x)=\displaystyle\sum_{n=0}^{\infty}\left(-1-\frac{1}{3^{n+1}}\right)x^{n},\qquad R=1}
\]
\newpage
%=========================================================
\section*{Ejercicio 3}

\subsection*{a) Representación en serie para \(\displaystyle f(x)=\frac{1}{(1+x)^{2}}\)}

\subsubsection*{Paso 1: Usamos la serie conocida}
Sabemos que
\[
\frac{1}{1+x}=\sum_{n=0}^{\infty}(-1)^n x^{\,n},
\qquad |x|<1.
\]

\subsubsection*{Paso 2: Derivamos ambos lados}
\[
\frac{d}{dx}\left(\frac{1}{1+x}\right)
= -\frac{1}{(1+x)^2},
\]
y
\[
\frac{d}{dx}\left(\sum_{n=0}^{\infty}(-1)^n x^{n}\right)
= \sum_{n=1}^{\infty}(-1)^n n x^{\,n-1}.
\]

\subsubsection*{Paso 3: Ajuste del signo}
\[
\frac{1}{(1+x)^2}
= \sum_{n=1}^{\infty}(-1)^{n+1} n x^{\,n-1}.
\]

\subsubsection*{Paso 4: Reindexación (serie que inicia en \(n=0\))}
Sea \(k=n-1\), entonces \(n=k+1\):
\[
\frac{1}{(1+x)^2}
=\sum_{k=0}^{\infty}(-1)^{k} (k+1) x^{\,k}.
\]

\subsubsection*{Conclusión}
\[
\boxed{\displaystyle
\frac{1}{(1+x)^2}
=\sum_{n=0}^{\infty}(-1)^{n}(n+1)x^{n}}, 
\qquad R=1.
\]


%=========================================================
\subsection*{b) Serie de potencias para \(\displaystyle g(x)=\frac{1}{(1+x)^{3}}\)}

\subsubsection*{Paso 1: Partimos del inciso anterior}
\[
\frac{1}{(1+x)^2}
=\sum_{n=0}^{\infty}(-1)^{n}(n+1)x^{n}.
\]

\subsubsection*{Paso 2: Derivamos nuevamente}
\[
\frac{d}{dx}\left(\frac{1}{(1+x)^2}\right)
=-\frac{2}{(1+x)^3},
\]
y
\[
\frac{d}{dx}\left(\sum_{n=0}^{\infty}(-1)^{n}(n+1)x^{n}\right)
=\sum_{n=1}^{\infty}(-1)^{n}(n+1) n\, x^{\,n-1}.
\]

\subsubsection*{Paso 3: Aislamos \(\displaystyle \frac{1}{(1+x)^3}\)}
\[
\frac{1}{(1+x)^3}
= -\frac12 \sum_{n=1}^{\infty}(-1)^{n}(n+1)n\, x^{\,n-1}.
\]

\subsubsection*{Paso 4: Reindexación para iniciar en \(n=0\)}
Sea \(k=n-1\), entonces
\[
\frac{1}{(1+x)^3}
= -\frac12 \sum_{k=0}^{\infty}(-1)^{k+1}(k+2)(k+1)\, x^{k}.
\]

Simplificando el signo:
\[
\frac{1}{(1+x)^3}
=\frac12 \sum_{k=0}^{\infty}(-1)^{k}(k+1)(k+2)\,x^{k}.
\]

\subsubsection*{Conclusión}
\[
\boxed{
\displaystyle
\frac{1}{(1+x)^3}
= \frac12 \sum_{n=0}^{\infty}(-1)^{n}(n+1)(n+2)x^{n}}, 
\qquad R=1.
\]
\newpage
%=========================================================
\section*{Ejercicio 4}

\subsection*{a) \(\displaystyle f(x)=\ln(5-x)\)}

\subsubsection*{Paso 1: Factorización}
\[
\ln(5-x)=\ln\left(5\left(1-\frac{x}{5}\right)\right)
=\ln(5)+\ln\left(1-\frac{x}{5}\right).
\]

\subsubsection*{Paso 2: Uso de la serie de \(\ln(1-u)\)}
\[
\ln(1-u)= -\sum_{n=1}^{\infty}\frac{u^{n}}{n},
\qquad |u|<1.
\]

Aquí \(u=\frac{x}{5}\):

\[
\ln(5-x)
=\ln(5)-\sum_{n=1}^{\infty}\frac{1}{n}\left(\frac{x}{5}\right)^{n}.
\]

\subsubsection*{Conclusión}
\[
\boxed{
\ln(5-x)
=\ln(5)-\sum_{n=1}^{\infty}\frac{x^{n}}{n\,5^{n}}
},
\qquad R=5.
\]


%=========================================================
\subsection*{b) \(\displaystyle f(x)=x^{2}\arctan(x^{3})\)}

\subsubsection*{Paso 1: Serie conocida}
\[
\arctan(u)=\sum_{n=0}^{\infty}\frac{(-1)^{n}}{2n+1}u^{2n+1},
\qquad |u|<1.
\]

Sustituimos \(u=x^{3}\):
\[
\arctan(x^{3})
=\sum_{n=0}^{\infty}\frac{(-1)^{n}}{2n+1}x^{3(2n+1)}
=\sum_{n=0}^{\infty}\frac{(-1)^{n}}{2n+1}x^{6n+3}.
\]

\subsubsection*{Paso 2: Multiplicamos por \(x^{2}\)}
\[
x^{2}\arctan(x^{3})
=\sum_{n=0}^{\infty}\frac{(-1)^{n}}{2n+1}x^{6n+5}.
\]

\subsubsection*{Conclusión}
\[
\boxed{
x^{2}\arctan(x^{3})
=\sum_{n=0}^{\infty}\frac{(-1)^{n}}{2n+1}x^{6n+5}
},
\qquad R=1.
\]


%=========================================================
\subsection*{c) \(\displaystyle f(x)=\frac{x}{(1+4x)^{2}}\)}

\subsubsection*{Paso 1: Serie base}
Usamos
\[
\frac{1}{1-u}=\sum_{n=0}^{\infty}u^{n},\qquad |u|<1.
\]

\subsubsection*{Paso 2: Obtener \(\frac{1}{(1+4x)^{2}}\)}
\[
\frac{1}{1+4x}
=\frac{1}{1-(-4x)}
=\sum_{n=0}^{\infty}(-4x)^{n}.
\]

Derivando:
\[
\frac{d}{dx}\left(\frac{1}{1+4x}\right)
= -\frac{4}{(1+4x)^{2}},
\]

\[
\frac{d}{dx}\left(\sum_{n=0}^{\infty}(-4x)^{n}\right)
=\sum_{n=1}^{\infty}n(-4)x^{n-1}.
\]

Entonces
\[
\frac{1}{(1+4x)^{2}}
=\frac14 \sum_{n=1}^{\infty}n(-4)x^{n-1}
=\sum_{n=1}^{\infty}n(-1)x^{n-1}.
\]

Reindexando (\(k=n-1\)):
\[
\frac{1}{(1+4x)^{2}}
=\sum_{k=0}^{\infty}(-1)^{k+1}(k+1)x^{k}.
\]

\subsubsection*{Paso 3: Multiplicación por \(x\)}
\[
f(x)=x\cdot\frac{1}{(1+4x)^{2}}
=\sum_{k=0}^{\infty}(-1)^{k+1}(k+1)x^{k+1}.
\]

\subsubsection*{Conclusión}
\[
\boxed{
\frac{x}{(1+4x)^{2}}
=\sum_{n=1}^{\infty}(-1)^{n}(n)x^{n}
},
\qquad R=\frac14.
\]
\newpage
%=========================================================
\section*{Ejercicio 5}

\subsection*{a) \(\displaystyle \int_{0}^{x}\frac{t}{1-t^{8}}\,dt\)}

\subsubsection*{Serie base}
\[
\frac{1}{1-t^{8}} = \sum_{n=0}^{\infty} t^{8n}, \qquad |t|<1.
\]

\subsubsection*{Multiplicación por \(t\)}
\[
\frac{t}{1-t^{8}} = \sum_{n=0}^{\infty} t^{8n+1}.
\]

\subsubsection*{Integración término a término}
\[
\int_{0}^{x}\frac{t}{1-t^{8}}\,dt
= \sum_{n=0}^{\infty}\int_{0}^{x} t^{8n+1}\,dt
= \sum_{n=0}^{\infty} \frac{x^{8n+2}}{8n+2}.
\]

\subsubsection*{Conclusión}
\[
\boxed{
\displaystyle \int_{0}^{x}\frac{t}{1-t^{8}}\,dt
= \sum_{n=0}^{\infty} \frac{x^{8n+2}}{8n+2}
},
\qquad R=1.
\]


%=========================================================
\subsection*{b) \(\displaystyle \int_{0}^{x} t^{2}\ln(1+t)\,dt\)}

\subsubsection*{Serie para \(\ln(1+t)\)}
\[
\ln(1+t)=\sum_{n=1}^{\infty}\frac{(-1)^{n-1}}{n} t^{n}.
\]

\subsubsection*{Multiplicación por \(t^{2}\)}
\[
t^{2}\ln(1+t)
=\sum_{n=1}^{\infty}\frac{(-1)^{n-1}}{n} t^{n+2}.
\]

\subsubsection*{Integración término a término}
\[
\int_{0}^{x} t^{2}\ln(1+t)\,dt
= \sum_{n=1}^{\infty}\frac{(-1)^{n-1}}{n}\frac{x^{n+3}}{n+3}.
\]

\subsubsection*{Conclusión}
\[
\boxed{
\displaystyle \int_{0}^{x} t^{2}\ln(1+t)\,dt
= \sum_{n=1}^{\infty}\frac{(-1)^{n-1}}{n(n+3)} x^{\,n+3}
},
\qquad R=1.
\]


%=========================================================
\subsection*{c) \(\displaystyle \int_{0}^{x} \frac{\arctan(t)}{t}\,dt\)}

\subsubsection*{Serie para \(\arctan(t)\)}
\[
\arctan(t)=\sum_{n=0}^{\infty}\frac{(-1)^{n}}{2n+1} t^{2n+1}.
\]

\subsubsection*{División entre \(t\)}
\[
\frac{\arctan(t)}{t}
= \sum_{n=0}^{\infty}\frac{(-1)^{n}}{2n+1} t^{2n}.
\]

\subsubsection*{Integración}
\[
\int_{0}^{x} \frac{\arctan(t)}{t}\,dt
= \sum_{n=0}^{\infty}\frac{(-1)^{n}}{2n+1}
\frac{x^{2n+1}}{2n+1}.
\]

\subsubsection*{Conclusión}
\[
\boxed{
\displaystyle \int_{0}^{x} \frac{\arctan(t)}{t}\,dt
= \sum_{n=0}^{\infty}\frac{(-1)^{n}}{(2n+1)^{2}} x^{2n+1}
},
\qquad R=1.
\]
\newpage
%=========================================================
\section*{Ejercicio 6}

Dado que
\[
f^{(n)}(0) = (n+1)!,
\]
la serie de Maclaurin es:
\[
f(x)=\sum_{n=0}^{\infty}\frac{f^{(n)}(0)}{n!} x^{n}
=\sum_{n=0}^{\infty}(n+1)x^{n}.
\]

\subsubsection*{Conclusión}
\[
\boxed{
f(x)=\sum_{n=0}^{\infty}(n+1)x^{n}
},
\qquad R=1.
\]
\newpage
%=========================================================
\section*{Ejercicio 7}

Dado que
\[
f^{(n)}(4)=\frac{(-1)^{n}n!}{3^{n}(n+1)},
\]
la serie de Taylor en \(x=4\) es:

\[
f(x)=\sum_{n=0}^{\infty}
\frac{f^{(n)}(4)}{n!}(x-4)^{n}
=\sum_{n=0}^{\infty}
\frac{(-1)^{n}}{3^{n}(n+1)}(x-4)^{n}.
\]

\subsubsection*{Conclusión}
\[
\boxed{
f(x)=\sum_{n=0}^{\infty}\frac{(-1)^{n}}{3^{n}(n+1)}(x-4)^{n}
},
\qquad R=3.
\]
\newpage
%=========================================================
\section*{Ejercicio 8}

\subsection*{a) \(\displaystyle f(x)=(1-x)^{-2}\), centro \(a=0\)}

Usamos
\[
\frac{1}{(1-x)^{2}} = \sum_{n=0}^{\infty}(n+1)x^{n}.
\]

\[
\boxed{
(1-x)^{-2} = \sum_{n=0}^{\infty}(n+1)x^{n}
},
\qquad R=1.
\]


%-------------------------------
\subsection*{b) \(\displaystyle f(x)=2^{x}\), centro \(a=0\)}
\[
2^{x}=e^{x\ln 2}
=\sum_{n=0}^{\infty}\frac{(\ln 2)^{n}}{n!}x^{n}.
\]

\[
\boxed{
2^{x}=\sum_{n=0}^{\infty}\frac{(\ln 2)^{n}}{n!}x^{n}
},
\qquad R=\infty.
\]


%-------------------------------
\subsection*{c) \(\displaystyle f(x)=\ln(x)\), centro \(a=2\)}

Reescribimos:
\[
\ln(x)=\ln(2)+\ln\left(1+\frac{x-2}{2}\right).
\]

Usamos
\[
\ln(1+u)=\sum_{n=1}^{\infty}\frac{(-1)^{n-1}}{n}u^{n}.
\]

\[
\boxed{
\ln(x)=\ln(2)+\sum_{n=1}^{\infty}
\frac{(-1)^{n-1}}{n2^{\,n}}(x-2)^{n}
},
\qquad R=2.
\]


%-------------------------------
\subsection*{d) \(\displaystyle f(x)=e^{2x}\), centro \(a=3\)}

\[
e^{2x} = e^{6}e^{2(x-3)}
= e^{6}\sum_{n=0}^{\infty}\frac{2^{n}}{n!}(x-3)^{n}.
\]

\[
\boxed{
e^{2x}=e^{6}\sum_{n=0}^{\infty}\frac{2^{n}}{n!}(x-3)^{n}
},
\qquad R=\infty.
\]


%-------------------------------
\subsection*{e) \(\displaystyle f(x)=\sin(x)\), centro \(a=\pi\)}

Usamos la serie:
\[
\sin(u)=\sum_{n=0}^{\infty}\frac{(-1)^{n}}{(2n+1)!}u^{2n+1}.
\]

Con \(u=x-\pi\):
\[
\boxed{
\sin(x)=\sum_{n=0}^{\infty}\frac{(-1)^{n}}{(2n+1)!}(x-\pi)^{2n+1}
},
\qquad R=\infty.
\]

\newpage
%=========================================================
\section*{Ejercicio 9}

\subsection*{a) \(\displaystyle f(x)=\arctan(x^{2})\)}

\subsubsection*{Serie base}
\[
\arctan(u)=\sum_{n=0}^{\infty}\frac{(-1)^{n}}{2n+1}u^{2n+1},
\qquad |u|<1.
\]

Sustituimos \(u=x^{2}\):
\[
\arctan(x^{2})
=\sum_{n=0}^{\infty}\frac{(-1)^{n}}{2n+1}x^{4n+2}.
\]

\subsubsection*{Radio de convergencia}
\(|x^{2}|<1 \Rightarrow |x|<1\).

\[
\boxed{
\arctan(x^{2})
= \sum_{n=0}^{\infty}\frac{(-1)^{n}}{2n+1}x^{4n+2}}, 
\qquad R=1.
\]


%=========================================================
\subsection*{b) \(\displaystyle f(x)=x\cos\left(\frac{x^{2}}{2}\right)\)}

\subsubsection*{Serie base}
\[
\cos(u)=\sum_{n=0}^{\infty}\frac{(-1)^{n}}{(2n)!}u^{2n}.
\]

Sustituimos \(u=\frac{x^{2}}{2}\):
\[
\cos\left(\frac{x^{2}}{2}\right)
=\sum_{n=0}^{\infty}\frac{(-1)^{n}}{(2n)!}
\left(\frac{x^{2}}{2}\right)^{2n}
=\sum_{n=0}^{\infty}\frac{(-1)^{n}}{(2n)!}
\frac{x^{4n}}{2^{2n}}.
\]

Multiplicamos por \(x\):
\[
x\cos\left(\frac{x^{2}}{2}\right)
=\sum_{n=0}^{\infty}\frac{(-1)^{n}}{(2n)!}
\frac{x^{4n+1}}{2^{2n}}.
\]

\subsubsection*{Radio de convergencia}
\(\cos\) tiene \(R=\infty\).

\[
\boxed{
x\cos\left(\frac{x^{2}}{2}\right)
=\sum_{n=0}^{\infty}\frac{(-1)^{n}}{(2n)!\,2^{2n}}
x^{4n+1}},\qquad R=\infty.
\]


%=========================================================
\subsection*{c) \(\displaystyle f(x)=\frac{x}{\sqrt{4+x^{2}}}\)}

\subsubsection*{Reescritura}
\[
\sqrt{4+x^{2}} = 2\sqrt{1+\frac{x^{2}}{4}}.
\]

Entonces
\[
f(x)=\frac{x}{2}\left(1+\frac{x^{2}}{4}\right)^{-1/2}.
\]

\subsubsection*{Serie binomial generalizada}
\[
(1+u)^{-1/2}=\sum_{n=0}^{\infty}\binom{-1/2}{n}u^{n},
\qquad |u|<1.
\]

Aquí \(u=\frac{x^{2}}{4}\):

\[
f(x)=\frac{x}{2}\sum_{n=0}^{\infty}\binom{-1/2}{n}
\left(\frac{x^{2}}{4}\right)^{n}.
\]

\[
f(x)=\sum_{n=0}^{\infty}\binom{-1/2}{n}
\frac{x^{2n+1}}{2^{2n+1}}.
\]

\subsubsection*{Radio de convergencia}
\[
\left|\frac{x^{2}}{4}\right|<1
\quad\Longrightarrow\quad |x|<2.
\]

\[
\boxed{
\frac{x}{\sqrt{4+x^{2}}}
=\sum_{n=0}^{\infty}\binom{-1/2}{n}
\frac{x^{2n+1}}{2^{2n+1}}}, 
\qquad R=2.
\]
\newpage
%=========================================================
\section*{Ejercicio 10}

\subsection*{a) \(\displaystyle \sum_{n=0}^{\infty}(-1)^{n}\frac{x^{4n}}{n!}\)}

\[
e^{-x^{4}}=\sum_{n=0}^{\infty}\frac{(-1)^{n}x^{4n}}{n!}.
\]

\[
\boxed{
\sum_{n=0}^{\infty}(-1)^{n}\frac{x^{4n}}{n!}
= e^{-x^{4}}
}.
\]


%=========================================================
\subsection*{b) \(\displaystyle \sum_{n=0}^{\infty}\frac{(-1)^{n}\pi^{2n}}{6^{2n}(2n)!}\)}

Observamos
\[
\cos(u)=\sum_{n=0}^{\infty}\frac{(-1)^{n}u^{2n}}{(2n)!}.
\]

Tomando \(u=\dfrac{\pi}{6}\):
\[
\boxed{
\sum_{n=0}^{\infty}\frac{(-1)^{n}\pi^{2n}}{6^{2n}(2n)!}
=\cos\left(\frac{\pi}{6}\right)
=\frac{\sqrt{3}}{2}.
}
\]


%=========================================================
\subsection*{c) \(\displaystyle \sum_{n=1}^{\infty}(-1)^{n-1}\frac{3^{n}}{n5^{n}}\)}

\[
(-1)^{n-1}\frac{3^{n}}{5^{n}}=\frac{(-3/5)^{n}}{n}.
\]

Usamos
\[
\sum_{n=1}^{\infty}\frac{u^{n}}{n}=-\ln(1-u).
\]

Con \(u=-\frac{3}{5}\):
\[
\sum_{n=1}^{\infty}\frac{(-3/5)^{n}}{n}
=-\ln\left(1+\frac{3}{5}\right)
= -\ln\left(\frac{8}{5}\right).
\]

\[
\boxed{
\sum_{n=1}^{\infty}(-1)^{n-1}\frac{3^{n}}{n5^{n}}
=\ln\left(\frac{8}{5}\right)
}.
\]


%=========================================================
\subsection*{d) \(\displaystyle \sum_{n=0}^{\infty}\frac{(-1)^{n}\pi^{2n+1}}{4^{2n+1}(2n+1)!}\)}

Sacamos factores constantes:

\[
\frac{\pi}{4}\sum_{n=0}^{\infty}\frac{(-1)^{n}}{(2n+1)!}
\left(\frac{\pi}{4}\right)^{2n}.
\]

Reconocemos la serie de \(\sin(u)\):
\[
\sin(u)=\sum_{n=0}^{\infty}\frac{(-1)^{n}u^{2n+1}}{(2n+1)!}.
\]

Tomamos \(u=\frac{\pi}{4}\):

\[
\boxed{
\sum_{n=0}^{\infty}\frac{(-1)^{n}\pi^{2n+1}}{4^{2n+1}(2n+1)!}
=\sin\left(\frac{\pi}{4}\right)
=\frac{\sqrt{2}}{2}.
}
\]
\newpage
%=========================================================
\section*{Ejercicio adicional}

Considere la serie de potencias
\[
\sum_{n=0}^{\infty} \frac{n+1}{(n+2)(3^n+1)}(2x-1)^n.
\]

\subsection*{1. Identificación de la forma general}

Se puede escribir como
\[
\sum_{n=0}^{\infty} a_n(2x-1)^n,
\qquad
a_n = \frac{n+1}{(n+2)(3^n+1)},
\]
que es una serie de potencias centrada en \(x=\frac12\).

\subsection*{2. Cálculo del radio de convergencia}

Aplicamos el criterio de la razón al término general completo
\[
u_n = a_n(2x-1)^n.
\]

\subsubsection*{Cociente}
\[
\left|\frac{u_{n+1}}{u_n}\right|
= |2x-1|
\left|\frac{a_{n+1}}{a_n}\right|,
\]
donde
\[
a_n = \frac{n+1}{(n+2)(3^n+1)},
\qquad
a_{n+1} = \frac{n+2}{(n+3)(3^{n+1}+1)}.
\]

Entonces
\[
\left|\frac{a_{n+1}}{a_n}\right|
= \frac{(n+2)^2(3^n+1)}{(n+3)(n+1)(3^{n+1}+1)}.
\]

\subsubsection*{Límite en \(n\to\infty\)}

Para \(n\) grande:
\[
3^n+1 \sim 3^n,
\qquad
3^{n+1}+1 \sim 3^{n+1}=3\cdot 3^n,
\]
y
\[
\frac{(n+2)^2}{(n+3)(n+1)} \longrightarrow 1.
\]

Por tanto
\[
\lim_{n\to\infty}\left|\frac{a_{n+1}}{a_n}\right|
= \frac{1}{3}.
\]

Así,
\[
\lim_{n\to\infty}\left|\frac{u_{n+1}}{u_n}\right|
= |2x-1|\cdot\frac{1}{3}
= \frac{|2x-1|}{3}.
\]

La condición de convergencia por la razón es
\[
\frac{|2x-1|}{3}<1
\quad\Longrightarrow\quad
|2x-1|<3.
\]

\subsubsection*{Radio en la variable \(x\)}

Dividiendo entre \(2\):
\[
|x-\tfrac12| < \tfrac32.
\]

Por tanto, el radio de convergencia en términos de \(x\) es
\[
\boxed{R=\frac32},
\]
con intervalo abierto
\[
\frac12-\frac32 < x < \frac12+\frac32
\quad\Longrightarrow\quad
-1 < x < 2.
\]

\subsection*{3. Análisis de los extremos}

\subsubsection*{a) Extremo \(x=-1\)}
\[
2x-1 = 2(-1)-1 = -3,
\]
de modo que
\[
u_n = \frac{n+1}{(n+2)(3^n+1)}(-3)^n.
\]

En valor absoluto,
\[
|u_n|
= \frac{n+1}{n+2}\cdot\frac{3^n}{3^n+1}
\longrightarrow 1 \neq 0.
\]

Como el término general no tiende a cero, la serie **diverge** en \(x=-1\).

\subsubsection*{b) Extremo \(x=2\)}
\[
2x-1 = 2(2)-1 = 3,
\]
y
\[
u_n = \frac{n+1}{(n+2)(3^n+1)}3^n.
\]

En valor absoluto,
\[
|u_n|
= \frac{n+1}{n+2}\cdot\frac{3^n}{3^n+1}
\longrightarrow 1 \neq 0.
\]

Nuevamente, el término general no tiende a cero, por lo que la serie **diverge** en \(x=2\).

\subsection*{4. Conclusión}

\[
\boxed{
\text{Centro } x_0=\frac12,\qquad
R=\frac32,\qquad
\text{intervalo de convergencia }(-1,2).
}
\]

Ninguno de los extremos pertenece al intervalo de convergencia.


\end{document}
