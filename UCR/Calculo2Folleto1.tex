\newcommand{\interp}{\textbf{Interpretación:}} 
\newcommand{\formula}[1]{\[ #1 \]}


\documentclass[a4paper]{article}
\newcommand{\calc}{\textbf{Cálculo:}}

% --- Paquetes ---
\usepackage[utf8]{inputenc}
\usepackage[T1]{fontenc}
\usepackage[spanish, es-nodecimaldot]{babel}
\usepackage{amsmath, amssymb, amsthm}
\usepackage{geometry}
\usepackage{calc}
\geometry{margin=2.5cm}
\usepackage{siunitx}
\sisetup{locale=DE, output-decimal-marker={,}}

% --- Títulos bonitos ---
\usepackage{titlesec}
\titleformat{\section}{\large\bfseries}{\thesection.}{0.5em}{}
\titleformat{\subsection}{\normalsize\bfseries}{\thesubsection}{0.5em}{}

% --- Documento ---
\newcommand{\datos}{\textbf{Datos:}}
\begin{document}

\begin{center}
    {\LARGE \textbf{Criterios de convergencia para integrales impropias}} \\[0.3cm]
    \textbf{Curso: Cálculo Integral} \\[0.2cm]
    \textbf{Profesor: Julián Arias Meza}
\end{center}

\section{Introducción}
Una \textbf{integral impropia} es aquella en la que:
\begin{itemize}
    \item El intervalo de integración es infinito.
    \item El integrando presenta discontinuidades en el intervalo.
\end{itemize}

Para estudiar si dichas integrales \emph{convergen} o \emph{divergen}, se aplican varios criterios. A continuación se presentan los más importantes junto con ejemplos.

\section{Criterio de comparación directa}
Si $0 \leq f(x) \leq g(x)$ para todo $x \geq a$, entonces:
\[
\int_a^{\infty} g(x)\, dx \ \text{ converge } \ \Rightarrow \ \int_a^{\infty} f(x)\, dx \ \text{ converge.}
\]
\[
\int_a^{\infty} f(x)\, dx \ \text{ diverge } \ \Rightarrow \ \int_a^{\infty} g(x)\, dx \ \text{ diverge.}
\]

\subsection*{Ejemplo 1}
Determinar si 
\[
I = \int_1^{\infty} \frac{1}{x^2+1}\, dx
\]
converge o diverge.

\textbf{Desarrollo:}
\begin{align*}
\text{Datos: } & f(x) = \frac{1}{x^2+1}, \quad g(x) = \frac{1}{x^2}. \\[0.2cm]
\text{Fórmula: } & 0 \leq f(x) \leq g(x). \\[0.2cm]
\text{Comparación: } & \frac{1}{x^2+1} \leq \frac{1}{x^2}, \quad x \geq 1. \\[0.2cm]
\text{Integral mayor: } & \int_1^{\infty} \frac{1}{x^2}\, dx = \left[ -\frac{1}{x} \right]_1^{\infty} = 1. \\[0.2cm]
\text{Conclusión: } & \text{Como la integral de } g(x) \text{ converge, también la de } f(x).
\end{align*}

\[
I = \int_1^{\infty} \frac{1}{x^2+1}\, dx \ \text{ converge.}
\]


\section{Criterio de convergencia absoluta}
Una integral impropia
\[
\int_a^{\infty} f(x)\, dx
\]
converge absolutamente si
\[
\int_a^{\infty} |f(x)|\, dx
\]
converge. Esto garantiza la convergencia de la integral original.

\subsection*{Ejemplo 2}
Analizar
\[
I = \int_1^{\infty} \frac{\sin(x)}{x^2}\, dx
\]

\textbf{Desarrollo:}
\begin{align*}
\text{Datos: } & f(x) = \frac{\sin(x)}{x^2}. \\[0.2cm]
\text{Fórmula: } & \left| f(x) \right| \leq \frac{1}{x^2}. \\[0.2cm]
\text{Comparación: } & \int_1^{\infty} \frac{1}{x^2}\, dx = 1 \quad \text{converge.} \\[0.2cm]
\text{Interpretación: } & \text{La integral original converge absolutamente.}
\end{align*}

\section{Criterio de Dirichlet}
Si $f(x)$ tiene integral impropia acotada y $g(x)$ es monótona decreciente tendiendo a 0, entonces:
\[
\int_a^{\infty} f(x)g(x)\, dx
\]
converge.

\subsection*{Ejemplo 3}
Analizar
\[
I = \int_1^{\infty} \frac{\sin(x)}{x}\, dx
\]

\textbf{Desarrollo:}
\begin{align*}
\text{Datos: } & f(x) = \sin(x), \quad g(x) = \frac{1}{x}. \\[0.2cm]
\text{Propiedades: } & f(x) \ \text{tiene integral acotada.} \\[0.2cm]
& g(x) \ \text{es decreciente y } \lim_{x\to\infty} g(x) = 0. \\[0.2cm]
\text{Conclusión: } & \text{Por el criterio de Dirichlet, la integral converge.}
\end{align*}

\section{Criterio de Abel}
Si $f(x)$ tiene integral impropia convergente y $g(x)$ es monótona y acotada, entonces:
\[
\int_a^{\infty} f(x)g(x)\, dx
\]
converge.

\subsection*{Ejemplo 4}
Analizar
\[
I = \int_0^{\infty} e^{-x} \cos(x)\, dx
\]

\textbf{Desarrollo:}
\begin{align*}
\text{Datos: } & f(x) = e^{-x}, \quad g(x) = \cos(x). \\[0.2cm]
\text{Propiedades: } & \int_0^{\infty} e^{-x}\, dx = 1 \quad \text{converge.} \\[0.2cm]
& g(x) = \cos(x) \quad \text{es acotada.} \\[0.2cm]
\text{Conclusión: } & \text{Por el criterio de Abel, la integral converge.}
\end{align*}

\section{Conclusión general}
Los criterios de comparación, convergencia absoluta, Dirichlet y Abel son herramientas fundamentales para determinar la convergencia de integrales impropias, especialmente cuando el integrando combina oscilaciones y decaimientos.


\section{Aplicación: Simulacro UCR — Soluciones desarrolladas}

\subsection*{1) Polinomio de Taylor de orden 3 en $a=-\pi$ para $f(x)=x\cos x+e^{x}$}
\datos $f(x)=x\cos x+e^{x}$,\; $a=-\pi$.\\
\formula{T_3(x)=\displaystyle \sum_{k=0}^{3}\frac{f^{(k)}(a)}{k!}(x-a)^k.}\\
%\calc 
\[
\begin{aligned}
&f'(x)=\cos x-x\sin x+e^x,\\
&f''(x)=-2\sin x - x\cos x + e^x,\\
&f^{(3)}(x)=-3\cos x + x\sin x + e^x.
\end{aligned}
\]
En $a=-\pi$ (nótese $\cos(-\pi)=-1$, $\sin(-\pi)=0$):
\[
\begin{aligned}
&f(a)=(-\pi)\cos(-\pi)+e^{-\pi}=-\pi+e^{-\pi},\\
&f'(a)=\cos(-\pi)-(-\pi)\sin(-\pi)+e^{-\pi}=-1+e^{-\pi},\\
&f''(a)=-2\sin(-\pi)-(-\pi)\cos(-\pi)+e^{-\pi}=-\pi+e^{-\pi},\\
&f^{(3)}(a)=-3\cos(-\pi)+(-\pi)\sin(-\pi)+e^{-\pi}=3+e^{-\pi}.
\end{aligned}
\]
Entonces
\[
\boxed{\,T_3(x)=(-\pi+e^{-\pi})+(-1+e^{-\pi})(x+\pi)+\tfrac{-\pi+e^{-\pi}}{2}(x+\pi)^2+\tfrac{3+e^{-\pi}}{6}(x+\pi)^3\,}\,.
\]
\interp Polinomio cúbico centrado en $-\pi$ con coeficientes exactos.

\subsection*{2) Límite por desarrollos limitados}
\[
\lim_{x\to 0}\frac{\ln(1+x^{4})-5x\,\arctan(x)+5x^{2}}{\,1-\cos(x^{2})+4x^{4}\,}
\]
\datos Series de Maclaurin hasta el orden necesario.\\
%\formula 
\[
\ln(1+u)=u-\tfrac{u^{2}}{2}+O(u^{3}),\quad
\arctan x=x-\tfrac{x^{3}}{3}+O(x^{5}),\quad
\cos u=1-\tfrac{u^{2}}{2}+O(u^{4}).
\]
Sustitución: Con $u=x^{4}$:
\[
\ln(1+x^{4})=x^{4}-\tfrac{x^{8}}{2}+O(x^{12}).
\]
Además,
\[
-5x\arctan x=-5x\Big(x-\tfrac{x^{3}}{3}+O(x^{5})\Big)=-5x^{2}+\tfrac{5}{3}x^{4}+O(x^{6}).
\]
Sumando el $+5x^{2}$:
\[
\text{Numerador}=\Big(1+\tfrac{5}{3}\Big)x^{4}+O(x^{6})=\tfrac{8}{3}x^{4}+O(x^{6}).
\]
Para el denominador, con $u=x^{2}$:
\[
1-\cos(x^{2})=\tfrac{x^{4}}{2}+O(x^{8}),\quad
\Rightarrow \text{Denominador}=\tfrac{9}{2}x^{4}+O(x^{8}).
\]
\calc 
\[
\lim_{x\to 0}\frac{\tfrac{8}{3}x^{4}+O(x^{6})}{\tfrac{9}{2}x^{4}+O(x^{8})}
=\frac{\tfrac{8}{3}}{\tfrac{9}{2}}=\boxed{\tfrac{16}{27}}.
\]
\interp El cociente de los términos líderes determina el valor del límite.

\subsection*{3) Convergencia (y valor si converge) de integrales impropias}
\paragraph{(a)} \(\displaystyle \int_{1/2}^{1}\frac{1}{x\,[\ln(x)]^{3}}\,dx\).\\[2mm]
\datos Singularidad en $x=1$ por $\ln x\to 0$. Sustitución $u=\ln x$.\\
\formula Con $u=\ln x$, $du=dx/x$, y $x\in[\tfrac12,1)$ da $u\in[\ln(1/2),0)$.\\
\calc 
\[
\int_{\ln(1/2)}^{0} u^{-3}\,du=\Big[-\tfrac{1}{2u^{2}}\Big]_{\ln(1/2)}^{0^-}=-\infty.
\]
\interp La integral \textbf{diverge} (tipo $u^{-3}$ cerca de $0$).

\paragraph{(b)} \(\displaystyle \int_{0}^{\infty}\frac{dx}{(x+2)(x+6)}\).\\[2mm]
\datos Integrando positivo y decreciente como $x^{-2}$. Fracciones parciales.\\
\(\tfrac{1}{(x+2)(x+6)}=\tfrac{1}{4}\Big(\tfrac{1}{x+2}-\tfrac{1}{x+6}\Big).\)

\calc 
\[
\int_{0}^{b}\frac{dx}{(x+2)(x+6)}
=\tfrac14\ln\!\Big(\tfrac{b+2}{b+6}\Big)-\tfrac14\ln\!\Big(\tfrac{2}{6}\Big).
\]
Al pasar $b\to\infty$, queda
\[
\boxed{\int_{0}^{\infty}\frac{dx}{(x+2)(x+6)}=\tfrac{1}{2}\ln 3}.
\]
\interp Converge (similar a $1/x^{2}$) y el valor es finito.

\subsection*{4) Decidir convergencia/divergencia}
\paragraph{(a)} \(\displaystyle \int_{0}^{2}\frac{2+x}{\sqrt{x+x^{5}}}\,dx\).\\[2mm]
\datos Singularidad en $x=0^+$.\\
\formula Para $0<x\le 1$, $\tfrac{2+x}{\sqrt{x+x^{5}}}\le \tfrac{3}{\sqrt{x}}$.\\
\calc Como $\int_{0}^{1}x^{-1/2}\,dx=2$ converge, la integral converge.\\
\interp \(\boxed{\text{Converge}}\).

\paragraph{(b)} \(\displaystyle \int_{1}^{\infty}\frac{x+|\sin(3x)|}{x^{2}+5x}\,dx\).\\[2mm]
\datos Comportamiento en infinito y \textbf{comparación por el límite} con \(\tfrac{1}{x}\).\\

\begin{equation*}
L \;=\; \lim_{x\to\infty}\frac{\dfrac{x+|\sin(3x)|}{x^{2}+5x}}{1/x}
   \;=\; \lim_{x\to\infty}\frac{x+|\sin(3x)|}{x+5}
   \;=\; 1,
\end{equation*}
pues \(|\sin(3x)|\le 1\).\\
\interp Como \(\displaystyle \int_{1}^{\infty}\tfrac{1}{x}\,dx\) \textbf{diverge} y \(L=1\in(0,\infty)\),
por el \emph{criterio de comparación por el límite} la integral dada también \textbf{diverge}.


\subsection*{5) Opcional}
\paragraph{(a) Aproximar \(\cos(1/9)\) con Maclaurin de orden 5 y acotar el residuo}
\datos Serie de Maclaurin: \(\cos z=1-\tfrac{z^{2}}{2!}+\tfrac{z^{4}}{4!}-\tfrac{z^{6}}{6!}+\cdots\).\\

%\formula 
\(T_{5}(z)=1-\frac{z^{2}}{2}+\frac{z^{4}}{24}\)

\[R_{5}(z)=\frac{f^{(6)}(\xi)}{6!}\,z^{6}=-\frac{\cos(\xi)}{6!}\,z^{6}\], con \(\xi\in(0,z)\).\\
\calc 
\[
T_{5}\!\Big(\tfrac{1}{9}\Big)=1-\tfrac{1}{162}+\tfrac{1}{157464}, 
\qquad
|R_{5}|\le \frac{1}{6!}\Big(\tfrac{1}{9}\Big)^{6}
= \frac{1}{720\cdot 9^{6}} \approx 2{,}6\times 10^{-9}.
\]
Como \(\xi\in(0,\tfrac{1}{9})\Rightarrow \cos(\xi)>0\), se tiene \(R_{5}<0\).\\
\interp 
\[
\boxed{\cos\!\Big(\tfrac{1}{9}\Big)\approx 1-\tfrac{1}{162}+\tfrac{1}{157464}},
\qquad 
\boxed{0>R_{5}>-\,2{,}7\times 10^{-9}}.
\]


\paragraph{(b) Convergencia de \(\displaystyle \int_{0}^{\infty}\frac{x\sin(3x)}{x^{2}+5}\,dx\)}
\datos Producto “oscilación \(\times\) decaimiento”. Queremos usar \textbf{Dirichlet}.\\
\formula 
Sea \(f(x)=\sin(3x)\) y \(g(x)=\dfrac{x}{x^{2}+5}\).
Entonces 
\(F(x)=\int_{0}^{x} f(t)\,dt=\dfrac{1-\cos(3x)}{3}\) es \textbf{acotada}; 
además \(g\) es eventualmente monótona decreciente y \(g(x)\to 0\).\\
\calc 
\[
F(x)=\frac{1-\cos(3x)}{3},\qquad
g'(x)=\frac{5-x^{2}}{(x^{2}+5)^{2}}\le 0\ \text{para }x\ge \sqrt{5},\qquad
\lim_{x\to\infty}g(x)=0.
\]
Dividimos la integral como \(\int_{0}^{\infty}=\int_{0}^{\sqrt{5}}+\int_{\sqrt{5}}^{\infty}\); la primera es propia y la segunda cumple las hipótesis de Dirichlet.\\
\interp 
\[
\boxed{\text{La integral converge (condicionalmente) por el criterio de Dirichlet.}}
\]
\newpage
\section*{Soluciones — Ejercicios 3 y 4}

% =========================
% EJERCICIO 3
% =========================
\subsection*{3) Mediante la definición, determine si cada integral impropia es convergente. Si converge, obtenga su valor.}

\paragraph{(a)} \(\displaystyle \int_{0}^{2}\frac{x}{\sqrt{4-x^{2}}}\,dx\).

\datos La impropiedad ocurre en \(x=2^{-}\) porque \(\sqrt{4-x^{2}}\to 0\).\\
\formula Definición de integral impropia con sustitución trigonométrica \(x=2\sin\theta\).\\
 Sea \(x=2\sin\theta\Rightarrow dx=2\cos\theta\,d\theta\) y \(\sqrt{4-x^{2}}=\sqrt{4-4\sin^{2}\theta}=2\cos\theta\).  
Cuando \(x\to 0^{+}\), \(\theta\to 0^{+}\); cuando \(x\to 2^{-}\), \(\theta\to \tfrac{\pi}{2}^{-}\).\\
\calc
\[
\int_{0}^{2}\frac{x}{\sqrt{4-x^{2}}}\,dx
=\lim_{b\to 2^{-}}\int_{0}^{b}\frac{x}{\sqrt{4-x^{2}}}\,dx
=\int_{0}^{\pi/2}\frac{2\sin\theta}{2\cos\theta}\,2\cos\theta\,d\theta
=\int_{0}^{\pi/2}2\sin\theta\,d\theta.
\]
\[
\int_{0}^{\pi/2}2\sin\theta\,d\theta=\Big[-2\cos\theta\Big]_{0}^{\pi/2}=(-2\cdot 0)-(-2\cdot 1)=2.
\]
\interp \(\boxed{\text{Converge y su valor es } 2}\).

\paragraph{(b)} \(\displaystyle \int_{4}^{\infty}\frac{dx}{x(x^{2}+1)}\).

\datos Impropia en \(+\infty\). El integrando es positivo y para \(x\) grande es \(\sim 1/x^{3}\) (sugiere convergencia).\\
\formula Fracciones parciales: \(\displaystyle \frac{1}{x(x^{2}+1)}=\frac{1}{x}-\frac{x}{x^{2}+1}\).\\
\calc
\[
\int\frac{dx}{x(x^{2}+1)}=\int\Big(\frac{1}{x}-\frac{x}{x^{2}+1}\Big)\,dx
=\ln x-\tfrac{1}{2}\ln(x^{2}+1)+C.
\]
Como impropia:
\[
\int_{4}^{\infty}\frac{dx}{x(x^{2}+1)}
=\lim_{b\to\infty}\Big[\ln x-\tfrac{1}{2}\ln(x^{2}+1)\Big]_{4}^{b}.
\]
Para \(b\to\infty\),
\[
\ln b-\tfrac{1}{2}\ln(b^{2}+1)
=\tfrac{1}{2}\ln\!\Big(\frac{b^{2}}{b^{2}+1}\Big)\longrightarrow \tfrac{1}{2}\ln 1=0.
\]
Por tanto,
\[
\int_{4}^{\infty}\frac{dx}{x(x^{2}+1)}
=0-\Big(\ln 4-\tfrac{1}{2}\ln 17\Big)
=\boxed{\tfrac{1}{2}\ln\!\Big(\tfrac{17}{16}\Big)}.
\]
\interp \(\boxed{\text{Converge y vale } \tfrac{1}{2}\ln\!\big(\tfrac{17}{16}\big)}\).

% =========================
% EJERCICIO 4
% =========================
\subsection*{4) Utilizando criterios adecuados, determine convergencia o divergencia}

\paragraph{(a)} \(\displaystyle \int_{0}^{1}\frac{x^{2}+3}{x+x^{2}}\,dx\).

\datos Impropia en \(x=0^{+}\) (denominador \(x(1+x)\)).\\
\formula Comparación directa cerca de \(0\). Para \(0<x\le 1\): \(x^{2}+3\ge 3\) y \(x+x^{2}\le 2x\).\\
\calc
\[
0\le \frac{x^{2}+3}{x+x^{2}}=\frac{x^{2}+3}{x(1+x)}
\;\ge\; \frac{3}{2x}\quad (0<x\le 1).
\]
Como \(\displaystyle \int_{0}^{1}\frac{1}{x}\,dx\) \emph{diverge}, entonces por comparación inferior la integral dada \emph{diverge}.  
(Alternativamente, con fracciones parciales:
\(\frac{x^{2}+3}{x+x^{2}}=1+\frac{3}{x}-\frac{4}{1+x}\) y
\(\displaystyle \int_{\varepsilon}^{1}\big(1+\frac{3}{x}-\frac{4}{1+x}\big)dx\to +\infty\) cuando \(\varepsilon\to 0^{+}\)).\\
\interp \(\boxed{\text{Diverge}}\).

\paragraph{(b)} \(\displaystyle I=\int_{1}^{\infty}\frac{2^{x}+5x^{4}}{3^{x}+2^{x}}\,dx\).

\datos Impropria en \(+\infty\). El denominador es positivo para \(x\ge 1\).
Para \(x\) grande, \(3^{x}\) domina a \(2^{x}\) y a \(x^{4}\).

\textbf{Comparación directa (cota superior):} para \(x\ge 1\),
\[
0\le \frac{2^{x}+5x^{4}}{3^{x}+2^{x}}
\le \frac{2^{x}+5x^{4}}{3^{x}}
=\Big(\frac{2}{3}\Big)^{x}+\frac{5x^{4}}{3^{x}}.
\]
Sea \(k=\ln\!\big(\tfrac{3}{2}\big)>0\) y \(m=\ln 3>0\). Entonces
\[
\Big(\frac{2}{3}\Big)^{x}=e^{-kx},\qquad 
\frac{5x^{4}}{3^{x}}=5\,x^{4}e^{-mx}.
\]
Por tanto, 
\[
0\le \frac{2^{x}+5x^{4}}{3^{x}+2^{x}}\le e^{-kx}+5\,x^{4}e^{-mx},\quad x\ge 1.
\]

 \(k=\ln(3/2)\), \(m=\ln 3\).

\calc 
\[
\int_{1}^{\infty} e^{-kx}\,dx=\frac{e^{-k}}{k}=\frac{\tfrac{2}{3}}{\ln(3/2)}
\approx 1{,}6442.
\]
Para el segundo término usamos que, para \(n\in\mathbb{N}\) y \(a>0\),
\[
\int_{0}^{\infty} x^{n}e^{-ax}\,dx=\frac{n!}{a^{\,n+1}}
\quad(\text{fórmula gamma}).
\]
Así,
\[
\int_{1}^{\infty} 5x^{4}e^{-mx}\,dx
\;\le\; 5\int_{0}^{\infty} x^{4}e^{-mx}\,dx
=5\,\frac{4!}{m^{5}}
= \frac{120}{(\ln 3)^{5}}
\approx 74{,}982.
\]
Por comparación con la suma de estas dos integrales (ambas finitas), \(I\) converge. Además, queda una cota numérica:
\[
0\le I \le \frac{\tfrac{2}{3}}{\ln(3/2)}+\frac{120}{(\ln 3)^{5}}
\approx 1{,}6442+74{,}982 \approx 76{,}626.
\]

\interp \(\boxed{\text{Converge (absolutamente) por comparación directa}}\).

% --- Alternativa breve: comparación por el límite ---
\medskip
\noindent\textit{Alternativa (comparación por el límite).}
Como
\[
\lim_{x\to\infty}\frac{\dfrac{2^{x}+5x^{4}}{3^{x}+2^{x}}}{\left(\tfrac{2}{3}\right)^{x}}
=\lim_{x\to\infty}\frac{2^{x}+5x^{4}}{3^{x}+2^{x}}\cdot\Big(\frac{3}{2}\Big)^{x}
=\lim_{x\to\infty}\Bigg[\frac{(2/3)^{x}}{1+(2/3)^{x}}
+\frac{5x^{4}(3/2)^{x}}{3^{x}\!\left(1+(2/3)^{x}\right)}\Bigg]=1,
\]
(y el segundo término tiende a \(0\) porque \(3^{x}\) domina a cualquier potencia),
y \(\displaystyle \int_{1}^{\infty}\!\!\left(\tfrac{2}{3}\right)^{x}dx\) converge, por el
\emph{criterio de comparación por el límite} también concluye que \(I\) converge.

\end{document}
