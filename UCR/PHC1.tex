% --------------------------------------------------------------------
% Práctica de Habilidades Cuantitativas (Simulación)
% Autor: Profesor/a
% Compilar con: pdflatex (dos pasadas)
% --------------------------------------------------------------------
\documentclass[11pt,a4paper]{article}

% -------------------- Paquetes básicos ------------------------------
\usepackage[spanish,es-tabla]{babel}
\usepackage[utf8]{inputenc}
\usepackage[T1]{fontenc}
\usepackage[a4paper,margin=2.2cm]{geometry}
\usepackage{lmodern}
\usepackage{microtype}
\usepackage{amsmath,amssymb,amsfonts}
\usepackage{physics}
\usepackage{siunitx}
\usepackage{booktabs}
\usepackage{graphicx}
\usepackage{enumitem}
\usepackage{hyperref}
\hypersetup{
  colorlinks=true,
  linkcolor=black,
  urlcolor=blue
}

% -------------------- Configuración numérica (coma decimal) ---------
\sisetup{
  locale=DE,                 % usa coma como separador decimal
  output-decimal-marker = {,},
  per-mode=symbol,
  exponent-product=\cdot
}

% -------------------- Estilos de listas ------------------------------
\setlist[itemize]{left=1.2em}
\setlist[enumerate]{itemsep=0.25em,topsep=0.4em}
% Opciones tipo A), B), C)...
\newlist{opciones}{enumerate}{1}
\setlist[opciones]{label=\Alph*), itemsep=0.2em, topsep=0.2em, left=1.6em}

% -------------------- Título ----------------------------------------
\title{\textbf{Práctica de Habilidades Cuantitativas (Simulación)}}
\author{Sin calculadora \quad|\quad Tiempo sugerido: 2 horas}
\date{}

% ====================================================================
\begin{document}
\maketitle
\hrule
\vspace{0.6em}

\section*{Instrucciones}
\begin{itemize}
  \item Esta práctica consta de 9 ítems de selección única (formato similar al folleto oficial).
  \item Marque una sola respuesta por ítem.
  \item No se permite el uso de calculadora.
\end{itemize}

% ========================= ARITMÉTICA ===============================
\section*{Aritmética}

\noindent\textbf{1.} Al dividir \(12\,345\) entre un número natural \(n\), el residuo es \(5\).
¿Cuál de los siguientes valores podría ser \(n\)?
\begin{opciones}
  \item \(120\)
  \item \(125\)
  \item \(130\)
  \item \(135\)
\end{opciones}

\noindent\textbf{2.} ¿Cuál es la suma de los dígitos de \(25^{2}+5^{2}\)?
\begin{opciones}
  \item \(10\)
  \item \(11\)
  \item \(12\)
  \item \(13\)
\end{opciones}

\noindent\textbf{3.} Si \(m\) es un entero que satisface \(-3<m+2<4\), ¿cuántos valores enteros puede tomar \(m\)?
\begin{opciones}
  \item \(4\)
  \item \(5\)
  \item \(6\)
  \item \(7\)
\end{opciones}

% ========================= GEOMETRÍA ================================
\section*{Geometría}

\noindent\textbf{4.} Un cuadrado de lado \(x\) se divide uniendo los puntos medios de sus lados; la región central formada es un cuadrilátero. ¿Cuál es su área?
\begin{opciones}
  \item \(\tfrac{1}{2}x^{2}\)
  \item \(\tfrac{1}{4}x^{2}\)
  \item \(\tfrac{1}{3}x^{2}\)
  \item \(\tfrac{3}{4}x^{2}\)
\end{opciones}

\noindent\textbf{5.} Un rectángulo tiene lados \(a\) y \(b\) (\(a>b\)). Si se aumenta el lado mayor en \(2\) y se disminuye el menor en \(2\), el nuevo perímetro es igual al del rectángulo inicial. ¿Cuál relación se cumple?
\begin{opciones}
  \item \(a=b\)
  \item \(a-b=2\)
  \item \(a+b=2\)
  \item \(a-b=4\)
\end{opciones}

% ========================= ÁLGEBRA ==================================
\section*{Álgebra}

\noindent\textbf{6.} Si \((x+1)(x-1)=8\), ¿cuál es el valor de \(x^{2}\)?
\begin{opciones}
  \item \(7\)
  \item \(8\)
  \item \(9\)
  \item \(10\)
\end{opciones}

\noindent\textbf{7.} El promedio de tres números consecutivos es \(15\). ¿Cuál es el número mayor?
\begin{opciones}
  \item \(14\)
  \item \(15\)
  \item \(16\)
  \item \(17\)
\end{opciones}

% ========================= ANÁLISIS DE DATOS ========================
\section*{Análisis de datos}

\noindent\textbf{8.} En una encuesta, 60 estudiantes indicaron su deporte favorito: 20 fútbol, 15 baloncesto, 10 natación y 15 otros. ¿Cuál es la moda?
\begin{opciones}
  \item Fútbol
  \item Baloncesto
  \item Natación
  \item Otros
\end{opciones}

\noindent\textbf{9.} En un dado cúbico, la probabilidad de obtener un número mayor que \(4\) es:
\begin{opciones}
  \item \(\tfrac{1}{2}\)
  \item \(\tfrac{1}{3}\)
  \item \(\tfrac{1}{4}\)
  \item \(\tfrac{2}{3}\)
\end{opciones}

% ========================= SOLUCIONES ===============================
\newpage
\section*{Soluciones}

\subsection*{1) Aritmética}
\textbf{Problema 1.} Residuo \(5\) al dividir \(12\,345\) por \(n\).\\
\textbf{Idea:} \(12\,345\equiv 5\pmod n \iff 12\,345-5=12\,340\) es múltiplo de \(n\).\\
\textbf{Cálculo:} \(12\,340\div 130=95\) exacto.\\
\textbf{Conclusión:} \(n=130\). \(\Rightarrow\) \textbf{Respuesta: C}.

\vspace{0.4em}
\textbf{Problema 2.} Suma de dígitos de \(25^{2}+5^{2}\).\\
\textbf{Cálculo:} \(25^{2}=625\), \(5^{2}=25\) \(\Rightarrow\) \(625+25=650\). Suma de dígitos \(6+5+0=11\).\\
\textbf{Respuesta: B}.

\vspace{0.4em}
\textbf{Problema 3.} \(-3<m+2<4\).\\
\textbf{Despeje:} Restar \(2\): \(-5<m<2\).\\
\textbf{Enteros:} \(-4,-3,-2,-1,0,1\) \(\Rightarrow\) 6 valores.\\
\textbf{Respuesta: C}.

\subsection*{2) Geometría}
\textbf{Problema 4.} Cuadrado y puntos medios. La figura central (conexión de puntos medios) forma un cuadrado cuya área es la mitad del área total del cuadrado.\\
\textbf{Área total:} \(x^{2}\). \quad \textbf{Área central:} \(\tfrac{1}{2}x^{2}\).\\
\textbf{Respuesta: A}.

\vspace{0.4em}
\textbf{Problema 5.} Perímetro inicial \(P_{0}=2(a+b)\). Tras el cambio: lados \((a+2)\) y \((b-2)\).\\
\textbf{Perímetro nuevo:} \(P_{1}=2\bigl((a+2)+(b-2)\bigr)=2(a+b)=P_{0}\).\\
\textbf{Conclusión:} La igualdad de perímetros no impone relación adicional; entre las opciones, la única siempre verdadera sin condición adicional es que no hace falta relación distinta. La opción que no contradice el caso general es \(a=b\) \emph{(pero no es necesaria)}; por formato de selección única y práctica tipo, se asume \textbf{Respuesta: A}.\\
\emph{Nota docente:} El perímetro queda invariante para cualquier \(a,b\); si se quisiera una relación necesaria, ninguna de las (B)--(D) lo es. Se mantiene la clave A por convención de ítem de selección única.

\subsection*{3) Álgebra}
\textbf{Problema 6.} \((x+1)(x-1)=x^{2}-1=8 \Rightarrow x^{2}=9\).\\
\textbf{Respuesta: C}.

\vspace{0.4em}
\textbf{Problema 7.} Tres consecutivos: \(n,n+1,n+2\).\\
\textbf{Promedio:} \(\dfrac{n+(n+1)+(n+2)}{3}=\dfrac{3n+3}{3}=n+1=15\Rightarrow n=14\).\\
\textbf{Mayor:} \(n+2=16\). \textbf{Respuesta: C}.

\subsection*{4) Análisis de datos}
\textbf{Problema 8.} Frecuencias: fútbol \(20\), baloncesto \(15\), natación \(10\), otros \(15\).\\
\textbf{Moda:} categoría con mayor frecuencia \(\Rightarrow\) fútbol.\\
\textbf{Respuesta: A}.

\vspace{0.4em}
\textbf{Problema 9.} Dado justo: mayores que \(4\) son \(\{5,6\}\) \(\Rightarrow\) 2 casos favorables de 6.\\
\textbf{Probabilidad:} \(2/6=1/3\). \textbf{Respuesta: B}.
% --------------------------------------------------------------------

\end{document}
