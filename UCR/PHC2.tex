% --------------------------------------------------------------------
% Práctica de Habilidades Cuantitativas (Simulación 2)
% Compilar con: pdflatex (dos pasadas)
% --------------------------------------------------------------------
\documentclass[11pt,a4paper]{article}

% -------------------- Paquetes básicos ------------------------------
\usepackage[spanish,es-tabla]{babel}
\usepackage[utf8]{inputenc}
\usepackage[T1]{fontenc}
\usepackage[a4paper,margin=2.2cm]{geometry}
\usepackage{lmodern}
\usepackage{microtype}
\usepackage{amsmath,amssymb,amsfonts}
\usepackage{siunitx}
\usepackage{graphicx}
\usepackage{booktabs}
\usepackage{enumitem}
\usepackage{hyperref}
\hypersetup{
  colorlinks=true,
  linkcolor=black,
  urlcolor=blue
}

% -------------------- Configuración numérica (coma decimal) ---------
\sisetup{
  locale=DE,                 % coma decimal
  output-decimal-marker = {,},
  per-mode=symbol,
  exponent-product=\cdot
}

% -------------------- Estilos de listas ------------------------------
\setlist[itemize]{left=1.2em}
\setlist[enumerate]{itemsep=0.25em,topsep=0.4em}
\newlist{opciones}{enumerate}{1}
\setlist[opciones]{label=\Alph*), itemsep=0.2em, topsep=0.2em, left=1.6em}

% -------------------- Título ----------------------------------------
\title{\textbf{Práctica de Habilidades Cuantitativas (Simulación 2)}}
\author{Sin calculadora \quad|\quad Tiempo sugerido: 2 horas}
\date{}

% ====================================================================
\begin{document}
\maketitle
\hrule
\vspace{0.6em}

\section*{Instrucciones}
\begin{itemize}
  \item Esta práctica consta de 12 ítems de selección única, distribuidos por áreas: Aritmética, Geometría, Álgebra y Análisis de datos.
  \item Marque una sola respuesta por ítem. No se permite el uso de calculadora.
\end{itemize}

% ========================= ARITMÉTICA ===============================
\section*{Aritmética}

\noindent\textbf{1.} Al dividir \(18\,437\) entre un número natural \(n\), el residuo es \(7\).
¿Cuál de los siguientes valores podría ser \(n\)?
\begin{opciones}
  \item \(1\,820\)
  \item \(1\,843\)
  \item \(1\,850\)
  \item \(2\,000\)
\end{opciones}

\noindent\textbf{2.} ¿Cuál de los siguientes números es un divisor de \(56^{2}+2\cdot 56\cdot 44+44^{2}\)?
\begin{opciones}
  \item \(25\)
  \item \(64\)
  \item \(125\)
  \item \(256\)
\end{opciones}

\noindent\textbf{3.} Sea \(k\) un entero que satisface \(1<2k-3\le 7\). ¿Cuántos valores enteros puede tomar \(k\)?
\begin{opciones}
  \item \(2\)
  \item \(3\)
  \item \(4\)
  \item \(5\)
\end{opciones}
\newpage
% ========================= GEOMETRÍA ================================
\section*{Geometría}

\noindent\textbf{4.} En un cuadrado de lado \(x\), se trazan los puntos medios \(M\) y \(N\) en dos lados \emph{adyacentes}. ¿Cuál es el área del triángulo formado por \(M\), \(N\) y el vértice opuesto a ambos?
\begin{opciones}
  \item \(\dfrac{1}{4}x^{2}\)
  \item \(\dfrac{3}{8}x^{2}\)
  \item \(\dfrac{1}{2}x^{2}\)
  \item \(\dfrac{5}{8}x^{2}\)
\end{opciones}

\noindent\textbf{5.} Sean \(P(1,2)\), \(Q(1,2k)\) y \(R(2k,1)\) con \(k>1\). ¿Cuál relación entre distancias es correcta?
\begin{opciones}
  \item \(PR=PQ\)
  \item \(PR>PQ\)
  \item \(PQ>PR\)
  \item \(PR=2\,PQ\)
\end{opciones}

\noindent\textbf{6.} En un triángulo rectángulo isósceles de catetos \(a\), se construye un cuadrado sobre la \emph{hipotenusa}. ¿Cuál es el área de ese cuadrado?
\begin{opciones}
  \item \(a^{2}\)
  \item \(2a^{2}\)
  \item \(4a^{2}\)
  \item \(\dfrac{a^{2}}{2}\)
\end{opciones}

% ========================= ÁLGEBRA ==================================
\section*{Álgebra}

\noindent\textbf{7.} Para \(y=mx-m^{2}-x\) y \(x=m-1\), ¿cuál es el valor de \(y\)?
\begin{opciones}
  \item \(y=1\)
  \item \(y=-2m+1\)
  \item \(y=m\)
  \item \(y=2m+1\)
\end{opciones}

\noindent\textbf{8.} Si
\[
k(x+3)+k(x+4)+\cdots+k(x+12)=105\,k,\quad k\ne 0,
\]
¿cuál es el valor de \(x\)?
\begin{opciones}
  \item \(x=3\)
  \item \(x=\dfrac{10}{3}\)
  \item \(x=30\)
  \item \(x=\dfrac{3}{k}\)
\end{opciones}

\noindent\textbf{9.} Sea \(z=x(x+2)(x-1)\). Si \(x\in[0,1]\), ¿qué condición cumple \(z\)?
\begin{opciones}
  \item Es estrictamente positivo.
  \item Es positivo en algunos casos y cero en otros.
  \item Es estrictamente negativo.
  \item Es negativo en algunos casos y cero en otros.
\end{opciones}

% ========================= ANÁLISIS DE DATOS ========================
\section*{Análisis de datos}

\noindent\textbf{10.} En una fábrica hay \(25\) cajas que pesan en conjunto \(\SI{80,0}{kg}\).
Si el peso de \emph{cada} caja se reduce en \(\SI{0,2}{kg}\), ¿cuál es el nuevo \emph{peso promedio} por caja?
\begin{opciones}
  \item \(\SI{3,0}{kg}\)
  \item \(\SI{3,2}{kg}\)
  \item \(\SI{3,5}{kg}\)
  \item \(\SI{2,8}{kg}\)
\end{opciones}

\noindent\textbf{11.} En una bolsa hay \(3\) rojas, \(2\) azules y \(1\) verde. Si se extrae una al azar, la probabilidad de \emph{no} sacar roja es:
\begin{opciones}
  \item \(\dfrac{1}{2}\)
  \item \(\dfrac{1}{3}\)
  \item \(\dfrac{2}{3}\)
  \item \(\dfrac{5}{6}\)
\end{opciones}

\noindent\textbf{12.} En una encuesta a \(200\) personas sobre transporte al trabajo se obtuvo:
45\% vehículo particular, 35\% bus, 10\% taxi y 10\% bicicleta
¿Cuál afirmación es \emph{verdadera con certeza}?
\begin{opciones}
  \item Exactamente \(90\) usan vehículo particular.
  \item Más de \(100\) usan un medio \emph{no} particular.
  \item La diferencia bus–particular es de \(20\) personas.
  \item Menos de \(100\) usan medios no particulares.
\end{opciones}

% ========================= SOLUCIONES ===============================
\newpage
\section*{Soluciones}

\subsection*{Aritmética}

\textbf{1.} Residuo \(7\) al dividir \(18\,437\) entre \(n\).
\[
18\,437\equiv 7 \pmod n \iff 18\,437-7=18\,430 \text{ es múltiplo de } n.
\]
Factorización simple: \(18\,430=2\cdot 5\cdot 1\,843\).
Así, \(n\) puede ser \(1\,843\). \\
\textbf{Respuesta: B}.

\vspace{0.4em}
\textbf{2.} Sea \(E=56^{2}+2\cdot 56\cdot 44+44^{2}=(56+44)^{2}=100^{2}=10\,000\).
Divisores de \(10\,000\) incluyen \(25\). \\
\textbf{Respuesta: A}.

\vspace{0.4em}
\textbf{3.} \(1<2k-3\le 7\).
\[
1<2k-3 \Rightarrow 4<2k \Rightarrow k>2 \Rightarrow k\ge 3,\qquad
2k-3\le 7 \Rightarrow 2k\le 10 \Rightarrow k\le 5.
\]
Valores posibles: \(k\in\{3,4,5\}\) \(\Rightarrow\) \(3\) valores. \\
\textbf{Respuesta: B}.

\subsection*{Geometría}

\textbf{4.} Cuadrado de vértices \((0,0),(x,0),(x,x),(0,x)\).
Sean \(M=(\tfrac{x}{2},0)\) y \(N=(0,\tfrac{x}{2})\). Tomando como tercer vértice \((x,x)\),
\[
\vec{v}_1=M-(x,x)=\Big(-\tfrac{x}{2},-x\Big),\quad
\vec{v}_2=N-(x,x)=\Big(-x,-\tfrac{x}{2}\Big).
\]
Área del triángulo:
\[
A=\frac{1}{2}\left|\det\begin{pmatrix}-\tfrac{x}{2} & -x\\[2pt] -x & -\tfrac{x}{2}\end{pmatrix}\right|
=\frac{1}{2}\left|\frac{x^{2}}{4}-x^{2}\right|
=\frac{1}{2}\cdot \frac{3x^{2}}{4}
=\frac{3}{8}x^{2}.
\]
\textbf{Respuesta: B}.

\vspace{0.4em}
\textbf{5.} \(PQ=|2k-2|=2(k-1)\).
\[
PR=\sqrt{(2k-1-1)^{2}+(1-2)^{2}}
=\sqrt{(2k-1)^{2}+1}.
\]
Comparamos cuadrados:
\[
(2k-1)^{2}+1 \;\mathop{\gtrless}\; \bigl(2(k-1)\bigr)^{2}
\iff 4k^{2}-4k+2 \;\mathop{\gtrless}\; 4k^{2}-8k+4
\iff 4k-2 \;\mathop{\gtrless}\; 0.
\]
Para \(k>1\), \(4k-2>0\Rightarrow PR>PQ\). \\
\textbf{Respuesta: B}.

\vspace{0.4em}
\textbf{6.} En triángulo rectángulo isósceles (catetos \(a\)), hipotenusa \(a\sqrt{2}\).
Área del cuadrado sobre la hipotenusa:
\[
\bigl(a\sqrt{2}\bigr)^{2}=2a^{2}.
\]
\textbf{Respuesta: B}.

\subsection*{Álgebra}

\textbf{7.} Sustituyendo \(x=m-1\) en \(y=mx-m^{2}-x\):
\[
y=m(m-1)-m^{2}-(m-1)=\bigl(m^{2}-m\bigr)-m^{2}-m+1=-2m+1.
\]
\textbf{Respuesta: B}.

\vspace{0.4em}
\textbf{8.} Suma de \(10\) términos: \(x+3,\dots,x+12\).
\[
\sum_{j=3}^{12}(x+j)=10x+\sum_{j=3}^{12}j=10x+(3+12)\cdot\frac{10}{2}=10x+75.
\]
Multiplicando por \(k\) y igualando a \(105k\) (con \(k\ne 0\)):
\[
10x+75=105 \Rightarrow 10x=30 \Rightarrow x=3.
\]
\textbf{Respuesta: A}.

\vspace{0.4em}
\textbf{9.} Para \(x\in[0,1]\): \(x\ge 0\), \(x+2>0\), \(x-1\le 0\).
Producto \(z=x(x+2)(x-1)\le 0\), y es \(0\) en \(x=0\) o \(x=1\). \\
\textbf{Respuesta: D}.

\subsection*{Análisis de datos}

\textbf{10.} Promedio inicial: \(\dfrac{\SI{80,0}{kg}}{25}=\SI{3,2}{kg}\).
Al reducir \(\SI{0,2}{kg}\) \emph{cada} caja, el nuevo promedio baja en \(\SI{0,2}{kg}\):
\[
\SI{3,2}{kg}-\SI{0,2}{kg}=\SI{3,0}{kg}.
\]
\textbf{Respuesta: A}.

\vspace{0.4em}
\textbf{11.} Total \(6\) bolas; no rojas \(=2+1=3\).
\[
P(\text{no roja})=\frac{3}{6}=\frac{1}{2}.
\]
\textbf{Respuesta: A}.

\vspace{0.4em}
\textbf{12.} No particular \(=\) bus + taxi + otros \(=35\%+10\%+10\%=55\%\) de \(200\Rightarrow 110\) personas \(>100\).
\textbf{Respuesta: B}.

% --------------------------------------------------------------------
\end{document}
