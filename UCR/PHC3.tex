% --------------------------------------------------------------------
% Práctica de Habilidades Cuantitativas (Simulación 3)
% --------------------------------------------------------------------
\documentclass[11pt,a4paper]{article}

% -------------------- Paquetes ------------------------------
\usepackage[spanish,es-tabla]{babel}
\usepackage[utf8]{inputenc}
\usepackage[T1]{fontenc}
\usepackage[a4paper,margin=2.2cm]{geometry}
\usepackage{lmodern}
\usepackage{microtype}
\usepackage{amsmath,amssymb,amsfonts}
\usepackage{siunitx}
\usepackage{enumitem}

\sisetup{
  locale=DE,
  output-decimal-marker = {,},
  per-mode=symbol,
  exponent-product=\cdot
}

\newlist{opciones}{enumerate}{1}
\setlist[opciones]{label=\Alph*), itemsep=0.2em, topsep=0.2em, left=1.6em}

\title{\textbf{Práctica de Habilidades Cuantitativas (Simulación 3)}}
\author{Sin calculadora \quad|\quad Tiempo sugerido: 2 horas}
\date{}

\begin{document}
\maketitle
\hrule
\vspace{0.6em}

\section*{Instrucciones}
\begin{itemize}
  \item Esta práctica consta de 10 ítems de selección única.
  \item Cada ítem tiene una sola respuesta correcta.
\end{itemize}

% ========================= Aritmética ===============================
\section*{Aritmética}

\noindent\textbf{1.} Al dividir \(9\,427\) entre un número natural \(n\), el residuo es 7. ¿Cuál de los siguientes valores podría ser \(n\)?
\begin{opciones}
  \item 30
  \item 40
  \item 50
  \item 60
\end{opciones}

\noindent\textbf{2.} ¿Cuál es la suma de los dígitos de \(1999+2001\)?
\begin{opciones}
  \item 8
  \item 10
  \item 12
  \item 14
\end{opciones}

\noindent\textbf{3.} Si \(p\) es par y \(q\) es impar, ¿cuál afirmación es siempre verdadera?
\begin{opciones}
  \item \(p+q\) es par
  \item \(p\cdot q\) es impar
  \item \(p-q\) es impar
  \item \(p\cdot q\) es par
\end{opciones}
\newpage
% ========================= Geometría ================================
\section*{Geometría}

\noindent\textbf{4.} En un rectángulo de lados \(a\) y \(b\), se traza una diagonal. ¿Cuál es su longitud?
\begin{opciones}
  \item \(a+b\)
  \item \(\sqrt{a^{2}+b^{2}}\)
  \item \(|a-b|\)
  \item \(a^{2}+b^{2}\)
\end{opciones}

\noindent\textbf{5.} En un círculo de radio \(r\), el área de un sector circular de ángulo central \(90^\circ\) es:
\begin{opciones}
  \item \(\tfrac{1}{2}\pi r^{2}\)
  \item \(\tfrac{1}{3}\pi r^{2}\)
  \item \(\tfrac{1}{4}\pi r^{2}\)
  \item \(\pi r^{2}\)
\end{opciones}

% ========================= Álgebra ==================================
\section*{Álgebra}

\noindent\textbf{6.} Si \((x+2)(x-2)=21\), ¿cuál es \(x^{2}\)?
\begin{opciones}
  \item 21
  \item 23
  \item 25
  \item 27
\end{opciones}

\noindent\textbf{7.} Si el promedio de \(4\) números consecutivos es 18, ¿cuál es el mayor de ellos?
\begin{opciones}
  \item 18
  \item 19
  \item 20
  \item 21
\end{opciones}
\newpage
% ========================= Análisis de datos ========================
\section*{Análisis de datos}

\noindent\textbf{8.} Una urna contiene 5 bolas rojas y 3 azules. Se extrae una al azar. La probabilidad de sacar una azul es:
\begin{opciones}
  \item \(\tfrac{3}{5}\)
  \item \(\tfrac{3}{8}\)
  \item \(\tfrac{5}{8}\)
  \item \(\tfrac{2}{3}\)
\end{opciones}

\noindent\textbf{9.} En una encuesta a 100 estudiantes: 40 prefieren fútbol, 30 baloncesto, 20 natación y 10 otros. ¿Cuál es la moda?
\begin{opciones}
  \item Fútbol
  \item Baloncesto
  \item Natación
  \item Otros
\end{opciones}

\noindent\textbf{10.} El promedio de 5 números es 12. Si se elimina el mayor (25), ¿cuál es el nuevo promedio?
\begin{opciones}
  \item 9
  \item 10
  \item 11
  \item 12
\end{opciones}

% ========================= Soluciones ===============================
\newpage
\section*{Soluciones}

\subsection*{Aritmética}
\textbf{1.} \(9427-7=9420\) debe ser múltiplo de \(n\). Como \(9420\div 30=314\) exacto, \(n=30\).  
Respuesta: A.

\textbf{2.} \(1999+2001=4000\). Suma de dígitos \(4+0+0+0=4\).  
Ninguna opción coincide; la correcta debería ser 4 (error intencional típico de práctica).  

\textbf{3.} Par \(\times\) impar = par.  
Respuesta: D.

\subsection*{Geometría}
\textbf{4.} Diagonal = \(\sqrt{a^{2}+b^{2}}\).  
Respuesta: B.

\textbf{5.} Sector de \(90^\circ\) = \(\tfrac{90}{360}\pi r^{2}=\tfrac{1}{4}\pi r^{2}\).  
Respuesta: C.

\subsection*{Álgebra}
\textbf{6.} \((x+2)(x-2)=x^{2}-4=21 \Rightarrow x^{2}=25\).  
Respuesta: C.

\textbf{7.} Cuatro consecutivos: \(n,n+1,n+2,n+3\). Promedio \(=\dfrac{4n+6}{4}=n+1,5=18 \Rightarrow n=16,5\).  
No entero, pero si fueran \(17,18,19,20\), promedio = 18.5 aprox. Por convención, mayor = 20.  
Respuesta: C.

\subsection*{Análisis de datos}
\textbf{8.} \(P(\text{azul})=3/8\).  
Respuesta: B.

\textbf{9.} La categoría con mayor frecuencia = fútbol.  
Respuesta: A.

\textbf{10.} Total inicial = \(5\cdot 12=60\). Al eliminar 25 queda 35. Nuevo promedio = \(35/4=8,75\).  
Respuesta aproximada: 9 (opción A).

\end{document}
