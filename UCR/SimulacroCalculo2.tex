
\documentclass[11pt,a4paper]{article}

\usepackage{amsmath}
% --- Idioma y formato numérico ---
\usepackage[T1]{fontenc}
\usepackage[utf8]{inputenc}
\usepackage[spanish, es-nodecimaldot]{babel}

% --- Matemática y SI ---
\usepackage{amsmath, amssymb, bm}
\usepackage{siunitx}
\sisetup{
  locale = DE,                 % coma decimal
  output-decimal-marker = {,}, % coma decimal
  per-mode = symbol,
  exponent-product = \cdot,
  group-minimum-digits = 4
}

% --- Página y figuras ---
\usepackage[a4paper,margin=2.4cm]{geometry}
\usepackage{graphicx}
\usepackage{xcolor}

\usepackage{tikz}
\usetikzlibrary{arrows.meta, calc, decorations.markings}

\renewcommand{\figurename}{Figura}
\renewcommand{\tablename}{Cuadro}

\begin{document}

\begin{center}
  {\Large \textbf{Simulacro segundo examen parcial }}\\[1ex]
  \textit{Sección cónica \(x^{2}+12x+9y^{2}=0\)}
\end{center}

% ==========================================================
\section*{ Pregunta 1}

\subsection*{Datos}
Ecuación general:
\[
x^{2}+12x+9y^{2}=0.
\]

\subsection*{Fórmula (completar cuadrado)}
\[
x^{2}+12x=(x+6)^{2}-36.
\]

\subsection*{Sustitución}
\[
x^{2}+12x+9y^{2}=0
\;\Longrightarrow\;
(x+6)^{2}-36+9y^{2}=0
\;\Longrightarrow\;
(x+6)^{2}+9y^{2}=36.
\]

\subsection*{Cálculo (forma canónica)}
Dividiendo entre \(36\):
\[
\boxed{\;\frac{(x+6)^{2}}{36}+\frac{y^{2}}{4}=1\;}
\]
que es una \textbf{elipse} de eje mayor horizontal con:
\[
\text{centro } C=(-6,\,0),\qquad a=6,\qquad b=2.
\]

\subsection*{Parámetros notables}
\[
c=\sqrt{a^{2}-b^{2}}=\sqrt{36-4}=4\sqrt{2},\qquad
e=\frac{c}{a}=\frac{4\sqrt{2}}{6}=\frac{2\sqrt{2}}{3}.
\]

\subsection*{Focos, vértices y covértices}
Eje mayor horizontal (\(x\)):
\[
F_{1,2}=(-6\mp 4\sqrt{2},\,0),\qquad
V_{1,2}=(-6\mp 6,\,0)=(-12,\,0)\ \text{y}\ (0,\,0).
\]
Eje menor (sobre \(y\)):
\[
\text{Covértices: }\;(-6,\,\pm 2).
\]

\subsection*{Intersecciones con los ejes}
\begin{itemize}
  \item Con el eje \(x\) (\(y=0\)): \(x^{2}+12x=0\Rightarrow x=0,-12\).\\
        Puntos: \((0,\,0)\) y \((-12,\,0)\) (coinciden con los vértices).
  \item Con el eje \(y\) (\(x=0\)): \(9y^{2}=0\Rightarrow y=0\).\\
        Punto: \((0,\,0)\) (tangencia en el vértice derecho).
\end{itemize}

\subsection*{Directrices}
Para el eje mayor horizontal:
\[
x = h \pm \frac{a}{e}
= -6 \pm \frac{6}{\tfrac{2\sqrt{2}}{3}}
= -6 \pm \frac{18}{2\sqrt{2}}
= -6 \pm \frac{9\sqrt{2}}{2}.
\]

\subsection*{Conclusión}
La curva es una \textbf{elipse} centrada en \((-6,\,0)\), con eje mayor horizontal de longitud \(2a=12\) (de \((-12,0)\) a \((0,0)\)) y eje menor de longitud \(2b=4\) (de \((-6,-2)\) a \((-6,2)\)). Los focos se ubican en \((-6\pm 4\sqrt{2},\,0)\).

% ==========================================================
\section*{Bosquejo (centro, ejes, vértices, focos e intersecciones)}

\begin{figure}[h!]
\centering
\shorthandoff{>}

\begin{tikzpicture}[scale=0.5,>=Stealth]
  % Ejes
  \draw[->,thin] (-16,0)--(4,0) node[below] {$x$};
  \draw[->,thin] (-6,-4)--(-6,4) node[left] {$y$};

  % Elipse (a=6, b=2, centro (-6,0))
  \draw[line width=1pt,blue]
    plot[domain=0:360,samples=200]
    ({-6 + 6*cos(\x)}, {2*sin(\x)});

  % Centro
  \filldraw[black] (-6,0) circle (2pt) node[below right] {$C(-6,0)$};

  % Vértices
  \filldraw[black] (-12,0) circle (2pt) node[below left] {$V_1(-12,0)$};
  \filldraw[black] (0,0)   circle (2pt) node[below right] {$V_2(0,0)$};

  % Covértices
  \filldraw[black] (-6,2)  circle (2pt) node[above right] {$(-6,2)$};
  \filldraw[black] (-6,-2) circle (2pt) node[below right] {$(-6,-2)$};

  % Focos (c = 4*sqrt(2) ≈ 5,657)
  \def\c{5.657}
  \filldraw[red] ({-6-\c},0) circle (2.2pt) node[above left] {$F_1(-6-4\sqrt{2},0)$};
  \filldraw[red] ({-6+\c},0) circle (2.2pt) node[above right] {$F_2(-6+4\sqrt{2},0)$};


\end{tikzpicture}
\caption{Elipse $\dfrac{(x+6)^{2}}{36}+\dfrac{y^{2}}{4}=1$.}
\end{figure}

% ==========================================================
% Pregunta 2. Coordenadas polares: r = cos(2θ),  r = sen^2(θ)
% Incisos (a) asociación y (b) a cartesianas
% ==========================================================
\newpage
\section*{2) Ecuaciones en coordenadas polares}
\[
r=\cos(2\theta),\qquad r=\sen^{2}\theta.
\]

% ------------------- (a) Asociación con las figuras -------------------
\subsection*{(a) Asociación con las figuras y argumento}
\textbf{Figura (b)} $\longleftrightarrow$ $r=\cos(2\theta)$.
\begin{itemize}
  \item $r=\cos(2\theta)$ genera una \emph{rosa} de 4 pétalos (par de $2$).
  \item Para $0\le\theta\le\pi$ se observan tres lóbulos: derecho ($\theta\approx 0$),
        izquierdo ($\theta\approx \pi$) y el inferior, que aparece porque $r<0$
        cerca de $\theta=\pi/2$ (el punto se dibuja en dirección opuesta).
\end{itemize}

\textbf{Figura (a)} $\longleftrightarrow$ $r=\sen^{2}\theta$.
\begin{itemize}
  \item $r=\sen^{2}\theta\ge 0$ para todo $\theta$, con máximo $r=1$ en $\theta=\pi/2$
        y $r=0$ en $\theta=0,\pi$.
  \item El trazo forma un lazo tipo “gota” \emph{sobre} el eje $x$ y es simétrico
        respecto del eje $y$, cerrando en el origen.
\end{itemize}

% ------------------- (b) Paso a cartesianas -------------------
\subsection*{(b) Expresión en coordenadas cartesianas}

\paragraph{Caso 1: $r=\cos(2\theta)$.}
\textbf{Datos.}
\[
x=r\cos\theta,\quad y=r\sen\theta,\quad r=\sqrt{x^{2}+y^{2}},\quad
\cos(2\theta)=\cos^{2}\theta-\sen^{2}\theta=\frac{x^{2}-y^{2}}{r^{2}}.
\]

\textbf{Sustitución.}
\[
r=\cos(2\theta)=\frac{x^{2}-y^{2}}{r^{2}}
\quad\Rightarrow\quad r^{3}=x^{2}-y^{2}.
\]

\textbf{Cálculo.} Reemplazando $r=\sqrt{x^{2}+y^{2}}$ y elevando al cuadrado para
eliminar la raíz:
\[
\bigl(\sqrt{x^{2}+y^{2}}\bigr)^{3}=x^{2}-y^{2}
\ \Longrightarrow\
\boxed{(x^{2}+y^{2})^{3}=(x^{2}-y^{2})^{2}}.
\]

\textbf{Conclusión.} Ecuación cartesiana implícita (grado 6) de la rosa de 4 pétalos.

\medskip

\paragraph{Caso 2: $r=\sen^{2}\theta$.}
\textbf{Datos.}
\[
\sen\theta=\frac{y}{r}\ \Rightarrow\ \sen^{2}\theta=\frac{y^{2}}{r^{2}},\qquad
r=\sqrt{x^{2}+y^{2}}.
\]

\textbf{Sustitución.}
\[
r=\frac{y^{2}}{r^{2}}
\quad\Rightarrow\quad r^{3}=y^{2}.
\]

\textbf{Cálculo.} Con $r=\sqrt{x^{2}+y^{2}}$ y cuadrando:
\[
\bigl(\sqrt{x^{2}+y^{2}}\bigr)^{3}=y^{2}
\ \Longrightarrow\
\boxed{(x^{2}+y^{2})^{3}=y^{4}}.
\]

\textbf{Conclusión.} Ecuación cartesiana implícita del lazo (“gota”) superior,
simétrica respecto del eje $y$.

% ------------------- Nota técnica opcional -------------------
\paragraph{Nota.}
Las formas sin cuadrar son
$(x^{2}+y^{2})^{3/2}=x^{2}-y^{2}$ y $(x^{2}+y^{2})^{3/2}=y^{2}$.
Para describir \emph{toda} la curva sin restricciones de signo en el miembro derecho,
se prefiere la versión \emph{cuadrada} mostrada en los recuadros.

% ==========================================================
% Pregunta 3. Sucesión: x_{1}=7,\; x_{n+1}=\sqrt{3x_{n}+4}
% Inducción: decreciente y acotada; convergencia y límite
% ==========================================================

\section*{3) Sucesión \(x_{1}=7,\; x_{n+1}=\sqrt{3x_{n}+4}\) para \(n\ge 1\)}

\subsection*{(a) Demostrar por inducción que es decreciente y que \(x_n\ge 3\) para todo \(n\)}

\paragraph{Paso 0 — Observación clave.}
Definimos \(f(x)=\sqrt{3x+4}\).
\[
f'(x)=\frac{3}{2\sqrt{3x+4}}>0 \quad(\text{$f$ es estrictamente creciente}).
\]
Además,
\[
f(x)\le x
\ \Longleftrightarrow\
\sqrt{3x+4}\le x
\ \Longleftrightarrow\
3x+4\le x^{2}
\ \Longleftrightarrow\
x^{2}-3x-4\ge 0
\ \Longleftrightarrow\
(x-4)(x+1)\ge 0.
\]
Como \(x>0\), si \(x\ge 4\) entonces \(f(x)\le x\).

\medskip
Demostraremos por inducción dos hechos:
\[
\boxed{\,\text{(i) }x_n\ge 4\ \Rightarrow\ x_n\ge 3\,}\qquad\text{y}\qquad
\boxed{\,\text{(ii) }x_{n+1}\le x_n\ (\text{decreciente}).\,}
\]

\paragraph{1. Inducción para la cota \(x_n\ge 4\).}
\textbf{Base:} \(x_1=7\ge 4\).

\textbf{Hipótesis de inducción:} Suponga \(x_n\ge 4\).

\textbf{Paso inductivo:}
Como \(f\) es creciente,
\[
x_{n+1}=f(x_n)\ \ge\ f(4)=\sqrt{3\cdot 4+4}=\sqrt{16}=4.
\]
Concluimos \(\boxed{x_{n}\ge 4\ \ \forall n}\). (En particular, \(x_n\ge 3\) para todo \(n\).)

\paragraph{2. Inducción para la monotonía decreciente.}
\textbf{Base:} 
\[
x_2=\sqrt{3x_1+4}=\sqrt{21+4}=\sqrt{25}=5\le 7=x_1.
\]

\textbf{Hipótesis de inducción:} Suponga \(x_n\ge 4\) (ya probado) y que \(x_n\ge x_{n-1}\) no es necesario; usaremos directamente \(x_n\ge 4\).

\textbf{Paso inductivo:}
Con \(x_n\ge 4\) tenemos \(f(x_n)\le x_n\) (por el análisis del Paso 0). Luego
\[
x_{n+1}=f(x_n)\le x_n.
\]
Concluimos \(\boxed{x_{n+1}\le x_n\ \ \forall n}\); es decir, \((x_n)\) es \textbf{decreciente}.

\paragraph{Conclusión del (a).}
Por inducción:
\[
\boxed{x_n\ge 4\ (\Rightarrow x_n\ge 3)\quad \text{y}\quad x_{n+1}\le x_n\ \text{ para todo }n.}
\]

% ----------------------------------------------------------

\subsection*{(b) Convergencia y cálculo del límite}

\paragraph{Monotonía + cota inferior.}
La sucesión es \textbf{decreciente} y está \textbf{acotada inferiormente} por \(4\).
Por el \emph{Teorema de convergencia monótona}, \((x_n)\) \textbf{converge}.

\paragraph{Cálculo del límite.}
Sea \(\displaystyle \lim_{n\to\infty} x_n = L\). Pasando a límite en la recurrencia:
\[
L=\sqrt{3L+4}
\ \Longrightarrow\
L^{2}=3L+4
\ \Longrightarrow\
L^{2}-3L-4=0
\ \Longrightarrow\
(L-4)(L+1)=0.
\]
Como \(x_n>0\), el único límite admisible es
\[
\boxed{L=4}.
\]

\paragraph{Interpretación.}
El punto fijo de \(f(x)=\sqrt{3x+4}\) en \((0,\infty)\) es \(x=4\).
Iniciando en \(x_1=7\), la iteración desciende monótonamente y queda atrapada por abajo en \(4\), por lo que converge exactamente a ese punto fijo.
% ==========================================================
% Pregunta 4. Sucesión a_n = (n/(n+1)) cos(nπ+1)
% Convergencia, monotonía y acotación
% ==========================================================
\newpage
\section*{4) Sea \(a_n=\dfrac{n}{n+1}\cos(n\pi+1)\). Analice convergencia, monotonía y acotación.}

\subsection*{Datos}
\[
a_n=\frac{n}{n+1}\cos(n\pi+1),\qquad n\in\mathbb{N}.
\]

\subsection*{Fórmula (descomposición trigonométrica)}
\[
\cos(n\pi+1)=\cos(n\pi)\cos 1-\sin(n\pi)\sin 1
= (-1)^n\cos 1.
\]

\subsection*{Sustitución}
\[
\boxed{\ a_n=(-1)^n\cos 1\cdot\frac{n}{n+1}\ }.
\]

\subsection*{Cálculo de propiedades}

\paragraph{Acotación.}
\[
|a_n|=|\cos 1|\cdot\frac{n}{n+1}<|\cos 1|<1.
\]
Luego \(\exists M=|\cos 1|\) tal que \(|a_n|\le M\) para todo \(n\).  
\[
\boxed{\text{La sucesión es acotada.}}
\]

\paragraph{Convergencia.}
Como \(\dfrac{n}{n+1}\to 1\) y el factor \((-1)^n\) alterna de signo,
\[
a_{2k}= \cos 1\cdot \frac{2k}{2k+1}\xrightarrow[k\to\infty]{}\cos 1,
\qquad
a_{2k+1}= -\cos 1\cdot \frac{2k+1}{2k+2}\xrightarrow[k\to\infty]{}-\cos 1.
\]
Las subsucesiones par e impar tienen \emph{límites distintos}. Por lo tanto,
\[
\boxed{\text{la sucesión } (a_n)\ \text{no converge}.}
\]

\paragraph{Monotonía.}
Debido al factor \((-1)^n\), los términos cambian de signo; en consecuencia,
\[
\boxed{\text{(a\_n) no es monótona en } \mathbb{N}.}
\]
Sin embargo, las \emph{subsucesiones} sí lo son:
\[
a_{2k}=\cos 1\cdot\frac{2k}{2k+1}\ \nearrow\ \cos 1,
\qquad
a_{2k+1}=-\cos 1\cdot\frac{2k+1}{2k+2}\ \searrow\ -\cos 1.
\]

\subsection*{Conclusión}
\[
\boxed{\text{Acotada: sí}} \quad
\boxed{\text{Monótona: no}} \quad
\boxed{\text{Convergente: no}}.
\]
% ==========================================================
% Pregunta 5. Series: resolver ambas y justificar el método
% ==========================================================

\section*{5) Series}

\subsection*{Serie (I): \(\displaystyle \sum_{n=0}^{\infty}(-1)^{n}\,\frac{2^{n}}{(1+\sqrt{2})^{n}}\)}

\textbf{Datos.}
\[
a_n=(-1)^n\left(\frac{2}{1+\sqrt{2}}\right)^{n}.
\]

\textbf{Razón geométrica.}
\[
r=-\,\frac{2}{1+\sqrt{2}}
=-\,\frac{2(1-\sqrt{2})}{(1+\sqrt{2})(1-\sqrt{2})}
=-\,2(\sqrt{2}-1),
\qquad |r|=2(\sqrt{2}-1)<1.
\]

\textbf{Criterio y suma.}
Como \(|r|<1\), la serie es geométrica convergente y
\[
S=\sum_{n=0}^{\infty}r^{n}=\frac{1}{1-r}
=\frac{1}{1+2(\sqrt{2}-1)}
=\frac{1}{2\sqrt{2}-1}
=\frac{2\sqrt{2}+1}{(2\sqrt{2})^{2}-1}
=\boxed{\frac{2\sqrt{2}+1}{7}}.
\]
\subsection*{Serie \(\displaystyle \sum_{n=0}^{\infty}(-1)^{n}\,\frac{2^{n}}{(1+\sqrt{2})^{n}}\)}
Razón geométrica:
\[
r=-\frac{2}{1+\sqrt{2}},\qquad |r|=\frac{2}{1+\sqrt{2}}<1 \Rightarrow \text{converge (geométrica, incluso absolutamente)}.
\]
Suma:
\[
S=\frac{1}{1-r}
=\frac{1}{1+\frac{2}{1+\sqrt{2}}}
=\frac{1+\sqrt{2}}{3+\sqrt{2}}
=\frac{(1+\sqrt{2})(3-\sqrt{2})}{9-2}
=\boxed{\frac{1+2\sqrt{2}}{7}}.
\]

% ----------------------------------------------------------

\subsection*{(II)\quad \(\displaystyle \sum_{n=3}^{\infty}\frac{1}{(2n+1)(2n+3)}\)}

\textbf{Datos.} Términos positivos y decrecientes. El patrón sugiere \emph{telescopía} vía fracciones parciales.

\textbf{Método.} Descomposición en fracciones parciales para obtener una diferencia de términos consecutivos
que cancele en la suma (serie telescópica). Además, esto también muestra convergencia por comparación con una \(p\)-serie.

\textbf{Fracciones parciales.}
\[
\frac{1}{(2n+1)(2n+3)}=\frac{A}{2n+1}+\frac{B}{2n+3}
\ \Longrightarrow\
A=\frac{1}{2},\; B=-\frac{1}{2}.
\]
\[
\Rightarrow\quad 
\frac{1}{(2n+1)(2n+3)}
=\frac{1}{2}\!\left(\frac{1}{2n+1}-\frac{1}{2n+3}\right).
\]

\textbf{Telescopía.} Suma parcial \(S_N=\displaystyle\sum_{n=3}^{N}\frac{1}{(2n+1)(2n+3)}\):
\[
S_N=\frac{1}{2}\Bigl(\tfrac{1}{7}-\tfrac{1}{9}+\tfrac{1}{9}-\tfrac{1}{11}+\cdots+\tfrac{1}{2N+1}-\tfrac{1}{2N+3}\Bigr)
=\frac{1}{2}\left(\frac{1}{7}-\frac{1}{2N+3}\right).
\]

\textbf{Límite de las sumas parciales.}
\[
\lim_{N\to\infty}S_N=\frac{1}{2}\cdot\frac{1}{7}
=\boxed{\frac{1}{14}}.
\]

\textbf{Justificación de convergencia (adicional).} Para \(n\ge 3\),
\[
\frac{1}{(2n+1)(2n+3)}<\frac{1}{(2n)^{2}}=\frac{1}{4n^{2}},
\]
y como \(\sum \dfrac{1}{n^{2}}\) converge, por comparación directa la serie dada también converge.
La telescopía, además, entrega la \emph{suma exacta}.
% ==========================================================
% Pregunta 6. Convergencia/divergencia con criterios estudiados
% ==========================================================

\section*{6) Determine la convergencia o divergencia usando criterios}

\subsection*{(i)\quad \(\displaystyle \sum_{n=1}^{\infty}\frac{n}{n^{2}+\cos(n)}\)}

\textbf{Datos.} Términos positivos; \(-1\le \cos n \le 1\).

\textbf{Criterio elegido:} \emph{Comparación por el límite} con \(b_n=\dfrac{1}{n}\) (serie armónica).

\textbf{Cálculo del cociente límite.}
\[
\lim_{n\to\infty}\frac{a_n}{b_n}
=\lim_{n\to\infty}\frac{\dfrac{n}{n^{2}+\cos n}}{\dfrac{1}{n}}
=\lim_{n\to\infty}\frac{n^{2}}{n^{2}+\cos n}
=1\in(0,\infty).
\]

\textbf{Conclusión.} Como \(\sum \frac{1}{n}\) \textbf{diverge} y el cociente límite es \(1\),
por el criterio de comparación límite:
\[
\boxed{\displaystyle \sum_{n=1}^{\infty}\frac{n}{n^{2}+\cos(n)} \text{ diverge}.}
\]

\medskip
\textit{(Observación alternativa)} Para \(n\ge 2\):
\[
\frac{n}{n^{2}+\cos n}\ge \frac{n}{n^{2}+1}\sim \frac{1}{n},
\]
lo que también da divergencia por comparación directa con la armónica.

% ----------------------------------------------------------

\subsection*{(ii)\quad \(\displaystyle \sum_{n=2}^{\infty}\frac{e^{-\sqrt{n}}}{\sqrt{n}}\)}

\textbf{Datos.} Términos positivos y decrecientes para \(n\ge 2\).

\textbf{Criterio elegido (opción A):} \emph{Criterio integral}.
Sea \(f(x)=\dfrac{e^{-\sqrt{x}}}{\sqrt{x}}\) continua, positiva y decreciente en \([2,\infty)\).

\textbf{Sustitución.} \(t=\sqrt{x}\Rightarrow x=t^{2},\; dx=2t\,dt\).

\subsection*{Serie \(\displaystyle \sum_{n=2}^{\infty}\frac{e^{-\sqrt{n}}}{\sqrt{n}}\)}

\[
a_n=\frac{e^{-\sqrt{n}}}{\sqrt{n}}, \quad n\ge2
\]

\[
f(x)=\frac{e^{-\sqrt{x}}}{\sqrt{x}}
\]

\begin{itemize}
  \item Positiva
  \item Continua
  \item Decreciente
\end{itemize}

\[
\int_{2}^{\infty}\frac{e^{-\sqrt{x}}}{\sqrt{x}}\,dx
=\lim_{b\to\infty}\int_{2}^{b}\frac{e^{-\sqrt{x}}}{\sqrt{x}}\,dx
\]

Sea \(u=\sqrt{x}\), entonces \(du=\dfrac{1}{2\sqrt{x}}dx\Rightarrow dx=2\sqrt{x}\,du\).

\[
\int_{2}^{b}\frac{e^{-\sqrt{x}}}{\sqrt{x}}\,dx
=\int_{\sqrt{2}}^{\sqrt{b}}\frac{e^{-u}}{u}\,(2u)\,du
=2\int_{\sqrt{2}}^{\sqrt{b}}e^{-u}\,du
\]

\[
\int e^{-u}\,du=-e^{-u}+C
\Rightarrow
\int_{\sqrt{2}}^{\sqrt{b}} 2e^{-u}\,du
=\left[-2e^{-u}\right]_{\sqrt{2}}^{\sqrt{b}}
=-2e^{-\sqrt{b}}+2e^{-\sqrt{2}}
\]

\[
\lim_{b\to\infty}\left(-2e^{-\sqrt{b}}+2e^{-\sqrt{2}}\right)
=2e^{-\sqrt{2}}
\]

\[
\boxed{\displaystyle \int_{2}^{\infty}\frac{e^{-\sqrt{x}}}{\sqrt{x}}\,dx
=2e^{-\sqrt{2}}<\infty}
\]

Por tanto, según el \textbf{criterio integral}:
\[
\boxed{\displaystyle \sum_{n=2}^{\infty}\frac{e^{-\sqrt{n}}}{\sqrt{n}} \text{ converge.}}
\]


\textbf{Conclusión.} La integral converge \(\Rightarrow\) por el criterio integral,
\[
\boxed{\displaystyle \sum_{n=2}^{\infty}\frac{e^{-\sqrt{n}}}{\sqrt{n}} \text{ converge}.}
\]

\medskip
\textbf{Criterio elegido (opción B, equivalente):} \emph{Comparación por bloques} con geométrica.  
Para \(m\in\mathbb{N}\), si \(n\in[m^{2},(m+1)^{2}-1]\) entonces \(\sqrt{n}\ge m\) y
\[
\sum_{n=m^{2}}^{(m+1)^{2}-1}\frac{e^{-\sqrt{n}}}{\sqrt{n}}
\le \sum_{n=m^{2}}^{(m+1)^{2}-1}\frac{e^{-m}}{m}
\le \frac{2m\,e^{-m}}{m}=2e^{-m}.
\]
Sumando en \(m\):
\[
\sum_{n=2}^{\infty}\frac{e^{-\sqrt{n}}}{\sqrt{n}}
\le \sum_{m=1}^{\infty}2e^{-m}
=\frac{2}{e-1}<\infty,
\]
luego converge por comparación con una geométrica.

\end{document}