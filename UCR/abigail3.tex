% ==========================================================
% Sucesión: x_{n+1} = sqrt(3 x_n + 4),  x_1 = 7
% Prueba: decreciente y acotada; convergencia y límite
% ==========================================================
\documentclass[11pt,letterpaper]{article}

% Idioma y matemática
\usepackage[T1]{fontenc}
\usepackage[utf8]{inputenc}
\usepackage[spanish, es-nodecimaldot]{babel}
\usepackage{amsmath, amssymb}

\begin{document}

\section*{Sucesión \(x_{1}=7,\; x_{n+1}=\sqrt{3x_{n}+4}\) para \(n\ge 1\)}

\subsection*{(a) Inducción: \( (x_n) \) es decreciente y \(x_n \ge 3\) para todo \(n\)}

\textbf{Datos.} \(x_{1}=7\), \quad \(x_{n+1}=f(x_n)\) con \(f(x)=\sqrt{3x+4}\).

\textbf{Paso 1 — Cota inferior \(x_n\ge 3\).}
\begin{itemize}
  \item \textit{Base:} \(x_1=7\ge 3\).
  \item \textit{Paso inductivo:} Suponga \(x_n\ge 3\). Entonces
  \[
  x_{n+1}=\sqrt{3x_n+4}\ \ge\ \sqrt{3\cdot 3+4}=\sqrt{13}>3.
  \]
  Por tanto, \(x_{n+1}\ge 3\).
\end{itemize}
Concluimos por inducción que \(\boxed{x_n\ge 3\ \forall n}\).
(De hecho, incluso \(x_n\ge 4\), ver abajo.)

\textbf{Paso 2 — Monotonía decreciente.}
Notamos que, para \(x>0\),
\[
f(x)\le x \iff \sqrt{3x+4}\le x \iff 3x+4\le x^{2}
\iff x^{2}-3x-4\ge 0 \iff (x-4)(x+1)\ge 0.
\]
Así, si \(x\ge 4\) entonces \(f(x)\le x\).

\textit{Mostramos por inducción que \(x_n\ge 4\).}
\begin{itemize}
  \item \textit{Base:} \(x_1=7\ge 4\).
  \item \textit{Paso inductivo:} Si \(x_n\ge 4\), entonces
  \[
  x_{n+1}=\sqrt{3x_n+4}\ \ge\ \sqrt{3\cdot 4+4}=\sqrt{16}=4.
  \]
\end{itemize}
Luego \(x_n\ge 4\) para todo \(n\). En consecuencia, usando la implicación anterior,
\[
x_{n+1}=f(x_n)\le x_n \quad \text{para todo } n,
\]
es decir, \(\boxed{(x_n)\ \text{es decreciente}.}\)

\textbf{Conclusión del (a).} La sucesión es \emph{decreciente} y está \emph{acotada inferiormente} por \(4\) (y, por tanto, por \(3\)).

\subsection*{(b) Convergencia y límite}

\textbf{Criterio de convergencia.}
Una sucesión monótona y acotada converge. Como \((x_n)\) es decreciente y \(x_n\ge 4\), resulta \(\boxed{(x_n)\ \text{converge}.}\)

\textbf{Cálculo del límite.}
Si \(\displaystyle \lim_{n\to\infty} x_n = L\), pasando al límite en la relación recursiva:
\[
L=\sqrt{3L+4}
\ \Longrightarrow\
L^{2}=3L+4
\ \Longrightarrow\
L^{2}-3L-4=0
\ \Longrightarrow\
(L-4)(L+1)=0.
\]
Como \(x_n>0\), el único límite admisible es \(\boxed{L=4}\).

\textbf{Interpretación.} La iteración \(f(x)=\sqrt{3x+4}\) deja invariante el punto \(x=4\) y, al iniciar en \(x_1=7\), la sucesión desciende de forma monótona y queda atrapada por debajo por \(4\), convergiendo precisamente a ese punto fijo.
\section*{Análisis de \(a_n=\dfrac{n}{n+1}\cos(n\pi+1)\)}
\[
\cos(n\pi+1)=\cos(n\pi)\cos 1-\sin(n\pi)\sin 1=(-1)^n\cos 1,
\]
\[
\Rightarrow\quad a_n=(-1)^n\cos 1\cdot\frac{n}{n+1}.
\]

\textbf{Acotación:}
\[
|a_n|=|\cos 1|\frac{n}{n+1}<|\cos 1|\quad\Rightarrow\quad -|\cos 1|<a_n<|\cos 1|.
\]

\textbf{Monotonía:} no es monótona (alterna de signo). Subsucesiones:
\[
a_{2k}=\cos 1\cdot\frac{2k}{2k+1}\nearrow \cos 1,\qquad
a_{2k+1}=-\cos 1\cdot\frac{2k+1}{2k+2}\searrow -\cos 1.
\]

\textbf{Convergencia:} no converge, pues
\[
\lim_{k\to\infty} a_{2k}=\cos 1\neq -\cos 1=\lim_{k\to\infty} a_{2k+1}.
\]

\end{document}
