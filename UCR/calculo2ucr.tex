\documentclass[12pt]{article}
\usepackage[utf8]{inputenc}
\usepackage[spanish]{babel}
\usepackage{amsmath, amssymb}
\usepackage{siunitx}
\sisetup{output-decimal-marker={,}}

\title{Ejemplos de criterios de convergencia de series}
\author{}
\date{}

\begin{document}

\maketitle


\section*{Ejemplo criterio de la integral}

Demuestre que la serie
\[
\sum_{n=1}^{\infty} \frac{1}{n^2 + 1}
\]
es convergente utilizando el \textbf{criterio de la integral}.

\section*{Paso 1: Verificar condiciones del criterio de la integral}

Sea \( f(x) = \frac{1}{x^2 + 1} \). Esta función cumple que:

\begin{itemize}
    \item \( f(x) > 0 \) para \( x \geq 1 \),
    \item \( f(x) \) es continua para \( x \geq 1 \),
    \item \( f(x) \) es decreciente para \( x \geq 1 \), ya que:
    \[
    f'(x) = \frac{d}{dx} \left( \frac{1}{x^2 + 1} \right) = \frac{-2x}{(x^2 + 1)^2} < 0 \quad \text{para } x > 0.
    \]
\end{itemize}

\section*{Paso 2: Aplicar el criterio de la integral}

Se evalúa la convergencia de la integral impropia:
\[
\int_1^{\infty} \frac{1}{x^2 + 1} \, dx.
\]

\section*{Paso 3: Calcular la integral}

\[
\int_1^{\infty} \frac{1}{x^2 + 1} \, dx
= \lim_{t \to \infty} \int_1^{t} \frac{1}{x^2 + 1} \, dx
= \lim_{t \to \infty} \left[ \arctan(x) \right]_1^t
= \lim_{t \to \infty} \left( \arctan(t) - \arctan(1) \right)
\]

\[
= \frac{\pi}{2} - \frac{\pi}{4} = \frac{\pi}{4}.
\]

\section*{Paso 4: Conclusión}

La integral impropia converge, por lo tanto, según el \textbf{criterio de la integral}, la serie
\[
\sum_{n=1}^{\infty} \frac{1}{n^2 + 1}
\]
también \textbf{converge}.
\section*{Ejemplo: Análisis de convergencia con el criterio de la integral}

Analicemos la convergencia de la serie
\[
\sum_{n=6}^{\infty} \frac{1}{n(\ln n - 1)^3}
\]
utilizando el \textbf{criterio de la integral}.

\subsection*{Paso 1: Verificar condiciones del criterio de la integral}

Sea la función:
\[
f(x) = \frac{1}{x(\ln x - 1)^3}
\]
para \( x \geq 6 \). Observamos que:

\begin{itemize}
    \item \( f(x) > 0 \) para \( x > e \), en particular para \( x \geq 6 \),
    \item \( f(x) \) es continua para \( x > e \), en particular en el intervalo de integración,
    \item \( f(x) \) es decreciente para \( x \geq 6 \) (esto se puede verificar derivando, pero basta con observar que el denominador crece más rápido que el numerador).
\end{itemize}

\subsection*{Paso 2: Aplicar el criterio de la integral}

Estudiamos la convergencia de la integral impropia:
\[
\int_6^{\infty} \frac{1}{x(\ln x - 1)^3} \, dx
\]

Usamos el cambio de variable:
\[
u = \ln x - 1 \quad \Rightarrow \quad \ln x = u + 1 \quad \Rightarrow \quad x = e^{u+1}
\]

Entonces:
\[
dx = e^{u+1} \, du \quad \text{y} \quad \frac{1}{x} = \frac{1}{e^{u+1}}
\]

Por tanto, la integral queda:
\[
\int_{u_0}^{\infty} \frac{1}{e^{u+1}} \cdot \frac{1}{u^3} \cdot e^{u+1} \, du = \int_{u_0}^{\infty} \frac{1}{u^3} \, du
\]

donde \( u_0 = \ln 6 - 1 > 0 \).

\subsection*{Paso 3: Evaluar convergencia de la nueva integral}

\[
\int_{u_0}^{\infty} \frac{1}{u^3} \, du = \left[ -\frac{1}{2u^2} \right]_{u_0}^{\infty} = 0 + \frac{1}{2u_0^2} < \infty
\]

\subsection*{Paso 4: Conclusión}

Como la integral impropia converge, entonces por el \textbf{criterio de la integral}, la serie

\[
\sum_{n=6}^{\infty} \frac{1}{n(\ln n - 1)^3}
\]

también \textbf{converge}.
\section*{Otra Forma del Ejemplo: Análisis de convergencia con el criterio de la integral}

Analicemos la convergencia de la serie
\[
\sum_{n=6}^{\infty} \frac{1}{n(\ln n - 1)^3}
\]
utilizando el \textbf{criterio de la integral}.

\subsection*{Paso 1: Verificar condiciones del criterio de la integral}

Sea la función:
\[
f(x) = \frac{1}{x(\ln x - 1)^3}
\]
para \( x \geq 6 \). Observamos que:

\begin{itemize}
    \item \( f(x) > 0 \) para \( x > e \), en particular para \( x \geq 6 \),
    \item \( f(x) \) es continua para \( x > e \),
    \item \( f(x) \) es decreciente para \( x \geq 6 \).
\end{itemize}

\subsection*{Paso 2: Aplicar el criterio de la integral}

Estudiamos la convergencia de la integral impropia:
\[
\int_6^{\infty} \frac{1}{x(\ln x - 1)^3} \, dx
\]

Hacemos el cambio de variable:
\[
u = \ln x - 1 \quad \Rightarrow \quad du = \frac{1}{x} dx
\]

Entonces:
\[
\int_6^{\infty} \frac{1}{x(\ln x - 1)^3} \, dx
= \int_{u_0}^{\infty} \frac{1}{u^3} \, du
\]

donde \( u_0 = \ln(6) - 1 \).

\subsection*{Paso 3: Evaluar convergencia de la nueva integral}

\[
\int_{u_0}^{\infty} \frac{1}{u^3} \, du = \left[ -\frac{1}{2u^2} \right]_{u_0}^{\infty}
= 0 + \frac{1}{2u_0^2} < \infty
\]

\subsection*{Paso 4: Conclusión}

Como la integral impropia converge, entonces por el \textbf{criterio de la integral}, la serie

\[
\sum_{n=6}^{\infty} \frac{1}{n(\ln n - 1)^3}
\]

también \textbf{converge}.
\section*{Ejemplo: Criterio de comparación directa}

Analicemos la convergencia de la serie:
\[
\sum_{n=0}^{\infty} \frac{\arctan(n)}{4n^3 + 3}
\]

\subsection*{Paso 1: Considerar una serie de comparación}

Sabemos que:
\[
\lim_{n \to \infty} \arctan(n) = \frac{\pi}{2}
\quad \Rightarrow \quad
\arctan(n) < \frac{\pi}{2} \text{ para todo } n \in \mathbb{N}
\]

Por tanto:
\[
\frac{\arctan(n)}{4n^3 + 3} < \frac{\pi/2}{4n^3} = \frac{C}{n^3}
\quad \text{con } C = \frac{\pi}{8}
\]

Entonces definimos una serie de comparación:
\[
b_n = \frac{1}{n^3}
\quad \text{(serie p con } p = 3 > 1 \text{, converge)}
\]

\subsection*{Paso 2: Aplicar el criterio de comparación directa}

Como
\[
0 \leq \frac{\arctan(n)}{4n^3 + 3} \leq \frac{C}{n^3}
\]
y
\[
\sum_{n=1}^{\infty} \frac{1}{n^3} \text{ converge},
\]
entonces, por el \textbf{criterio de comparación directa}, la serie
\[
\sum_{n=1}^{\infty} \frac{\arctan(n)}{4n^3 + 3}
\]
también \textbf{converge}.

\subsection*{Nota:}
El término con \( n = 0 \) es finito:
\[
\frac{\arctan(0)}{4(0)^3 + 3} = 0,
\]
por lo tanto no afecta la convergencia.

\subsection*{Conclusión:}

La serie completa
\[
\sum_{n=0}^{\infty} \frac{\arctan(n)}{4n^3 + 3}
\]
es \textbf{convergente}.

\newpage
\section*{Ejemplo: Criterio de comparación directa}

Analicemos la convergencia de la serie:
\[
\sum_{n=1}^{\infty} \frac{n^3 + 1}{n^5 + 4n^3 - 1}
\]

\subsection*{Paso 1: Análisis asintótico de los términos}

Para \( n \) grande, los términos dominantes del numerador y denominador son:
\[
n^3 + 1 \sim n^3, \quad n^5 + 4n^3 - 1 \sim n^5
\]

Por tanto, para \( n \) suficientemente grande,
\[
\frac{n^3 + 1}{n^5 + 4n^3 - 1} \sim \frac{n^3}{n^5} = \frac{1}{n^2}
\]

\subsection*{Paso 2: Definimos la serie de comparación}

Tomamos como serie de comparación:
\[
b_n = \frac{1}{n^2}
\quad \text{(serie p con \( p = 2 > 1 \), converge)}
\]

\subsection*{Paso 3: Verificamos la desigualdad}

Para \( n \geq 1 \), se cumple que:
\[
\frac{n^3 + 1}{n^5 + 4n^3 - 1} \leq \frac{2n^3}{n^5} = \frac{2}{n^2}
\]

Esto es válido ya que:
\begin{itemize}
  \item \( n^3 + 1 \leq 2n^3 \),
  \item \( n^5 + 4n^3 - 1 \geq n^5 \)
\end{itemize}

\subsection*{Paso 4: Aplicar el criterio de comparación directa}

Como
\[
0 \leq \frac{n^3 + 1}{n^5 + 4n^3 - 1} \leq \frac{2}{n^2}
\quad \text{y} \quad \sum_{n=1}^{\infty} \frac{1}{n^2} \text{ converge},
\]
entonces, por el \textbf{criterio de comparación directa}, la serie
\[
\sum_{n=1}^{\infty} \frac{n^3 + 1}{n^5 + 4n^3 - 1}
\]
también \textbf{converge}.


\section*{Ejemplo: Criterio del cociente de límites}

Analicemos la convergencia de la serie:
\[
\sum_{n=1}^{\infty} \frac{\sqrt{4n + 1}}{2n^2 + \sqrt[3]{n} + 2}
\]

\subsection*{Paso 1: Análisis asintótico del término general}

Para \( n \to \infty \), se tiene:
\[
\sqrt{4n + 1} \sim \sqrt{4n} = 2\sqrt{n}, \quad
2n^2 + \sqrt[3]{n} + 2 \sim 2n^2
\]

Entonces:
\[
a_n = \frac{\sqrt{4n + 1}}{2n^2 + \sqrt[3]{n} + 2}
\sim \frac{2\sqrt{n}}{2n^2} = \frac{1}{n^{3/2}}
\]

\subsection*{Paso 2: Definir la serie de comparación}

Tomamos como serie de comparación:
\[
b_n = \frac{1}{n^{3/2}}
\quad \text{(serie p con } p = \frac{3}{2} > 1, \text{ converge)}
\]

\subsection*{Paso 3: Aplicar el criterio del límite}

Calculamos:
\[
\lim_{n \to \infty} \frac{a_n}{b_n}
= \lim_{n \to \infty}
\frac{\dfrac{\sqrt{4n + 1}}{2n^2 + \sqrt[3]{n} + 2}}{\dfrac{1}{n^{3/2}}}
= \lim_{n \to \infty}
\frac{\sqrt{4n + 1} \cdot n^{3/2}}{2n^2 + \sqrt[3]{n} + 2}
\]

Como:
\[
\sqrt{4n + 1} \sim 2\sqrt{n}, \quad \text{entonces:} \quad
\sqrt{4n + 1} \cdot n^{3/2} \sim 2n^2
\]

Y el denominador:
\[
2n^2 + \sqrt[3]{n} + 2 \sim 2n^2
\]

Por lo tanto:
\[
\lim_{n \to \infty} \frac{a_n}{b_n} = \lim_{n \to \infty} \frac{2n^2}{2n^2} = 1
\]

\subsection*{Paso 4: Conclusión}

El límite es:
\[
L = \lim_{n \to \infty} \frac{a_n}{b_n} = 1 \neq 0
\]

Como \( L \in (0, \infty) \) y \( \sum b_n = \sum \frac{1}{n^{3/2}} \) es convergente, entonces por el \textbf{criterio del límite de cocientes}, la serie
\[
\sum_{n=1}^{\infty} \frac{\sqrt{4n + 1}}{2n^2 + \sqrt[3]{n} + 2}
\]
también \textbf{converge}.
\section*{Resumen: Criterio del cociente de límites}

Sean \( a_n > 0 \) y \( b_n > 0 \) para \( n \) suficientemente grande. Definimos:

\[
L = \lim_{n \to \infty} \frac{a_n}{b_n}
\]

Entonces se tienen los siguientes casos:

\begin{enumerate}
    \item \textbf{Si } \( 0 < L < \infty \), ambas series tienen el mismo comportamiento:
    \begin{itemize}
        \item Si \( \sum b_n \) converge, entonces \( \sum a_n \) converge.
        \item Si \( \sum b_n \) diverge, entonces \( \sum a_n \) diverge.
    \end{itemize}

    \item \textbf{Si } \( L = 0 \):
    \begin{itemize}
        \item Si \( \sum b_n \) converge, entonces \( \sum a_n \) también converge.
        \item Si \( \sum b_n \) diverge, \textbf{no se puede concluir nada} sobre \( \sum a_n \).
    \end{itemize}

    \item \textbf{Si } \( L = +\infty \):
    \begin{itemize}
        \item Si \( \sum b_n \) diverge, entonces \( \sum a_n \) también diverge.
        \item Si \( \sum b_n \) converge, \textbf{no se puede concluir nada} sobre \( \sum a_n \).
    \end{itemize}
\end{enumerate}

\textbf{Nota:} Este criterio es útil cuando \( a_n \) y \( b_n \) son difíciles de comparar directamente pero tienen comportamiento asintótico similar.
\section*{Ejemplo: Criterio de Leibniz para series alternadas}

Considere la serie:
\[
\sum_{n=1}^{\infty} \frac{(-1)^n}{n}
\]

Esta es una \textbf{serie alternada}, ya que los signos de los términos cambian de forma alterna: positivo, negativo, positivo, etc.

Sea \( a_n = \dfrac{1}{n} \). Aplicamos el \textbf{criterio de Leibniz} (criterio de las series alternadas), que establece que la serie
\[
\sum_{n=1}^{\infty} (-1)^n a_n
\]
converge si se cumplen:

\begin{enumerate}
    \item \( \displaystyle \lim_{n \to \infty} a_n = 0 \)
    
    En este caso:
    \[
    \lim_{n \to \infty} \frac{1}{n} = 0
    \]

    \item La sucesión \( a_n \) es decreciente:

    Como:
    \[
    \frac{1}{n+1} < \frac{1}{n} \quad \text{para todo } n \geq 1,
    \]
    se concluye que \( a_n \) es decreciente.
\end{enumerate}

\subsection*{Conclusión:}

Como se cumplen ambas condiciones del criterio de Leibniz, se concluye que la serie

\[
\sum_{n=1}^{\infty} \frac{(-1)^n}{n}
\]

es \textbf{convergente}.


\section*{Ejemplo: Criterio de Leibniz para series alternadas}

Considere la serie:
\[
\sum_{n=1}^{\infty} (-1)^n \sin\left(\frac{1}{n}\right)
\]

Esta es una \textbf{serie alternada}, ya que los signos de los términos cambian de forma alterna.

Definimos:
\[
a_n = \sin\left(\frac{1}{n}\right)
\]

Aplicamos el \textbf{criterio de Leibniz}, que establece que la serie
\[
\sum (-1)^n a_n
\]
converge si se cumplen las siguientes condiciones:

\begin{enumerate}
    \item \textbf{Límite nulo:}
    \[
    \lim_{n \to \infty} \sin\left(\frac{1}{n}\right) = \sin(0) = 0
    \]

    \item \textbf{Monotonía decreciente:} Como la función seno es creciente en \( (0, \pi/2) \) y \( \frac{1}{n} \) es decreciente, entonces:
    \[
    \frac{1}{n+1} < \frac{1}{n} \quad \Rightarrow \quad \sin\left(\frac{1}{n+1}\right) < \sin\left(\frac{1}{n}\right)
    \]
    Por tanto, la sucesión \( a_n = \sin\left(\frac{1}{n}\right) \) es decreciente.
\end{enumerate}

\subsection*{Conclusión:}

Se cumplen ambas condiciones del \textbf{criterio de Leibniz}, por lo tanto, la serie:

\[
\sum_{n=1}^{\infty} (-1)^n \sin\left(\frac{1}{n}\right)
\]

es \textbf{convergente}.


\section*{Resumen: Criterio de Leibniz (series alternadas)}

Una \textbf{serie alternada} tiene la forma:
\[
\sum_{n=1}^{\infty} (-1)^n a_n \quad \text{o} \quad \sum_{n=1}^{\infty} (-1)^{n+1} a_n
\]
donde \( a_n > 0 \).

\subsection*{Condiciones del criterio de Leibniz}

La serie alternada \( \sum (-1)^n a_n \) es \textbf{convergente} si:

\begin{enumerate}
    \item \( \displaystyle \lim_{n \to \infty} a_n = 0 \)
    \item La sucesión \( a_n \) es decreciente: \( a_{n+1} \leq a_n \)
\end{enumerate}

\subsection*{Qué ocurre si no se cumplen las condiciones}

\begin{itemize}
    \item Si \( \lim a_n \neq 0 \), la serie \textbf{diverge}.
    \item Si \( a_n \) no es decreciente, el criterio de Leibniz no puede aplicarse, y se necesita otro método.
\end{itemize}

\subsection*{Observación}

El criterio de Leibniz prueba \textbf{convergencia condicional}, no absoluta.

Por ejemplo:
\[
\sum_{n=1}^{\infty} \frac{(-1)^n}{n} \text{ converge por Leibniz, pero no absolutamente.}
\]
Ya que:
\[
\sum_{n=1}^{\infty} \left| \frac{(-1)^n}{n} \right| = \sum_{n=1}^{\infty} \frac{1}{n}
\]
es la serie armónica, que diverge.

\section*{Ejemplo: Convergencia absoluta o condicional}

Analicemos la convergencia de la serie:
\[
\sum_{n=1}^{\infty} \frac{\cos(n)}{n^2 + 1}
\]

\subsection*{Paso 1: Definir el término general}

Sea:
\[
a_n = \frac{\cos(n)}{n^2 + 1}
\]

Esta es una serie **no alternada** pero **no positiva**, porque \( \cos(n) \) oscila entre \(-1\) y \(1\), cambiando de signo.

\subsection*{Paso 2: Estudiar convergencia absoluta}

Analizamos la serie con valor absoluto:
\[
\sum_{n=1}^{\infty} \left| \frac{\cos(n)}{n^2 + 1} \right| = \sum_{n=1}^{\infty} \frac{|\cos(n)|}{n^2 + 1}
\]

Sabemos que \( |\cos(n)| \leq 1 \), entonces:
\[
\frac{|\cos(n)|}{n^2 + 1} \leq \frac{1}{n^2}
\]

Como la serie \( \sum_{n=1}^{\infty} \frac{1}{n^2} \) es una serie \( p \)-con \( p = 2 > 1 \), **converge**.

Por el \textbf{criterio de comparación directa}, la serie con valor absoluto también converge.

\subsection*{Paso 3: Conclusión}

Como:
\[
\sum_{n=1}^{\infty} \left| \frac{\cos(n)}{n^2 + 1} \right| \text{ converge},
\]
entonces la serie
\[
\sum_{n=1}^{\infty} \frac{\cos(n)}{n^2 + 1}
\]
\textbf{converge absolutamente}.

\section*{Ejemplo: Convergencia condicional vs absoluta}

Sea \( a_n = \dfrac{1}{\sqrt{n}} \)

\begin{enumerate}
    \item[\textbf{1.}] \textbf{Muestre que \( a_n \) es decreciente}

    Para \( n \in \mathbb{N} \), se tiene que:
    \[
    a_n = \frac{1}{\sqrt{n}}, \quad a_{n+1} = \frac{1}{\sqrt{n+1}}
    \]

    Como \( \sqrt{n+1} > \sqrt{n} \Rightarrow \frac{1}{\sqrt{n+1}} < \frac{1}{\sqrt{n}} \), entonces \( a_n \) es decreciente.

    \item[\textbf{2.}] \textbf{Calcule el límite de \( a_n \):}
    \[
    \lim_{n \to \infty} \frac{1}{\sqrt{n}} = 0
    \]

    \item[\textbf{3.}] \textbf{Justifique por qué la serie alternada}
    \[
    \sum_{n=1}^{\infty} (-1)^n \frac{1}{\sqrt{n}}
    \]
    \textbf{es convergente}

    Aplicamos el \textbf{criterio de Leibniz}:
    \begin{itemize}
        \item \( a_n \to 0 \) 
        \item \( a_n \) decrece 
    \end{itemize}

    Por tanto, la serie **converge condicionalmente**.

    \item[\textbf{4.}] \textbf{Justifique por qué la serie}
    \[
    \sum_{n=1}^{\infty} \frac{1}{\sqrt{n}}
    \quad \textbf{diverge}
    \]

    Es una serie tipo \( p \)-serie con \( p = \frac{1}{2} < 1 \), por lo tanto, **diverge**.

    \item[\textbf{5.}] \textbf{¿Qué tipo de convergencia presenta la serie en (3)?}

    Como:
    \[
    \sum_{n=1}^{\infty} (-1)^n \frac{1}{\sqrt{n}} \text{ converge}, \quad
    \sum_{n=1}^{\infty} \left| \frac{(-1)^n}{\sqrt{n}} \right| = \sum_{n=1}^{\infty} \frac{1}{\sqrt{n}} \text{ diverge},
    \]
    entonces, la serie (3) **converge condicionalmente**.
\end{enumerate}
\section*{Ejemplo: Criterio de la razón (D'Alembert)}

Consideremos la serie:
\[
\sum_{n=1}^{\infty} \frac{n!}{p(p+1)(p+2)\cdots(p+n-1)}
\quad \text{con } p > 2
\]

Definimos:
\[
a_n = \frac{n!}{p(p+1)(p+2)\cdots(p+n-1)}
\]

\subsection*{Paso 1: Aplicar el criterio de la razón}

Calculamos:
\[
L = \lim_{n \to \infty} \left| \frac{a_{n+1}}{a_n} \right|
\]

Tenemos:
\[
a_{n+1} = \frac{(n+1)!}{p(p+1)\cdots(p+n)} \\
a_n = \frac{n!}{p(p+1)\cdots(p+n-1)}
\]

Entonces:

\[
a_n = \frac{n!}{\prod_{k=0}^{n-1} (p + k)} = \frac{n!}{p(p+1)(p+2)\cdots(p+n-1)}
\]

\[
a_{n+1} = \frac{(n+1)!}{\prod_{k=0}^{n} (p + k)} = \frac{(n+1)!}{p(p+1)(p+2)\cdots(p+n)}
\]

Entonces:

\[
\frac{a_{n+1}}{a_n}
= \frac{(n+1)!}{p(p+1)\cdots(p+n)} \cdot \frac{p(p+1)\cdots(p+n-1)}{n!}
\]

Descomponemos \( (n+1)! = (n+1) \cdot n! \):

\[
= \frac{(n+1) \cdot n!}{p(p+1)\cdots(p+n)} \cdot \frac{p(p+1)\cdots(p+n-1)}{n!}
\]

Cancelamos \( n! \) y los productos comunes:

\[
= (n+1) \cdot \frac{1}{p+n}
= \frac{n+1}{p+n}
\]

Por tanto:
\[
L = \lim_{n \to \infty} \frac{n+1}{p+n}
= \lim_{n \to \infty} \frac{1 + \frac{1}{n}}{1 + \frac{p}{n}} = 1
\]

\subsection*{Paso 2: El criterio de la razón no decide}

Como:
\[
L = 1,
\]
el criterio de D’Alembert no permite concluir nada sobre la convergencia.

\subsection*{Paso 3: Usar comparación con una serie p-convergente}

Notamos que:
\[
\frac{n!}{p(p+1)\cdots(p+n-1)} = \prod_{k=0}^{n-1} \frac{n-k}{p+k}
\]

Cada fracción del producto es menor que 1 para \( p > n \), pero cuando \( p > 2 \), se puede usar la desigualdad:
\[
a_n \leq \frac{n!}{p^n}
\]

Luego aplicamos el criterio de comparación con:
\[
b_n = \frac{n!}{p^n}
\]

Y con el criterio de la razón sobre \( b_n \):
\[
\frac{b_{n+1}}{b_n} = \frac{(n+1)! / p^{n+1}}{n! / p^n} = \frac{n+1}{p}
\]

Entonces:
\[
\lim_{n \to \infty} \frac{b_{n+1}}{b_n} = \infty \text{ si } p < \infty
\]

Pero como \( \frac{n!}{p^n} \) decrece rápidamente, la serie:
\[
\sum \frac{n!}{p^n}
\]
converge (esto se puede justificar por otros criterios como la razón).

\subsection*{Conclusión}

La serie original:
\[
\sum_{n=1}^{\infty} \frac{n!}{p(p+1)\cdots(p+n-1)}
\]
\textbf{converge absolutamente} si \( p > 2 \).
\section*{Ejemplo: Aplicación del criterio de Raabe}

Analicemos la convergencia de la serie:
\[
\sum_{n=1}^{\infty} \frac{2^n n!}{n^n}
\]

Definimos el término general:
\[
a_n = \frac{2^n n!}{n^n}
\]

Aplicamos el \textbf{criterio de Raabe}, que establece que:

\[
L = \lim_{n \to \infty} n \cdot \left(1 - \left| \frac{a_{n+1}}{a_n} \right| \right)
\]

\subsection*{Paso 1: Calcular \( \dfrac{a_{n+1}}{a_n} \)}

\[
a_{n+1} = \frac{2^{n+1} (n+1)!}{(n+1)^{n+1}}, \quad a_n = \frac{2^n n!}{n^n}
\]

Entonces:

\[
\frac{a_{n+1}}{a_n}
= \frac{2^{n+1} (n+1)!}{(n+1)^{n+1}} \cdot \frac{n^n}{2^n n!}
= \frac{2 \cdot (n+1) \cdot n^n}{(n+1)^{n+1}}
= \frac{2n^n}{(n+1)^n}
\]

\subsection*{Paso 2: Sustituir en la fórmula de Raabe}

\[
L = \lim_{n \to \infty} n \cdot \left(1 - \frac{2n^n}{(n+1)^n} \right)
\]

Usamos que:
\[
\left( \frac{n}{n+1} \right)^n \longrightarrow \frac{1}{e}
\quad \Rightarrow \quad
\frac{n^n}{(n+1)^n} \longrightarrow \frac{1}{e}
\]

Entonces:
\[
\frac{2n^n}{(n+1)^n} \longrightarrow \frac{2}{e}
\quad \Rightarrow \quad
L = \lim_{n \to \infty} n \cdot \left(1 - \frac{2}{e} \right)
= n \cdot \left( \frac{e - 2}{e} \right)
\]

Como \( \dfrac{e - 2}{e} > 0 \), el límite tiende a infinito:

\[
L = \infty > 1
\]

\subsection*{Conclusión}

Como \( L > 1 \), por el \textbf{criterio de Raabe}, la serie

\[
\sum_{n=1}^{\infty} \frac{2^n n!}{n^n}
\]

\textbf{converge absolutamente}.

\section*{Resumen: Criterio de Raabe}

El \textbf{criterio de Raabe} es una versión más precisa del criterio de D'Alembert. Se aplica a una serie:

\[
\sum_{n=k}^{\infty} a_n
\]

donde \( a_n \neq 0 \) para \( n \) suficientemente grande. Se define el siguiente límite:

\[
L = \lim_{n \to \infty} n \cdot \left(1 - \left| \frac{a_{n+1}}{a_n} \right| \right)
\]

---

\subsection*{Interpretación del resultado:}

\begin{itemize}
    \item[] \textbf{1.} Si \( L > 1 \), entonces la serie \textbf{converge absolutamente}.
    \item[] \textbf{2.} Si \( L < 1 \), entonces la serie \textbf{diverge}.
    \item[] \textbf{3.} Si \( L = 1 \), el criterio es \textbf{inconcluso}.
\end{itemize}

---

\subsection*{Ventaja del criterio}

Este criterio es útil especialmente en series donde aparece el factorial o potencias tipo \( n^n \), donde el criterio de la razón usual da \( L = 1 \) y no permite decidir.

\textbf{Nota:} A veces se llama también \emph{criterio de Raabe-Duhamel}.

\section*{Ejemplo: Aplicación del criterio de Raabe}

Analicemos la convergencia de la serie:
\[
\sum_{n=2}^{\infty} \frac{1}{n \cdot \ln(n)^p}
\quad \text{con } p > 0
\]

(Notamos que el índice empieza en \( n = 2 \) para evitar la indeterminación en \( \ln(1) \)).

Definimos:
\[
a_n = \frac{1}{n \cdot \ln(n)^p}
\]

Aplicamos el \textbf{criterio de Raabe}, calculando:

\[
L = \lim_{n \to \infty} n \cdot \left( 1 - \left| \frac{a_{n+1}}{a_n} \right| \right)
\]

\subsection*{Paso 1: Cociente \( \dfrac{a_{n+1}}{a_n} \)}

\[
\frac{a_{n+1}}{a_n}
= \frac{1}{(n+1) \cdot \ln(n+1)^p} \cdot \frac{n \cdot \ln(n)^p}{1}
= \frac{n}{n+1} \cdot \left( \frac{\ln(n)}{\ln(n+1)} \right)^p
\]

\subsection*{Paso 2: Desarrollo del límite}

Sabemos que:
\[
\frac{n}{n+1} \longrightarrow 1^{-}, \quad \frac{\ln(n)}{\ln(n+1)} \longrightarrow 1^{-}
\]

Entonces:

\[
\left( \frac{\ln(n)}{\ln(n+1)} \right)^p \approx 1 - \frac{p}{n \cdot \ln(n)} + \text{términos menores}
\]

Y:

\[
\frac{n}{n+1} \approx 1 - \frac{1}{n} + \text{términos menores}
\]

Por lo tanto:

\[
\frac{a_{n+1}}{a_n} \approx \left( 1 - \frac{1}{n} \right) \left( 1 - \frac{p}{n \cdot \ln(n)} \right)
\approx 1 - \frac{1}{n} - \frac{p}{n \cdot \ln(n)}
\]

Entonces:

\[
L = n \cdot \left( 1 - \left( 1 - \frac{1}{n} - \frac{p}{n \cdot \ln(n)} \right) \right)
= 1 + \frac{p}{\ln(n)} \longrightarrow 1
\]

\subsection*{Conclusión:}

\[
L = \lim_{n \to \infty} n \cdot \left( 1 - \frac{a_{n+1}}{a_n} \right) = 1
\]

Como \( L = 1 \), el \textbf{criterio de Raabe es inconcluso}.

---

\subsection*{Observación final:}

Este ejemplo es útil para mostrar que incluso el criterio de Raabe puede fallar cuando \( L = 1 \), como también sucede con el criterio de la razón.

Para decidir la convergencia de esta serie se requiere otro enfoque, como el **criterio de la integral**:

- La serie \( \sum \frac{1}{n \ln(n)^p} \) converge si \( p > 1 \), y diverge si \( 0 < p \leq 1 \).

\newpage

\section*{Ejemplo: Criterio de la raíz (Cauchy)}

Analicemos la convergencia de la serie:
\[
\sum_{n=1}^{\infty} \left(1 + \frac{2}{n}\right)^{-n^2}
\]

Sea:
\[
a_n = \left(1 + \frac{2}{n}\right)^{-n^2}
\]

Aplicamos el \textbf{criterio de la raíz}, que consiste en calcular:

\[
L = \lim_{n \to \infty} \sqrt[n]{|a_n|} = \lim_{n \to \infty} \left[\left(1 + \frac{2}{n}\right)^{-n^2}\right]^{1/n}
= \lim_{n \to \infty} \left(1 + \frac{2}{n}\right)^{-n}
\]

Recordamos el límite notable:
\[
\lim_{n \to \infty} \left(1 + \frac{x}{n} \right)^n = e^x
\quad \Rightarrow \quad
\left(1 + \frac{2}{n}\right)^n \to e^2
\]

Entonces:
\[
\left(1 + \frac{2}{n}\right)^{-n} \to \frac{1}{e^2}
\]

Por tanto:
\[
L = \frac{1}{e^2} < 1
\]

\subsection*{Conclusión}

Como:
\[
L < 1
\]

la serie

\[
\sum_{n=1}^{\infty} \left(1 + \frac{2}{n}\right)^{-n^2}
\]

\textbf{converge absolutamente} por el \textbf{criterio de la raíz (Cauchy)}.
\section*{Ejemplo: Aplicación del criterio de la raíz (Cauchy)}

Analicemos la convergencia de la serie:
\[
\sum_{n=1}^{\infty} \left( \frac{5n^2}{3n^2 + 9} \right)^{-n}
\]

Definimos:
\[
a_n = \left( \frac{5n^2}{3n^2 + 9} \right)^{-n}
\]

Aplicamos el \textbf{criterio de la raíz}, que requiere calcular:

\[
L = \lim_{n \to \infty} \sqrt[n]{|a_n|} = \lim_{n \to \infty} \left( \frac{5n^2}{3n^2 + 9} \right)^{-1}
\]

Simplificamos el límite:

\[
L = \lim_{n \to \infty} \left( \frac{5n^2}{3n^2 + 9} \right)^{-1}
= \lim_{n \to \infty} \frac{3n^2 + 9}{5n^2}
= \lim_{n \to \infty} \frac{3 + \frac{9}{n^2}}{5}
= \frac{3}{5}
\]

\subsection*{Conclusión}

Como:
\[
L = \frac{3}{5} < 1,
\]
la serie
\[
\sum_{n=1}^{\infty} \left( \frac{5n^2}{3n^2 + 9} \right)^{-n}
\]
\textbf{converge absolutamente} por el \textbf{criterio de la raíz (Cauchy)}.

\newpage
\section*{Ejemplo: Aplicación de la fórmula de Stirling}

Analicemos la convergencia de la serie:
\[
\sum_{n=2}^{\infty} \frac{n^n}{n! \cdot 5^n}
\]

Sea:
\[
a_n = \frac{n^n}{n! \cdot 5^n}
\]

Aplicamos la \textbf{fórmula de Stirling} para estimar el factorial:
\[
n! \sim \left( \frac{n}{e} \right)^n \sqrt{2\pi n}
\]

Entonces:
\[
a_n \sim \frac{n^n}{\left( \frac{n}{e} \right)^n \sqrt{2\pi n} \cdot 5^n}
= \frac{n^n}{\left( \frac{n^n}{e^n} \right) \cdot \sqrt{2\pi n} \cdot 5^n}
= \frac{e^n}{\sqrt{2\pi n} \cdot 5^n}
\]

Por tanto:
\[
a_n \sim \frac{1}{\sqrt{2\pi n}} \left( \frac{e}{5} \right)^n
\]

\subsection*{Análisis del término general}

Sabemos que \( \frac{e}{5} < 1 \), por lo tanto \( \left( \frac{e}{5} \right)^n \) decrece exponencialmente. Además, el factor \( \frac{1}{\sqrt{n}} \) es decreciente y acotado.

\subsection*{Comparación con serie geométrica}

Como:
\[
a_n \sim \frac{1}{\sqrt{n}} \cdot r^n \quad \text{con } 0 < r = \frac{e}{5} < 1
\]

Entonces:
\[
a_n \leq \frac{C}{\sqrt{n}} \cdot r^n
\]

Y la serie:
\[
\sum_{n=2}^{\infty} \frac{1}{\sqrt{n}} \cdot r^n
\]

\textbf{converge} por el criterio de comparación con una serie geométrica.

\subsection*{Conclusión}

Por lo tanto, la serie
\[
\sum_{n=2}^{\infty} \frac{n^n}{n! \cdot 5^n}
\quad \textbf{converge absolutamente}.
\]

\section*{Resolución de ejercicios — Comparación directa}

\subsection*{Ejercicio (a)}

Analicemos la convergencia de la serie:
\[
\sum_{n=1}^{\infty} \frac{\sin^2(n^3)}{n^4 + 3}
\]

\subsubsection*{Paso 1: Acotación del numerador}

Sabemos que:
\[
0 \leq \sin^2(n^3) \leq 1 \quad \text{para todo } n \in \mathbb{N}
\]

Entonces:
\[
0 \leq \frac{\sin^2(n^3)}{n^4 + 3} \leq \frac{1}{n^4 + 3}
< \frac{1}{n^4}
\quad \text{para todo } n \geq 1
\]

\subsubsection*{Paso 2: Serie de comparación}

Consideramos la serie:
\[
\sum_{n=1}^{\infty} \frac{1}{n^4}
\]

Esta es una serie \( p \)-serie con \( p = 4 > 1 \), por tanto, \textbf{converge}.

\subsubsection*{Paso 3: Aplicación del criterio de comparación directa}

Dado que:
\[
0 \leq \frac{\sin^2(n^3)}{n^4 + 3} \leq \frac{1}{n^4}
\quad \text{y } \sum \frac{1}{n^4} \text{ converge}
\]

Entonces, por el \textbf{criterio de comparación directa}, la serie
\[
\sum_{n=1}^{\infty} \frac{\sin^2(n^3)}{n^4 + 3}
\quad \textbf{converge}.
\]


\subsection*{Ejercicio (c)}

Analice la convergencia de la serie:
\[
\sum_{n=1}^{\infty} \frac{a_n}{10^n}, \quad \text{donde } 0 < a_n < 10
\]

\subsubsection*{Paso 1: Acotación del término general}

Dado que:
\[
0 < a_n < 10,
\quad \text{entonces} \quad
0 < \frac{a_n}{10^n} < \frac{10}{10^n} = \left(\frac{1}{10}\right)^{n - 1}
\]

\subsubsection*{Paso 2: Serie de comparación}

Consideremos la serie:
\[
\sum_{n=1}^{\infty} \frac{10}{10^n}
= 10 \sum_{n=1}^{\infty} \left( \frac{1}{10} \right)^n
\]

Esta es una serie **geométrica** con razón \( r = \frac{1}{10} < 1 \), por lo tanto, \textbf{converge}.

\subsubsection*{Paso 3: Aplicación del criterio de comparación directa}

Como:
\[
0 < \frac{a_n}{10^n} < \frac{10}{10^n}
\quad \text{y } \sum \frac{10}{10^n} \text{ converge},
\]

entonces, por el \textbf{criterio de comparación directa}, la serie:
\[
\sum_{n=1}^{\infty} \frac{a_n}{10^n}
\quad \textbf{converge}.
\]
\section*{Ejercicio 2(a): Criterio de comparación al límite}

Analicemos la convergencia de la serie:
\[
\sum_{n=1}^{\infty} \frac{4n^2 + 5n - 2}{n \sqrt{(n^2 + 1)^3}}
\]

\subsection*{Paso 1: Definir el término general}

Sea:
\[
a_n = \frac{4n^2 + 5n - 2}{n \sqrt{(n^2 + 1)^3}}
\]

\subsection*{Paso 2: Elegir una serie de comparación}

Para \( n \gg 1 \), tenemos:
\[
4n^2 + 5n - 2 \sim 4n^2, \quad \text{y} \quad \sqrt{(n^2 + 1)^3} \sim \sqrt{n^6} = n^3
\]

Entonces:
\[
a_n \sim \frac{4n^2}{n \cdot n^3} = \frac{4n^2}{n^4} = \frac{4}{n^2}
\]

Tomamos como comparación la serie \( b_n = \dfrac{1}{n^2} \), que es una \( p \)-serie con \( p = 2 > 1 \), por lo tanto, \textbf{converge}.

\subsection*{Paso 3: Aplicar el criterio de comparación al límite}

\[
\lim_{n \to \infty} \frac{a_n}{b_n}
= \lim_{n \to \infty} \frac{\dfrac{4n^2 + 5n - 2}{n \sqrt{(n^2 + 1)^3}}}{\dfrac{1}{n^2}}
= \lim_{n \to \infty} \frac{(4n^2 + 5n - 2) \cdot n^2}{n \cdot \sqrt{(n^2 + 1)^3}}
\]

Simplificamos:
\[
= \lim_{n \to \infty} \frac{(4n^2 + 5n - 2) \cdot n^2}{n \cdot \sqrt{n^6 \left(1 + \frac{1}{n^2}\right)^3}}
= \lim_{n \to \infty} \frac{(4n^2 + 5n - 2) \cdot n^2}{n \cdot n^3 \cdot \left(1 + \frac{1}{n^2} \right)^{3/2}}
\]

\[
= \lim_{n \to \infty} \frac{(4n^2 + 5n - 2)}{n^2 \cdot \left(1 + \frac{1}{n^2} \right)^{3/2}} \to \frac{4}{1} = 4
\]

\subsection*{Paso 4: Conclusión}

Como:
\[
\lim_{n \to \infty} \frac{a_n}{b_n} = 4 \neq 0,
\quad \text{y } \sum b_n = \sum \frac{1}{n^2} \text{ converge},
\]

entonces, por el \textbf{criterio de comparación al límite}, la serie
\[
\sum_{n=1}^{\infty} \frac{4n^2 + 5n - 2}{n \sqrt{(n^2 + 1)^3}}
\quad \textbf{converge}.
\]


\subsection*{Ejercicio 2(c): Comparación al límite con parámetros \( a, b > 0 \)}

Analicemos la convergencia de la serie:
\[
\sum_{n=1}^{\infty} \frac{n^{2n}}{(n + a)^n (n + b)^{n + a}}, \quad a, b > 0
\]

\subsubsection*{Paso 1: Análisis asintótico}

Para \( n \gg 1 \), se cumple:
\[
n + a \sim n, \quad n + b \sim n
\]

Entonces:
\[
(n + a)^n \sim n^n, \quad (n + b)^{n + a} \sim n^{n + a}
\]

Por tanto, el término general se comporta asintóticamente como:
\[
a_n \sim \frac{n^{2n}}{n^n \cdot n^{n + a}} = \frac{n^{2n}}{n^{2n + a}} = \frac{1}{n^a}
\]

\subsubsection*{Paso 2: Serie de comparación}

Definimos la serie comparativa:
\[
b_n = \frac{1}{n^a}
\]

Es una \( p \)-serie que:

- **Converge** si \( a > 1 \)
- **Diverge** si \( 0 < a \leq 1 \)

\subsubsection*{Paso 3: Aplicar el criterio de comparación al límite}

Calculamos:
\[
\lim_{n \to \infty} \frac{a_n}{b_n} = \lim_{n \to \infty} \frac{\dfrac{n^{2n}}{(n + a)^n (n + b)^{n + a}}}{\dfrac{1}{n^a}} 
= \lim_{n \to \infty} \frac{n^{2n + a}}{(n + a)^n (n + b)^{n + a}}
\]

Como:
\[
(n + a)^n \sim n^n, \quad (n + b)^{n + a} \sim n^{n + a}
\]

Entonces:
\[
\lim_{n \to \infty} \frac{n^{2n + a}}{n^n \cdot n^{n + a}} = \lim_{n \to \infty} \frac{n^{2n + a}}{n^{2n + a}} = 1
\]

\subsubsection*{Paso 4: Conclusión}

Como:
\[
\lim_{n \to \infty} \frac{a_n}{b_n} = 1 \neq 0, \infty,
\quad \text{y la serie } \sum b_n = \sum \frac{1}{n^a} \text{ converge si } a > 1,
\]

entonces, por el \textbf{criterio de comparación al límite}, la serie
\[
\sum_{n=1}^{\infty} \frac{n^{2n}}{(n + a)^n (n + b)^{n + a}}
\quad \textbf{converge si } a > 1.
\]

\textbf{Diverge si } \( 0 < a \leq 1 \).

\newpage
\subsection*{Ejercicio 2(c): Comparación al límite con parámetros \( a, b > 0 \)}

Analicemos la convergencia de la serie:
\[
\sum_{n=1}^{\infty} \frac{n^{2n}}{(n + a)^n (n + b)^{n + a}}, \quad a, b > 0
\]

\subsubsection*{Paso 1: Análisis asintótico}

Para \( n \gg 1 \), se cumple:
\[
n + a \sim n, \quad n + b \sim n
\]

Entonces:
\[
(n + a)^n \sim n^n, \quad (n + b)^{n + a} \sim n^{n + a}
\]

Por tanto, el término general se comporta asintóticamente como:
\[
a_n \sim \frac{n^{2n}}{n^n \cdot n^{n + a}} = \frac{n^{2n}}{n^{2n + a}} = \frac{1}{n^a}
\]

\subsubsection*{Paso 2: Serie de comparación}

Definimos la serie comparativa:
\[
b_n = \frac{1}{n^a}
\]

Es una \( p \)-serie que:

- **Converge** si \( a > 1 \)
- **Diverge** si \( 0 < a \leq 1 \)

\subsubsection*{Paso 3: Aplicar el criterio de comparación al límite}

Calculamos:
\[
\lim_{n \to \infty} \frac{a_n}{b_n} = \lim_{n \to \infty} \frac{\dfrac{n^{2n}}{(n + a)^n (n + b)^{n + a}}}{\dfrac{1}{n^a}} 
= \lim_{n \to \infty} \frac{n^{2n + a}}{(n + a)^n (n + b)^{n + a}}
\]

Como:
\[
(n + a)^n \sim n^n, \quad (n + b)^{n + a} \sim n^{n + a}
\]

Entonces:
\[
\lim_{n \to \infty} \frac{n^{2n + a}}{n^n \cdot n^{n + a}} = \lim_{n \to \infty} \frac{n^{2n + a}}{n^{2n + a}} = 1
\]

\subsubsection*{Paso 4: Conclusión}

Como:
\[
\lim_{n \to \infty} \frac{a_n}{b_n} = 1 \neq 0, \infty,
\quad \text{y la serie } \sum b_n = \sum \frac{1}{n^a} \text{ converge si } a > 1,
\]

entonces, por el \textbf{criterio de comparación al límite}, la serie
\[
\sum_{n=1}^{\infty} \frac{n^{2n}}{(n + a)^n (n + b)^{n + a}}
\quad \textbf{converge si } a > 1.
\]

\textbf{Diverge si } \( 0 < a \leq 1 \).


\section*{Ejercicio 3(b): Criterio de la integral}

Analicemos la convergencia de la serie:
\[
\sum_{n=2}^{\infty} \frac{1}{n \ln(n)(\ln \ln n)^p}, \quad \text{con } p > 0
\]

Definimos:
\[
f(x) = \frac{1}{x \ln(x)(\ln \ln x)^p}, \quad x \geq 2
\]

Esta función es continua, positiva y decreciente para \( x > e \), por lo tanto, podemos aplicar el \textbf{criterio de la integral}.

\subsection*{Paso 1: Aplicamos el criterio de la integral}

Estudiamos la convergencia de:
\[
\int_2^{\infty} \frac{1}{x \ln(x)(\ln \ln x)^p} \, dx
\]

Usamos el cambio de variable:
\[
u = \ln \ln x, \quad \text{entonces: } du = \frac{1}{\ln x} \cdot \frac{1}{x} \, dx
\Rightarrow dx = x \ln x \, du
\]

Sustituyendo:

\[
\int_2^{\infty} \frac{1}{x \ln x (\ln \ln x)^p} \, dx 
= \int_{\ln \ln 2}^{\infty} \frac{1}{u^p} \, du
\]

\subsection*{Paso 2: Evaluamos la integral resultante}

La integral:
\[
\int_{\ln \ln 2}^{\infty} \frac{1}{u^p} \, du
\]

- **Converge** si \( p > 1 \)
- **Diverge** si \( 0 < p \leq 1 \)

\subsection*{Conclusión}

Por el \textbf{criterio de la integral}, la serie:
\[
\sum_{n=2}^{\infty} \frac{1}{n \ln(n)(\ln \ln n)^p}
\]

\textbf{converge si} \( p > 1 \)  
\textbf{diverge si} \( 0 < p \leq 1 \)


\subsection*{Ejercicio 3(c): Criterio de la integral}

Analicemos la convergencia de la serie:
\[
\sum_{n=1}^{\infty} \frac{e^{\arctan n}}{n^2 + 1}
\]

Definimos la función:
\[
f(x) = \frac{e^{\arctan x}}{x^2 + 1}, \quad x \geq 1
\]

La función es continua, positiva y decreciente en \( [1, \infty) \), ya que:

- \( \arctan x \) es creciente y acotada: \( \arctan x < \frac{\pi}{2} \), por lo que \( e^{\arctan x} \) también es creciente y acotada por \( e^{\pi/2} \),
- \( x^2 + 1 \) crece más rápido que \( e^{\arctan x} \), haciendo que \( f(x) \) decrezca.

Entonces, podemos aplicar el **criterio de la integral**.

\subsubsection*{Paso 1: Acotación de la función}

Sabemos que:
\[
\arctan x < \frac{\pi}{2} \Rightarrow e^{\arctan x} < e^{\pi/2}
\Rightarrow f(x) < \frac{e^{\pi/2}}{x^2 + 1}
\]

Y además:
\[
\frac{1}{x^2 + 1} < \frac{1}{x^2}, \quad \text{para } x \geq 1
\]

Entonces:
\[
f(x) < \frac{e^{\pi/2}}{x^2}
\]

\subsubsection*{Paso 2: Comparación con una integral conocida}

Como:
\[
\int_1^{\infty} \frac{1}{x^2} \, dx = 1 < \infty
\]

entonces:
\[
\int_1^{\infty} f(x) \, dx \leq \int_1^{\infty} \frac{e^{\pi/2}}{x^2} \, dx = e^{\pi/2} < \infty
\]

\subsubsection*{Conclusión}

Por el **criterio de la integral**, la serie:
\[
\sum_{n=1}^{\infty} \frac{e^{\arctan n}}{n^2 + 1}
\quad \textbf{converge}.
\]



\section*{Ejercicio 4(c): Criterio de series alternadas (criterio de Leibniz)}

Analicemos la convergencia de la serie:
\[
\sum_{n=1}^{\infty} \frac{\sin\left(\dfrac{(2n - 1)\pi}{2}\right)}{n \ln(n + 3)}
\]

\subsection*{Paso 1: Analizar el signo del numerador}

Recordemos que:
\[
\sin\left( \frac{(2n - 1)\pi}{2} \right) =
\begin{cases}
1 & \text{si } n = 1, 5, 9, \ldots \ (\text{n} \equiv 1 \mod 4) \\
-1 & \text{si } n = 3, 7, 11, \ldots \ (\text{n} \equiv 3 \mod 4) \\
0 & \text{si } n \equiv 0,2 \mod 4 \Rightarrow \text{¡Nunca sucede!}
\end{cases}
\]

Entonces, para todo \( n \), el numerador vale \( \pm 1 \), alternando signos. Por tanto, la serie es alternante:

\[
a_n = \frac{(-1)^{n+1}}{n \ln(n + 3)}
\quad \text{o equivalente}.
\]

\subsection*{Paso 2: Verificamos las condiciones del criterio de Leibniz}

Queremos ver si \( b_n = \dfrac{1}{n \ln(n + 3)} \) cumple:

\begin{itemize}
    \item \textbf{(i) \( b_n > 0 \)} 
    \item \textbf{(ii) Decreciente:}

    Calculamos la derivada de \( f(x) = \dfrac{1}{x \ln(x + 3)} \), que resulta negativa para \( x > 1 \), entonces la sucesión es decreciente para \( n \geq 1 \). 

    \item \textbf{(iii) \( \lim_{n \to \infty} b_n = 0 \):}

    \[
    \lim_{n \to \infty} \frac{1}{n \ln(n + 3)} = 0 \quad 
    \]
\end{itemize}

\subsection*{Conclusión}

La serie:
\[
\sum_{n=1}^{\infty} \frac{\sin\left(\dfrac{(2n - 1)\pi}{2}\right)}{n \ln(n + 3)}
\quad \textbf{converge} \text{ por el criterio de Leibniz (series alternadas).}
 \]


\subsection*{Cálculo del número de términos para una precisión de \( 10^{-5} \)}

Queremos encontrar el menor \( N \in \mathbb{N} \) tal que:
\[
\left| R_N \right| \leq \frac{1}{(N+1) \ln(N+4)} < 10^{-5}
\]

Probamos por tanteo:

\begin{itemize}
    \item \( N = 500 \): \( \frac{1}{501 \cdot \ln(504)} \approx \frac{1}{501 \cdot 6,22} \approx 3,2 \times 10^{-4} \)
    \item \( N = 1500 \): \( \frac{1}{1501 \cdot \ln(1504)} \approx \frac{1}{1501 \cdot 7,32} \approx 9,1 \times 10^{-5} \)
    \item \( N = 2500 \): \( \frac{1}{2501 \cdot \ln(2504)} \approx \frac{1}{2501 \cdot 7,83} \approx 5,1 \times 10^{-5} \)
    \item \( N = 3500 \): \( \frac{1}{3501 \cdot \ln(3504)} \approx \frac{1}{3501 \cdot 8,16} \approx 3,5 \times 10^{-5} \)
    \item \( N = 5000 \): \( \frac{1}{5001 \cdot \ln(5004)} \approx \frac{1}{5001 \cdot 8,52} \approx 2,3 \times 10^{-5} \)
    \item \( N = 7000 \): \( \frac{1}{7001 \cdot \ln(7004)} \approx 1,5 \times 10^{-5} \)
    \item \( N = 9000 \): \( \frac{1}{9001 \cdot \ln(9004)} \approx 1,1 \times 10^{-5} \)
    \item \( N = 10000 \): \( \frac{1}{10001 \cdot \ln(10004)} \approx 9,9 \times 10^{-6} \)
\end{itemize}

\subsubsection*{Conclusión}

Para que el error sea menor a \( 10^{-5} \), se deben tomar al menos:
\[
\boxed{10\,000 \text{ términos.}}
\]

---

\subsection*{Cota del error con suma parcial de los primeros 5 términos}

Usamos:
\[
R_5 \leq b_6 = \frac{1}{6 \cdot \ln(9)} \approx \frac{1}{6 \cdot 2,20} \approx \boxed{0,0757}
\]

Entonces, si sumamos los primeros 5 términos, la cota del error es:
\[
\boxed{R_5 \leq 0,0757}
\]




\subsection*{Ejercicio 4(b): Criterio de series alternadas (criterio de Leibniz)}

Analicemos la serie:
\[
\sum_{n=1}^{\infty} \frac{(-1)^n}{\ln(n+1)}
\]

Sea \( b_n = \frac{1}{\ln(n+1)} \). Verificamos las condiciones del criterio de Leibniz:

\begin{itemize}
    \item \textbf{(i) Positividad:} \( b_n > 0 \) para todo \( n \geq 1 \). 
    \item \textbf{(ii) Monotonía:} \( b_n \) es decreciente porque \( \ln(n+1) \) es creciente. 
    \item \textbf{(iii) Límite:}
    \[
    \lim_{n \to \infty} \frac{1}{\ln(n+1)} = 0 \quad 
    \]
\end{itemize}

\subsubsection*{Conclusión}

La serie
\[
\sum_{n=1}^{\infty} \frac{(-1)^n}{\ln(n+1)} \quad \textbf{converge} \text{ por el criterio de Leibniz.}
\]

\subsection*{Ejercicio 5(b): Criterio de la raíz n-ésima}

Consideramos la serie:
\[
\sum_{n=1}^{\infty} \left( \frac{n!}{(n+2)!} \right)^{n^2}
\]

\subsubsection*{Paso 1: Simplificamos el término general}

Recordamos que:
\[
(n+2)! = (n+2)(n+1)n!
\quad \Rightarrow \quad
\frac{n!}{(n+2)!} = \frac{n!}{(n+2)(n+1)n!} = \frac{1}{(n+1)(n+2)}
\]

Entonces el término general queda:
\[
a_n = \left( \frac{1}{(n+1)(n+2)} \right)^{n^2}
\]

\subsubsection*{Paso 2: Aplicamos el criterio de la raíz n-ésima}

Recordamos que si:
\[
L = \lim_{n \to \infty} \sqrt[n]{|a_n|} < 1 \Rightarrow \text{la serie converge absolutamente}
\]

Entonces:
\[
\sqrt[n]{a_n} = \left[ \left( \frac{1}{(n+1)(n+2)} \right)^{n^2} \right]^{1/n}
= \left( \frac{1}{(n+1)(n+2)} \right)^n
\]

Ahora aplicamos logaritmos para evaluar el límite:
\[
\ln\left( \left( \frac{1}{(n+1)(n+2)} \right)^n \right)
= -n \ln[(n+1)(n+2)] = -n \ln(n^2 + 3n + 2)
\]

Así que:
\[
\lim_{n \to \infty} \left( \frac{1}{(n+1)(n+2)} \right)^n
= \lim_{n \to \infty} e^{-n \ln(n^2 + 3n + 2)} = 0
\]

\subsubsection*{Conclusión}

Como:
\[
\lim_{n \to \infty} \sqrt[n]{a_n} = 0 < 1
\]

entonces, por el \textbf{criterio de la raíz \( n \)-ésima}, la serie:
\[
\sum_{n=1}^{\infty} \left( \frac{n!}{(n+2)!} \right)^{n^2}
\quad \textbf{converge absolutamente.}
\]

\subsection*{Ejercicio 6(a): Criterio de la razón (D’Alembert)}

Analizamos la serie:
\[
\sum_{n=1}^{\infty} \frac{2 \cdot 4 \cdot \cdots \cdot (2n)}{5 \cdot 7 \cdot \cdots \cdot (2n + 3)} \cdot \frac{a^n}{\pi^n}
\]

\subsubsection*{Paso 1: Notación simplificada}

Observamos que:
- El numerador es el producto de los primeros \( n \) números pares:  
\[
2 \cdot 4 \cdot \cdots \cdot 2n = 2^n \cdot n!
\]
- El denominador es el producto de los impares desde 5 hasta \( 2n+3 \), es decir, los últimos \( n+1 \) impares a partir de 5. Lo escribimos como:
\[
\prod_{k=1}^{n+1} (2k + 3) = D_n
\]

Por tanto, el término general es:
\[
a_n = \frac{2^n n!}{D_n} \cdot \left( \frac{a}{\pi} \right)^n
\]

\subsubsection*{Paso 2: Aplicamos el criterio de la razón}

Usamos:
\[
L = \lim_{n \to \infty} \left| \frac{a_{n+1}}{a_n} \right|
\]

Calculamos:
\[
a_{n+1} = \frac{2^{n+1} (n+1)!}{D_{n+1}} \cdot \left( \frac{a}{\pi} \right)^{n+1}, \quad
a_n = \frac{2^n n!}{D_n} \cdot \left( \frac{a}{\pi} \right)^n
\]

Entonces:
\[
\left| \frac{a_{n+1}}{a_n} \right|
= \frac{2^{n+1} (n+1)!}{D_{n+1}} \cdot \frac{D_n}{2^n n!} \cdot \frac{a}{\pi}
= \frac{2 \cdot (n+1)}{d_{n+1}} \cdot \frac{a}{\pi}
\]

donde \( d_{n+1} = \frac{D_{n+1}}{D_n} = 2(n+2) + 3 = 2n + 7 \), el nuevo término que se agrega en el denominador.

\[
\Rightarrow \left| \frac{a_{n+1}}{a_n} \right| = \frac{2(n+1)}{2n + 7} \cdot \frac{a}{\pi}
\]

\subsubsection*{Paso 3: Límite final}

\[
L = \lim_{n \to \infty} \left| \frac{a_{n+1}}{a_n} \right| = \lim_{n \to \infty} \frac{2(n+1)}{2n+7} \cdot \frac{a}{\pi} = \frac{2}{2} \cdot \frac{a}{\pi} = \frac{a}{\pi}
\]

\subsubsection*{Conclusión}

Por el criterio de la razón:

- La serie \textbf{converge absolutamente} si \( \displaystyle \frac{a}{\pi} < 1 \Rightarrow \boxed{a < \pi} \)
- La serie \textbf{diverge} si \( \displaystyle \frac{a}{\pi} > 1 \Rightarrow \boxed{a > \pi} \)
- El criterio es \textbf{inconcluso} si \( a = \pi \)

\subsection*{Ejercicio 6(b): Criterio de la razón (D’Alembert)}

Estudiamos la convergencia de la serie:
\[
\sum_{n=1}^{\infty} \frac{(2n)!}{(n!)^2} \cdot \left(\frac{1}{4}\right)^n
\]

Observamos que el término general es:
\[
a_n = \frac{(2n)!}{(n!)^2} \cdot \left(\frac{1}{4}\right)^n
\]

Esta expresión aparece con frecuencia en combinatoria: 
\[
\frac{(2n)!}{(n!)^2} = \binom{2n}{n}
\quad \Rightarrow \quad
a_n = \binom{2n}{n} \cdot \left(\frac{1}{4}\right)^n
\]

\subsubsection*{Paso 1: Aplicamos el criterio de la razón}

Calculamos:
\[
L = \lim_{n \to \infty} \left| \frac{a_{n+1}}{a_n} \right|
= \lim_{n \to \infty} \frac{\binom{2(n+1)}{n+1} \cdot \left( \frac{1}{4} \right)^{n+1}}{\binom{2n}{n} \cdot \left( \frac{1}{4} \right)^n}
= \lim_{n \to \infty} \frac{\binom{2n + 2}{n+1}}{\binom{2n}{n}} \cdot \frac{1}{4}
\]

\subsubsection*{Paso 2: Usamos una aproximación conocida}

Existe la siguiente relación asintótica para grandes \( n \):

\[
\binom{2n}{n} \sim \frac{4^n}{\sqrt{\pi n}}
\quad \Rightarrow \quad
a_n \sim \frac{4^n}{\sqrt{\pi n}} \cdot \left( \frac{1}{4} \right)^n = \frac{1}{\sqrt{\pi n}}
\]

Por tanto,
\[
a_n \sim \frac{1}{\sqrt{\pi n}} \quad \Rightarrow \quad a_n \not\to 0 \text{ rápidamente, pero sí } \to 0
\]

Ahora sí, usando el criterio de la razón más directamente, y expandiendo:

\[
\binom{2n}{n} = \frac{(2n)!}{(n!)^2}, \quad
\binom{2n+2}{n+1} = \frac{(2n+2)!}{[(n+1)!]^2}
\]

Entonces:
\[
\left| \frac{a_{n+1}}{a_n} \right| = \left( \frac{(2n+2)!}{[(n+1)!]^2} \cdot \frac{(n!)^2}{(2n)!} \cdot \frac{1}{4} \right)
= \frac{(2n+2)(2n+1)}{(n+1)^2} \cdot \frac{1}{4}
\]

\subsubsection*{Paso 3: Calculamos el límite}

\[
L = \lim_{n \to \infty} \frac{(2n+2)(2n+1)}{4(n+1)^2}
= \lim_{n \to \infty} \frac{4n^2 + 6n + 2}{4(n^2 + 2n + 1)}
= \frac{4}{4} = 1
\]

\subsubsection*{Conclusión}

Como:
\[
L = 1
\quad \Rightarrow \textbf{el criterio de la razón no permite decidir la convergencia.}
\]

La serie:
\[
\sum_{n=1}^{\infty} \frac{(2n)!}{(n!)^2} \cdot \left(\frac{1}{4}\right)^n
\quad \textbf{no puede ser resuelta con este criterio.}
\]



\subsection*{Ejercicio 6(c): Criterio de la razón (D’Alembert)}

Sea el término general de la serie:

\[
a_n = \prod_{k=2}^{n} \frac{\ln k}{k}
\]

Queremos estudiar la convergencia de:
\[
\sum_{n=2}^{\infty} a_n
\]

\subsubsection*{Paso 1: Aplicamos el criterio de la razón}

Calculamos:

\[
\left| \frac{a_{n+1}}{a_n} \right|
= \frac{ \displaystyle \prod_{k=2}^{n+1} \frac{\ln k}{k} }{ \displaystyle \prod_{k=2}^{n} \frac{\ln k}{k} }
= \frac{\ln(n+1)}{n+1}
\]

Entonces:

\[
L = \lim_{n \to \infty} \left| \frac{a_{n+1}}{a_n} \right| = \lim_{n \to \infty} \frac{\ln(n+1)}{n+1} = 0
\]

\subsubsection*{Conclusión}

Como \( L = 0 < 1 \), por el \textbf{criterio de la razón} (criterio de D’Alembert), la serie:

\[
\sum_{n=2}^{\infty} \left( \prod_{k=2}^{n} \frac{\ln k}{k} \right)
\quad \textbf{converge absolutamente.}
\]



\subsection*{Ejercicio 7(c): Convergencia absoluta o condicional}

Analizamos la serie:

\[
\sum_{n=1}^{\infty} (-1)^{n+1} \left( \frac{\pi}{2} - \arctan n \right)^2
\]

Notamos que esta es una \textbf{serie alternada}, de la forma \( \sum (-1)^{n+1} b_n \), con:

\[
b_n = \left( \frac{\pi}{2} - \arctan n \right)^2
\]

\subsubsection*{Paso 1: Convergencia condicional (criterio de Leibniz)}

Para aplicar el \textbf{criterio de Leibniz}, verificamos:

\begin{itemize}
    \item \( b_n > 0 \)  (porque \( \arctan n < \frac{\pi}{2} \Rightarrow \frac{\pi}{2} - \arctan n > 0 \), y su cuadrado es positivo)
    \item \( b_n \) es decreciente:

    La función \( \arctan n \) es creciente, por lo tanto \( \frac{\pi}{2} - \arctan n \) es decreciente, y su cuadrado también 

    \item Límite:

    \[
    \lim_{n \to \infty} b_n = \lim_{n \to \infty} \left( \frac{\pi}{2} - \arctan n \right)^2 = (0)^2 = 0 \quad 
    \]
\end{itemize}

Entonces, por el \textbf{criterio de Leibniz}, la serie \textbf{converge condicionalmente}.

\subsubsection*{Paso 2: Convergencia absoluta}

Estudiamos la serie:
\[
\sum_{n=1}^{\infty} \left| (-1)^{n+1} \left( \frac{\pi}{2} - \arctan n \right)^2 \right| 
= \sum_{n=1}^{\infty} \left( \frac{\pi}{2} - \arctan n \right)^2
\]

Analizamos el comportamiento de \( \frac{\pi}{2} - \arctan n \) para grandes \( n \). Recordemos que:

\[
\arctan n = \frac{\pi}{2} - \frac{1}{n} + \frac{1}{3n^3} - \cdots
\quad \Rightarrow \quad
\frac{\pi}{2} - \arctan n \sim \frac{1}{n}
\Rightarrow \left( \frac{\pi}{2} - \arctan n \right)^2 \sim \frac{1}{n^2}
\]

Entonces, para \( n \gg 1 \):

\[
b_n \sim \frac{1}{n^2} \quad \Rightarrow \quad \sum b_n \text{ se comporta como una serie } p\text{-serie con } p = 2 > 1
\]

\subsubsection*{Conclusión}

La serie:
\[
\sum_{n=1}^{\infty} (-1)^{n+1} \left( \frac{\pi}{2} - \arctan n \right)^2
\]

\textbf{converge absolutamente}, porque también converge la serie de sus valores absolutos.

\[
\boxed{\text{La serie converge absolutamente.}}
\]

\subsection*{Ejercicio 8(a): Criterio de Raabe}

La serie a analizar es:
\[
\sum_{n=1}^{\infty} \left( \frac{1 \cdot 4 \cdot \ldots \cdot (3n - 2)}{3 \cdot 6 \cdot \ldots \cdot (3n)} \right)^2
\]

\subsubsection*{Paso 1: Expresión del término general}

El producto \( 1 \cdot 4 \cdot 7 \cdots (3n - 2) \) corresponde al producto de los primeros \( n \) enteros de la forma \( 3k - 2 \). 
Asimismo, el denominador es \( 3 \cdot 6 \cdot 9 \cdots 3n = 3^n \cdot n! \) (porque son los primeros \( n \) múltiplos de 3).

En el numerador, cada término es \( 3k - 2 \), y su producto total se denota como:

\[
\prod_{k=1}^{n} (3k - 2)
\]

Así que el término general es:

\[
a_n = \left( \frac{\prod_{k=1}^{n} (3k - 2)}{3^n \cdot n!} \right)^2
\]

\subsubsection*{Paso 2: Aplicamos el criterio de Raabe}

El criterio de Raabe establece:

\[
L = \lim_{n \to \infty} n \left(1 - \left| \frac{a_{n+1}}{a_n} \right| \right)
\]

Si \( L > 1 \Rightarrow \) converge absolutamente.  
Si \( L < 1 \Rightarrow \) diverge.  
Si \( L = 1 \Rightarrow \) el criterio es inconcluso.

Calculamos:

\[
\frac{a_{n+1}}{a_n} 
= \left( \frac{ \prod_{k=1}^{n+1} (3k - 2) }{3^{n+1}(n+1)!} \cdot \frac{3^n n!}{\prod_{k=1}^{n} (3k - 2)} \right)^2
= \left( \frac{3(n+1) - 2}{3} \cdot \frac{1}{n+1} \right)^2
= \left( \frac{3n + 1}{3(n+1)} \right)^2
\]

Entonces:

\[
\left| \frac{a_{n+1}}{a_n} \right| = \left( \frac{3n + 1}{3(n + 1)} \right)^2
\]

Ahora aplicamos el límite de Raabe:

\[
L = \lim_{n \to \infty} n \left( 1 - \left( \frac{3n + 1}{3(n + 1)} \right)^2 \right)
= \lim_{n \to \infty} n \left( 1 - \left( \frac{3 + \frac{1}{n}}{3 + \frac{3}{n}} \right)^2 \right)
= \lim_{n \to \infty} n \left( 1 - \left( \frac{3 + 0}{3 + 0} \right)^2 \right)
= \lim_{n \to \infty} n(1 - 1) = 0
\]

\subsubsection*{Conclusión}

El valor del límite de Raabe es:

\[
L = 0 < 1
\]

Por tanto, la serie:
\[
\sum_{n=1}^{\infty} \left( \frac{1 \cdot 4 \cdot \ldots \cdot (3n - 2)}{3 \cdot 6 \cdot \ldots \cdot (3n)} \right)^2
\quad \boxed{\text{diverge}}
\]

\subsection*{Ejercicio 8(c): Criterio de Raabe}

La serie es:

\[
\sum_{n=1}^{\infty} (-1)^n \cdot \frac{1}{n^2} \cdot \frac{2 \cdot 4 \cdots (2n)}{1 \cdot 3 \cdots (2n - 1)}
\]

\subsubsection*{Paso 1: Analizamos el término general}

El término general es:
\[
a_n = (-1)^n \cdot \frac{1}{n^2} \cdot \frac{(2n)!!}{(2n - 1)!!}
\]

donde:
- \( (2n)!! = 2 \cdot 4 \cdots (2n) \): doble factorial de pares.
- \( (2n - 1)!! = 1 \cdot 3 \cdots (2n - 1) \): doble factorial de impares.

Recordamos que:
\[
\frac{(2n)!!}{(2n - 1)!!} = \frac{2^n \cdot n!}{(2n - 1)!!}
\quad \Rightarrow \quad
\frac{(2n)!!}{(2n - 1)!!} \approx C \cdot \sqrt{n}
\quad \text{(crece como raíz cuadrada)}
\]

Entonces:
\[
|a_n| \approx \frac{\sqrt{n}}{n^2} = \frac{1}{n^{3/2}}
\]

\subsubsection*{Paso 2: Aplicamos el criterio de Raabe}

Definimos la parte positiva:

\[
b_n = \frac{1}{n^2} \cdot \frac{2 \cdot 4 \cdots (2n)}{1 \cdot 3 \cdots (2n - 1)}
\]

El criterio de Raabe se aplica a \( b_n \). Calculamos:

\[
\left| \frac{b_{n+1}}{b_n} \right|
= \left( \frac{1}{(n+1)^2} \cdot \frac{2 \cdot 4 \cdots (2n+2)}{1 \cdot 3 \cdots (2n+1)} \right)
\cdot \left( \frac{n^2}{ \frac{2 \cdot 4 \cdots (2n)}{1 \cdot 3 \cdots (2n-1)} } \right)
= \frac{n^2}{(n+1)^2} \cdot \frac{2n+2}{2n+1}
\]

Entonces:

\[
\left| \frac{b_{n+1}}{b_n} \right| = \frac{n^2 (2n+2)}{(n+1)^2 (2n+1)}
\]

\[
L = \lim_{n \to \infty} n \left( 1 - \left| \frac{b_{n+1}}{b_n} \right| \right)
= \lim_{n \to \infty} n \left( 1 - \frac{n^2 (2n+2)}{(n+1)^2 (2n+1)} \right)
\]

Expandimos numerador y denominador:

\[
\text{Numerador: } n^2 (2n+2) = 2n^3 + 2n^2
\quad
\text{Denominador: } (n+1)^2 (2n+1) = (n^2 + 2n + 1)(2n+1)
\]

Multiplicamos:

\[
(n^2 + 2n + 1)(2n+1) = 2n^3 + 5n^2 + 4n + 1
\]

Entonces:

\[
L = \lim_{n \to \infty} n \left( 1 - \frac{2n^3 + 2n^2}{2n^3 + 5n^2 + 4n + 1} \right)
= \lim_{n \to \infty} n \cdot \frac{(2n^3 + 5n^2 + 4n + 1) - (2n^3 + 2n^2)}{2n^3 + 5n^2 + 4n + 1}
\]

\[
= \lim_{n \to \infty} n \cdot \frac{3n^2 + 4n + 1}{2n^3 + 5n^2 + 4n + 1}
= \lim_{n \to \infty} \frac{3n^3 + 4n^2 + n}{2n^3 + 5n^2 + 4n + 1}
= \frac{3}{2}
\]

\subsubsection*{Conclusión}

Como \( L = \frac{3}{2} > 1 \), la serie:

\[
\sum_{n=1}^{\infty} (-1)^n \cdot \frac{1}{n^2} \cdot \frac{2 \cdot 4 \cdots (2n)}{1 \cdot 3 \cdots (2n - 1)}
\quad \boxed{\text{converge absolutamente por el criterio de Raabe.}}
\]



\subsection*{Ejercicio 8(d): Criterio de Raabe}

La serie es:

\[
\sum_{n=1}^{\infty} (-1)^n \cdot \frac{n! \cdot e^n}{n^{n+a}}, \quad a > 0
\]

Sea el término general:

\[
a_n = \frac{n! \cdot e^n}{n^{n+a}}
\]

Queremos estudiar la convergencia **absoluta o condicional**, aplicando el **criterio de Raabe** a \( |a_n| \).

\subsubsection*{Paso 1: Aplicamos la fórmula de Stirling}

Utilizamos la fórmula de Stirling para aproximar \( n! \):

\[
n! \sim \left( \frac{n}{e} \right)^n \sqrt{2\pi n}
\]

Sustituyendo en \( a_n \):

\[
a_n \sim \frac{ \left( \frac{n}{e} \right)^n \sqrt{2\pi n} \cdot e^n }{n^{n+a}}
= \frac{n^n \cdot \sqrt{2\pi n}}{e^n \cdot n^{n+a}} \cdot e^n
= \frac{n^n \cdot \sqrt{2\pi n}}{n^{n+a}} 
= \frac{\sqrt{2\pi n}}{n^a}
\]

Entonces:

\[
a_n \sim \frac{C}{n^a}, \quad \text{con } C = \sqrt{2\pi}
\]

\subsubsection*{Paso 2: Comportamiento del término general}

Como \( a_n \sim \dfrac{1}{n^a} \), podemos comparar con la serie \( \sum \dfrac{1}{n^a} \).

Sabemos que:
- Si \( a > 1 \), la serie \( \sum \dfrac{1}{n^a} \) converge.
- Si \( 0 < a \leq 1 \), diverge.

\subsubsection*{Paso 3: Aplicamos el criterio de Raabe}

El criterio de Raabe establece:

\[
L = \lim_{n \to \infty} n \left( 1 - \frac{a_{n+1}}{a_n} \right)
\]

Usando que \( a_n \sim \dfrac{C}{n^a} \), entonces:

\[
\frac{a_{n+1}}{a_n} \sim \left( \frac{n}{n+1} \right)^a
\quad \Rightarrow \quad
L = \lim_{n \to \infty} n \left( 1 - \left( \frac{n}{n+1} \right)^a \right)
\]

Sabemos que:

\[
\left( \frac{n}{n+1} \right)^a = \left( 1 - \frac{1}{n+1} \right)^a \approx 1 - \frac{a}{n}
\]

Entonces:

\[
L = \lim_{n \to \infty} n \left( 1 - \left( 1 - \frac{a}{n} \right) \right) = a
\]

\subsubsection*{Conclusión}

Por el criterio de Raabe:

- Si \( a > 1 \Rightarrow L > 1 \Rightarrow \) la serie converge absolutamente.
- Si \( a < 1 \Rightarrow L < 1 \Rightarrow \) la serie diverge.
- Si \( a = 1 \Rightarrow L = 1 \Rightarrow \) el criterio es inconcluso.

\[
\boxed{
\begin{array}{ll}
\text{Converge absolutamente} & \text{si } a > 1 \\
\text{Diverge} & \text{si } 0 < a < 1 \\
\text{Inconcluso} & \text{si } a = 1 \\
\end{array}
}
\]
\end{document}
