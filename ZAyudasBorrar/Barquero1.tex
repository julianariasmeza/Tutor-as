\documentclass[12pt]{article}

\usepackage{amsmath}
\usepackage{siunitx}

\begin{document}

\section*{Problema 1}

\subsection*{Datos}

\[
P = 600\ \text{lb},\qquad
E = 29\times 10^{6}\ \text{psi},
\qquad
A = 0,125\ \text{pulg}^2
\]

\[
L_{BC} = 5\ \text{pulg},\qquad
L_{DE} = 5\ \text{pulg}
\]

\[
S_C = 6\theta,\qquad
S_D = 4\theta,\qquad
S_A = 10\theta
\]

\subsection*{Fórmulas}

\[
F = \frac{EA\,S}{L}
\]

\[
\sum M_F = 0
\]

\subsection*{Fuerza en BC}

\[
F_{BC} = \frac{EA(-6\theta)}{5}
\]

\[
F_{BC} = \frac{(29\times 10^{6})(0,125)(-6\theta)}{5}
\]

\[
F_{BC} = -4,35\times 10^{6}\theta
\]

\subsection*{Fuerza en ED}

\[
F_{ED} = \frac{EA(4\theta)}{5}
\]

\[
F_{ED} = \frac{(29\times 10^{6})(0,125)(4\theta)}{5}
\]

\[
F_{ED} = 2,90\times 10^{6}\theta
\]

\subsection*{Cálculo del ángulo $\theta$}

\[
10P - 6F_{BC} - 4F_{ED} = 0
\]

\[
10(600)
- 6(-4,35\times 10^{6}\theta)
- 4(2,90\times 10^{6}\theta)
= 0
\]

\[
6000 + 26,1\times 10^{6}\theta - 11,6\times 10^{6}\theta = 0
\]

\[
14,5\times 10^{6}\theta = -6000
\]

\[
\theta = -4,14\times 10^{-4}\ \text{rad}
\]

\[
\theta = 4,14\times 10^{-4}\ \text{rad}
\]

\subsection*{Fuerzas finales}

\[
F_{BC} = -4,35\times 10^{6}(4,14\times 10^{-4})
\]

\[
F_{BC} = -1800\ \text{lb}
\]

\[
F_{ED} = 2,90\times 10^{6}(4,14\times 10^{-4})
\]

\[
F_{ED} = 1200\ \text{lb}
\]

\subsection*{Deflexión en A}

\[
S_A = 10\theta
\]

\[
S_A = 10(4,14\times 10^{-4})
\]

\[
S_A = 4,14\times 10^{-3}\ \text{pulg}
\]

\subsection*{Resultados}

\[
\boxed{F_{BC}=1800\ \text{lb}}
\]

\[
\boxed{F_{ED}=1200\ \text{lb}}
\]

\[
\boxed{\delta_A = 0,00414\ \text{pulg}}
\]
\newpage
\section*{Problema 2}

\subsection*{Datos}

\[
d_{AB}=0,6\ \text{in},\qquad
d_{BC}=0,75\ \text{in},\qquad
d_{CD}=0,9\ \text{in}
\]

\[
d_h = 0,30\ \text{in},\qquad
c_1 = \frac{d_h}{2} = 0,15\ \text{in}
\]

\[
T_A = 400\ \text{lb·in},\qquad
T_B = 1200\ \text{lb·in},\qquad
T_C = 500\ \text{lb·in}
\]

\subsection*{Torques internos}

\[
T_{AB}=400\ \text{lb·in}
\]

\[
T_{BC}=1200 - 400 = 800\ \text{lb·in}
\]

\[
T_{CD}=1200 - 400 + 500 = 1300\ \text{lb·in}
\]

\subsection*{Momento polar de inercia}

\[
J=\frac{\pi}{32}(d_o^4 - d_i^4)
\]

\[
d_i^4 = (0,30)^4 = 0,0081
\]

\subsubsection*{Tramo AB}

\[
d_o = 0,6,\qquad
d_o^4 = (0,6)^4 = 0,1296
\]

\[
J_{AB}=\frac{\pi}{32}(0,1296 - 0,0081)
\]

\[
J_{AB}=\frac{3,1416}{32}(0,1215)
\]

\[
J_{AB}=0,01193\ \text{in}^4
\]

\subsubsection*{Tramo BC}

\[
d_o = 0,75,\qquad
d_o^4 = (0,75)^4 = 0,3164
\]

\[
J_{BC}=\frac{\pi}{32}(0,3164 - 0,0081)
\]

\[
J_{BC}=\frac{3,1416}{32}(0,3083)
\]

\[
J_{BC}=0,03030\ \text{in}^4
\]

\subsubsection*{Tramo CD}

\[
d_o = 0,9,\qquad
d_o^4 = (0,9)^4 = 0,6561
\]

\[
J_{CD}=\frac{\pi}{32}(0,6561 - 0,0081)
\]

\[
J_{CD}=\frac{3,1416}{32}(0,6480)
\]

\[
J_{CD}=0,06360\ \text{in}^4
\]

\subsection*{Esfuerzo cortante máximo}

\[
\tau_{\max} = \frac{T\,c_2}{J}
\]

\textbf{Donde}

\[
c_2 = \frac{d_o}{2}
\]

\subsubsection*{Tramo AB}

\[
c_{AB} = 0,3\ \text{in}
\]

\[
\tau_{AB} = \frac{400(0,3)}{0,01193}
           = \frac{120}{0,01193}
\]

\[
\tau_{AB} = 10060\ \text{psi}
\]

\[
\tau_{AB} = 10,06\ \text{ksi}
\]

\subsubsection*{Tramo BC}

\[
c_{BC}=0,375\ \text{in}
\]

\[
\tau_{BC} = \frac{800(0,375)}{0,03030}
           = \frac{300}{0,03030}
\]

\[
\tau_{BC}=9900\ \text{psi}
\]

\[
\tau_{BC}=9,90\ \text{ksi}
\]

\subsubsection*{Tramo CD}

\[
c_{CD}=0,45\ \text{in}
\]

\[
\tau_{CD}=\frac{1300(0,45)}{0,06360}
         =\frac{585}{0,06360}
\]

\[
\tau_{CD}=9198\ \text{psi}
\]

\[
\tau_{CD}=9,20\ \text{ksi}
\]

\subsection*{Resultados}

\[
\boxed{\tau_{\max}\ \text{ocurre en el tramo AB}}
\]

\[
\boxed{\tau_{\max} \approx 1,006\times 10^{4}\ \text{psi} = 10,06\ \text{ksi}}
\]

\newpage
\section*{Problema 3}

\subsection*{Datos}

\[
T_B = 400\ \text{N·m},\qquad
T_C = 900\ \text{N·m},\qquad
T_D = 500\ \text{N·m}
\]

\[
G = 27\times 10^{9}\ \text{Pa}
\]

\[
L_{AB} = 0,6\ \text{m},\quad
L_{BC} = 0,8\ \text{m},\quad
L_{CD} = 1,0\ \text{m},\quad
L_{DE} = 0,5\ \text{m}
\]

\[
d_{BC} = 30\ \text{mm} = 0,030\ \text{m},\qquad
d_{CD} = 36\ \text{mm} = 0,036\ \text{m}
\]

\subsection*{Torques internos}

\[
T_{BC} = 400\ \text{N·m}
\]

\[
T_{CD} = 400 - 900 = -500\ \text{N·m}
\]

\subsection*{Propiedades de sección}

\[
c_{BC} = \frac{d_{BC}}{2}
       = \frac{0,030}{2}
       = 0,015\ \text{m}
\]

\[
J_{BC} = \frac{\pi}{2} c_{BC}^{4}
       = \frac{\pi}{2} (0,015)^{4}
       = \frac{\pi}{2} (5,0625\times 10^{-8})
       = 79,522\times 10^{-9}\ \text{m}^{4}
\]

\[
c_{CD} = \frac{d_{CD}}{2}
       = \frac{0,036}{2}
       = 0,018\ \text{m}
\]

\[
J_{CD} = \frac{\pi}{2} c_{CD}^{4}
       = \frac{\pi}{2} (0,018)^{4}
       = \frac{\pi}{2} (1,04976\times 10^{-7})
       = 164,896\times 10^{-9}\ \text{m}^{4}
\]

\subsection*{Ángulo de giro entre C y B}

\[
\varphi_{BC} = \frac{T_{BC} L_{BC}}{G J_{BC}}
\]

\[
\varphi_{BC} =
\frac{(400)(0,8)}
     {(27\times 10^{9})(79,522\times 10^{-9})}
\]

\[
\varphi_{BC} =
\frac{320}
     {2,147094\times 10^{3}}
= 0,1499\ \text{rad}
\]

\[
\varphi_{BC}\left( \frac{180^\circ}{\pi} \right)
= 0,1499\left( \frac{180^\circ}{\pi} \right)
= 8,54^\circ
\]

\[
\boxed{\varphi_{CB} = 0,1499\ \text{rad} = 8,54^\circ}
\]

\subsection*{Ángulo de giro entre D y B}

\[
\varphi_{CD} = \frac{T_{CD} L_{CD}}{G J_{CD}}
\]

\[
\varphi_{CD} =
\frac{(-500)(1,0)}
     {(27\times 10^{9})(164,896\times 10^{-9})}
\]

\[
\varphi_{CD} =
\frac{-500}
     {4,452192\times 10^{3}}
= -0,1123\ \text{rad}
\]

\[
\varphi_{DB} = \varphi_{BC} + \varphi_{CD}
\]

\[
\varphi_{DB} = 0,1499 - 0,1123
             = 0,0370\ \text{rad}
\]

\[
\varphi_{DB}\left( \frac{180^\circ}{\pi} \right)
= 0,0370\left( \frac{180^\circ}{\pi} \right)
= 2,12^\circ
\]

\[
\boxed{\varphi_{DB} \approx 0,037\ \text{rad} \approx 2,12^\circ}
\]
\newpage
\section*{Problema 4}

\subsection*{Datos}

\[
T_A = 750\ \text{lb·in}
\]

\[
d = \tfrac{3}{4}\ \text{in} = 0,75\ \text{in},\qquad
c = \frac{d}{2} = 0,375\ \text{in}
\]

\[
G = 11,2\times 10^{6}\ \text{psi}
\]

\[
L_{FE} = 8\ \text{in},\qquad
L_{BA} = 6 + 5 = 11\ \text{in}
\]

Relación de radios entre engranes:
\[
\frac{r_E}{r_B} = \frac{4}{3}
\]

\subsection*{Torque transmitido al eje FE}

\[
T_{FE} = \frac{r_E}{r_B}\,T_A
       = \frac{4}{3}(750)
       = 1000\ \text{lb·in}
\]

\subsection*{Momento polar de inercia de los ejes}

\[
J = \frac{\pi}{2}c^{4}
\]

\[
c^{4} = (0,375)^{4} = 0,019775
\]

\[
J = \frac{\pi}{2}(0,019775)
  = 0,031063\ \text{in}^{4}
\]

\subsection*{Giro del eje FE}

\[
\varphi_{FE} = \frac{T_{FE} L_{FE}}{GJ}
\]

\[
\varphi_{FE} =
\frac{(1000)(8)}
     {(11,2\times 10^{6})(0,031063)}
\]

\[
\varphi_{FE} = 22,995\times 10^{-3}\ \text{rad}
\]

\[
\varphi_E = \varphi_{FE} = 22,995\times 10^{-3}\ \text{rad}
\]

\subsection*{Rotación en B por efecto del engrane}

Desplazamientos tangenciales:
\[
r_E \varphi_E = r_B \varphi_B
\]

\[
\varphi_B = \frac{r_E}{r_B}\,\varphi_E
          = \frac{4}{3}(22,995\times 10^{-3})
\]

\[
\varphi_B = 30,660\times 10^{-3}\ \text{rad}
\]

\subsection*{Giro del eje BA}

\[
\varphi_{BA} = \frac{T_A L_{BA}}{GJ}
\]

\[
\varphi_{BA} =
\frac{(750)(11)}
     {(11,2\times 10^{6})(0,031063)}
\]

\[
\varphi_{BA} = 27,713\times 10^{-3}\ \text{rad}
\]

\subsection*{Rotación en A}

\[
\varphi_A = \varphi_B + \varphi_{BA}
\]

\[
\varphi_A = 30,660\times 10^{-3}
          + 27,713\times 10^{-3}
\]

\[
\varphi_A = 54,373\times 10^{-3}\ \text{rad}
\]

\[
\varphi_A = 0,05437\ \text{rad}
\]

\[
\varphi_A(^\circ)
= 0,05437\left(\frac{180^\circ}{\pi}\right)
= 3,12^\circ
\]

\subsection*{Resultado}

\[
\boxed{\varphi_A \approx 0,0544\ \text{rad} \;\approx\ 3,12^\circ}
\]
\newpage
\section*{Problema 5}

\subsection*{Datos}

\[
\tau_{\text{adm}} = 8500\ \text{psi},\qquad
G = 11,2\times 10^{6}\ \text{psi}
\]

\[
L = 125\ \text{ft} = 1500\ \text{in}
\]

\[
d_o = 16\ \text{in},\qquad
d_i = 8\ \text{in}
\]

\[
c_2 = \frac{d_o}{2} = 8\ \text{in},\qquad
c_1 = \frac{d_i}{2} = 4\ \text{in}
\]

\[
\text{rpm} = 165,\qquad
f = \frac{165}{60} = 2,75\ \text{Hz}
\]

\subsection*{Momento polar}

\[
J=\frac{\pi}{2}(c_2^{4}-c_1^{4})
\]

\[
c_2^{4} = 8^{4} = 4096,\qquad
c_1^{4} = 4^{4} = 256
\]

\[
c_2^{4}-c_1^{4} = 4096 - 256 = 3840
\]

\[
J = \frac{\pi}{2}(3840)
\]

\[
J = 6031,8579\ \text{in}^{4}
\]

\subsection*{Torque máximo}

\[
\tau = \frac{Tc}{J}
\]

\[
T = \frac{\tau J}{c}
\]

\[
T = \frac{(8500)(6031,8579)}{8}
\]

\[
T = 6408844,0188\ \text{lb·in}
\]

\subsection*{Potencia máxima}

\[
P = 2\pi f T
\]

\[
P = 2\pi(2,75)(6408844,0188)
\]

\[
P = 110736961,47\ \frac{\text{lb·in}}{s}
\]

Conversión a HP:

\[
1\ \text{HP} = 6600\ \frac{\text{lb·in}}{s}
\]

\[
P = \frac{110736961,47}{6600}
\]

\[
P = 16778,3275\ \text{HP}
\]

\subsection*{Ángulo de giro}

\[
\varphi = \frac{T L}{G J}
\]

\[
\varphi = 
\frac{(6408844,0188)(1500)}
     {(11,2\times 10^{6})(6031,8579)}
\]

\[
\varphi = 0,1423\ \text{rad}
\]

\[
\varphi =
0,1423 \left(\frac{180^\circ}{\pi}\right)
=
8,1532^\circ
\]

\subsection*{Resultados}

\[
\boxed{
P_{\max} = 1,678\times 10^{4}\ \text{HP}
}
\]

\[
\boxed{
\varphi = 0,1423\ \text{rad} \;\; (8,1532^\circ)
}
\]
\newpage
\section*{Problema 6}

\subsection*{Datos}

\[
w = 8\ \text{kN/m},\qquad
P = 10\ \text{kN}
\]

\[
x = 2\ \text{m}
\]

\[
L = 6\ \text{m}
\]

\subsection*{Reacciones}

\[
\sum M_A = 0
\]

\[
F_B(6) - 10(5) - 8(3)\left(1+\frac{3}{2}\right) = 0
\]

\[
F_B = 18,3333\ \text{kN}
\]

\[
\sum F_y = 0
\]

\[
F_A + 8(3) - 10 - 18,3333 = 0
\]

\[
F_A = 15,6667\ \text{kN}
\]

\subsection*{Momento flector a 2 m}

\[
\sum M = 0
\]

\[
-15,6667(2) + 8(1)\left(\frac{1}{2}\right) + M = 0
\]

\[
M = 27,3334\ \text{kN·m}
\]

\subsection*{Propiedades geométricas}

\[
b = 10\ \text{mm},\quad
h = 50\ \text{mm}
\]

\[
\bar y =
\frac{(10)(50)(35)^2 + 50(10)(5)}{(10)(50)^3}
\]

\[
\bar y = 25\ \text{mm}
\]

\[
I_1 = \frac{1}{12}(10)(50)^3 + 50(10)(35-25)^2
\]

\[
I_1 = 154166,6667\ \text{mm}^4
\]

\[
I_2 = I_1
\]

\[
I_3 = \frac{1}{12}(50)(10)^3 + 50(10)(25-5)^2
\]

\[
I_3 = 204166,6667\ \text{mm}^4
\]

\[
I_T = I_1 + I_2 + I_3
\]

\[
I_T = 512500\ \text{mm}^4
\]

\subsection*{Esfuerzo máximo}

\[
\sigma = \frac{M c}{I}
\]

\subsection*{En compresión}

\[
c = 35\ \text{mm}
\]

\[
\sigma_{\max} =
\frac{(27,3334)(10^6)(35)}
     {512500}
\]

\[
\sigma_{\max} = 1,8667\ \text{GPa}
\]

\subsection*{En tensión}

\[
c = 25\ \text{mm}
\]

\[
\sigma_{\max} =
\frac{(27,3334)(10^6)(25)}
     {512500}
\]

\[
\sigma_{\max} = 1,3333\ \text{GPa}
\]

\subsection*{Resultados}

\[
\boxed{\sigma_{\text{comp}} = 1,8667\ \text{GPa}}
\]

\[
\boxed{\sigma_{\text{tens}} = 1,3333\ \text{GPa}}
\]

\end{document}
