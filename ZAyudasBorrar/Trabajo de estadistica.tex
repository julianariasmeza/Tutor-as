\documentclass[12pt]{article}
\usepackage[utf8]{inputenc}
\usepackage[margin=2.5cm]{geometry}
\usepackage{amsmath, amssymb}
\usepackage{graphicx}
\usepackage{booktabs}

\title{Compendio de Ejercicios de Estadística}
\author{Nombre del Estudiante \\ Curso de Estadística}
\date{2025}

\begin{document}

\maketitle

\tableofcontents
\newpage

% -------------------------------------------------------------------
% -----------------------   TEMA 1   --------------------------------
% -------------------------------------------------------------------

\section{Población Infinita}

\subsection*{Ejercicio 1}
\textbf{Enunciado.}\\
Se desea estimar la proporción de hogares que consumen café diariamente en una ciudad grande. 
Su población es considerada \emph{prácticamente infinita}. 
Calcule el tamaño mínimo de la muestra necesaria si se desea un nivel de confianza del 95\% y un margen de error de 5\%. 
Suponga proporción esperada $p = 0.5$.

\textbf{Desarrollo.}\\
Para poblaciones infinitas, el tamaño de la muestra se obtiene mediante:
\[
n=\frac{Z^{2} p q}{E^{2}},
\]
donde:\\
$Z = 1.96$ (95\%), $p = 0.5$, $q = 1-p = 0.5$, $E = 0.05$.\\[6pt]

Sustitución:
\[
n=\frac{(1.96)^{2} (0.5)(0.5)}{(0.05)^{2}}.
\]

Cálculo:
\[
(1.96)^{2}=3.8416,
\quad
(0.5)(0.5)=0.25,
\quad
(0.05)^{2}=0.0025.
\]
\[
n=\frac{3.8416 \times 0.25}{0.0025}
= \frac{0.9604}{0.0025}
= 384.16.
\]

\textbf{Resultado.}\\
Se requiere una muestra de aproximadamente **385 personas**.

\textbf{Fuente (APA):}\\
Scribd. (2023). \textit{Ejercicios de cálculo de población finita e infinita}. Recuperado de https://es.scribd.com/document/621107138/Ejercicios-de-calculo-de-poblacion-finita-e-infinita


\subsection*{Ejercicio 2}
\textbf{Enunciado.}\\
Una empresa desea estimar el porcentaje de clientes que estarían interesados en un nuevo servicio digital. 
Dado que la población de usuarios supera ampliamente los 2 millones, se considera \emph{población infinita}. 
Calcule el tamaño de muestra requerido con $95\%$ de confianza y $E = 3\%$; 
suponga $p = 0.4$.

\textbf{Desarrollo.}\\
Fórmula para población infinita:
\[
n=\frac{Z^{2} p q}{E^{2}}.
\]

Datos: $Z = 1.96$, $p = 0.4$, $q = 0.6$, $E = 0.03$.\\[6pt]

Sustitución:
\[
n=\frac{(1.96)^{2}(0.4)(0.6)}{(0.03)^{2}}.
\]

Cálculo:
\[
(1.96)^{2}=3.8416,
\quad
(0.4)(0.6)=0.24,
\quad
(0.03)^{2}=0.0009.
\]
\[
n=\frac{3.8416 \times 0.24}{0.0009}
= \frac{0.9220}{0.0009}
\approx 1024.4.
\]

\textbf{Resultado.}\\
La muestra necesaria es de aproximadamente **1025 personas**.

\textbf{Fuente (APA):}\\
Daniel, W. (2012). \textit{Bioestadística}. McGraw-Hill. (Método clásico para muestreo infinito)

\newpage

% -------------------------------------------------------------------
% -----------------------   TEMA 2   --------------------------------
% -------------------------------------------------------------------

\section{Muestra Aleatoria}

\subsection*{Ejercicio 1}
\textbf{Enunciado.}\\
En un colegio se desea seleccionar una muestra aleatoria simple de 10 estudiantes entre un total de 300 para aplicar una encuesta de salud. 
Explique el procedimiento y seleccione una posible muestra utilizando números aleatorios.

\textbf{Desarrollo.}\\
Definición: en la \emph{muestra aleatoria simple}, todos los individuos tienen la misma probabilidad de ser elegidos.\\[6pt]

Procedimiento:
\begin{enumerate}
    \item Numerar los 300 estudiantes del 1 al 300.
    \item Usar una tabla de números aleatorios o un generador digital.
    \item Seleccionar 10 números distintos dentro del rango.
\end{enumerate}

Ejemplo usando generador digital (valores simulados):
\[
\{12,\; 45,\; 88,\; 101,\; 134,\; 167,\; 199,\; 203,\; 244,\; 289\}.
\]

\textbf{Resultado.}\\
Una muestra aleatoria simple válida es:  
12, 45, 88, 101, 134, 167, 199, 203, 244 y 289.

\textbf{Fuente (APA):}\\
Lind, D., Marchal, W. \& Wathen, S. (2018). \textit{Estadística aplicada a los negocios y la economía}. McGraw-Hill.


\subsection*{Ejercicio 2}
\textbf{Enunciado.}\\
Una empresa desea seleccionar una muestra aleatoria simple de 5 productos de una línea de producción compuesta por 120 unidades. 
Explique cómo obtener la muestra usando una tabla de números aleatorios y muestre un posible resultado.

\textbf{Desarrollo.}\\
Procedimiento general:
\begin{enumerate}
    \item Enumerar las 120 unidades del 001 al 120.
    \item Leer números de tres dígitos desde una tabla o generador.
    \item Aceptar solo valores entre 001 y 120 sin repetir.
\end{enumerate}

Ejemplo de lectura (números simulados):
\[
\{004,\; 017,\; 056,\; 099,\; 113\}.
\]

\textbf{Resultado.}\\
Una muestra aleatoria simple posible es: 4, 17, 56, 99 y 113.

\textbf{Fuente (APA):}\\
Wackerly, D., Mendenhall, W. \& Scheaffer, R. (2014). \textit{Matemáticas para la estadística}. Cengage Learning.

\newpage

% -------------------------------------------------------------------
% -----------------------   TEMA 3   --------------------------------
% -------------------------------------------------------------------

\section{Frecuencia}

\subsection*{Ejercicio 1}
\textbf{Enunciado.}\\
La siguiente tabla muestra el número de llamadas recibidas en un call center en intervalos de 10 minutos. Complete la columna de frecuencias absolutas según los datos registrados:  
\[
\{12,\; 15,\; 18,\; 22,\; 15,\; 19,\; 16,\; 20,\; 18,\; 22,\; 21,\; 19\}.
\]

\textbf{Desarrollo.}\\
Se cuentan las repeticiones de cada valor:
\[
12(1),\; 15(2),\; 16(1),\; 18(2),\; 19(2),\; 20(1),\; 21(1),\; 22(2).
\]

\textbf{Resultado.}\\
La frecuencia absoluta queda representada así:
\[
\{1,\;2,\;1,\;2,\;2,\;1,\;1,\;2\}.
\]

\textbf{Fuente (APA):}\\
Matemóvil. (2022). \textit{Histogramas: ejemplos y ejercicios}. Recuperado de https://matemovil.com/histogramas-ejemplos-y-ejercicios/


\subsection*{Ejercicio 2}
\textbf{Enunciado.}\\
En una muestra de ventas diarias de una tienda de chocolates se obtuvieron los siguientes valores (unidades):  
\[
\{15,\; 18,\; 19,\; 20,\; 18,\; 17,\; 16,\; 19,\; 20,\; 18\}.
\]
Determine las frecuencias absolutas.

\textbf{Desarrollo.}\\
Se agrupan las repeticiones:
\[
15(1),\; 16(1),\; 17(1),\; 18(3),\; 19(2),\; 20(2).
\]

\textbf{Resultado.}\\
Frecuencias absolutas:
\[
\{1,\;1,\;1,\;3,\;2,\;2\}.
\]

\textbf{Fuente (APA):}\\
Scribd. (2023). \textit{Histogramas: ejercicios resueltos}. Recuperado de https://es.scribd.com/document/703612592/Histogramas-Ejercicios-Resueltos

\newpage

% -------------------------------------------------------------------
% -----------------------   TEMA 4   --------------------------------
% -------------------------------------------------------------------

\section{Frecuencia Relativa}

\subsection*{Ejercicio 1}
\textbf{Enunciado.}\\
Con los datos:  
\[
\{10,\; 12,\; 12,\; 14,\; 16,\; 16,\; 16,\; 18\},
\]
determine la frecuencia relativa de cada valor.

\textbf{Desarrollo.}\\
Total de datos: $N = 8$.  
Frecuencias:  
10(1), 12(2), 14(1), 16(3), 18(1).

Frecuencia relativa:
\[
f_r=\frac{f}{N}.
\]

\[
10:\frac{1}{8}=0.125,\;
12:\frac{2}{8}=0.25,\;
14:\frac{1}{8}=0.125,\;
16:\frac{3}{8}=0.375,\;
18:\frac{1}{8}=0.125.
\]

\textbf{Resultado.}\\
Frecuencias relativas:  
\[
\{0.125,\; 0.25,\; 0.125,\; 0.375,\; 0.125\}.
\]

\textbf{Fuente (APA):}\\
Lind, D., Marchal, W. \& Wathen, S. (2018). \textit{Estadística aplicada a los negocios y la economía}. McGraw-Hill.


\subsection*{Ejercicio 2}
\textbf{Enunciado.}\\
Los siguientes datos representan las ventas semanales de un producto:  
\[
\{5,\; 7,\; 7,\; 8,\; 10,\; 10,\; 10,\; 12\}.
\]
Calcule la frecuencia relativa para cada valor distinto.

\textbf{Desarrollo.}\\
\[
N=8.
\]
Frecuencias:  
5(1), 7(2), 8(1), 10(3), 12(1).

\[
f_r=\frac{f}{8}.
\]

Cálculos:
\[
5:0.125,\;
7:0.25,\;
8:0.125,\;
10:0.375,\;
12:0.125.
\]

\textbf{Resultado.}\\
Frecuencias relativas:  
\[
0.125,\; 0.25,\; 0.125,\; 0.375,\; 0.125.
\]

\textbf{Fuente (APA):}\\
Triola, M. (2014). \textit{Estadística}. Pearson Educación.


\newpage

% -------------------------------------------------------------------
% -----------------------   TEMA 5   --------------------------------
% -------------------------------------------------------------------

\section{Clases Cerradas y Abiertas}

\subsection*{Ejercicio 1}
\textbf{Enunciado.}\\
Un conjunto de datos registra la altura (cm) de 40 personas. Construya clases cerradas de amplitud 5 cm iniciando en 155 cm.

\textbf{Desarrollo.}\\
Clases cerradas incluyen ambos límites: \([a,b]\).  
Amplitud = 5.

Clases:
\[
[155,160],\;
[160,165],\;
[165,170],\;
[170,175],\;
[175,180].
\]

\textbf{Resultado.}\\
Las clases cerradas válidas son las anteriores.

\textbf{Fuente (APA):}\\
Scribd. (2023). \textit{Histogramas: ejercicios resueltos}. Recuperado de https://es.scribd.com/document/703612592/Histogramas-Ejercicios-Resueltos


\subsection*{Ejercicio 2}
\textbf{Enunciado.}\\
Una base de datos de tiempos de espera (min) debe organizarse en clases abiertas. Construya clases de amplitud 10 min desde 0 min.

\textbf{Desarrollo.}\\
Clases abiertas:  
\[
[0,10),\;
[10,20),\;
[20,30),\;
[30,40),\;
[40,50).
\]

\textbf{Resultado.}\\
Clases abiertas definidas según amplitud 10 min.

\textbf{Fuente (APA):}\\
Triola, M. (2014). \textit{Estadística}. Pearson Educación.


\newpage

% -------------------------------------------------------------------
% -----------------------   TEMA 6   --------------------------------
% -------------------------------------------------------------------

\section{Gráficos de Distribución de Frecuencia}

\subsection*{Ejercicio 1}
\textbf{Enunciado.}\\
La siguiente tabla corresponde a los tiempos de llamadas recibidas (min). Construya el gráfico de distribución de frecuencias (histograma).

\begin{center}
\begin{tabular}{c c}
\toprule
Intervalo & Frecuencia \\
\midrule
0--10 & 3 \\
10--20 & 8 \\
20--30 & 12 \\
30--40 & 6 \\
40--50 & 3 \\
\bottomrule
\end{tabular}
\end{center}

\textbf{Desarrollo.}\\
El histograma se construye colocando los intervalos en el eje horizontal y las frecuencias en el vertical.  
Cada barra tiene amplitud 10 y altura igual a su frecuencia.

\textbf{Resultado.}\\
Se obtiene un histograma con barras de alturas:  
3, 8, 12, 6, 3.

\textbf{Fuente (APA):}\\
Matemóvil. (2022). \textit{Histogramas: ejemplos y ejercicios}. Recuperado de https://matemovil.com/histogramas-ejemplos-y-ejercicios/


\subsection*{Ejercicio 2}
\textbf{Enunciado.}\\
Los datos de ventas de chocolates mostrados en la tabla siguiente corresponden a un ejercicio clásico de distribución de frecuencias. Dibuje el gráfico correspondiente.

\begin{center}
\begin{tabular}{c c}
\toprule
Unidades & Frecuencia \\
\midrule
15 & 4 \\
16 & 2 \\
17 & 3 \\
18 & 5 \\
19 & 4 \\
20 & 3 \\
\bottomrule
\end{tabular}
\end{center}

\textbf{Desarrollo.}\\
El gráfico de barras se construye asignando cada valor de unidades a una barra cuya altura coincide con su frecuencia.

\textbf{Resultado.}\\
Barras con alturas:
\[
4,\;2,\;3,\;5,\;4,\;3.
\]

\textbf{Fuente (APA):}\\
Scribd. (2023). \textit{Histogramas: ejercicios resueltos}. Recuperado de https://es.scribd.com/document/703612592/Histogramas-Ejercicios-Resueltos


\newpage

% -------------------------------------------------------------------
% -----------------------   TEMA 7   --------------------------------
% -------------------------------------------------------------------

\section{Variable Continua}

\subsection*{Ejercicio 1}
\textbf{Enunciado.}\\
La variable “tiempo de atención en ventanilla (minutos)” se midió para 20 clientes. Determine si esta variable es continua y justifique con base en la definición estadística.

\textbf{Desarrollo.}\\
Una variable es continua si puede tomar infinitos valores en un intervalo real.  
El tiempo puede medirse con precisión arbitraria (min, s, ms), por lo que no se limita a valores discretos.

\textbf{Resultado.}\\
El tiempo de atención es una variable continua porque puede tomar cualquier valor real dentro de un intervalo.

\textbf{Fuente (APA):}\\
Triola, M. (2014). \textit{Estadística}. Pearson Educación.


\subsection*{Ejercicio 2}
\textbf{Enunciado.}\\
Se registra la temperatura (°C) en una planta industrial cada 10 minutos durante una hora. Determine si la variable temperatura es continua y explique por qué.

\textbf{Desarrollo.}\\
La temperatura puede tomar infinitos valores reales, aunque la medición sea redondeada.  
En estadística, cualquier variable que admita subdivisión indefinida pertenece a las variables continuas.

\textbf{Resultado.}\\
La temperatura es una variable continua, ya que admite infinitos valores posibles en un intervalo.

\textbf{Fuente (APA):}\\
Montgomery, D. (2013). \textit{Probabilidad y estadística aplicadas a la ingeniería}. Wiley.


\newpage

% -------------------------------------------------------------------
% -----------------------   TEMA 8   --------------------------------
% -------------------------------------------------------------------

\section{Ojivas}

\subsection*{Ejercicio 1}
\textbf{Enunciado.}\\
Dada la siguiente tabla de frecuencias acumuladas de tiempos de espera (min), construya la ojiva correspondiente.

\begin{center}
\begin{tabular}{c c}
\toprule
Intervalo & Frecuencia acumulada \\
\midrule
0--5 & 4 \\
5--10 & 10 \\
10--15 & 17 \\
15--20 & 22 \\
20--25 & 25 \\
\bottomrule
\end{tabular}
\end{center}

\textbf{Desarrollo.}\\
La ojiva representa puntos en el extremo superior de cada clase contra su frecuencia acumulada y posteriormente se unen mediante líneas.

Puntos:
\[
(5,4),\; (10,10),\; (15,17),\; (20,22),\; (25,25).
\]

\textbf{Resultado.}\\
La ojiva se construye uniendo los puntos anteriores con líneas rectas.

\textbf{Fuente (APA):}\\
Scribd. (2023). \textit{Frecuencia acumulada: ejercicio en clase}. Recuperado de https://es.scribd.com/document/557677562/Frecuencia-Acumulada-Ejercicio-en-Clase


\subsection*{Ejercicio 2}
\textbf{Enunciado.}\\
La tabla siguiente muestra frecuencias acumuladas de ventas semanales. Construya la ojiva.

\begin{center}
\begin{tabular}{c c}
\toprule
Unidades acumuladas & Frecuencia acumulada \\
\midrule
10 & 3 \\
20 & 9 \\
30 & 15 \\
40 & 18 \\
50 & 20 \\
\bottomrule
\end{tabular}
\end{center}

\textbf{Desarrollo.}\\
Se grafican los puntos:
\[
(10,3),\; (20,9),\; (30,15),\; (40,18),\; (50,20).
\]

\textbf{Resultado.}\\
La ojiva se obtiene uniendo los puntos de manera ascendente.

\textbf{Fuente (APA):}\\
Matemóvil. (2022). \textit{Histogramas y distribución acumulada}. Recuperado de https://matemovil.com/histogramas-ejemplos-y-ejercicios/


\newpage

% -------------------------------------------------------------------
% -----------------------   TEMA 9   --------------------------------
% -------------------------------------------------------------------

\section{Medidas de Tendencia Central}

\subsection*{Ejercicio 1}
\textbf{Enunciado.}\\
Los siguientes datos representan salarios semanales (en miles):  
\[
\{220,\; 240,\; 250,\; 260,\; 300\}.
\]
Calcule la media, mediana y moda.

\textbf{Desarrollo.}\\
Media:
\[
\bar{x}=\frac{220+240+250+260+300}{5}=\frac{1270}{5}=254.
\]

Mediana: dato central → 250.  

Moda: no hay repeticiones, por lo tanto no existe una moda.

\textbf{Resultado.}\\
Media = 254  
Mediana = 250  
Moda = No existe

\textbf{Fuente (APA):}\\
Lind, D., Marchal, W. \& Wathen, S. (2018). \textit{Estadística aplicada a los negocios}. McGraw-Hill.


\subsection*{Ejercicio 2}
\textbf{Enunciado.}\\
Encuentre la media, mediana y moda de los datos:  
\[
\{12,\; 15,\; 15,\; 18,\; 20,\; 21\}.
\]

\textbf{Desarrollo.}\\
Media:
\[
\bar{x}=\frac{12+15+15+18+20+21}{6}
=\frac{101}{6}
\approx 16.83.
\]

Mediana:  
Datos ordenados: ya ordenados.  
Posición 3 y 4:
\[
\frac{15+18}{2}=16.5.
\]

Moda: 15.

\textbf{Resultado.}\\
Media $\approx$ 16.83  
Mediana = 16.5  
Moda = 15

\textbf{Fuente (APA):}\\
Triola, M. (2014). \textit{Estadística}. Pearson Educación.


\newpage

% -------------------------------------------------------------------
% -----------------------   TEMA 10  --------------------------------
% -------------------------------------------------------------------

\section{Media Armónica}

\subsection*{Ejercicio 1}
\textbf{Enunciado.}\\
Un vehículo recorrió la mitad del trayecto a 40 km/h y la otra mitad a 60 km/h. Calcule la media armónica de las velocidades.

\textbf{Desarrollo.}\\
Fórmula:
\[
H=\frac{n}{\sum_{i=1}^{n}\frac{1}{x_i}}.
\]

\[
H=\frac{2}{\frac{1}{40}+\frac{1}{60}}
=\frac{2}{0.025+0.01667}
=\frac{2}{0.04167}
\approx 48.
\]

\textbf{Resultado.}\\
La media armónica es aproximadamente 48 km/h.

\textbf{Fuente (APA):}\\
Wackerly, D., Mendenhall, W. \& Scheaffer, R. (2014). \textit{Matemáticas para la estadística}. Cengage Learning.


\subsection*{Ejercicio 2}
\textbf{Enunciado.}\\
Un trabajador produce 5 unidades por hora durante la primera mitad del turno y 8 unidades por hora en la segunda mitad. Calcule la media armónica de productividad.

\textbf{Desarrollo.}\\
\[
H=\frac{2}{\frac{1}{5}+\frac{1}{8}}
=\frac{2}{0.2+0.125}
=\frac{2}{0.325}
\approx 6.15.
\]

\textbf{Resultado.}\\
La media armónica es aproximadamente 6.15 unidades por hora.

\textbf{Fuente (APA):}\\
Montgomery, D. (2013). \textit{Probabilidad y estadística para ingeniería}. Wiley.


\newpage

% -------------------------------------------------------------------
% -----------------------   TEMA 11  --------------------------------
% -------------------------------------------------------------------

\section{Mediana}

\subsection*{Ejercicio 1}
\textbf{Enunciado.}\\
Los siguientes tiempos de atención (min) fueron registrados en un centro de servicio:  
\[
\{4,\; 6,\; 8,\; 9,\; 10,\; 12,\; 15\}.
\]
Calcule la mediana.

\textbf{Desarrollo.}\\
Los datos ya están ordenados.  
Número de datos: $n = 7$ (impar).  
La mediana es el valor en la posición:
\[
\frac{n+1}{2} = \frac{7+1}{2} = 4.
\]
El cuarto dato es 9.

\textbf{Resultado.}\\
La mediana es **9 minutos**.

\textbf{Fuente (APA):}\\
Triola, M. (2014). \textit{Estadística}. Pearson Educación.


\subsection*{Ejercicio 2}
\textbf{Enunciado.}\\
Calcule la mediana de los datos:  
\[
\{12,\; 14,\; 16,\; 18,\; 20,\; 22\}.
\]

\textbf{Desarrollo.}\\
$n=6$ (par).  
La mediana es el promedio entre las posiciones 3 y 4.

\[
\text{Mediana} = \frac{16 + 18}{2} = 17.
\]

\textbf{Resultado.}\\
La mediana es **17**.

\textbf{Fuente (APA):}\\
Lind, D., Marchal, W. \& Wathen, S. (2018). \textit{Estadística aplicada a los negocios}. McGraw-Hill.


\newpage

% -------------------------------------------------------------------
% -----------------------   TEMA 12  --------------------------------
% -------------------------------------------------------------------

\section{Desviación Estándar}

\subsection*{Ejercicio 1}
\textbf{Enunciado.}\\
Calcule la desviación estándar para los datos:  
\[
\{5,\; 7,\; 7,\; 8,\; 10\}.
\]

\textbf{Desarrollo.}\\
Media:
\[
\bar{x}=\frac{5+7+7+8+10}{5}=7.4.
\]

Fórmula:
\[
s=\sqrt{\frac{\sum (x_i - \bar{x})^2}{n-1}}.
\]

Cálculos:
\[
(5-7.4)^2 = 5.76,\;
(7-7.4)^2 = 0.16,\;
(7-7.4)^2=0.16,
\]
\[
(8-7.4)^2 = 0.36,\;
(10-7.4)^2 = 6.76.
\]

\[
\sum = 13.2.
\]

\[
s=\sqrt{\frac{13.2}{4}}= \sqrt{3.3} \approx 1.82.
\]

\textbf{Resultado.}\\
La desviación estándar es aproximadamente **1.82**.

\textbf{Fuente (APA):}\\
Triola, M. (2014). \textit{Estadística}. Pearson Educación.


\subsection*{Ejercicio 2}
\textbf{Enunciado.}\\
Calcule la desviación estándar de:  
\[
\{10,\; 12,\; 15,\; 18\}.
\]

\textbf{Desarrollo.}\\
Media:
\[
\bar{x}=\frac{10+12+15+18}{4}=13.75.
\]

\[
s=\sqrt{\frac{(10-13.75)^2+(12-13.75)^2+(15-13.75)^2+(18-13.75)^2}{3}}.
\]

Cálculos:
\[
(10-13.75)^2=14.06,\;
(12-13.75)^2=3.06,
\]
\[
(15-13.75)^2=1.56,\;
(18-13.75)^2=18.06.
\]

\[
\sum=36.74.
\]

\[
s=\sqrt{\frac{36.74}{3}}=\sqrt{12.25}=3.5.
\]

\textbf{Resultado.}\\
La desviación estándar es **3.5**.

\textbf{Fuente (APA):}\\
Montgomery, D. (2013). \textit{Probabilidad y estadística aplicada a la ingeniería}. Wiley.


\newpage

% -------------------------------------------------------------------
% -----------------------   TEMA 13  --------------------------------
% -------------------------------------------------------------------

\section{Distribución No Nominal}

\subsection*{Ejercicio 1}
\textbf{Enunciado.}\\
Una encuesta clasifica el nivel de satisfacción en:  
1 = Muy bajo,  
2 = Bajo,  
3 = Medio,  
4 = Alto,  
5 = Muy alto.  
Los resultados fueron:  
\[
\{3,\;4,\;4,\;5,\;2,\;3,\;4,\;5\}.
\]
Determine el tipo de variable y su distribución.

\textbf{Desarrollo.}\\
La variable tiene categorías ordenadas → es **ordinal (no nominal)**.  
Frecuencias:
\[
2(1),\; 3(2),\; 4(3),\; 5(2).
\]

\textbf{Resultado.}\\
Distribución ordinal con mayor frecuencia en la categoría 4 (alto).

\textbf{Fuente (APA):}\\
Triola, M. (2014). \textit{Estadística}. Pearson Educación.


\subsection*{Ejercicio 2}
\textbf{Enunciado.}\\
Una evaluación académica clasifica el desempeño como:  
1 = Insuficiente,  
2 = Regular,  
3 = Bueno,  
4 = Excelente.  
Datos:  
\[
\{2,\;3,\;3,\;4,\;3,\;2,\;1,\;3\}.
\]

\textbf{Desarrollo.}\\
Variable categórica ordenada → ordinal.  

Frecuencias:
\[
1(1),\; 2(2),\; 3(4),\; 4(1).
\]

\textbf{Resultado.}\\
La categoría más frecuente es **3 (Bueno)**.

\textbf{Fuente (APA):}\\
Lind, D., Marchal, W. \& Wathen, S. (2018). \textit{Estadística aplicada a los negocios}. McGraw-Hill.


\newpage

% -------------------------------------------------------------------
% -----------------------   TEMA 14  --------------------------------
% -------------------------------------------------------------------

\section{Distribución Hipergeométrica}

\subsection*{Ejercicio 1}
\textbf{Enunciado.}\\
Una caja contiene 10 componentes buenos y 5 defectuosos.  
Si se extraen 4 sin reemplazo, calcule la probabilidad de obtener exactamente 2 defectuosos.

\textbf{Desarrollo.}\\
Fórmula:
\[
P(X=k)=\frac{\binom{D}{k}\binom{B}{n-k}}{\binom{N}{n}}.
\]

Datos: $D=5$, $B=10$, $N=15$, $n=4$, $k=2$.

\[
P=\frac{\binom{5}{2}\binom{10}{2}}{\binom{15}{4}}
=\frac{10 \times 45}{1365}
=\frac{450}{1365}
\approx 0.3296.
\]

\textbf{Resultado.}\\
La probabilidad es aproximadamente **0.3296**.

\textbf{Fuente (APA):}\\
Wackerly, D., Mendenhall, W. \& Scheaffer, R. (2014). \textit{Matemáticas para la estadística}. Cengage Learning.


\subsection*{Ejercicio 2}
\textbf{Enunciado.}\\
Una urna contiene 12 bolas azules y 8 rojas. Si se extraen 5 sin reemplazo, calcule la probabilidad de obtener 3 azules.

\textbf{Desarrollo.}\\
\[
P=\frac{\binom{12}{3}\binom{8}{2}}{\binom{20}{5}}.
\]

\[
\binom{12}{3}=220,\quad \binom{8}{2}=28,\quad \binom{20}{5}=15504.
\]

\[
P=\frac{220 \times 28}{15504}=\frac{6160}{15504}\approx 0.397.
\]

\textbf{Resultado.}\\
La probabilidad es **0.397**.

\textbf{Fuente (APA):}\\
Montgomery, D. (2013). \textit{Probabilidad y estadística para ingeniería}. Wiley.


\newpage

% -------------------------------------------------------------------
% -----------------------   TEMA 15  --------------------------------
% -------------------------------------------------------------------

\section{Distribución Exponencial}

\subsection*{Ejercicio 1}
\textbf{Enunciado.}\\
El tiempo entre llegadas de clientes a un banco sigue una distribución exponencial con $\lambda = 0.2$.  
Calcule la probabilidad de que el siguiente cliente llegue después de 10 minutos.

\textbf{Desarrollo.}\\
\[
P(X>t)=e^{-\lambda t}.
\]

\[
P(X>10)=e^{-0.2(10)}=e^{-2}\approx 0.1353.
\]

\textbf{Resultado.}\\
La probabilidad es **0.1353**.

\textbf{Fuente (APA):}\\
Ross, S. (2014). \textit{Introducción a la probabilidad y estadística}. Academic Press.


\subsection*{Ejercicio 2}
\textbf{Enunciado.}\\
Si el tiempo de vida de un componente electrónico sigue una distribución exponencial con $\lambda=0.05$, calcule la probabilidad de que dure menos de 30 horas.

\textbf{Desarrollo.}\\
\[
P(X<t)=1-e^{-\lambda t}.
\]

\[
P(X<30)=1-e^{-0.05(30)}=1-e^{-1.5}\approx 1-0.2231=0.7769.
\]

\textbf{Resultado.}\\
La probabilidad es **0.7769**.

\textbf{Fuente (APA):}\\
Rice, J. (2006). \textit{Mathematical statistics and data analysis}. Cengage Learning.


\newpage

% -------------------------------------------------------------------
% --------------------   BIBLIOGRAFÍA   -----------------------------
% -------------------------------------------------------------------

\section*{Referencias}

Daniel, W. (2012). \textit{Bioestadística}. McGraw-Hill.

Lind, D., Marchal, W., \& Wathen, S. (2018). \textit{Estadística aplicada a los negocios y la economía}. McGraw-Hill.

Matemóvil. (2022). \textit{Histogramas: ejemplos y ejercicios}. Recuperado de https://matemovil.com/histogramas-ejemplos-y-ejercicios/

Montgomery, D. (2013). \textit{Probabilidad y estadística aplicadas a la ingeniería}. Wiley.

Rice, J. (2006). \textit{Mathematical statistics and data analysis}. Cengage Learning.

Ross, S. (2014). \textit{Introducción a la probabilidad y estadística}. Academic Press.

Scribd. (2023). \textit{Ejercicios de cálculo de población finita e infinita}. Recuperado de https://es.scribd.com/document/621107138/Ejercicios-de-calculo-de-poblacion-finita-e-infinita

Scribd. (2023). \textit{Frecuencia acumulada: ejercicio en clase}. Recuperado de https://es.scribd.com/document/557677562/Frecuencia-Acumulada-Ejercicio-en-Clase

Scribd. (2023). \textit{Histogramas: ejercicios resueltos}. Recuperado de https://es.scribd.com/document/703612592/Histogramas-Ejercicios-Resueltos

Triola, M. (2014). \textit{Estadística}. Pearson Educación.

Wackerly, D., Mendenhall, W., \& Scheaffer, R. (2014). \textit{Matemáticas para la estadística}. Cengage Learning.


\end{document}
