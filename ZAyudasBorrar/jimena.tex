\documentclass[12pt]{article}

% ---------------------------------------------------------------
% PAQUETES
% ---------------------------------------------------------------
\usepackage[spanish]{babel}
\usepackage[utf8]{inputenc}
\usepackage{amsmath, amssymb}
\usepackage{geometry}
\usepackage{hyperref}
\usepackage{setspace}
\usepackage{graphicx}
\usepackage{siunitx}

\sisetup{
    locale = DE,                % coma como separador decimal
    output-decimal-marker = {,}
}

\geometry{
    a4paper,
    left=2.5cm,
    right=2.5cm,
    top=2.5cm,
    bottom=2.5cm
}

\onehalfspacing

\title{Análisis de Optimización en una Función de Producción}
\author{Ejemplo Aplicado para Cálculo Multivariable}
\date{}

\begin{document}

\maketitle

\section*{Introducción}

El análisis de funciones de producción de dos variables permite identificar
combinaciones de insumos que potencian la eficiencia en procesos industriales.
Mediante derivadas parciales, puntos críticos y el uso de la matriz Hessiana
es posible clasificar dichos puntos como máximos, mínimos o puntos de silla.
En este documento se desarrolla, paso a paso, un ejemplo de aplicación en el
contexto de la producción de galletas gourmet.

\section*{Planteamiento del problema}

Una empresa produce galletas gourmet utilizando dos insumos principales:
\begin{itemize}
    \item \(x\): cantidad de harina (en kg),
    \item \(y\): cantidad de mantequilla (en kg).
\end{itemize}

La función de producción se modela mediante:
\[
P(x,y) = -2x^{2} - y^{2} + 4xy + 20x + 15y.
\]

El objetivo es:
\begin{quote}
    Determinar los puntos críticos de \(P(x,y)\) y clasificarlos
    (máximo local, mínimo local o punto de silla) utilizando el criterio
    de la matriz Hessiana.
\end{quote}

% ===============================================================
\section*{Cálculo de derivadas parciales de primer orden}

\subsection*{Derivada parcial respecto a \(x\)}

Partimos de la función:
\[
P(x,y) = -2x^{2} - y^{2} + 4xy + 20x + 15y.
\]

Derivamos término a término respecto a \(x\), tratando \(y\) como constante:

\begin{align*}
\frac{\partial}{\partial x}(-2x^{2}) &= -4x,\\[4pt]
\frac{\partial}{\partial x}(-y^{2})  &= 0 \quad (\text{no depende de } x),\\[4pt]
\frac{\partial}{\partial x}(4xy)    &= 4y,\\[4pt]
\frac{\partial}{\partial x}(20x)    &= 20,\\[4pt]
\frac{\partial}{\partial x}(15y)    &= 0 \quad (\text{no depende de } x).
\end{align*}

Por lo tanto,
\[
P_x(x,y) = -4x + 4y + 20.
\]

\subsection*{Derivada parcial respecto a \(y\)}

Nuevamente, partimos de:
\[
P(x,y) = -2x^{2} - y^{2} + 4xy + 20x + 15y.
\]

Derivamos término a término respecto a \(y\), tratando \(x\) como constante:

\begin{align*}
\frac{\partial}{\partial y}(-2x^{2}) &= 0 \quad (\text{no depende de } y),\\[4pt]
\frac{\partial}{\partial y}(-y^{2})  &= -2y,\\[4pt]
\frac{\partial}{\partial y}(4xy)    &= 4x,\\[4pt]
\frac{\partial}{\partial y}(20x)    &= 0 \quad (\text{no depende de } y),\\[4pt]
\frac{\partial}{\partial y}(15y)    &= 15.
\end{align*}

Entonces,
\[
P_y(x,y) = -2y + 4x + 15.
\]

% ===============================================================
\section*{Determinación de puntos críticos}

Los puntos críticos se obtienen resolviendo el sistema:
\[
P_x(x,y) = 0,
\qquad
P_y(x,y) = 0.
\]

Es decir:
\[
\begin{cases}
-4x + 4y + 20 = 0,\\[4pt]
-2y + 4x + 15 = 0.
\end{cases}
\]

\subsection*{Resolución del sistema paso a paso}

\textbf{Primera ecuación:}
\[
-4x + 4y + 20 = 0.
\]

Restamos \(20\) a ambos lados:
\[
-4x + 4y = -20.
\]

Podemos dividir toda la ecuación entre \(4\):
\[
-x + y = -5.
\]

Reordenando:
\[
y = x - 5.
\]

\medskip

\textbf{Sustitución en la segunda ecuación:}

La segunda ecuación es:
\[
-2y + 4x + 15 = 0.
\]

Sustituimos \(y = x - 5\):
\begin{align*}
-2(x - 5) + 4x + 15 &= 0,\\[4pt]
-2x + 10 + 4x + 15 &= 0,\\[4pt]
( -2x + 4x ) + (10 + 15) &= 0,\\[4pt]
2x + 25 &= 0.
\end{align*}

Despejamos \(x\):
\[
2x = -25
\quad\Rightarrow\quad
x = -\frac{25}{2}.
\]

En forma decimal:
\[
x = -\num{12.5} \approx -\num{12,5}.
\]

Ahora sustituimos en \(y = x - 5\):
\begin{align*}
y &= -\frac{25}{2} - 5,\\[4pt]
  &= -\frac{25}{2} - \frac{10}{2},\\[4pt]
  &= -\frac{35}{2}.
\end{align*}

En forma decimal:
\[
y = -\num{17.5} \approx -\num{17,5}.
\]

\subsection*{Punto crítico encontrado}

El sistema presenta un único punto crítico:
\[
(x^{\ast},y^{\ast}) = \left(-\frac{25}{2}, -\frac{35}{2}\right)
\approx \left(\num{-12,5},\, \num{-17,5}\right).
\]

% ===============================================================
\section*{Derivadas de segundo orden y matriz Hessiana}

Calculamos ahora las derivadas de segundo orden.

\subsection*{Segunda derivada respecto a \(x\)}

Partimos de:
\[
P_x(x,y) = -4x + 4y + 20.
\]

Derivamos respecto a \(x\):
\[
P_{xx} = \frac{\partial}{\partial x}(-4x + 4y + 20) = -4.
\]

\subsection*{Segunda derivada respecto a \(y\)}

Partimos de:
\[
P_y(x,y) = -2y + 4x + 15.
\]

Derivamos respecto a \(y\):
\[
P_{yy} = \frac{\partial}{\partial y}(-2y + 4x + 15) = -2.
\]

\subsection*{Derivada mixta}

Podemos derivar \(P_x\) respecto a \(y\) o \(P_y\) respecto a \(x\).

\[
P_x(x,y) = -4x + 4y + 20
\quad\Rightarrow\quad
P_{xy} = \frac{\partial}{\partial y}(-4x + 4y + 20) = 4.
\]

De forma equivalente:
\[
P_y(x,y) = -2y + 4x + 15
\quad\Rightarrow\quad
P_{yx} = \frac{\partial}{\partial x}(-2y + 4x + 15) = 4.
\]

Por simetría en funciones suficientemente suaves,
\[
P_{xy} = P_{yx}.
\]

\subsection*{Matriz Hessiana}

La matriz Hessiana \(H\) de \(P\) es:
\[
H(x,y) =
\begin{pmatrix}
P_{xx} & P_{xy} \\
P_{yx} & P_{yy}
\end{pmatrix}
=
\begin{pmatrix}
-4 & 4 \\
4  & -2
\end{pmatrix}.
\]

El discriminante (determinante de la matriz Hessiana) se define como:
\[
D = P_{xx}P_{yy} - (P_{xy})^{2}.
\]

Sustituimos:
\begin{align*}
D &= (-4)(-2) - (4)^{2},\\[4pt]
  &= 8 - 16,\\[4pt]
  &= -8.
\end{align*}

% ===============================================================
\section*{Clasificación del punto crítico}

El criterio de la matriz Hessiana establece:

\begin{itemize}
    \item Si \(D > 0\) y \(P_{xx} < 0\), el punto crítico es un \textbf{máximo local}.
    \item Si \(D > 0\) y \(P_{xx} > 0\), el punto crítico es un \textbf{mínimo local}.
    \item Si \(D < 0\), el punto crítico es un \textbf{punto de silla}.
    \item Si \(D = 0\), el criterio es inconcluso.
\end{itemize}

En nuestro caso:
\[
D = -8 < 0.
\]

Por lo tanto, el punto crítico
\[
\left(-\frac{25}{2}, -\frac{35}{2}\right)
\]
es un:
\[
\boxed{\text{punto de silla}}.
\]

Esto significa que la función \(P(x,y)\) tiene direcciones en las que aumenta
y otras en las que disminuye alrededor del punto crítico; por ello, no se
trata ni de un máximo ni de un mínimo local.

% ===============================================================
\section*{Valor de la función en el punto crítico}

Aunque el punto crítico es de silla, podemos evaluar la función para
completar el análisis.

\[
P(x,y) = -2x^{2} - y^{2} + 4xy + 20x + 15y.
\]

Sustituimos \(x = -\dfrac{25}{2}\) y \(y = -\dfrac{35}{2}\).

\subsection*{Sustitución detallada}

\begin{align*}
P\left(-\frac{25}{2}, -\frac{35}{2}\right)
&= -2\left(-\frac{25}{2}\right)^{2}
   -\left(-\frac{35}{2}\right)^{2}
   +4\left(-\frac{25}{2}\right)\left(-\frac{35}{2}\right)
   +20\left(-\frac{25}{2}\right)
   +15\left(-\frac{35}{2}\right).
\end{align*}

Calculamos cada término:

\[
\left(-\frac{25}{2}\right)^{2}
= \frac{625}{4},
\qquad
\left(-\frac{35}{2}\right)^{2}
= \frac{1225}{4}.
\]

Entonces:
\begin{align*}
-2\left(-\frac{25}{2}\right)^{2}
&= -2 \cdot \frac{625}{4}
= -\frac{1250}{4}
= -\frac{625}{2},
\\[6pt]
-\left(-\frac{35}{2}\right)^{2}
&= -\frac{1225}{4},
\\[6pt]
4\left(-\frac{25}{2}\right)\left(-\frac{35}{2}\right)
&= 4 \cdot \frac{875}{4}
= 875,
\\[6pt]
20\left(-\frac{25}{2}\right)
&= -\frac{500}{2}
= -250,
\\[6pt]
15\left(-\frac{35}{2}\right)
&= -\frac{525}{2}.
\end{align*}

Agrupamos todos los términos:
\begin{align*}
P\left(-\frac{25}{2}, -\frac{35}{2}\right)
&= \left(-\frac{625}{2}\right)
   + \left(-\frac{1225}{4}\right)
   + 875
   - 250
   - \frac{525}{2}.
\end{align*}

Llevamos todo a denominador común \(4\):

\[
-\frac{625}{2} = -\frac{1250}{4},
\qquad
-\frac{525}{2} = -\frac{1050}{4},
\qquad
875 = \frac{3500}{4},
\qquad
-250 = -\frac{1000}{4}.
\]

Entonces:
\begin{align*}
P\left(-\frac{25}{2}, -\frac{35}{2}\right)
&= -\frac{1250}{4}
   -\frac{1225}{4}
   +\frac{3500}{4}
   -\frac{1000}{4}
   -\frac{1050}{4}.
\end{align*}

Sumamos numeradores:
\[
-1250 - 1225 + 3500 - 1000 - 1050
= -3025 + 3500 - 2050
= 475 - 2050
= -1575.
\]

Por lo tanto:
\[
P\left(-\frac{25}{2}, -\frac{35}{2}\right)
= -\frac{1575}{4}
= -393{,}75.
\]

(En la práctica, el valor exacto es menos relevante que la clasificación
del punto crítico como punto de silla.)

% ===============================================================
\section*{Interpretación y comentario económico}

Desde un punto de vista económico, las cantidades de insumos
\(x = \num{-12,5}\,\text{kg}\) y \(y = \num{-17,5}\,\text{kg}\) carecen de sentido físico,
ya que representan valores negativos. Esto sugiere que, además del análisis
matemático, es necesario imponer restricciones al modelo, por ejemplo:
\[
x \geq 0,
\qquad
y \geq 0.
\]

Bajo estas condiciones, el problema de optimización debería replantearse
considerando también la frontera del dominio (métodos de optimización
con restricciones), lo cual es habitual en modelos de producción reales.

% ===============================================================
\section*{Conclusión}

El desarrollo detallado de la función de producción
\[
P(x,y) = -2x^{2} - y^{2} + 4xy + 20x + 15y
\]
permitió:
\begin{enumerate}
    \item Calcular de manera sistemática las derivadas parciales de primer
          y segundo orden.
    \item Determinar el punto crítico resolviendo un sistema lineal
          de dos ecuaciones con dos incógnitas.
    \item Construir la matriz Hessiana y evaluar su determinante
          para aplicar el criterio de clasificación.
    \item Concluir que el punto crítico encontrado es un
          \textbf{punto de silla}, por lo que no corresponde a un máximo
          ni a un mínimo local.
    \item Discutir la importancia de considerar restricciones en el
          dominio para obtener soluciones económicamente viables.
\end{enumerate}

Este ejemplo ilustra cómo las herramientas del cálculo multivariable
son fundamentales para interpretar y tomar decisiones en modelos de
producción y optimización de recursos.

% ===============================================================
\section*{Referencias bibliográficas}

\begin{itemize}
    \item Chiang, A., \& Wainwright, K. (2005).
          \textit{Fundamentos de economía matemática}. McGraw-Hill.
    \item Stewart, J. (2016).
          \textit{Cálculo de varias variables}. Cengage Learning.
    \item Lay, D. (2015).
          \textit{Álgebra Lineal y sus aplicaciones}. Pearson.
\end{itemize}

\end{document}