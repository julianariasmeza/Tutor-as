\documentclass[11pt,a4paper]{article}

\usepackage[utf8]{inputenc}
\usepackage[T1]{fontenc}
\usepackage[spanish]{babel}
\usepackage[left=2.5cm,right=2.5cm,top=2.5cm,bottom=2.5cm]{geometry}
\usepackage{enumitem}
\usepackage{parskip}
\usepackage{titlesec}
\usepackage{hyperref}
\usepackage{xcolor}
\usepackage{multicol}
\usepackage{fontawesome}

\titleformat{\section}{\Large\bfseries\sffamily\color{blue!60!black}}{}{0em}{}
\titleformat{\subsection}{\normalsize\bfseries\sffamily\color{black}}{}{0em}{}

\hypersetup{
    colorlinks=true,
    linkcolor=blue!70!black,
    urlcolor=blue!70!black
}

\renewcommand{\labelitemi}{\faAngleRight}

\begin{document}

\begin{center}
    {\LARGE \textbf{Julián Arias Meza}} \\
    \vspace{1mm}
    
    Paraíso, Cartago \\
    \href{mailto:juli.cone@gmail.com}{juli.cone@gmail.com} $\mid$ \href{tel:+50670769371}{+506 7076 9371}
\end{center}

\hrule \vspace{1em}

\section*{\faBullseye\ Objetivo profesional}
Aplicar mis conocimientos en matemática, física y metodologías de enseñanza para potenciar el aprendizaje de estudiantes de secundaria y universitarios, fomentando el razonamiento lógico, el pensamiento crítico y el uso de herramientas digitales innovadoras.

\vspace{0.5em}

\section*{\faCheckCircle\ Aptitudes}
\begin{multicols}{2}
\begin{itemize}[leftmargin=*]
    \item Trabajo autónomo y colaborativo
    \item Capacidad de análisis
    \item Paciencia y empatía docente
    \item Adaptabilidad pedagógica
    \item Resolución de problemas
    \item Comunicación efectiva
\end{itemize}
\end{multicols}

\vspace{0.5em}

\section*{\faCogs\ Habilidades y competencias técnicas}
\begin{itemize}[leftmargin=*]
    \item Lenguajes: Python, R, SQL, LaTeX, MATLAB
    \item Herramientas: Excel avanzado, Google Colab
    \item Competencias: análisis de datos, redacción científica, docencia virtual
    \item Creación de material didáctico con IA
    \item Diseño de dashboards y actividades interactivas
\end{itemize}

\vspace{0.5em}

\section*{\faWrench\ Proyectos destacados}
\textbf{Asistente educativo con IA (en desarrollo)}\\
Sistema para responder consultas educativas por WhatsApp con Python y ChatGPT.

\vspace{0.5em}

\section*{\faBriefcase\ Experiencia laboral}
\subsection*{Tutor privado independiente}
\textit{2008 -- presente}\\
He brindado tutorías personalizadas en matemática, física y química para estudiantes de secundaria y universidad. Asistencia en cursos como:
\begin{itemize}
    \item Matemática elemental, cálculo, álgebra lineal, estadística
    \item Ecuaciones diferenciales y física universitaria
    \item Preparación para pruebas de admisión y nivelación
    \item Diseño de materiales didácticos con \LaTeX, Python y Excel
\end{itemize}

\subsection*{Colegio Bilingüe Jorge Volio Jiménez}
\textit{2013 -- 2018}\\
Profesor de matemáticas y apoyo interdisciplinario en ciencias para secundaria.

\subsection*{Ministerio de Educación Pública (MEP)}
\textit{Educación para Adultos, 2017 -- 2021}\\
Profesor de matemática en modalidad nocturna.

\subsection*{Universidad de Costa Rica (UCR)}
\textit{Recinto de Paraíso, 2010 -- 2013}\\
\begin{itemize}
    \item Impartí tutorías de nivelación en matemática para estudiantes de primer ingreso.
    \item Fui profesor asistente en cursos de matemática discreta y cálculo.
    \item Colaboré como asistente académico en la Escuela de Matemática.
\end{itemize}

\vspace{0.5em}

\section*{\faGraduationCap\ Formación académica}
\subsection*{Universidad de Costa Rica (UCR)}
\begin{itemize}
    \item Título de Profesorado en Enseñanza de la Matemática (2021)
    \item Estudiante de Bachillerato en Enseñanza de la Matemática (próximo a concluir)
    \item Estudios de Bachillerato en Física
\end{itemize}

\subsection*{Tecnológico de Costa Rica (TEC)}
Estudiante de Ingeniería Física desde 2019

\subsection*{Educación primaria y secundaria}
\begin{itemize}
    \item Escuela Fray José Antonio de Liendo y Goicoechea
    \item Liceo de Paraíso
\end{itemize}

\end{document}
