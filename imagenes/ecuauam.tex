\documentclass[12pt]{article}
\usepackage[utf8]{inputenc}
\usepackage[spanish]{babel}
\usepackage{amsmath, amssymb}
\usepackage{geometry}
\geometry{margin=2.5cm}
\usepackage{fancyhdr}
\pagestyle{fancy}
\fancyhf{}
\rhead{Ecuaciones Diferenciales}
\lhead{Prueba Práctica \#1}
\cfoot{\thepage}

\title{Parte 1: Orden y grado}
\author{Nombre del estudiante: \underline{\hspace{6cm}}}
\date{}

\begin{document}

\maketitle

\section*{Parte 1. Orden y grado de ecuaciones diferenciales (6 puntos)}

Indique el orden y el grado de las siguientes ecuaciones diferenciales:

\begin{enumerate}
    \item[a)] 
    \[
    \left(\frac{d^2 y}{dx^2}\right)^2 \cdot \sin x + 3x y \frac{dy}{dx} + y e^x = 0
    \]

    \textbf{Orden:} 2 \\
    \textbf{Grado:} 2 \\
    \textit{Justificación:} La derivada de mayor orden es \( \frac{d^2 y}{dx^2} \), y está elevada al cuadrado. No hay raíces ni fracciones con derivadas.

    \item[b)] 
    \[
    \frac{d^2 y}{dx^2} + 5x^2 \left( \frac{dy}{dx} \right)^3 + 6y = 0
    \]

    \textbf{Orden:} 2 \\
    \textbf{Grado:} 1 \\
    \textit{Justificación:} La derivada de mayor orden es \( \frac{d^2 y}{dx^2} \) y aparece en primer grado. Aunque \( \frac{dy}{dx} \) está al cubo, el grado se define respecto a la derivada de mayor orden.

    \item[c)] 
    \[
    x^2 \sqrt{y''} + 3x - 2y' = 3 \sin x
    \]

    \textbf{Orden:} 2 \\
    \textbf{Grado:} No definido \\
    \textit{Justificación:} La derivada de segundo orden aparece dentro de una raíz cuadrada: \( \sqrt{y''} = (y'')^{1/2} \), por lo que el grado no es entero.
\end{enumerate}
\newpage
\section*{Parte 2. Verificación de soluciones (5 puntos)}

Dada la ecuación diferencial:

\[
x^2 y'' - 4x y' + 6y = 0
\]

\begin{enumerate}
    \item[a)] Demostrar que \( y_1(x) = x^2 \) y \( y_2(x) = x^3 \) son soluciones de la ecuación.

    \textbf{Para \( y_1(x) = x^2 \):}
    \begin{align*}
        y_1(x) &= x^2 \\
        y_1'(x) &= 2x \\
        y_1''(x) &= 2 \\
        \\
        x^2 y_1'' - 4x y_1' + 6y_1 &= x^2(2) - 4x(2x) + 6(x^2) \\
        &= 2x^2 - 8x^2 + 6x^2 \\
        &= (2 - 8 + 6)x^2 = 0
    \end{align*}
    Por lo tanto, \( y_1(x) \) es solución.

    \textbf{Para \( y_2(x) = x^3 \):}
    \begin{align*}
        y_2(x) &= x^3 \\
        y_2'(x) &= 3x^2 \\
        y_2''(x) &= 6x \\
        \\
        x^2 y_2'' - 4x y_2' + 6y_2 &= x^2(6x) - 4x(3x^2) + 6(x^3) \\
        &= 6x^3 - 12x^3 + 6x^3 \\
        &= (6 - 12 + 6)x^3 = 0
    \end{align*}
    Por lo tanto, \( y_2(x) \) también es solución.

    \item[b)] Verificar si \( y(x) = y_1(x) + y_2(x) = x^2 + x^3 \) es solución.

    \begin{align*}
        y(x) &= x^2 + x^3 \\
        y'(x) &= 2x + 3x^2 \\
        y''(x) &= 2 + 6x \\
        \\
        x^2 y'' - 4x y' + 6y &= x^2(2 + 6x) - 4x(2x + 3x^2) + 6(x^2 + x^3) \\
        &= (2x^2 + 6x^3) - (8x^2 + 12x^3) + (6x^2 + 6x^3) \\
        &= (2 - 8 + 6)x^2 + (6 - 12 + 6)x^3 \\
        &= 0x^2 + 0x^3 = 0
    \end{align*}

    Por lo tanto, \( y(x) = x^2 + x^3 \) también es solución.
\end{enumerate}
\section*{Parte 3. Método de separación de variables (17 puntos)}

\subsection*{Inciso a) \quad Valor: 7 puntos}

Resolver la ecuación diferencial:

\[
(xy^2 - x)\, dx + (x^2y + y)\, dy = 0
\]

\textbf{Paso 1: Reescribir y despejar \(\dfrac{dy}{dx}\)}

\[
(xy^2 - x)\, dx + (x^2y + y)\, dy = 0
\quad \Rightarrow \quad \frac{dy}{dx} = -\frac{xy^2 - x}{x^2y + y}
\]

\textbf{Paso 2: Factorizar y simplificar}

\[
\frac{dy}{dx} = -\frac{x(y^2 - 1)}{y(x^2 + 1)}
\]

\textbf{Paso 3: Separar las variables}

\[
y \cdot \frac{dy}{y^2 - 1} = -\frac{x\, dx}{x^2 + 1}
\]

\textbf{Paso 4: Integrar ambos lados}

\underline{Lado izquierdo:}
\[
\int \frac{y}{y^2 - 1}\, dy
\]

Realizamos un cambio de variable:

\[
u = y^2 - 1 \quad \Rightarrow \quad du = 2y\, dy \quad \Rightarrow \quad \frac{1}{2} du = y\, dy
\]

Entonces la integral se transforma en:

\[
\int \frac{y}{y^2 - 1}\, dy = \int \frac{1}{u} \cdot \frac{1}{2} du = \frac{1}{2} \int \frac{1}{u}\, du = \frac{1}{2} \ln|u| + C = \frac{1}{2} \ln|y^2 - 1| + C
\]

\underline{Lado derecho:}
\[
\int \frac{-x}{x^2 + 1}\, dx
\]

Realizamos el cambio:

\[
u = x^2 + 1 \quad \Rightarrow \quad du = 2x\, dx \quad \Rightarrow \quad \frac{1}{2} du = x\, dx
\]

Entonces la integral se transforma en:

\[
\int \frac{-x}{x^2 + 1}\, dx = -\int \frac{1}{u} \cdot \frac{1}{2} du = -\frac{1}{2} \int \frac{1}{u}\, du = -\frac{1}{2} \ln|u| + C = -\frac{1}{2} \ln(x^2 + 1) + C
\]

---

\textbf{Paso 5: Solución general}

Igualamos ambos resultados:

\[
\frac{1}{2} \ln|y^2 - 1| = -\frac{1}{2} \ln(x^2 + 1) + C
\]

Multiplicamos toda la ecuación por 2:

\[
\ln|y^2 - 1| = -\ln(x^2 + 1) + C_1
\]

Aplicamos propiedad de logaritmos:

\[
\ln\left( |y^2 - 1| \cdot (x^2 + 1) \right) = C_1
\Rightarrow 
|y^2 - 1| \cdot (x^2 + 1) = C
\]

\textbf{Solución final:}
\[
\boxed{(x^2 + 1)\,|y^2 - 1| = C}
\]
\subsection*{Inciso b) \quad Valor: 5 puntos}

Resolver la ecuación:

\[
\frac{\sqrt{1 + t^2}}{1 + y} \cdot y' = -t
\]

\textbf{Paso 1: Reescribir como derivada}

\[
\frac{\sqrt{1 + t^2}}{1 + y} \cdot \frac{dy}{dt} = -t
\]

\textbf{Paso 2: Separar las variables}

Multiplicamos ambos lados por \( 1 + y \) y dividimos por \( \sqrt{1 + t^2} \):

\[
\frac{dy}{1 + y} = \frac{-t}{\sqrt{1 + t^2}}\, dt
\]

\textbf{Paso 3: Integrar ambos lados}

\underline{Lado izquierdo:}
\[
\int \frac{dy}{1 + y}
= \ln|1 + y|
\]

\underline{Lado derecho:}

\[
\int \frac{-t}{\sqrt{1 + t^2}}\, dt
\]

Sustitución:
\[
u = 1 + t^2 \quad \Rightarrow \quad du = 2t\, dt \quad \Rightarrow \quad \frac{1}{2} du = t\, dt
\]

Entonces:
\[
\int \frac{-t}{\sqrt{1 + t^2}}\, dt = -\int \frac{1}{\sqrt{u}} \cdot \frac{1}{2} du
= -\frac{1}{2} \int u^{-1/2}\, du
\]

\[
= -\frac{1}{2} \cdot \frac{u^{1/2}}{1/2} = -u^{1/2} = -\sqrt{1 + t^2}
\]

\textbf{Paso 4: Solución general}

\[
\ln|1 + y| = -\sqrt{1 + t^2} + C
\]


\subsection*{Inciso c) \quad Resolver \(\dfrac{dy}{dx} = e^{3x + 2y}\) por separación de variables}

\textbf{Paso 1: Reescribir y separar variables}

\[
\frac{dy}{dx} = e^{3x + 2y} = e^{3x} \cdot e^{2y}
\]

\[
\Rightarrow \frac{dy}{e^{2y}} = e^{3x} dx
\]

---

\textbf{Paso 2: Integrar ambos lados}

\underline{Lado izquierdo:}

\[
\int \frac{dy}{e^{2y}} = \int e^{-2y} dy = \frac{e^{-2y}}{-2}
\]

\underline{Lado derecho:}

\[
\int e^{3x} dx = \frac{e^{3x}}{3}
\]

---

\textbf{Paso 3: Igualamos las integrales}

\[
\frac{e^{-2y}}{-2} = \frac{e^{3x}}{3} + C
\quad \Rightarrow \quad
e^{-2y} = -\frac{2}{3} e^{3x} + C_1
\]

---

\textbf{Paso 4: Solución explícita (opcional)}

\[
-2y = \ln\left(-\frac{2}{3} e^{3x} + C_1\right)
\quad \Rightarrow \quad
y = -\frac{1}{2} \ln\left(-\frac{2}{3} e^{3x} + C_1\right)
\]



\end{document}
